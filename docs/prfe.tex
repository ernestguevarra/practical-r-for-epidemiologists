\documentclass[]{book}
\usepackage{lmodern}
\usepackage{amssymb,amsmath}
\usepackage{ifxetex,ifluatex}
\usepackage{fixltx2e} % provides \textsubscript
\ifnum 0\ifxetex 1\fi\ifluatex 1\fi=0 % if pdftex
  \usepackage[T1]{fontenc}
  \usepackage[utf8]{inputenc}
\else % if luatex or xelatex
  \ifxetex
    \usepackage{mathspec}
  \else
    \usepackage{fontspec}
  \fi
  \defaultfontfeatures{Ligatures=TeX,Scale=MatchLowercase}
\fi
% use upquote if available, for straight quotes in verbatim environments
\IfFileExists{upquote.sty}{\usepackage{upquote}}{}
% use microtype if available
\IfFileExists{microtype.sty}{%
\usepackage{microtype}
\UseMicrotypeSet[protrusion]{basicmath} % disable protrusion for tt fonts
}{}
\usepackage[margin=1in]{geometry}
\usepackage{hyperref}
\hypersetup{unicode=true,
            pdftitle={Practical R for Epidemiologists},
            pdfauthor={Mark Myatt},
            pdfborder={0 0 0},
            breaklinks=true}
\urlstyle{same}  % don't use monospace font for urls
\usepackage{natbib}
\bibliographystyle{apalike}
\usepackage{color}
\usepackage{fancyvrb}
\newcommand{\VerbBar}{|}
\newcommand{\VERB}{\Verb[commandchars=\\\{\}]}
\DefineVerbatimEnvironment{Highlighting}{Verbatim}{commandchars=\\\{\}}
% Add ',fontsize=\small' for more characters per line
\usepackage{framed}
\definecolor{shadecolor}{RGB}{248,248,248}
\newenvironment{Shaded}{\begin{snugshade}}{\end{snugshade}}
\newcommand{\KeywordTok}[1]{\textcolor[rgb]{0.13,0.29,0.53}{\textbf{#1}}}
\newcommand{\DataTypeTok}[1]{\textcolor[rgb]{0.13,0.29,0.53}{#1}}
\newcommand{\DecValTok}[1]{\textcolor[rgb]{0.00,0.00,0.81}{#1}}
\newcommand{\BaseNTok}[1]{\textcolor[rgb]{0.00,0.00,0.81}{#1}}
\newcommand{\FloatTok}[1]{\textcolor[rgb]{0.00,0.00,0.81}{#1}}
\newcommand{\ConstantTok}[1]{\textcolor[rgb]{0.00,0.00,0.00}{#1}}
\newcommand{\CharTok}[1]{\textcolor[rgb]{0.31,0.60,0.02}{#1}}
\newcommand{\SpecialCharTok}[1]{\textcolor[rgb]{0.00,0.00,0.00}{#1}}
\newcommand{\StringTok}[1]{\textcolor[rgb]{0.31,0.60,0.02}{#1}}
\newcommand{\VerbatimStringTok}[1]{\textcolor[rgb]{0.31,0.60,0.02}{#1}}
\newcommand{\SpecialStringTok}[1]{\textcolor[rgb]{0.31,0.60,0.02}{#1}}
\newcommand{\ImportTok}[1]{#1}
\newcommand{\CommentTok}[1]{\textcolor[rgb]{0.56,0.35,0.01}{\textit{#1}}}
\newcommand{\DocumentationTok}[1]{\textcolor[rgb]{0.56,0.35,0.01}{\textbf{\textit{#1}}}}
\newcommand{\AnnotationTok}[1]{\textcolor[rgb]{0.56,0.35,0.01}{\textbf{\textit{#1}}}}
\newcommand{\CommentVarTok}[1]{\textcolor[rgb]{0.56,0.35,0.01}{\textbf{\textit{#1}}}}
\newcommand{\OtherTok}[1]{\textcolor[rgb]{0.56,0.35,0.01}{#1}}
\newcommand{\FunctionTok}[1]{\textcolor[rgb]{0.00,0.00,0.00}{#1}}
\newcommand{\VariableTok}[1]{\textcolor[rgb]{0.00,0.00,0.00}{#1}}
\newcommand{\ControlFlowTok}[1]{\textcolor[rgb]{0.13,0.29,0.53}{\textbf{#1}}}
\newcommand{\OperatorTok}[1]{\textcolor[rgb]{0.81,0.36,0.00}{\textbf{#1}}}
\newcommand{\BuiltInTok}[1]{#1}
\newcommand{\ExtensionTok}[1]{#1}
\newcommand{\PreprocessorTok}[1]{\textcolor[rgb]{0.56,0.35,0.01}{\textit{#1}}}
\newcommand{\AttributeTok}[1]{\textcolor[rgb]{0.77,0.63,0.00}{#1}}
\newcommand{\RegionMarkerTok}[1]{#1}
\newcommand{\InformationTok}[1]{\textcolor[rgb]{0.56,0.35,0.01}{\textbf{\textit{#1}}}}
\newcommand{\WarningTok}[1]{\textcolor[rgb]{0.56,0.35,0.01}{\textbf{\textit{#1}}}}
\newcommand{\AlertTok}[1]{\textcolor[rgb]{0.94,0.16,0.16}{#1}}
\newcommand{\ErrorTok}[1]{\textcolor[rgb]{0.64,0.00,0.00}{\textbf{#1}}}
\newcommand{\NormalTok}[1]{#1}
\usepackage{longtable,booktabs}
\usepackage{graphicx,grffile}
\makeatletter
\def\maxwidth{\ifdim\Gin@nat@width>\linewidth\linewidth\else\Gin@nat@width\fi}
\def\maxheight{\ifdim\Gin@nat@height>\textheight\textheight\else\Gin@nat@height\fi}
\makeatother
% Scale images if necessary, so that they will not overflow the page
% margins by default, and it is still possible to overwrite the defaults
% using explicit options in \includegraphics[width, height, ...]{}
\setkeys{Gin}{width=\maxwidth,height=\maxheight,keepaspectratio}
\IfFileExists{parskip.sty}{%
\usepackage{parskip}
}{% else
\setlength{\parindent}{0pt}
\setlength{\parskip}{6pt plus 2pt minus 1pt}
}
\setlength{\emergencystretch}{3em}  % prevent overfull lines
\providecommand{\tightlist}{%
  \setlength{\itemsep}{0pt}\setlength{\parskip}{0pt}}
\setcounter{secnumdepth}{5}
% Redefines (sub)paragraphs to behave more like sections
\ifx\paragraph\undefined\else
\let\oldparagraph\paragraph
\renewcommand{\paragraph}[1]{\oldparagraph{#1}\mbox{}}
\fi
\ifx\subparagraph\undefined\else
\let\oldsubparagraph\subparagraph
\renewcommand{\subparagraph}[1]{\oldsubparagraph{#1}\mbox{}}
\fi

%%% Use protect on footnotes to avoid problems with footnotes in titles
\let\rmarkdownfootnote\footnote%
\def\footnote{\protect\rmarkdownfootnote}

%%% Change title format to be more compact
\usepackage{titling}

% Create subtitle command for use in maketitle
\newcommand{\subtitle}[1]{
  \posttitle{
    \begin{center}\large#1\end{center}
    }
}

\setlength{\droptitle}{-2em}

  \title{Practical R for Epidemiologists}
    \pretitle{\vspace{\droptitle}\centering\huge}
  \posttitle{\par}
    \author{Mark Myatt}
    \preauthor{\centering\large\emph}
  \postauthor{\par}
      \predate{\centering\large\emph}
  \postdate{\par}
    \date{2018-04-19}

\usepackage{booktabs}
\usepackage{color}
\usepackage{tcolorbox}
\usepackage{float}
\graphicspath{ {images/} }

\newenvironment{rmdremind}
  {\begin{tcolorbox}[width=\textwidth, 
                     colback = {white}, 
                     title = {\textbf{Remember}}, 
                     colbacktitle = lightgray,
                     coltitle = black]
  \begin{includegraphics}[scale = 1]{remind.png}
  \begin{itemize}}
  {\end{itemize}
  \end{includegraphics}
  \end{tcolorbox}}

\newenvironment{rmdnote}
  {\begin{tcolorbox}[width=\textwidth, 
                     colback = {white}, 
                     title = {\textbf{Note}}, 
                     colbacktitle = lightgray,
                     coltitle = black]
  \begin{includegraphics}[scale = 1]{pencil.png}}
  {\end{includegraphics}
  \end{tcolorbox}}
  
\newenvironment{rmdexercise}
  {\begin{tcolorbox}[width=\textwidth, 
                     colback = {white}, 
                     title = {\textbf{Exercise}}, 
                     colbacktitle = lightgray,
                     coltitle = black]
  \begin{includegraphics}[scale = 1]{exercise.png}}
  {\end{includegraphics}
  \end{tcolorbox}}
  
\newenvironment{rmdinfo}
  {\begin{tcolorbox}[width=\textwidth, 
                     colback = {white}, 
                     title = {\textbf{Info}}, 
                     colbacktitle = lightgray,
                     coltitle = black]
  \begin{includegraphics}[scale = 1]{info.png}}
  {\end{includegraphics}
  \end{tcolorbox}}  
  
\newenvironment{rmdwarning}
  {\begin{tcolorbox}[width=\textwidth, 
                     colback = {white}, 
                     title = {\textbf{Warning}}, 
                     colbacktitle = lightgray,
                     coltitle = black]
  \begin{includegraphics}[scale = 1]{warning.png}}
  {\end{includegraphics}
  \end{tcolorbox}}

\newenvironment{rmddownload}
  {\begin{tcolorbox}[width=\textwidth, 
                     colback = {white}, 
                     title = {\textbf{Download}}, 
                     colbacktitle = lightgray,
                     coltitle = black]
  \begin{includegraphics}[scale = 1]{download.png}}
  {\end{includegraphics}
  \end{tcolorbox}}

\usepackage{amsthm}
\newtheorem{theorem}{Theorem}[chapter]
\newtheorem{lemma}{Lemma}[chapter]
\theoremstyle{definition}
\newtheorem{definition}{Definition}[chapter]
\newtheorem{corollary}{Corollary}[chapter]
\newtheorem{proposition}{Proposition}[chapter]
\theoremstyle{definition}
\newtheorem{example}{Example}[chapter]
\theoremstyle{definition}
\newtheorem{exercise}{Exercise}[chapter]
\theoremstyle{remark}
\newtheorem*{remark}{Remark}
\newtheorem*{solution}{Solution}
\begin{document}
\maketitle

{
\setcounter{tocdepth}{1}
\tableofcontents
}
\hypertarget{chapter1}{%
\chapter{Introduction}\label{chapter1}}

\hypertarget{introducting-r}{%
\section{Introducting R}\label{introducting-r}}

\texttt{R} is a system for data manipulation, calculation, and graphics.
It provides:

\begin{itemize}
\item
  Facilities for data handling and storage
\item
  A large collection of tools for data analysis
\item
  Graphical facilities for data analysis and display
\item
  A simple but powerful programming language
\end{itemize}

\texttt{R} is often described as an environment for working with data.
This is in contrast to a \emph{package} which is a collection of very
specific tools. \texttt{R} is not strictly a statistics system but a
system that provides many classical and modern statistical procedures as
part of a broader data-analysis tool. This is an important difference
between \texttt{R} and other statistical systems. In \texttt{R} a
statistical analysis is usually performed as a series of steps with
intermediate results being stored in objects. Systems such as
\texttt{SPSS} and \texttt{SAS} provide copious output from (e.g.) a
regression analysis whereas \texttt{R} will give minimal output and
store the results of a fit for subsequent interrogation or use with
other \texttt{R} functions. This means that \texttt{R} can be tailored
to produce exactly the analysis and results that you want rather than
produce an analysis designed to fit all situations.

\texttt{R} is a language based product. This means that you interact
with \texttt{R} by typing commands such as:

\begin{Shaded}
\begin{Highlighting}[]
\KeywordTok{table}\NormalTok{(SEX, LIFE)}
\end{Highlighting}
\end{Shaded}

rather than by using menus, dialog boxes, selection lists, and buttons.
This may seem to be a drawback but it means that the system is
considerably more flexible than one that relies on menus, buttons, and
boxes. It also means that every stage of your data management and
analysis can be recorded and edited and re-run at a later date. It also
provides an audit trail for quality control purposes.

\texttt{R} is available under UNIX (including Linux), the Macintosh
operating system OS X, and Microsoft Windows. The method used for
starting \texttt{R} will vary from system to system. On UNIX systems you
may need to issue the \texttt{R} command in a terminal session or click
on an icon or menu option if your system has a windowing system. On
Macintosh systems \texttt{R} will be available as an application but can
also be run in a terminal session. On Microsoft Windows systems there
will usually be an icon on the Start menu or the desktop.

\texttt{R} is an open source system and is available under the \emph{GNU
general public license} (GPL) which means that it is available for free
but that there are some restrictions on how you are allowed to
distribute the system and how you may charge for bespoke data analysis
solutions written using the \texttt{R} system. Details of the general
public license are available from
\url{http://www.gnu.org/copyleft/gpl.html}.

\texttt{R} is available for download from
\url{http://www.r-project.org/}.

This is also the best place to get extension packages and documentation.
You may also subscribe to the \texttt{R} mailing lists from this site.
\texttt{R} is supported through mailing lists. The level of support is
at least as good as for commercial packages. It is typical to have
queries answered in a matter of a few hours.

Even though \texttt{R} is a free package it is more powerful than most
commercial packages. Many of the modern procedures found in commercial
packages were first developed and tested using \texttt{R} or
\textbf{S-Plus} (the commercial equivalent of \texttt{R}).

When you start \texttt{R} it will issue a prompt when it expects user
input. The default prompt is:

\begin{verbatim}
>
\end{verbatim}

This is where you type commands that call functions that instruct
\texttt{R} to (e.g.) read a data file, recode data, produce a table, or
fit a regression. For example:

\begin{verbatim}
   > table(SEX, LIFE)
\end{verbatim}

If a command you type is not complete then the prompt will change to:

\begin{verbatim}
+
\end{verbatim}

on subsequent lines until the command is complete:

\begin{verbatim}
> table(
+ SEX, LIFE +)
\end{verbatim}

The \texttt{\textgreater{}} and \texttt{+} prompts are not shown in the
example commands in the rest of this material.

The example commands in this material are often broken into shorter
lines and indented for ease of understanding. The code still works as
lines are split in places where \texttt{R} knows that a line is not
complete. For example:

\begin{Shaded}
\begin{Highlighting}[]
\KeywordTok{table}\NormalTok{(SEX,}
\NormalTok{      LIFE)}
\end{Highlighting}
\end{Shaded}

could be entered on a single line as:

\begin{Shaded}
\begin{Highlighting}[]
\KeywordTok{table}\NormalTok{(SEX, LIFE)}
\end{Highlighting}
\end{Shaded}

In this example \texttt{R} knows that the command is not complete until
the brackets are closed. The following example could also be written on
one line:

\begin{Shaded}
\begin{Highlighting}[]
\NormalTok{salex.lreg.coeffs <-}
\StringTok{  }\KeywordTok{coef}\NormalTok{(}\KeywordTok{summary}\NormalTok{(salex.lreg))}
\end{Highlighting}
\end{Shaded}

In this case \texttt{R} knows that the \texttt{\textless{}-} operator at
the end of the first line needs further input.

\texttt{R} maintains a history of previous commands. These can be
recalled and edited using the up and down arrow keys.

Output that has scrolled off the top of the output / command window can
be recalled using the window or terminal scroll bars.

Output can be saved using the \texttt{sink()} function with a file name:
\texttt{sink("results.out")} to start recording output. Use the
\texttt{sink()} function without a file name to stop recording output:
\texttt{sink()}

You can also use clipboard functions such as copy and paste to (e.g.)
copy and then paste selected chunks of output into an editor or word
processor running alongside \texttt{R}.

All the sample data files used in the exercises in this manual are space
delimited text files using the general format:

\begin{verbatim}
   ID AGE IQ
   1 39 94
   2 41 89
   3 42 83
   4 30 99
   5 35 94
   6 44 90
   7 31 94
   8 39 87
\end{verbatim}

\texttt{R} has facilities for working with files in different formats
including (through the use of extension packages) \textbf{ODBC} (open
database connectivity) and \textbf{SQL} data sources, \textbf{EpiInfo},
\textbf{EpiData}, \textbf{Minitab}, \textbf{SPSS}, \textbf{SAS},
\textbf{S-Plus}, and \textbf{Stata} format files.

\hypertarget{retrieving-data}{%
\section{Retrieving data}\label{retrieving-data}}

All of the exercises in this manual assume that the necessary data files
are located in the current working directory. All of the data files that
you require to follow this material are in a ZIP archive that can be
downloaded from:

\url{http://www.brixtonhealth.com/prfe/prfe.zip}

A command such as:

\begin{Shaded}
\begin{Highlighting}[]
\KeywordTok{read.table}\NormalTok{(}\StringTok{"data/fem.dat"}\NormalTok{, }\DataTypeTok{header =} \OtherTok{TRUE}\NormalTok{)}
\end{Highlighting}
\end{Shaded}

retrieves the data stored in the file named \texttt{fem.dat} which is
stored in the \texttt{data} folder.

To retrieve data that is stored in files outside a different directory
you need to specify the full path to the file. For example:

\begin{Shaded}
\begin{Highlighting}[]
\KeywordTok{read.table}\NormalTok{(}\StringTok{"~/example/fem.dat"}\NormalTok{, }\DataTypeTok{header =} \OtherTok{TRUE}\NormalTok{)}
\end{Highlighting}
\end{Shaded}

will retrieve the data stored in the file named \texttt{fem.dat} stored
in the \texttt{example} directory under the user's home directory on
UNIX, Linux, and OS X systems.

\texttt{R} follows many UNIX operating and naming conventions including
the use of the backslash (\texttt{\textbackslash{}}) character to
specify special characters in strings (e.g.~using
\texttt{\textbackslash{}n} to specify a new line in printed output).
Windows uses the backslash (\texttt{\textbackslash{}}) character to
separate directory and file names in paths. This means that Windows
users need to escape any backslashes in file paths using an additional
backslash character. For example:

\begin{Shaded}
\begin{Highlighting}[]
\KeywordTok{read.table}\NormalTok{(}\StringTok{"c:}\CharTok{\textbackslash{}\textbackslash{}}\StringTok{example}\CharTok{\textbackslash{}\textbackslash{}}\StringTok{fem.dat"}\NormalTok{, }\DataTypeTok{header =} \OtherTok{TRUE}\NormalTok{)}
\end{Highlighting}
\end{Shaded}

will retrieve the data that is stored in the file named \texttt{fem.dat}
which is stored in the \texttt{example} directory off the root directory
of the \texttt{C:} drive. The Windows version of \texttt{R} also allows
you to specify UNIX-style path names (i.e.~using the forward slash
(\texttt{/}) character as a separator in file paths). For example:

\begin{Shaded}
\begin{Highlighting}[]
\KeywordTok{read.table}\NormalTok{(}\StringTok{"c:/exampmle/fem.dat"}\NormalTok{, }\DataTypeTok{header =} \OtherTok{TRUE}\NormalTok{)}
\end{Highlighting}
\end{Shaded}

Path names may include shortcut characters such as:

\begin{longtable}[]{@{}cl@{}}
\toprule
\texttt{.} & The current working directory\tabularnewline
\texttt{..} & Up one level in the directory tree\tabularnewline
\texttt{\textasciitilde{}} & The user's home directory (on UNIX-based
systems)\tabularnewline
\bottomrule
\end{longtable}

\texttt{R} also allows you to retrieve files from any location that may
be represented by a standard \texttt{uniform\ resource\ locator\ (URL)}
string. For example:

\begin{Shaded}
\begin{Highlighting}[]
\KeywordTok{read.table}\NormalTok{(}\StringTok{"file://~/example/fem.dat"}\NormalTok{, }\DataTypeTok{header =} \OtherTok{TRUE}\NormalTok{)}
\end{Highlighting}
\end{Shaded}

will retrieve the data stored in the file named \texttt{fem.dat} stored
in the \texttt{example} directory under the users home directory on
UNIX-based systems.

All of the data files used in this section are stored in the
\texttt{/data} directory of this guide's GitLab repository
(\url{https://git.validmeasures.org/datahub/datahubguide/tree/master/data}).
This means, for example, that you can use the \texttt{read.table()}
function specifying

``\url{https://git.validmeasures.org/datahub/datahubguide/tree/master/data/fem.dat}''

as the \texttt{URL} to retrieve the data that is stored in the file
named \texttt{fem.da}t which is stored in the \texttt{/data} directory
of this guide's GitLab repository.

\hypertarget{section}{%
\chapter{}\label{section}}

\bibliography{book.bib}


\end{document}
