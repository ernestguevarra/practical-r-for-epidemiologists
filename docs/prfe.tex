\documentclass[12pt,a4paper]{book}
\usepackage{lmodern}
\usepackage{amssymb,amsmath}
\usepackage{ifxetex,ifluatex}
\usepackage{fixltx2e} % provides \textsubscript
\ifnum 0\ifxetex 1\fi\ifluatex 1\fi=0 % if pdftex
  \usepackage[T1]{fontenc}
  \usepackage[utf8]{inputenc}
\else % if luatex or xelatex
  \ifxetex
    \usepackage{mathspec}
  \else
    \usepackage{fontspec}
  \fi
  \defaultfontfeatures{Ligatures=TeX,Scale=MatchLowercase}
\fi
% use upquote if available, for straight quotes in verbatim environments
\IfFileExists{upquote.sty}{\usepackage{upquote}}{}
% use microtype if available
\IfFileExists{microtype.sty}{%
\usepackage{microtype}
\UseMicrotypeSet[protrusion]{basicmath} % disable protrusion for tt fonts
}{}
\usepackage[margin=2cm]{geometry}
\usepackage{hyperref}
\PassOptionsToPackage{usenames,dvipsnames}{color} % color is loaded by hyperref
\hypersetup{unicode=true,
            pdftitle={Practical R for Epidemiologists},
            pdfauthor={Mark Myatt and Ernest Guevarra},
            colorlinks=true,
            linkcolor=Maroon,
            citecolor=Blue,
            urlcolor=Blue,
            breaklinks=true}
\urlstyle{same}  % don't use monospace font for urls
\usepackage{natbib}
\bibliographystyle{apalike}
\usepackage{color}
\usepackage{fancyvrb}
\newcommand{\VerbBar}{|}
\newcommand{\VERB}{\Verb[commandchars=\\\{\}]}
\DefineVerbatimEnvironment{Highlighting}{Verbatim}{commandchars=\\\{\}}
% Add ',fontsize=\small' for more characters per line
\usepackage{framed}
\definecolor{shadecolor}{RGB}{248,248,248}
\newenvironment{Shaded}{\begin{snugshade}}{\end{snugshade}}
\newcommand{\KeywordTok}[1]{\textcolor[rgb]{0.13,0.29,0.53}{\textbf{#1}}}
\newcommand{\DataTypeTok}[1]{\textcolor[rgb]{0.13,0.29,0.53}{#1}}
\newcommand{\DecValTok}[1]{\textcolor[rgb]{0.00,0.00,0.81}{#1}}
\newcommand{\BaseNTok}[1]{\textcolor[rgb]{0.00,0.00,0.81}{#1}}
\newcommand{\FloatTok}[1]{\textcolor[rgb]{0.00,0.00,0.81}{#1}}
\newcommand{\ConstantTok}[1]{\textcolor[rgb]{0.00,0.00,0.00}{#1}}
\newcommand{\CharTok}[1]{\textcolor[rgb]{0.31,0.60,0.02}{#1}}
\newcommand{\SpecialCharTok}[1]{\textcolor[rgb]{0.00,0.00,0.00}{#1}}
\newcommand{\StringTok}[1]{\textcolor[rgb]{0.31,0.60,0.02}{#1}}
\newcommand{\VerbatimStringTok}[1]{\textcolor[rgb]{0.31,0.60,0.02}{#1}}
\newcommand{\SpecialStringTok}[1]{\textcolor[rgb]{0.31,0.60,0.02}{#1}}
\newcommand{\ImportTok}[1]{#1}
\newcommand{\CommentTok}[1]{\textcolor[rgb]{0.56,0.35,0.01}{\textit{#1}}}
\newcommand{\DocumentationTok}[1]{\textcolor[rgb]{0.56,0.35,0.01}{\textbf{\textit{#1}}}}
\newcommand{\AnnotationTok}[1]{\textcolor[rgb]{0.56,0.35,0.01}{\textbf{\textit{#1}}}}
\newcommand{\CommentVarTok}[1]{\textcolor[rgb]{0.56,0.35,0.01}{\textbf{\textit{#1}}}}
\newcommand{\OtherTok}[1]{\textcolor[rgb]{0.56,0.35,0.01}{#1}}
\newcommand{\FunctionTok}[1]{\textcolor[rgb]{0.00,0.00,0.00}{#1}}
\newcommand{\VariableTok}[1]{\textcolor[rgb]{0.00,0.00,0.00}{#1}}
\newcommand{\ControlFlowTok}[1]{\textcolor[rgb]{0.13,0.29,0.53}{\textbf{#1}}}
\newcommand{\OperatorTok}[1]{\textcolor[rgb]{0.81,0.36,0.00}{\textbf{#1}}}
\newcommand{\BuiltInTok}[1]{#1}
\newcommand{\ExtensionTok}[1]{#1}
\newcommand{\PreprocessorTok}[1]{\textcolor[rgb]{0.56,0.35,0.01}{\textit{#1}}}
\newcommand{\AttributeTok}[1]{\textcolor[rgb]{0.77,0.63,0.00}{#1}}
\newcommand{\RegionMarkerTok}[1]{#1}
\newcommand{\InformationTok}[1]{\textcolor[rgb]{0.56,0.35,0.01}{\textbf{\textit{#1}}}}
\newcommand{\WarningTok}[1]{\textcolor[rgb]{0.56,0.35,0.01}{\textbf{\textit{#1}}}}
\newcommand{\AlertTok}[1]{\textcolor[rgb]{0.94,0.16,0.16}{#1}}
\newcommand{\ErrorTok}[1]{\textcolor[rgb]{0.64,0.00,0.00}{\textbf{#1}}}
\newcommand{\NormalTok}[1]{#1}
\usepackage{longtable,booktabs}
\usepackage{graphicx,grffile}
\makeatletter
\def\maxwidth{\ifdim\Gin@nat@width>\linewidth\linewidth\else\Gin@nat@width\fi}
\def\maxheight{\ifdim\Gin@nat@height>\textheight\textheight\else\Gin@nat@height\fi}
\makeatother
% Scale images if necessary, so that they will not overflow the page
% margins by default, and it is still possible to overwrite the defaults
% using explicit options in \includegraphics[width, height, ...]{}
\setkeys{Gin}{width=\maxwidth,height=\maxheight,keepaspectratio}
\IfFileExists{parskip.sty}{%
\usepackage{parskip}
}{% else
\setlength{\parindent}{0pt}
\setlength{\parskip}{6pt plus 2pt minus 1pt}
}
\setlength{\emergencystretch}{3em}  % prevent overfull lines
\providecommand{\tightlist}{%
  \setlength{\itemsep}{0pt}\setlength{\parskip}{0pt}}
\setcounter{secnumdepth}{5}
% Redefines (sub)paragraphs to behave more like sections
\ifx\paragraph\undefined\else
\let\oldparagraph\paragraph
\renewcommand{\paragraph}[1]{\oldparagraph{#1}\mbox{}}
\fi
\ifx\subparagraph\undefined\else
\let\oldsubparagraph\subparagraph
\renewcommand{\subparagraph}[1]{\oldsubparagraph{#1}\mbox{}}
\fi

%%% Use protect on footnotes to avoid problems with footnotes in titles
\let\rmarkdownfootnote\footnote%
\def\footnote{\protect\rmarkdownfootnote}

%%% Change title format to be more compact
\usepackage{titling}

% Create subtitle command for use in maketitle
\newcommand{\subtitle}[1]{
  \posttitle{
    \begin{center}\large#1\end{center}
    }
}

\setlength{\droptitle}{-2em}
  \title{Practical R for Epidemiologists}
  \pretitle{\vspace{\droptitle}\centering\huge}
  \posttitle{\par}
\subtitle{Second Edition}
  \author{Mark Myatt and Ernest Guevarra}
  \preauthor{\centering\large\emph}
  \postauthor{\par}
  \predate{\centering\large\emph}
  \postdate{\par}
  \date{2018-05-08}

\usepackage{booktabs}
\usepackage{color}
\usepackage{tcolorbox}
\usepackage{float}
\usepackage{setspace}

\onehalfspacing

\graphicspath{ {images/} }

\newenvironment{rmdremind}
  {\begin{tcolorbox}[width=\textwidth, 
                     colback = {white}, 
                     title = {\textbf{Remember}}, 
                     colbacktitle = lightgray,
                     coltitle = black]
  \begin{includegraphics}[scale = 1]{remind.png}
  \begin{itemize}}
  {\end{itemize}
  \end{includegraphics}
  \end{tcolorbox}}

\newenvironment{rmdnote}
  {\begin{tcolorbox}[width=\textwidth, 
                     colback = {white}, 
                     title = {\textbf{Note}}, 
                     colbacktitle = lightgray,
                     coltitle = black]
  \begin{includegraphics}[scale = 1]{pencil.png}}
  {\end{includegraphics}
  \end{tcolorbox}}
  
\newenvironment{rmdexercise}
  {\begin{tcolorbox}[width=\textwidth, 
                     colback = {white}, 
                     title = {\textbf{Exercise}}, 
                     colbacktitle = lightgray,
                     coltitle = black]
  \begin{includegraphics}[scale = 1]{exercise.png}}
  {\end{includegraphics}
  \end{tcolorbox}}
  
\newenvironment{rmdinfo}
  {\begin{tcolorbox}[width=\textwidth, 
                     colback = {white}, 
                     title = {\textbf{Info}}, 
                     colbacktitle = lightgray,
                     coltitle = black]
  \begin{includegraphics}[scale = 1]{info.png}}
  {\end{includegraphics}
  \end{tcolorbox}}  
  
\newenvironment{rmdwarning}
  {\begin{tcolorbox}[width=\textwidth, 
                     colback = {white}, 
                     title = {\textbf{Warning}}, 
                     colbacktitle = lightgray,
                     coltitle = black]
  \begin{includegraphics}[scale = 1]{warning.png}}
  {\end{includegraphics}
  \end{tcolorbox}}

\newenvironment{rmddownload}
  {\begin{tcolorbox}[width=\textwidth, 
                     colback = {white}, 
                     title = {\textbf{Download}}, 
                     colbacktitle = lightgray,
                     coltitle = black]
  \begin{includegraphics}[scale = 1]{download.png}}
  {\end{includegraphics}
  \end{tcolorbox}}

\usepackage{amsthm}
\newtheorem{theorem}{Theorem}[chapter]
\newtheorem{lemma}{Lemma}[chapter]
\theoremstyle{definition}
\newtheorem{definition}{Definition}[chapter]
\newtheorem{corollary}{Corollary}[chapter]
\newtheorem{proposition}{Proposition}[chapter]
\theoremstyle{definition}
\newtheorem{example}{Example}[chapter]
\theoremstyle{definition}
\newtheorem{exercise}{Exercise}[chapter]
\theoremstyle{remark}
\newtheorem*{remark}{Remark}
\newtheorem*{solution}{Solution}
\begin{document}
\maketitle

{
\hypersetup{linkcolor=black}
\setcounter{tocdepth}{1}
\tableofcontents
}
\hypertarget{welcome}{%
\chapter*{Welcome}\label{welcome}}
\addcontentsline{toc}{chapter}{Welcome}

\includegraphics{images/bookcover_medium.jpg}

This is the website for \emph{Practical R for Epidemiologists}. Visit
the
\href{https://github.com/ernestguevarra/practical-r-for-epidemiologists}{GitHub
repository for this site} or buy it as a
\href{https://www.amazon.co.uk/Practical-R-Epidemiologists-Mark-Myatt-ebook/dp/B00DQATKIE/ref=sr_1_1?ie=UTF8\&qid=1524423427\&sr=8-1\&keywords=practical+r+for+epidemiologists}{Kindle
ebook on Amazon}.

\hypertarget{introduction}{%
\chapter*{Introduction}\label{introduction}}
\addcontentsline{toc}{chapter}{Introduction}

These notes are intended as a practical introduction to using the
\texttt{R} environment for data analysis and graphics to work with
epidemiological data. Topics covered include univariate statistics,
simple statistical inference, charting data, two-by-two tables,
stratified analysis, chi-square test for trend, logistic regression,
survival analysis, computer-intensive methods, and extending \texttt{R}
using user-provided functions. You should be able to follow the material
if you are reasonably familiar with the mechanics of statistical
estimation (e.g.~calculation of odds ratios and confidence intervals)
and require a system that can perform simple or complex analyses to your
exact specifications.

These notes are split into ten sections:

\textbf{Introduction}: You are reading this section now!

\textbf{Introducing R}: Some information about the \texttt{R} system,
the way the \texttt{R} system works, how to get a copy of \texttt{R},
and how to start \texttt{R}.

\textbf{Exercise 1}: Read a dataset, producing descriptive statistics,
charts, and perform simple statistical inference. The aim of the
exercise is for you to become familiar with \texttt{R} and some basic
\texttt{R} functions and objects.

\textbf{Exercise 2}: In this exercise we explore how to manipulate
\texttt{R} objects and how to write functions that can manipulate and
extract data and information from \texttt{R} objects and produce useful
analyses.

\textbf{Exercise 3}: In this exercise we explore how \texttt{R} handles
generalised linear models using the example of logistic regression as
well as seeing how \texttt{R} can perform stratified
(i.e.~Mantel-Haenszel) analysis as well as analysing data arising from
matched case-control studies.

\textbf{Exercise 4}: In this exercise we use \texttt{R} to analyse a
small dataset using the methods introduced in the previous exercises.

\textbf{Exercise 5}: In this exercise we explore how \texttt{R} can be
extended using add-in packages. Specifically, we will use an add-in
package to perform a survival analysis.

\textbf{Exercise 6}: In this exercise we explore how to make your own
\texttt{R} functions behave like \texttt{R} objects so that they return
a data-structure that can be manipulated or interrogated by other
\texttt{R} functions.

\textbf{Exercise 7}: In this exercise we explore how you can use
\texttt{R} to produce custom graphical functions.

\textbf{Exercise 8}: In this exercise we explore some more graphical
functions and create custom graphical functions that produce two
variable plots, pyramid charts, Pareto charts, charts with error bars,
and simple mesh-maps.

\textbf{Exercise 9}: In this exercise we explore ways of implementing
computer-intensive methods, such as the bootstrap and computer based
simulation, using standard \texttt{R} functions.

If you are interested in a system that is flexible, can be tailored to
produce exactly the analysis you want, provides modern analytical
facilities, and have a basic understanding of the mechanics of
hypothesis testing and estimation then you should consider following
this material.

\hypertarget{introducing-r}{%
\chapter*{Introducing R}\label{introducing-r}}
\addcontentsline{toc}{chapter}{Introducing R}

\texttt{R} is a system for data manipulation, calculation, and graphics.
It provides:

\begin{itemize}
\item
  Facilities for data handling and storage
\item
  A large collection of tools for data analysis
\item
  Graphical facilities for data analysis and display
\item
  A simple but powerful programming language
\end{itemize}

\texttt{R} is often described as an environment for working with data.
This is in contrast to a \emph{package} which is a collection of very
specific tools. \texttt{R} is not strictly a statistics system but a
system that provides many classical and modern statistical procedures as
part of a broader data-analysis tool. This is an important difference
between \texttt{R} and other statistical systems. In \texttt{R} a
statistical analysis is usually performed as a series of steps with
intermediate results being stored in objects. Systems such as
\texttt{SPSS} and \texttt{SAS} provide copious output from (e.g.) a
regression analysis whereas \texttt{R} will give minimal output and
store the results of a fit for subsequent interrogation or use with
other \texttt{R} functions. This means that \texttt{R} can be tailored
to produce exactly the analysis and results that you want rather than
produce an analysis designed to fit all situations.

\texttt{R} is a language based product. This means that you interact
with \texttt{R} by typing commands such as:

\begin{Shaded}
\begin{Highlighting}[]
\KeywordTok{table}\NormalTok{(SEX, LIFE)}
\end{Highlighting}
\end{Shaded}

rather than by using menus, dialog boxes, selection lists, and buttons.
This may seem to be a drawback but it means that the system is
considerably more flexible than one that relies on menus, buttons, and
boxes. It also means that every stage of your data management and
analysis can be recorded and edited and re-run at a later date. It also
provides an audit trail for quality control purposes.

\texttt{R} is available under UNIX (including Linux), the Macintosh
operating system OS X, and Microsoft Windows. The method used for
starting \texttt{R} will vary from system to system. On UNIX systems you
may need to issue the \texttt{R} command in a terminal session or click
on an icon or menu option if your system has a windowing system. On
Macintosh systems \texttt{R} will be available as an application but can
also be run in a terminal session. On Microsoft Windows systems there
will usually be an icon on the Start menu or the desktop.

\texttt{R} is an open source system and is available under the \emph{GNU
general public license} (GPL) which means that it is available for free
but that there are some restrictions on how you are allowed to
distribute the system and how you may charge for bespoke data analysis
solutions written using the \texttt{R} system. Details of the general
public license are available from
\url{http://www.gnu.org/copyleft/gpl.html}.

\texttt{R} is available for download from
\url{http://www.r-project.org/}.

This is also the best place to get extension packages and documentation.
You may also subscribe to the \texttt{R} mailing lists from this site.
\texttt{R} is supported through mailing lists. The level of support is
at least as good as for commercial packages. It is typical to have
queries answered in a matter of a few hours.

Even though \texttt{R} is a free package it is more powerful than most
commercial packages. Many of the modern procedures found in commercial
packages were first developed and tested using \texttt{R} or
\textbf{S-Plus} (the commercial equivalent of \texttt{R}).

When you start \texttt{R} it will issue a prompt when it expects user
input. The default prompt is:

\begin{Shaded}
\begin{Highlighting}[]
\OperatorTok{>}
\end{Highlighting}
\end{Shaded}

This is where you type commands that call functions that instruct
\texttt{R} to (e.g.) read a data file, recode data, produce a table, or
fit a regression. For example:

\begin{Shaded}
\begin{Highlighting}[]
\OperatorTok{>}\StringTok{ }\KeywordTok{table}\NormalTok{(SEX, LIFE)}
\end{Highlighting}
\end{Shaded}

If a command you type is not complete then the prompt will change to:

\begin{Shaded}
\begin{Highlighting}[]
\OperatorTok{+}
\end{Highlighting}
\end{Shaded}

on subsequent lines until the command is complete:

\begin{Shaded}
\begin{Highlighting}[]
\OperatorTok{>}\StringTok{ }\KeywordTok{table}\NormalTok{(}
\OperatorTok{+}\StringTok{ }\NormalTok{SEX, LIFE }\OperatorTok{+}\NormalTok{)}
\end{Highlighting}
\end{Shaded}

The \texttt{\textgreater{}} and \texttt{+} prompts are not shown in the
example commands in the rest of this material.

The example commands in this material are often broken into shorter
lines and indented for ease of understanding. The code still works as
lines are split in places where \texttt{R} knows that a line is not
complete. For example:

\begin{Shaded}
\begin{Highlighting}[]
\KeywordTok{table}\NormalTok{(SEX,}
\NormalTok{      LIFE)}
\end{Highlighting}
\end{Shaded}

could be entered on a single line as:

\begin{Shaded}
\begin{Highlighting}[]
\KeywordTok{table}\NormalTok{(SEX, LIFE)}
\end{Highlighting}
\end{Shaded}

In this example \texttt{R} knows that the command is not complete until
the brackets are closed. The following example could also be written on
one line:

\begin{Shaded}
\begin{Highlighting}[]
\NormalTok{salex.lreg.coeffs <-}
\StringTok{  }\KeywordTok{coef}\NormalTok{(}\KeywordTok{summary}\NormalTok{(salex.lreg))}
\end{Highlighting}
\end{Shaded}

In this case \texttt{R} knows that the \texttt{\textless{}-} operator at
the end of the first line needs further input.

\texttt{R} maintains a history of previous commands. These can be
recalled and edited using the up and down arrow keys.

Output that has scrolled off the top of the output / command window can
be recalled using the window or terminal scroll bars.

Output can be saved using the \texttt{sink()} function with a file name:
\texttt{sink("results.out")} to start recording output. Use the
\texttt{sink()} function without a file name to stop recording output:
\texttt{sink()}

You can also use clipboard functions such as copy and paste to (e.g.)
copy and then paste selected chunks of output into an editor or word
processor running alongside \texttt{R}.

All the sample data files used in the exercises in this manual are space
delimited text files using the general format:

\begin{verbatim}
   ID AGE IQ
   1 39 94
   2 41 89
   3 42 83
   4 30 99
   5 35 94
   6 44 90
   7 31 94
   8 39 87
\end{verbatim}

\texttt{R} has facilities for working with files in different formats
including (through the use of extension packages) \textbf{ODBC} (open
database connectivity) and \textbf{SQL} data sources, \textbf{EpiInfo},
\textbf{EpiData}, \textbf{Minitab}, \textbf{SPSS}, \textbf{SAS},
\textbf{S-Plus}, and \textbf{Stata} format files.

\newpage

\hypertarget{retrieving-data}{%
\section*{Retrieving data}\label{retrieving-data}}
\addcontentsline{toc}{section}{Retrieving data}

All of the exercises in this manual assume that the necessary data files
are located in the current working directory. All of the data files that
you require to follow this material are in a ZIP archive that can be
downloaded from:

\url{http://www.brixtonhealth.com/prfe/prfe.zip}

A command such as:

\begin{Shaded}
\begin{Highlighting}[]
\KeywordTok{read.table}\NormalTok{(}\StringTok{"data/fem.dat"}\NormalTok{, }\DataTypeTok{header =} \OtherTok{TRUE}\NormalTok{)}
\end{Highlighting}
\end{Shaded}

retrieves the data stored in the file named \texttt{fem.dat} which is
stored in the current working directory.

To retrieve data that is stored in files outside a different directory
you need to specify the full path to the file. For example:

\begin{Shaded}
\begin{Highlighting}[]
\KeywordTok{read.table}\NormalTok{(}\StringTok{"~/prfe/fem.dat"}\NormalTok{, }\DataTypeTok{header =} \OtherTok{TRUE}\NormalTok{)}
\end{Highlighting}
\end{Shaded}

will retrieve the data stored in the file named \texttt{fem.dat} stored
in the \texttt{prfe} directory under the user's home directory on UNIX,
Linux, and OS X systems.

\texttt{R} follows many UNIX operating and naming conventions including
the use of the backslash (\texttt{\textbackslash{}}) character to
specify special characters in strings (e.g.~using
\texttt{\textbackslash{}n} to specify a new line in printed output).
Windows uses the backslash (\texttt{\textbackslash{}}) character to
separate directory and file names in paths. This means that Windows
users need to escape any backslashes in file paths using an additional
backslash character. For example:

\begin{Shaded}
\begin{Highlighting}[]
\KeywordTok{read.table}\NormalTok{(}\StringTok{"c:}\CharTok{\textbackslash{}\textbackslash{}}\StringTok{prfe}\CharTok{\textbackslash{}\textbackslash{}}\StringTok{fem.dat"}\NormalTok{, }\DataTypeTok{header =} \OtherTok{TRUE}\NormalTok{)}
\end{Highlighting}
\end{Shaded}

will retrieve the data that is stored in the file named \texttt{fem.dat}
which is stored in the \texttt{prfe} directory off the root directory of
the \texttt{C:} drive. The Windows version of \texttt{R} also allows you
to specify UNIX-style path names (i.e.~using the forward slash
(\texttt{/}) character as a separator in file paths). For example:

\begin{Shaded}
\begin{Highlighting}[]
\KeywordTok{read.table}\NormalTok{(}\StringTok{"c:/prfe/fem.dat"}\NormalTok{, }\DataTypeTok{header =} \OtherTok{TRUE}\NormalTok{)}
\end{Highlighting}
\end{Shaded}

Path names may include shortcut characters such as:

\begin{longtable}[]{@{}cl@{}}
\toprule
\texttt{.} & The current working directory\tabularnewline
\texttt{..} & Up one level in the directory tree\tabularnewline
\texttt{\textasciitilde{}} & The user's home directory (on UNIX-based
systems)\tabularnewline
\bottomrule
\end{longtable}

\texttt{R} also allows you to retrieve files from any location that may
be represented by a standard \texttt{uniform\ resource\ locator\ (URL)}
string. For example:

\begin{Shaded}
\begin{Highlighting}[]
\KeywordTok{read.table}\NormalTok{(}\StringTok{"file://~/prfe/fem.dat"}\NormalTok{, }\DataTypeTok{header =} \OtherTok{TRUE}\NormalTok{)}
\end{Highlighting}
\end{Shaded}

will retrieve the data stored in the file named \texttt{fem.dat} stored
in the \texttt{prfe} directory under the users home directory on
UNIX-based systems.

All of the data files used in this section are stored in the
\texttt{/prfe} directory Brixton Health's website. This means, for
example, that you can use the \texttt{read.table()} function specifying

``\url{http://www.brixtonhealth.com/prfe/fem.dat}''

as the \texttt{URL} to retrieve the data that is stored in the file
named \texttt{fem.dat} which is stored in the \texttt{data/} directory
of this guide's GitHub repository.

\hypertarget{exercise1}{%
\chapter{Getting acquainted with R}\label{exercise1}}

In this exercise we will use \texttt{R} to read a dataset and produce
some descriptive statistics, produce some charts, and perform some
simple statistical inference. The aim of the exercise is for you to
become familiar with \texttt{R} and some basic \texttt{R} functions and
objects.

The first thing we will do, after starting \texttt{R}, is issue a
command to retrieve an example dataset:

\begin{Shaded}
\begin{Highlighting}[]
\NormalTok{fem <-}\StringTok{ }\KeywordTok{read.table}\NormalTok{(}\StringTok{"fem.dat"}\NormalTok{, }\DataTypeTok{header =} \OtherTok{TRUE}\NormalTok{)}
\end{Highlighting}
\end{Shaded}

This command illustrates some key things about the way \texttt{R} works.

We are instructing \texttt{R} to assign (using the \texttt{\textless{}-}
operator) the output of the \texttt{read.table()} function to an object
called \texttt{fem}.

The \texttt{fem} object will contain the data held in the file
\texttt{fem.dat} as an \texttt{R} data.frame object:

\begin{Shaded}
\begin{Highlighting}[]
\KeywordTok{class}\NormalTok{(fem)}
\end{Highlighting}
\end{Shaded}

\begin{verbatim}
## [1] "data.frame"
\end{verbatim}

You can inspect the contents of the \texttt{fem} data.frame (or any
other \texttt{R} object) just by typing its name:

\begin{Shaded}
\begin{Highlighting}[]
\NormalTok{fem}
\end{Highlighting}
\end{Shaded}

\begin{verbatim}
##   ID AGE IQ ANX DEP SLP SEX LIFE    WT
## 1  1  39 94   2   2   2   1    1  2.23
## 2  2  41 89   2   2   2   1    1  1.00
## 3  3  42 83   3   3   2   1    1  1.82
## 4  4  30 99   2   2   2   1    1 -1.18
## 5  5  35 94   2   1   1   1    2 -0.14
## 6  6  44 90  NA   1   2   2    2  0.41
\end{verbatim}

Note that the \texttt{fem} object is built from other objects. These are
the named vectors (columns) in the dataset:

\begin{Shaded}
\begin{Highlighting}[]
\KeywordTok{names}\NormalTok{(fem)}
\end{Highlighting}
\end{Shaded}

\begin{verbatim}
## [1] "ID"   "AGE"  "IQ"   "ANX"  "DEP"  "SLP"  "SEX"  "LIFE" "WT"
\end{verbatim}

The \texttt{{[}1{]}} displayed before the column names refers to the
numbered position of the first name in the output. These positions are
known as indexes and can be used to refer to individual items. For
example:

\begin{Shaded}
\begin{Highlighting}[]
\KeywordTok{names}\NormalTok{(fem)[}\DecValTok{1}\NormalTok{]}
\end{Highlighting}
\end{Shaded}

\begin{verbatim}
## [1] "ID"
\end{verbatim}

\begin{Shaded}
\begin{Highlighting}[]
\KeywordTok{names}\NormalTok{(fem)[}\DecValTok{8}\NormalTok{]}
\end{Highlighting}
\end{Shaded}

\begin{verbatim}
## [1] "LIFE"
\end{verbatim}

\begin{Shaded}
\begin{Highlighting}[]
\KeywordTok{names}\NormalTok{(fem)[}\DecValTok{2}\OperatorTok{:}\DecValTok{4}\NormalTok{]}
\end{Highlighting}
\end{Shaded}

\begin{verbatim}
## [1] "AGE" "IQ"  "ANX"
\end{verbatim}

The data consist of 118 records:

\begin{Shaded}
\begin{Highlighting}[]
\KeywordTok{nrow}\NormalTok{(fem)}
\end{Highlighting}
\end{Shaded}

\begin{verbatim}
## [1] 118
\end{verbatim}

each with nine variables:

\begin{Shaded}
\begin{Highlighting}[]
\KeywordTok{ncol}\NormalTok{(fem)}
\end{Highlighting}
\end{Shaded}

\begin{verbatim}
## [1] 9
\end{verbatim}

for female psychiatric patients.

The columns in the dataset are:

\begin{longtable}[]{@{}ll@{}}
\toprule
\begin{minipage}[b]{0.14\columnwidth}\raggedright
\textbf{ID}\strut
\end{minipage} & \begin{minipage}[b]{0.69\columnwidth}\raggedright
Patient ID\strut
\end{minipage}\tabularnewline
\midrule
\endhead
\begin{minipage}[t]{0.14\columnwidth}\raggedright
\textbf{AGE}\strut
\end{minipage} & \begin{minipage}[t]{0.69\columnwidth}\raggedright
Age in years\strut
\end{minipage}\tabularnewline
\begin{minipage}[t]{0.14\columnwidth}\raggedright
\textbf{IQ}\strut
\end{minipage} & \begin{minipage}[t]{0.69\columnwidth}\raggedright
IQ score\strut
\end{minipage}\tabularnewline
\begin{minipage}[t]{0.14\columnwidth}\raggedright
\textbf{ANX}\strut
\end{minipage} & \begin{minipage}[t]{0.69\columnwidth}\raggedright
Anxiety (1=none, 2=mild, 3=moderate, 4=severe)\strut
\end{minipage}\tabularnewline
\begin{minipage}[t]{0.14\columnwidth}\raggedright
\textbf{DEP}\strut
\end{minipage} & \begin{minipage}[t]{0.69\columnwidth}\raggedright
Depression (1=none, 2=mild, 3=moderate or severe)\strut
\end{minipage}\tabularnewline
\begin{minipage}[t]{0.14\columnwidth}\raggedright
\textbf{SLP}\strut
\end{minipage} & \begin{minipage}[t]{0.69\columnwidth}\raggedright
Sleeping normally (1=yes, 2=no)\strut
\end{minipage}\tabularnewline
\begin{minipage}[t]{0.14\columnwidth}\raggedright
\textbf{SEX}\strut
\end{minipage} & \begin{minipage}[t]{0.69\columnwidth}\raggedright
Lost interest in sex (1=yes, 2=no)\strut
\end{minipage}\tabularnewline
\begin{minipage}[t]{0.14\columnwidth}\raggedright
\textbf{LIFE}\strut
\end{minipage} & \begin{minipage}[t]{0.69\columnwidth}\raggedright
Considered suicide (1=yes, 2=no)\strut
\end{minipage}\tabularnewline
\begin{minipage}[t]{0.14\columnwidth}\raggedright
\textbf{WT}\strut
\end{minipage} & \begin{minipage}[t]{0.69\columnwidth}\raggedright
Weight change (kg) in previous 6 months\strut
\end{minipage}\tabularnewline
\bottomrule
\end{longtable}

\newpage

The first ten records of the \texttt{fem} data.frame are:

\begin{verbatim}
##    ID AGE  IQ ANX DEP SLP SEX LIFE    WT
## 1   1  39  94   2   2   2   1    1  2.23
## 2   2  41  89   2   2   2   1    1  1.00
## 3   3  42  83   3   3   2   1    1  1.82
## 4   4  30  99   2   2   2   1    1 -1.18
## 5   5  35  94   2   1   1   1    2 -0.14
## 6   6  44  90  NA   1   2   2    2  0.41
## 7   7  31  94   2   2  NA   1    1 -0.68
## 8   8  39  87   3   2   2   1    2  1.59
## 9   9  35 -99   3   2   2   1    1 -0.55
## 10 10  33  92   2   2   2   1    1  0.36
\end{verbatim}

~

You may check this by asking \texttt{R} to display all columns of the
first ten records in the \texttt{fem} data.frame:

\begin{Shaded}
\begin{Highlighting}[]
\NormalTok{fem[}\DecValTok{1}\OperatorTok{:}\DecValTok{10}\NormalTok{, ]}
\end{Highlighting}
\end{Shaded}

\begin{verbatim}
##    ID AGE  IQ ANX DEP SLP SEX LIFE    WT
## 1   1  39  94   2   2   2   1    1  2.23
## 2   2  41  89   2   2   2   1    1  1.00
## 3   3  42  83   3   3   2   1    1  1.82
## 4   4  30  99   2   2   2   1    1 -1.18
## 5   5  35  94   2   1   1   1    2 -0.14
## 6   6  44  90  NA   1   2   2    2  0.41
## 7   7  31  94   2   2  NA   1    1 -0.68
## 8   8  39  87   3   2   2   1    2  1.59
## 9   9  35 -99   3   2   2   1    1 -0.55
## 10 10  33  92   2   2   2   1    1  0.36
\end{verbatim}

~

The space after the comma is optional. You can think of it as a
\emph{placeholder} for where you would specify the indexes for columns
you wanted to display. For example:

\begin{Shaded}
\begin{Highlighting}[]
\NormalTok{fem[}\DecValTok{1}\OperatorTok{:}\DecValTok{10}\NormalTok{,}\DecValTok{2}\OperatorTok{:}\DecValTok{4}\NormalTok{]}
\end{Highlighting}
\end{Shaded}

~

displays the first ten rows and the second, third and fourth columns of
the \texttt{fem} data.frame:

\begin{verbatim}
##    AGE  IQ ANX
## 1   39  94   2
## 2   41  89   2
## 3   42  83   3
## 4   30  99   2
## 5   35  94   2
## 6   44  90  NA
## 7   31  94   2
## 8   39  87   3
## 9   35 -99   3
## 10  33  92   2
\end{verbatim}

~

\texttt{NA} is a special value meaning \emph{not available} or
\emph{missing}.

You can access the contents of a single column by name:

\begin{Shaded}
\begin{Highlighting}[]
\NormalTok{fem}\OperatorTok{$}\NormalTok{IQ}
\end{Highlighting}
\end{Shaded}

\begin{verbatim}
##   [1]  94  89  83  99  94  90  94  87 -99  92  92  94  91  86  90 -99  91
##  [18]  82  86  88  97  96  95  87 103 -99  91  87  91  89  92  84  94  92
##  [35]  96  96  86  92 102  82  92  90  92  88  98  93  90  91 -99  92  92
##  [52]  91  91  86  95  91  96 100  99  89  89  98  98 103  91  91  94  91
##  [69]  85  92  96  90  87  95  95  87  95  88  94 -99 -99  87  92  86  93
##  [86]  92 106  93  95  95  92  98  92  88  85  92  84  92  91  86  92  89
## [103] -99  96  97  92  92  98  91  91  89  94  90  96  87  86  89 -99
\end{verbatim}

~

\begin{Shaded}
\begin{Highlighting}[]
\NormalTok{fem}\OperatorTok{$}\NormalTok{IQ[}\DecValTok{1}\OperatorTok{:}\DecValTok{10}\NormalTok{]}
\end{Highlighting}
\end{Shaded}

\begin{verbatim}
##  [1]  94  89  83  99  94  90  94  87 -99  92
\end{verbatim}

~

The \texttt{\$} sign is used to separate the name of the data.frame and
the name of the column of interest. Note that \texttt{R} is
case-sensitive so that \texttt{IQ} and \texttt{iq} are
\textbf{\emph{not}} the same.

You can also access rows, columns, and individual cells by specifying
row and column positions. For example, the \texttt{IQ} column is the
third column in the \texttt{fem} data.frame:

\begin{Shaded}
\begin{Highlighting}[]
\NormalTok{fem[ ,}\DecValTok{3}\NormalTok{]}
\end{Highlighting}
\end{Shaded}

\begin{verbatim}
##   [1]  94  89  83  99  94  90  94  87 -99  92  92  94  91  86  90 -99  91
##  [18]  82  86  88  97  96  95  87 103 -99  91  87  91  89  92  84  94  92
##  [35]  96  96  86  92 102  82  92  90  92  88  98  93  90  91 -99  92  92
##  [52]  91  91  86  95  91  96 100  99  89  89  98  98 103  91  91  94  91
##  [69]  85  92  96  90  87  95  95  87  95  88  94 -99 -99  87  92  86  93
##  [86]  92 106  93  95  95  92  98  92  88  85  92  84  92  91  86  92  89
## [103] -99  96  97  92  92  98  91  91  89  94  90  96  87  86  89 -99
\end{verbatim}

~

\begin{Shaded}
\begin{Highlighting}[]
\NormalTok{fem[}\DecValTok{9}\NormalTok{, ]}
\end{Highlighting}
\end{Shaded}

\begin{verbatim}
##   ID AGE  IQ ANX DEP SLP SEX LIFE    WT
## 9  9  35 -99   3   2   2   1    1 -0.55
\end{verbatim}

~

\begin{Shaded}
\begin{Highlighting}[]
\NormalTok{fem[}\DecValTok{9}\NormalTok{,}\DecValTok{3}\NormalTok{]}
\end{Highlighting}
\end{Shaded}

\begin{verbatim}
## [1] -99
\end{verbatim}

~

There are missing values in the \texttt{IQ} column which are all coded
as \textbf{-99}. Before proceeding we must set these to the special
\texttt{NA} value:

\begin{Shaded}
\begin{Highlighting}[]
\NormalTok{fem}\OperatorTok{$}\NormalTok{IQ[fem}\OperatorTok{$}\NormalTok{IQ }\OperatorTok{==}\StringTok{ }\DecValTok{-99}\NormalTok{] <-}\StringTok{ }\OtherTok{NA}
\end{Highlighting}
\end{Shaded}

~

The term inside the square brackets is also an index. This type of index
is used to refer to subsets of data held in an object that meet a
particular condition. In this case we are instructing \texttt{R} to set
the contents of the \texttt{IQ} variable to \texttt{NA} if the contents
of the \texttt{IQ} variable is \textbf{-99}.

Check that this has worked:

\begin{Shaded}
\begin{Highlighting}[]
\NormalTok{fem}\OperatorTok{$}\NormalTok{IQ}
\end{Highlighting}
\end{Shaded}

\begin{verbatim}
##   [1]  94  89  83  99  94  90  94  87  NA  92  92  94  91  86  90  NA  91
##  [18]  82  86  88  97  96  95  87 103  NA  91  87  91  89  92  84  94  92
##  [35]  96  96  86  92 102  82  92  90  92  88  98  93  90  91  NA  92  92
##  [52]  91  91  86  95  91  96 100  99  89  89  98  98 103  91  91  94  91
##  [69]  85  92  96  90  87  95  95  87  95  88  94  NA  NA  87  92  86  93
##  [86]  92 106  93  95  95  92  98  92  88  85  92  84  92  91  86  92  89
## [103]  NA  96  97  92  92  98  91  91  89  94  90  96  87  86  89  NA
\end{verbatim}

~

We can now compare the groups who have and have not considered suicide.
For example:

\begin{Shaded}
\begin{Highlighting}[]
\KeywordTok{by}\NormalTok{(fem}\OperatorTok{$}\NormalTok{IQ, fem}\OperatorTok{$}\NormalTok{LIFE, summary)}
\end{Highlighting}
\end{Shaded}

~

Look at the help for the \texttt{by()} function:

\begin{Shaded}
\begin{Highlighting}[]
\KeywordTok{help}\NormalTok{(by)}
\end{Highlighting}
\end{Shaded}

~

Note that you may use \texttt{?by} as a shortcut for \texttt{help(by)}.

The \texttt{by()} function applies another function (in this case the
\texttt{summary()} function) to a column in a data.frame (in this case
\texttt{fem\$IQ}) split by the value of another variable (in this case
\texttt{fem\$LIFE}).

It can be tedious to always have to specify a data.frame each time we
want to use a particular variable. We can fix this problem by
`attaching' the data.frame:

\begin{Shaded}
\begin{Highlighting}[]
\KeywordTok{attach}\NormalTok{(fem)}
\end{Highlighting}
\end{Shaded}

\begin{verbatim}
## The following objects are masked from fem (pos = 3):
## 
##     AGE, ANX, DEP, ID, IQ, LIFE, SEX, SLP, WT
\end{verbatim}

\begin{verbatim}
## The following objects are masked from fem (pos = 5):
## 
##     AGE, ANX, DEP, ID, IQ, LIFE, SEX, SLP, WT
\end{verbatim}

\begin{verbatim}
## The following objects are masked from fem (pos = 6):
## 
##     AGE, ANX, DEP, ID, IQ, LIFE, SEX, SLP, WT
\end{verbatim}

\begin{verbatim}
## The following objects are masked from fem (pos = 11):
## 
##     AGE, ANX, DEP, ID, IQ, LIFE, SEX, SLP, WT
\end{verbatim}

~

We can now refer to the columns in the \texttt{fem} data.frame without
having to specify the name of the data.frame. This time we will produce
summary statistics for \texttt{WT} by \texttt{LIFE}:

\begin{Shaded}
\begin{Highlighting}[]
\KeywordTok{by}\NormalTok{(WT, LIFE, summary)}
\end{Highlighting}
\end{Shaded}

\begin{verbatim}
## LIFE: 1
##    Min. 1st Qu.  Median    Mean 3rd Qu.    Max.    NA's 
## -2.2300 -0.2700  1.0000  0.7867  1.7300  3.7700       4 
## -------------------------------------------------------- 
## LIFE: 2
##    Min. 1st Qu.  Median    Mean 3rd Qu.    Max.    NA's 
## -1.6800 -0.4500  0.6400  0.6404  1.5000  2.9500       7
\end{verbatim}

~

We can view the same data as a box and whisker plot:

\begin{Shaded}
\begin{Highlighting}[]
\KeywordTok{boxplot}\NormalTok{(WT }\OperatorTok{~}\StringTok{ }\NormalTok{LIFE)}
\end{Highlighting}
\end{Shaded}

\begin{center}\includegraphics{prfe_files/figure-latex/unnamed-chunk-39-1} \end{center}

~

We can add axis labels and a title to the graph:

\begin{Shaded}
\begin{Highlighting}[]
\KeywordTok{boxplot}\NormalTok{(WT }\OperatorTok{~}\StringTok{ }\NormalTok{LIFE,}
        \DataTypeTok{xlab =} \StringTok{"Life"}\NormalTok{,}
        \DataTypeTok{ylab =} \StringTok{"Weight"}\NormalTok{,}
        \DataTypeTok{main =} \StringTok{"Weight BY Life"}\NormalTok{)}
\end{Highlighting}
\end{Shaded}

\begin{center}\includegraphics{prfe_files/figure-latex/unnamed-chunk-40-1} \end{center}

~

A more descriptive title might be ``Weight Change BY Considered
Suicide''.

The groups do not seem to differ much in their medians and the
distributions appear to be reasonably symmetrical about their medians
with a similar spread of values.

We can look at the distribution as histograms:

\begin{Shaded}
\begin{Highlighting}[]
\KeywordTok{hist}\NormalTok{(WT[LIFE }\OperatorTok{==}\StringTok{ }\DecValTok{1}\NormalTok{])}
\end{Highlighting}
\end{Shaded}

\begin{center}\includegraphics{prfe_files/figure-latex/unnamed-chunk-41-1} \end{center}

~

\begin{Shaded}
\begin{Highlighting}[]
\KeywordTok{hist}\NormalTok{(WT[LIFE }\OperatorTok{==}\StringTok{ }\DecValTok{2}\NormalTok{])}
\end{Highlighting}
\end{Shaded}

\begin{center}\includegraphics{prfe_files/figure-latex/unnamed-chunk-42-1} \end{center}

~

and check the assumption of normality using quantile-quantile plots:

\begin{Shaded}
\begin{Highlighting}[]
\KeywordTok{qqnorm}\NormalTok{(WT[LIFE }\OperatorTok{==}\StringTok{ }\DecValTok{1}\NormalTok{])}
\KeywordTok{qqline}\NormalTok{(WT[LIFE }\OperatorTok{==}\StringTok{ }\DecValTok{1}\NormalTok{])}
\end{Highlighting}
\end{Shaded}

\begin{center}\includegraphics{prfe_files/figure-latex/unnamed-chunk-43-1} \end{center}

~

\begin{Shaded}
\begin{Highlighting}[]
\KeywordTok{qqnorm}\NormalTok{(WT[LIFE }\OperatorTok{==}\StringTok{ }\DecValTok{2}\NormalTok{])}
\KeywordTok{qqline}\NormalTok{(WT[LIFE }\OperatorTok{==}\StringTok{ }\DecValTok{2}\NormalTok{])}
\end{Highlighting}
\end{Shaded}

\begin{center}\includegraphics{prfe_files/figure-latex/unnamed-chunk-44-1} \end{center}

~

or by using a formal test:

\begin{Shaded}
\begin{Highlighting}[]
\KeywordTok{shapiro.test}\NormalTok{(WT[LIFE }\OperatorTok{==}\StringTok{ }\DecValTok{1}\NormalTok{])}
\end{Highlighting}
\end{Shaded}

\begin{verbatim}
## 
##  Shapiro-Wilk normality test
## 
## data:  WT[LIFE == 1]
## W = 0.98038, p-value = 0.4336
\end{verbatim}

~

\begin{Shaded}
\begin{Highlighting}[]
\KeywordTok{shapiro.test}\NormalTok{(WT[LIFE }\OperatorTok{==}\StringTok{ }\DecValTok{2}\NormalTok{])}
\end{Highlighting}
\end{Shaded}

\begin{verbatim}
## 
##  Shapiro-Wilk normality test
## 
## data:  WT[LIFE == 2]
## W = 0.97155, p-value = 0.3292
\end{verbatim}

~

Remember that we can use the \texttt{by()} function to apply a function
to a data.frame, including statistical functions such as
\texttt{shapiro.test()}:

\begin{Shaded}
\begin{Highlighting}[]
\KeywordTok{by}\NormalTok{(WT, LIFE, shapiro.test)}
\end{Highlighting}
\end{Shaded}

\begin{verbatim}
## LIFE: 1
## 
##  Shapiro-Wilk normality test
## 
## data:  dd[x, ]
## W = 0.98038, p-value = 0.4336
## 
## -------------------------------------------------------- 
## LIFE: 2
## 
##  Shapiro-Wilk normality test
## 
## data:  dd[x, ]
## W = 0.97155, p-value = 0.3292
\end{verbatim}

~

We can also test whether the variances differ significantly using
\emph{Bartlett's test} for the homogeneity of variances:

\begin{Shaded}
\begin{Highlighting}[]
\KeywordTok{bartlett.test}\NormalTok{(WT, LIFE)}
\end{Highlighting}
\end{Shaded}

\begin{verbatim}
## 
##  Bartlett test of homogeneity of variances
## 
## data:  WT and LIFE
## Bartlett's K-squared = 0.32408, df = 1, p-value = 0.5692
\end{verbatim}

~

There is no significant difference between the two variances.

Many functions in \texttt{R} have a \emph{formula interface} that may be
used to specify multiple variables and the relations between multiple
variables. We could have used the formula interface with the
\texttt{bartlett.test()} function:

\begin{Shaded}
\begin{Highlighting}[]
\KeywordTok{bartlett.test}\NormalTok{(WT }\OperatorTok{~}\StringTok{ }\NormalTok{LIFE)}
\end{Highlighting}
\end{Shaded}

\begin{verbatim}
## 
##  Bartlett test of homogeneity of variances
## 
## data:  WT by LIFE
## Bartlett's K-squared = 0.32408, df = 1, p-value = 0.5692
\end{verbatim}

~

Having checked the normality and homogeneity of variance assumptions we
can proceed to carry out a \texttt{t-test}:

\begin{Shaded}
\begin{Highlighting}[]
\KeywordTok{t.test}\NormalTok{(WT }\OperatorTok{~}\StringTok{ }\NormalTok{LIFE, }\DataTypeTok{var.equal =} \OtherTok{TRUE}\NormalTok{)}
\end{Highlighting}
\end{Shaded}

\begin{verbatim}
## 
##  Two Sample t-test
## 
## data:  WT by LIFE
## t = 0.59869, df = 104, p-value = 0.5507
## alternative hypothesis: true difference in means is not equal to 0
## 95 percent confidence interval:
##  -0.3382365  0.6307902
## sample estimates:
## mean in group 1 mean in group 2 
##       0.7867213       0.6404444
\end{verbatim}

~

There is no evidence that the two groups differ in weight change in the
previous six months.

We could still have performed a \texttt{t-test} if the variances were
not homogenous by setting the \textbf{var.equal} parameter of the
\texttt{t.test()} function to \textbf{FALSE}:

\begin{Shaded}
\begin{Highlighting}[]
\KeywordTok{t.test}\NormalTok{(WT }\OperatorTok{~}\StringTok{ }\NormalTok{LIFE, }\DataTypeTok{var.equal =} \OtherTok{FALSE}\NormalTok{)}
\end{Highlighting}
\end{Shaded}

\begin{verbatim}
## 
##  Welch Two Sample t-test
## 
## data:  WT by LIFE
## t = 0.60608, df = 98.866, p-value = 0.5459
## alternative hypothesis: true difference in means is not equal to 0
## 95 percent confidence interval:
##  -0.3326225  0.6251763
## sample estimates:
## mean in group 1 mean in group 2 
##       0.7867213       0.6404444
\end{verbatim}

~

or performed a non-parametric test:

\begin{Shaded}
\begin{Highlighting}[]
\KeywordTok{wilcox.test}\NormalTok{(WT }\OperatorTok{~}\StringTok{ }\NormalTok{LIFE)}
\end{Highlighting}
\end{Shaded}

\begin{verbatim}
## 
##  Wilcoxon rank sum test with continuity correction
## 
## data:  WT by LIFE
## W = 1488, p-value = 0.4622
## alternative hypothesis: true location shift is not equal to 0
\end{verbatim}

~

An alternative, and more general, non-parametric test is:

\begin{Shaded}
\begin{Highlighting}[]
\KeywordTok{kruskal.test}\NormalTok{(WT }\OperatorTok{~}\StringTok{ }\NormalTok{LIFE)}
\end{Highlighting}
\end{Shaded}

\begin{verbatim}
## 
##  Kruskal-Wallis rank sum test
## 
## data:  WT by LIFE
## Kruskal-Wallis chi-squared = 0.54521, df = 1, p-value = 0.4603
\end{verbatim}

~

We can use the \texttt{table()} function to examine the differences in
depression between the two groups:

\begin{Shaded}
\begin{Highlighting}[]
\KeywordTok{table}\NormalTok{(DEP, LIFE)}
\end{Highlighting}
\end{Shaded}

\begin{verbatim}
##    LIFE
## DEP  1  2
##   1  0 26
##   2 42 24
##   3 16  1
\end{verbatim}

~

The two distributions look very different from each other. We can test
this using a chi-square test on the table:

\begin{Shaded}
\begin{Highlighting}[]
\KeywordTok{chisq.test}\NormalTok{(}\KeywordTok{table}\NormalTok{(DEP, LIFE))}
\end{Highlighting}
\end{Shaded}

\begin{verbatim}
## 
##  Pearson's Chi-squared test
## 
## data:  table(DEP, LIFE)
## X-squared = 43.876, df = 2, p-value = 2.968e-10
\end{verbatim}

~

Note that we passed the output of the \texttt{table()} function directly
to the \texttt{chisq.test()} function. We could have saved the table as
an object first and then passed the object to the \texttt{chisq.test()}
function:

\begin{Shaded}
\begin{Highlighting}[]
\NormalTok{tab <-}\StringTok{ }\KeywordTok{table}\NormalTok{(DEP, LIFE)}
\KeywordTok{chisq.test}\NormalTok{(tab)}
\end{Highlighting}
\end{Shaded}

\begin{verbatim}
## 
##  Pearson's Chi-squared test
## 
## data:  tab
## X-squared = 43.876, df = 2, p-value = 2.968e-10
\end{verbatim}

~

The \texttt{tab} object contains the output of the \texttt{table()}
function:

\begin{Shaded}
\begin{Highlighting}[]
\KeywordTok{class}\NormalTok{(tab)}
\end{Highlighting}
\end{Shaded}

\begin{verbatim}
## [1] "table"
\end{verbatim}

\begin{Shaded}
\begin{Highlighting}[]
\NormalTok{tab}
\end{Highlighting}
\end{Shaded}

\begin{verbatim}
##    LIFE
## DEP  1  2
##   1  0 26
##   2 42 24
##   3 16  1
\end{verbatim}

~

We can pass this table object to another function. For example:

\begin{Shaded}
\begin{Highlighting}[]
\KeywordTok{fisher.test}\NormalTok{(tab)}
\end{Highlighting}
\end{Shaded}

\begin{verbatim}
## 
##  Fisher's Exact Test for Count Data
## 
## data:  tab
## p-value = 1.316e-12
## alternative hypothesis: two.sided
\end{verbatim}

~

When we are finished with the tab object we can delete it using the
\texttt{rm()} function:

\begin{Shaded}
\begin{Highlighting}[]
\KeywordTok{rm}\NormalTok{(tab)}
\end{Highlighting}
\end{Shaded}

~

You can see a list of available objects using the \texttt{ls()}
function:

\begin{Shaded}
\begin{Highlighting}[]
\KeywordTok{ls}\NormalTok{()}
\end{Highlighting}
\end{Shaded}

\begin{verbatim}
## [1] "fem"
\end{verbatim}

~

This should just show the \texttt{fem} object.

We can examine the association between loss of interest in sex and
considering suicide in the same way:

\begin{Shaded}
\begin{Highlighting}[]
\NormalTok{tab <-}\StringTok{ }\KeywordTok{table}\NormalTok{(SEX, LIFE)}
\NormalTok{tab}
\end{Highlighting}
\end{Shaded}

\begin{verbatim}
##    LIFE
## SEX  1  2
##   1 58 38
##   2  5 12
\end{verbatim}

\begin{Shaded}
\begin{Highlighting}[]
\KeywordTok{fisher.test}\NormalTok{(tab)}
\end{Highlighting}
\end{Shaded}

\begin{verbatim}
## 
##  Fisher's Exact Test for Count Data
## 
## data:  tab
## p-value = 0.03175
## alternative hypothesis: true odds ratio is not equal to 1
## 95 percent confidence interval:
##   1.080298 14.214482
## sample estimates:
## odds ratio 
##   3.620646
\end{verbatim}

~

Note that with a two-by-two table the \texttt{fisher.test()} function
produces an estimate of, and confidence intervals for, the odds ratio.
Again, we will delete the \texttt{tab} object:

\begin{Shaded}
\begin{Highlighting}[]
\KeywordTok{rm}\NormalTok{(tab)}
\end{Highlighting}
\end{Shaded}

~

We could have performed the Fisher exact test without creating the tab
object by passing the output of the \texttt{table()} function directly
to the \texttt{fisher.test()} function:

\begin{Shaded}
\begin{Highlighting}[]
\KeywordTok{fisher.test}\NormalTok{(}\KeywordTok{table}\NormalTok{(SEX, LIFE))}
\end{Highlighting}
\end{Shaded}

\begin{verbatim}
## 
##  Fisher's Exact Test for Count Data
## 
## data:  table(SEX, LIFE)
## p-value = 0.03175
## alternative hypothesis: true odds ratio is not equal to 1
## 95 percent confidence interval:
##   1.080298 14.214482
## sample estimates:
## odds ratio 
##   3.620646
\end{verbatim}

~

Choose whichever method you find easiest but remember that it is easy to
save the results of any function for later use.

We can explore the correlation between two variables using the
\texttt{cor()} function:

\begin{Shaded}
\begin{Highlighting}[]
\KeywordTok{cor}\NormalTok{(IQ, WT, }\DataTypeTok{use =} \StringTok{"pairwise.complete.obs"}\NormalTok{)}
\end{Highlighting}
\end{Shaded}

\begin{verbatim}
## [1] -0.2917158
\end{verbatim}

~

or by using a scatter plot:

\begin{Shaded}
\begin{Highlighting}[]
\KeywordTok{plot}\NormalTok{(IQ, WT)}
\end{Highlighting}
\end{Shaded}

\begin{center}\includegraphics{prfe_files/figure-latex/unnamed-chunk-64-1} \end{center}

~

and by a formal test:

\begin{Shaded}
\begin{Highlighting}[]
\KeywordTok{cor.test}\NormalTok{(IQ, WT)}
\end{Highlighting}
\end{Shaded}

\begin{verbatim}
## 
##  Pearson's product-moment correlation
## 
## data:  IQ and WT
## t = -3.0192, df = 98, p-value = 0.003231
## alternative hypothesis: true correlation is not equal to 0
## 95 percent confidence interval:
##  -0.4616804 -0.1010899
## sample estimates:
##        cor 
## -0.2917158
\end{verbatim}

~

With some functions you can pass an entire data.frame rather than a list
of variables:

\begin{Shaded}
\begin{Highlighting}[]
\KeywordTok{cor}\NormalTok{(fem, }\DataTypeTok{use =} \StringTok{"pairwise.complete.obs"}\NormalTok{)}
\end{Highlighting}
\end{Shaded}

\begin{verbatim}
##               ID         AGE            IQ         ANX           DEP
## ID    1.00000000  0.03069077  0.0370598672 -0.02941825 -0.0554147209
## AGE   0.03069077  1.00000000 -0.4345435680  0.06734300 -0.0387049246
## IQ    0.03705987 -0.43454357  1.0000000000 -0.02323787 -0.0001307404
## ANX  -0.02941825  0.06734300 -0.0232378691  1.00000000  0.5437946347
## DEP  -0.05541472 -0.03870492 -0.0001307404  0.54379463  1.0000000000
## SLP  -0.07268743  0.02606547  0.0812993104  0.22317875  0.5248724551
## SEX   0.08999634  0.10609216 -0.0536558660 -0.21062493 -0.3058422258
## LIFE -0.05604349 -0.10300193 -0.0915396469 -0.34211268 -0.6139017253
## WT    0.02640131  0.41574411 -0.2917157832  0.11817532  0.0233742465
##               SLP         SEX        LIFE           WT
## ID   -0.072687434  0.08999634 -0.05604349  0.026401310
## AGE   0.026065468  0.10609216 -0.10300193  0.415744109
## IQ    0.081299310 -0.05365587 -0.09153965 -0.291715783
## ANX   0.223178752 -0.21062493 -0.34211268  0.118175321
## DEP   0.524872455 -0.30584223 -0.61390173  0.023374247
## SLP   1.000000000 -0.29053971 -0.35186578 -0.009259774
## SEX  -0.290539709  1.00000000  0.22316967 -0.027826514
## LIFE -0.351865775  0.22316967  1.00000000 -0.058605326
## WT   -0.009259774 -0.02782651 -0.05860533  1.000000000
\end{verbatim}

\begin{Shaded}
\begin{Highlighting}[]
\KeywordTok{pairs}\NormalTok{(fem)}
\end{Highlighting}
\end{Shaded}

\includegraphics{prfe_files/figure-latex/unnamed-chunk-66-1.pdf}

~

The output can be a little confusing particularly if it includes
categorical or record identifying variables. To avoid this we can create
a new object that contains only the columns we are interested in using
the column binding \texttt{cbind()} function:

\begin{Shaded}
\begin{Highlighting}[]
\NormalTok{newfem <-}\StringTok{ }\KeywordTok{cbind}\NormalTok{(AGE, IQ, WT)}
\KeywordTok{cor}\NormalTok{(newfem, }\DataTypeTok{use =} \StringTok{"pairwise.complete.obs"}\NormalTok{)}
\end{Highlighting}
\end{Shaded}

\begin{verbatim}
##            AGE         IQ         WT
## AGE  1.0000000 -0.4345436  0.4157441
## IQ  -0.4345436  1.0000000 -0.2917158
## WT   0.4157441 -0.2917158  1.0000000
\end{verbatim}

\begin{Shaded}
\begin{Highlighting}[]
\KeywordTok{pairs}\NormalTok{(newfem)}
\end{Highlighting}
\end{Shaded}

\includegraphics{prfe_files/figure-latex/unnamed-chunk-67-1.pdf}

~

When we have finished with the \texttt{newfem} object we can delete it:

\begin{Shaded}
\begin{Highlighting}[]
\KeywordTok{rm}\NormalTok{(newfem)}
\end{Highlighting}
\end{Shaded}

~

There was no real need to create the \texttt{newfem} object as we could
have fed the output of the \texttt{cbind()} function directly to the
\texttt{cor()} or \texttt{pairs()} function:

\begin{Shaded}
\begin{Highlighting}[]
\KeywordTok{cor}\NormalTok{(}\KeywordTok{cbind}\NormalTok{(AGE, IQ, WT), }\DataTypeTok{use =} \StringTok{"pairwise.complete.obs"}\NormalTok{)}
\end{Highlighting}
\end{Shaded}

\begin{verbatim}
##            AGE         IQ         WT
## AGE  1.0000000 -0.4345436  0.4157441
## IQ  -0.4345436  1.0000000 -0.2917158
## WT   0.4157441 -0.2917158  1.0000000
\end{verbatim}

\begin{Shaded}
\begin{Highlighting}[]
\KeywordTok{pairs}\NormalTok{(}\KeywordTok{cbind}\NormalTok{(AGE, IQ, WT))}
\end{Highlighting}
\end{Shaded}

\includegraphics{prfe_files/figure-latex/unnamed-chunk-69-1.pdf}

~

It is, however, easier to work with the \texttt{newfem} object rather
than having to retype the \texttt{cbind()} function. This is
particularly true if you wanted to continue with an analysis of just the
three variables.

The relationship between \texttt{AGE} and \texttt{WT} can be plotted
using the \texttt{plot()} function:

\begin{Shaded}
\begin{Highlighting}[]
\KeywordTok{plot}\NormalTok{(AGE, WT)}
\end{Highlighting}
\end{Shaded}

\begin{center}\includegraphics{prfe_files/figure-latex/unnamed-chunk-70-1} \end{center}

~

And tested using the \texttt{cor()} and \texttt{cor.test()} functions:

\begin{Shaded}
\begin{Highlighting}[]
\KeywordTok{cor}\NormalTok{(AGE, WT, }\DataTypeTok{use =} \StringTok{"pairwise.complete.obs"}\NormalTok{)}
\end{Highlighting}
\end{Shaded}

\begin{verbatim}
## [1] 0.4157441
\end{verbatim}

\begin{Shaded}
\begin{Highlighting}[]
\KeywordTok{cor.test}\NormalTok{(AGE, WT)}
\end{Highlighting}
\end{Shaded}

\begin{verbatim}
## 
##  Pearson's product-moment correlation
## 
## data:  AGE and WT
## t = 4.6841, df = 105, p-value = 8.457e-06
## alternative hypothesis: true correlation is not equal to 0
## 95 percent confidence interval:
##  0.2452434 0.5612979
## sample estimates:
##       cor 
## 0.4157441
\end{verbatim}

~

Or by using the linear modelling \texttt{lm()} function:

\begin{Shaded}
\begin{Highlighting}[]
\KeywordTok{summary}\NormalTok{(}\KeywordTok{lm}\NormalTok{(WT }\OperatorTok{~}\StringTok{ }\NormalTok{AGE))}
\end{Highlighting}
\end{Shaded}

\begin{verbatim}
## 
## Call:
## lm(formula = WT ~ AGE)
## 
## Residuals:
##      Min       1Q   Median       3Q      Max 
## -3.10678 -0.85922 -0.05453  0.71434  2.70874 
## 
## Coefficients:
##             Estimate Std. Error t value Pr(>|t|)    
## (Intercept) -3.25405    0.85547  -3.804  0.00024 ***
## AGE          0.10592    0.02261   4.684 8.46e-06 ***
## ---
## Signif. codes:  0 '***' 0.001 '**' 0.01 '*' 0.05 '.' 0.1 ' ' 1
## 
## Residual standard error: 1.128 on 105 degrees of freedom
##   (11 observations deleted due to missingness)
## Multiple R-squared:  0.1728, Adjusted R-squared:  0.165 
## F-statistic: 21.94 on 1 and 105 DF,  p-value: 8.457e-06
\end{verbatim}

~

We use the \texttt{summary()} function here to extract summary
information from the output of the \texttt{lm()} function.

It is often more useful to use \texttt{lm()} to create an object:

\begin{Shaded}
\begin{Highlighting}[]
\NormalTok{fem.lm <-}\StringTok{ }\KeywordTok{lm}\NormalTok{(WT }\OperatorTok{~}\StringTok{ }\NormalTok{AGE)}
\end{Highlighting}
\end{Shaded}

~

And use the output in other functions:

\begin{Shaded}
\begin{Highlighting}[]
\KeywordTok{summary}\NormalTok{(fem.lm)}
\end{Highlighting}
\end{Shaded}

\begin{verbatim}
## 
## Call:
## lm(formula = WT ~ AGE)
## 
## Residuals:
##      Min       1Q   Median       3Q      Max 
## -3.10678 -0.85922 -0.05453  0.71434  2.70874 
## 
## Coefficients:
##             Estimate Std. Error t value Pr(>|t|)    
## (Intercept) -3.25405    0.85547  -3.804  0.00024 ***
## AGE          0.10592    0.02261   4.684 8.46e-06 ***
## ---
## Signif. codes:  0 '***' 0.001 '**' 0.01 '*' 0.05 '.' 0.1 ' ' 1
## 
## Residual standard error: 1.128 on 105 degrees of freedom
##   (11 observations deleted due to missingness)
## Multiple R-squared:  0.1728, Adjusted R-squared:  0.165 
## F-statistic: 21.94 on 1 and 105 DF,  p-value: 8.457e-06
\end{verbatim}

\newpage

\begin{Shaded}
\begin{Highlighting}[]
\KeywordTok{plot}\NormalTok{(AGE, WT)}
\KeywordTok{abline}\NormalTok{(fem.lm)}
\end{Highlighting}
\end{Shaded}

\begin{center}\includegraphics{prfe_files/figure-latex/unnamed-chunk-75-1} \end{center}

~

In this case we are passing the intercept and slope information held in
the \texttt{fem.lm} object to the \texttt{abline()} function which draws
a regression line. The \texttt{abline()} function adds to an existing
plot. This means that you need to keep the scatter plot of \texttt{AGE}
and \texttt{WT} open before issuing the \texttt{abline()} function call.

A useful function to apply to the \texttt{fem.lm} object is
\texttt{plot()} which produces diagnostic plots of the linear model:

\begin{Shaded}
\begin{Highlighting}[]
\KeywordTok{plot}\NormalTok{(fem.lm)}
\end{Highlighting}
\end{Shaded}

\begin{center}\includegraphics{prfe_files/figure-latex/unnamed-chunk-76-1} \end{center}

\begin{center}\includegraphics{prfe_files/figure-latex/unnamed-chunk-76-2} \end{center}

\begin{center}\includegraphics{prfe_files/figure-latex/unnamed-chunk-76-3} \end{center}

\begin{center}\includegraphics{prfe_files/figure-latex/unnamed-chunk-76-4} \end{center}

\newpage

Objects created by the \texttt{lm()} function (or any of the modelling
functions) can use up a lot of memory so we should remove them when we
no longer need them:

\begin{Shaded}
\begin{Highlighting}[]
\KeywordTok{rm}\NormalTok{(fem.lm)}
\end{Highlighting}
\end{Shaded}

It might be interesting to see whether a similar relationship exists
between \texttt{AGE} and \texttt{WT} for those who have and have not
considered suicide. This can be done using the \texttt{coplot()}
function:

\begin{Shaded}
\begin{Highlighting}[]
\KeywordTok{coplot}\NormalTok{(WT }\OperatorTok{~}\StringTok{ }\NormalTok{AGE }\OperatorTok{|}\StringTok{ }\KeywordTok{as.factor}\NormalTok{(LIFE))}
\end{Highlighting}
\end{Shaded}

\begin{center}\includegraphics{prfe_files/figure-latex/unnamed-chunk-78-1} \end{center}

\begin{verbatim}
## 
##  Missing rows: 21, 22, 31, 43, 44, 45, 69, 81, 101, 104, 114, 115
\end{verbatim}

\newpage

The two plots looks similar. We could also use \texttt{coplot()} to
investigate the relationship between \texttt{AGE} and \texttt{WT} for
categories of both \texttt{LIFE} and \texttt{SEX}:

\begin{Shaded}
\begin{Highlighting}[]
\KeywordTok{coplot}\NormalTok{(WT }\OperatorTok{~}\StringTok{ }\NormalTok{AGE }\OperatorTok{|}\StringTok{ }\KeywordTok{as.factor}\NormalTok{(LIFE) }\OperatorTok{*}\StringTok{ }\KeywordTok{as.factor}\NormalTok{(SEX))}
\end{Highlighting}
\end{Shaded}

\begin{center}\includegraphics{prfe_files/figure-latex/unnamed-chunk-79-1} \end{center}

\begin{verbatim}
## 
##  Missing rows: 12, 17, 21, 22, 31, 43, 44, 45, 66, 69, 81, 101, 104, 105, 114, 115
\end{verbatim}

~

although the numbers are too small for this to be useful here.

We used the \texttt{as.factor()} function with the \texttt{coplot()}
function to ensure that \texttt{R} was aware that the \texttt{LIFE} and
\texttt{SEX} columns hold categorical data.

We can check the way variables are stored using the
\texttt{data.class()} function:

\begin{Shaded}
\begin{Highlighting}[]
\KeywordTok{data.class}\NormalTok{(fem}\OperatorTok{$}\NormalTok{SEX)}
\end{Highlighting}
\end{Shaded}

\begin{verbatim}
## [1] "numeric"
\end{verbatim}

~

We can `apply' this function to all columns in a data.frame using the
\texttt{sapply()} function:

\begin{Shaded}
\begin{Highlighting}[]
\KeywordTok{sapply}\NormalTok{(fem, data.class)}
\end{Highlighting}
\end{Shaded}

\begin{verbatim}
##        ID       AGE        IQ       ANX       DEP       SLP       SEX 
## "numeric" "numeric" "numeric" "numeric" "numeric" "numeric" "numeric" 
##      LIFE        WT 
## "numeric" "numeric"
\end{verbatim}

~

The \texttt{sapply()} function is part of a group of functions that
apply a specified function to data objects:

\begin{longtable}[]{@{}ll@{}}
\toprule
\begin{minipage}[b]{0.27\columnwidth}\raggedright
\textbf{Function(s)}\strut
\end{minipage} & \begin{minipage}[b]{0.67\columnwidth}\raggedright
\textbf{Applies a function to \ldots{}}\strut
\end{minipage}\tabularnewline
\midrule
\endhead
\begin{minipage}[t]{0.27\columnwidth}\raggedright
\texttt{apply()}\strut
\end{minipage} & \begin{minipage}[t]{0.67\columnwidth}\raggedright
rows and columns of matrices, arrays, and tables\strut
\end{minipage}\tabularnewline
\begin{minipage}[t]{0.27\columnwidth}\raggedright
\texttt{lapply()}\strut
\end{minipage} & \begin{minipage}[t]{0.67\columnwidth}\raggedright
components of lists and data.frames\strut
\end{minipage}\tabularnewline
\begin{minipage}[t]{0.27\columnwidth}\raggedright
\texttt{sapply()}\strut
\end{minipage} & \begin{minipage}[t]{0.67\columnwidth}\raggedright
components of lists and data.frames\strut
\end{minipage}\tabularnewline
\begin{minipage}[t]{0.27\columnwidth}\raggedright
\texttt{mapply()}\strut
\end{minipage} & \begin{minipage}[t]{0.67\columnwidth}\raggedright
components of lists and data.frames\strut
\end{minipage}\tabularnewline
\begin{minipage}[t]{0.27\columnwidth}\raggedright
\texttt{tapply()}\strut
\end{minipage} & \begin{minipage}[t]{0.67\columnwidth}\raggedright
subsets of data\strut
\end{minipage}\tabularnewline
\bottomrule
\end{longtable}

Related functions are \texttt{aggregate()} which compute summary
statistics for subsets of data, \texttt{by()} which applies a function
to a data.frame split by factors, and \texttt{sweep()} which applies a
function to an array.

The parameters of most \texttt{R} functions have default values. These
are usually the most used and most useful parameter values for each
function. The \texttt{cor.test()} function, for example, calculates
\emph{Pearson's product moment correlation coefficient} by default. This
is an appropriate measure for data from a bivariate normal distribution.
The \texttt{DEP} and \texttt{ANX} variables contain ordered data. An
appropriate measure of correlation between \texttt{DEP} and \texttt{ANX}
is \emph{Kendall's tau}. This can be obtained using:

\begin{Shaded}
\begin{Highlighting}[]
\KeywordTok{cor.test}\NormalTok{(DEP, ANX, }\DataTypeTok{method =} \StringTok{"kendall"}\NormalTok{)}
\end{Highlighting}
\end{Shaded}

\begin{verbatim}
## 
##  Kendall's rank correlation tau
## 
## data:  DEP and ANX
## z = 5.5606, p-value = 2.689e-08
## alternative hypothesis: true tau is not equal to 0
## sample estimates:
##       tau 
## 0.4950723
\end{verbatim}

~

Before we finish we should save the \texttt{fem} data.frame so that next
time we want to use it we will not have to bother with recoding the
missing values to the special \texttt{NA} value. This is done with the
\texttt{write.table()} function:

\begin{Shaded}
\begin{Highlighting}[]
\KeywordTok{write.table}\NormalTok{(fem, }\DataTypeTok{file =} \StringTok{"newfem.dat"}\NormalTok{, }\DataTypeTok{row.names =} \OtherTok{FALSE}\NormalTok{)}
\end{Highlighting}
\end{Shaded}

~

Everything in \texttt{R} is either a function or an object. Even the
command to quit \texttt{R} is a function:

\begin{Shaded}
\begin{Highlighting}[]
\KeywordTok{q}\NormalTok{()}
\end{Highlighting}
\end{Shaded}

~

When you call the \texttt{q()} function you will be asked if you want to
save the workspace image. If you save the workspace image then all of
the objects and functions currently available to you will be saved.
These will then be automatically restored the next time you start
\texttt{R} in the current working directory.

For this exercise there is no need to save the workspace image so click
the \textbf{No} or \textbf{Don't Save} button (GUI) or enter \texttt{n}
when prompted to save the workspace image (terminal).

\hypertarget{summary}{%
\section{Summary}\label{summary}}

\begin{itemize}
\tightlist
\item
  \texttt{R} is a functional system. Everything is done by calling
  functions.
\item
  \texttt{R} provides a large set of functions for descriptive
  statistics, charting, and statistical inference.
\item
  Functions can be chained together so that the output of one function
  is the input of another function.
\item
  \texttt{R} is an object oriented system. We can use functions to
  create objects that can then be manipulated or passed to other
  functions for subsequent analysis.
\end{itemize}

\hypertarget{exercise2}{%
\chapter{Manipulating objects and creating new
functions}\label{exercise2}}

In this exercise we will explore how to manipulate \texttt{R} objects
and how to write functions that can manipulate and extract data and
information from \texttt{R} objects and produce useful analyses.

Before we go any further we should start \texttt{R} and retrieve a
dataset:

\begin{Shaded}
\begin{Highlighting}[]
\NormalTok{salex <-}\StringTok{ }\KeywordTok{read.table}\NormalTok{(}\StringTok{"salex.dat"}\NormalTok{, }\DataTypeTok{header =} \OtherTok{TRUE}\NormalTok{, }\DataTypeTok{na.strings =} \StringTok{"9"}\NormalTok{)}
\end{Highlighting}
\end{Shaded}

Missing values are coded as 9 throughout this dataset so we can use the
\texttt{na.strings} parameter of the \texttt{read.table()} function to
replace all 9's with the special \texttt{NA} code when we retrieve the
dataset. Check that this works by examining the \texttt{salex}
data.frame:

\begin{Shaded}
\begin{Highlighting}[]
\NormalTok{salex}
\end{Highlighting}
\end{Shaded}

\begin{verbatim}
##    ILL HAM BEEF EGGS MUSHROOM PEPPER PORKPIE PASTA RICE LETTUCE TOMATO
## 1    1   1    1    1        1      1       2     2    2       2      2
## 2    1   1    1    1        2      2       1     2    2       2      1
## 3    1   1    1    1        1      1       1     1    1       1      2
## 4    1   1    1    1        2      2       2     2    2       1      1
## 5    1   1    1    1        1      1       1     1    1       1      1
## 6    1   1    1    1        2      2       2     2    2       2      1
## 7    1   1    1    1        1      1       1     2    2       2      2
## 8    1   1    2    1        1      1       2     1    1       1      2
## 9    1   1    1    1        2      1       1     2    1       2      2
## 10   1   1    1    1        2      1       1     1    1       1      1
## 11   1   2    2    1        1      1       2     2    2       1      1
## 12   1   1    1    1        2      2       2     2    2       2      2
## 13   2   2    1    2        2      2       1     2    2       2      1
## 14   1   1    1    1        2      2       2     1    1       2      1
## 15   1   1    1    1        1      1       2     1    1       2      2
## 16   1   1    1    1        1      1       1     2    2       2      2
## 17   1   1    1    1        1      1       1     1    1       1      1
## 18   2   1    1    2        2      2       2     2    2       2      2
## 19   2   1    1    1        1      2       2     1    1       2      1
## 20   2   1    1    2        2      2       2     2    2       2      2
## 21   2   2    2    2        2      2       2     2    2       2      2
## 22   1   1    1    1        2      2       2     2    2       1      1
## 23   1   2    1    2        2      2       2     1    1       2      1
## 24   1   1    1    1        2      1       2     1    1       2      2
## 25   1   1    1    2        1      1       1     1    1       1      1
## 26   1   1    2    1        1      1       2     2    2       1      1
## 27   1   1    1    1        2      2       1     2    1       1      1
## 28   1   1    1    1        1      1       2     1    1       2      2
## 29   1   2    1    1        1     NA       2     1    1       1      1
## 30   1   1    1    2        2      2       1     2    2       2      2
## 31   1   1    1    1        1      2       2     1    1       2      2
## 32   1   1    1    1        1      2      NA     2    1       1      1
## 33   1   1    1    1        2      2       2     1    2       2      2
## 34   1   1    1    1        1      2       2     2    2       1      1
## 35   1   1    1    1        1      1       1     1    2       2      1
## 36   2   2    1    2        2      2       2     2    2       2      2
## 37   1   1    1    1        1      1       2     1    1       1      1
## 38   1   1    1    2        2      2       1     1    1       1      2
## 39   1   1    1    1        1      1       1     2    2       1      2
## 40   1   1    1    1        1      1       1     2    2       1      1
## 41   1   1    1    2        2      1       2     1    1       1      1
## 42   1   1    1    2        2      2       2     2    2       2      2
## 43   1   1    1    1        1      1       2     1    1       1      1
## 44   1   2    1    2        2      2       1     2    2       1      2
## 45   1   1    1    1        1      2       2     2    1       1      1
## 46   1   1    1    2        2      2       2     1    1       1      1
## 47   1   1    1    1        2      2       2     2    1       1      2
## 48   1   1    1    1        1     NA       1     1    1       2      2
## 49   1   1    1    1        2      1       2     2    1       1      1
## 50   1   2    1    1        2      2       2     1    2       2      1
## 51   2   2    1    2        2      2       2     2    2       2      2
## 52   2   1    1    2        2      2       2     1    2       2      1
## 53   2   1    1    2        2      2       1     2    2       2      1
## 54   2   1    1    2        1      2       1     2    2       2      1
## 55   2   1    1    1        1      1       2     2    1       2      2
## 56   2   1    1    2        2      2       2     2    2       2      1
## 57   2   1    1    1        1      1       1     2    2       2      2
## 58   2   1    1    1        2      2       1     2    1       2      2
## 59   2   1    1    2        2      2       2     2    2       2      2
## 60   2   2    2    2        2      2       1     2    2       2      2
## 61   2   1    1    2        2      2       1     2    2       2      2
## 62   2   1    2    2        2      2       2     2    2       1      1
## 63   1   1    1    1        1      1       2     2    2       2      1
## 64   2   1    1    2        2      2       2     2    2       2      2
## 65   2   1    1    1        1      2       1     2    1       2      2
## 66   2   2    1    2        2      2       2     2    2       2      2
## 67   2   2    1    2        2      2       2     2    2       2      2
## 68   2   1    1    2        1      1       1     1    2       2      1
## 69   2   2    1    2        2      2       2     2    2       2      2
## 70   2   2    1    2        2      2       2     2    2       2      2
## 71   1   1    2    2        2      2       1     2    1       2      2
## 72   2   1    2    1       NA     NA       2     2    2       2      1
## 73   1   1    1    1        2      2       1     2    2       2      2
## 74   1   1    2    1       NA     NA       2     1    1       1      1
## 75   1   1    2    2        2      1       2     1    2       1      1
## 76   1   1    1    1        2      2       1     1    2       2      2
## 77   1   1    1   NA       NA     NA       1     2    1       1      1
##    COLESLAW CRISPS PEACHCAKE CHOCOLATE FRUIT TRIFLE ALMONDS
## 1         2      2         2         2     2      2       2
## 2         2      2         2         2     2      2       2
## 3         2      1         2         1     2      2       2
## 4         2      2         2         1     2      2       2
## 5         1      2         2         1     2      1       2
## 6         1      1         2         1     2      2       2
## 7         1      1         1         2     2      2       2
## 8         1      1         2         2     2      1       2
## 9         2      2         2         2     2      1       2
## 10        1      1         2         2     2      1       1
## 11        2      2         2         2     2      2      NA
## 12        2      1         2         1     2      2       2
## 13        2      1         2         2     1      2      NA
## 14        1      1         2         2     2      1       2
## 15        1      1         2         2     2      1       1
## 16        1      2         2         2     2      2       2
## 17        1      2         2         2     2      2       2
## 18        2      2         2         2     2      2       2
## 19        1      1         2         2     1      2       2
## 20        2      2         2         1     2      2       2
## 21        2      2         2         2     2      2       2
## 22        2      1         2         1     2      2       2
## 23        1      2         2         2     2      2      NA
## 24        1      1         2         2     2      1       2
## 25        1      2         2         2     2      1      NA
## 26        1      2         2         2     2      1       2
## 27        1      1         1         1     2      1       2
## 28        2      1         2         2     2      2      NA
## 29        1      1         2         2     2      2      NA
## 30        2      2         2         2     2      2       2
## 31        2      2         2         2     2      2       2
## 32        2      2         2         2     2      2       2
## 33        1      2         2         2     2      2       2
## 34        1      2         2         2     2      1       2
## 35        1      2         2         2     2      1       2
## 36        2      2         2         2     2      2      NA
## 37        1      1         2         1     2      1       2
## 38        2      2         2         2     2      2       2
## 39        2      2         2         1     2      2       2
## 40        1      2         2         2     2      2       2
## 41        1      1         2         2    NA      1      NA
## 42        2      2         2         2     2      2      NA
## 43        1      1         2         2     2      2      NA
## 44        2      2         2         2     2      2       2
## 45        1      2         2         2     2      1       2
## 46        1      2         2         2     2      1       2
## 47        2      2         2        NA     2      1       2
## 48        2      1         2         2     2      2       2
## 49        1      1         2         2     2      1       2
## 50       NA      2         2         1     2      1       1
## 51        2      2         2         2     2      2      NA
## 52        2      2         2         1     2      2       1
## 53        2      2         2         2     1      2       2
## 54        2      2         2         2     2      2       2
## 55        2      1         2         2     2      2       2
## 56        2      2         2         2     1      2       2
## 57        1      2         2         2     2      2       1
## 58        2      1         1         2     2      2       2
## 59        2      1         1         2     2      1       2
## 60        2      2         2         2     2      2       2
## 61        2      1         2         2     2      1       1
## 62        2      1         2         2     2      2       2
## 63        1      2         2         1     1      2       2
## 64        2      2         2         2     2      2       2
## 65        1      1         2         2     2      1       2
## 66        2      2         2         2     2      1      NA
## 67        2      1         2         2     2      2       2
## 68        2      2         2         2     2      2       2
## 69        2      1         2         2     2      2       2
## 70        2      2         2         2     2      2       2
## 71        2      2         2         2     2      2       2
## 72        2      2         2         2     2      1       2
## 73        2      2         2         2     2      2       2
## 74        1      1         2         1     2      2       2
## 75        1      1         2         2     2      2      NA
## 76        2      2         2         2     2      2      NA
## 77        1      1         2         2     2      2       2
\end{verbatim}

\begin{Shaded}
\begin{Highlighting}[]
\KeywordTok{names}\NormalTok{(salex)}
\end{Highlighting}
\end{Shaded}

\begin{verbatim}
##  [1] "ILL"       "HAM"       "BEEF"      "EGGS"      "MUSHROOM" 
##  [6] "PEPPER"    "PORKPIE"   "PASTA"     "RICE"      "LETTUCE"  
## [11] "TOMATO"    "COLESLAW"  "CRISPS"    "PEACHCAKE" "CHOCOLATE"
## [16] "FRUIT"     "TRIFLE"    "ALMONDS"
\end{verbatim}

This data comes from a food-borne outbreak. On Saturday 17th October
1992, eighty-two people attended a buffet meal at a sports club. Within
fourteen to twenty-four hours, fifty-one of the participants developed
diarrhoea, with nausea, vomiting, abdominal pain and fever.

The columns in the dataset are as follows:

\begin{longtable}[]{@{}ll@{}}
\toprule
\begin{minipage}[t]{0.21\columnwidth}\raggedright
\textbf{ILL}\strut
\end{minipage} & \begin{minipage}[t]{0.30\columnwidth}\raggedright
Ill or not-ill\strut
\end{minipage}\tabularnewline
\begin{minipage}[t]{0.21\columnwidth}\raggedright
\textbf{HAM}\strut
\end{minipage} & \begin{minipage}[t]{0.30\columnwidth}\raggedright
Baked ham\strut
\end{minipage}\tabularnewline
\begin{minipage}[t]{0.21\columnwidth}\raggedright
\textbf{BEEF}\strut
\end{minipage} & \begin{minipage}[t]{0.30\columnwidth}\raggedright
Roast beef\strut
\end{minipage}\tabularnewline
\begin{minipage}[t]{0.21\columnwidth}\raggedright
\textbf{EGGS}\strut
\end{minipage} & \begin{minipage}[t]{0.30\columnwidth}\raggedright
Eggs\strut
\end{minipage}\tabularnewline
\begin{minipage}[t]{0.21\columnwidth}\raggedright
\textbf{MUSHROOM}\strut
\end{minipage} & \begin{minipage}[t]{0.30\columnwidth}\raggedright
Mushroom flan\strut
\end{minipage}\tabularnewline
\begin{minipage}[t]{0.21\columnwidth}\raggedright
\textbf{PEPPER}\strut
\end{minipage} & \begin{minipage}[t]{0.30\columnwidth}\raggedright
Pepper flan\strut
\end{minipage}\tabularnewline
\begin{minipage}[t]{0.21\columnwidth}\raggedright
\textbf{PORKPIE}\strut
\end{minipage} & \begin{minipage}[t]{0.30\columnwidth}\raggedright
Pork pie\strut
\end{minipage}\tabularnewline
\begin{minipage}[t]{0.21\columnwidth}\raggedright
\textbf{PASTA}\strut
\end{minipage} & \begin{minipage}[t]{0.30\columnwidth}\raggedright
Pasta salad\strut
\end{minipage}\tabularnewline
\begin{minipage}[t]{0.21\columnwidth}\raggedright
\textbf{RICE}\strut
\end{minipage} & \begin{minipage}[t]{0.30\columnwidth}\raggedright
Rice salad\strut
\end{minipage}\tabularnewline
\begin{minipage}[t]{0.21\columnwidth}\raggedright
\textbf{LETTUCE}\strut
\end{minipage} & \begin{minipage}[t]{0.30\columnwidth}\raggedright
Lettuce\strut
\end{minipage}\tabularnewline
\begin{minipage}[t]{0.21\columnwidth}\raggedright
\textbf{TOMATO}\strut
\end{minipage} & \begin{minipage}[t]{0.30\columnwidth}\raggedright
Tomato salad\strut
\end{minipage}\tabularnewline
\begin{minipage}[t]{0.21\columnwidth}\raggedright
\textbf{COLESLAW}\strut
\end{minipage} & \begin{minipage}[t]{0.30\columnwidth}\raggedright
Coleslaw\strut
\end{minipage}\tabularnewline
\begin{minipage}[t]{0.21\columnwidth}\raggedright
\textbf{CRISPS}\strut
\end{minipage} & \begin{minipage}[t]{0.30\columnwidth}\raggedright
Crisps\strut
\end{minipage}\tabularnewline
\begin{minipage}[t]{0.21\columnwidth}\raggedright
\textbf{PEACHCAKE}\strut
\end{minipage} & \begin{minipage}[t]{0.30\columnwidth}\raggedright
Peach cake\strut
\end{minipage}\tabularnewline
\begin{minipage}[t]{0.21\columnwidth}\raggedright
\textbf{CHOCOLATE}\strut
\end{minipage} & \begin{minipage}[t]{0.30\columnwidth}\raggedright
Chocolate cake\strut
\end{minipage}\tabularnewline
\begin{minipage}[t]{0.21\columnwidth}\raggedright
\textbf{FRUIT}\strut
\end{minipage} & \begin{minipage}[t]{0.30\columnwidth}\raggedright
Tropical fruit salad\strut
\end{minipage}\tabularnewline
\begin{minipage}[t]{0.21\columnwidth}\raggedright
\textbf{TRIFLE}\strut
\end{minipage} & \begin{minipage}[t]{0.30\columnwidth}\raggedright
Trifle\strut
\end{minipage}\tabularnewline
\begin{minipage}[t]{0.21\columnwidth}\raggedright
\textbf{ALMONDS}\strut
\end{minipage} & \begin{minipage}[t]{0.30\columnwidth}\raggedright
Almonds\strut
\end{minipage}\tabularnewline
\bottomrule
\end{longtable}

Data is available for seventy-seven of the eighty-two people who
attended the sports club buffet. All of the variables are coded 1=yes,
2=no.

We can use the \texttt{attach()} function to make it easier to access
our data:

\begin{Shaded}
\begin{Highlighting}[]
\KeywordTok{attach}\NormalTok{(salex)}
\end{Highlighting}
\end{Shaded}

\begin{verbatim}
## The following objects are masked from salex (pos = 11):
## 
##     ALMONDS, BEEF, CHOCOLATE, COLESLAW, CRISPS, EGGS, FRUIT, HAM,
##     ILL, LETTUCE, MUSHROOM, PASTA, PEACHCAKE, PEPPER, PORKPIE,
##     RICE, TOMATO, TRIFLE
\end{verbatim}

The two-by-two table is a basic epidemiological tool. In analysing data
from a food-borne outbreak collected as a retrospective cohort study,
for example, we would tabulate each exposure (suspect foodstuffs)
against the outcome (illness) and calculate risk ratios and confidence
intervals. \texttt{R} has no explicit function to calculate risk ratios
from two-by-two tables but we can easily write one ourselves.

The first step in writing such a function would be to create the
two-by-two table. This can be done with the \texttt{table()} function.
We will use a table of \texttt{HAM} by \texttt{ILL} as an illustration:

\begin{Shaded}
\begin{Highlighting}[]
\KeywordTok{table}\NormalTok{(HAM, ILL)}
\end{Highlighting}
\end{Shaded}

This command produces the following output:

\begin{verbatim}
##    ILL
## HAM  1  2
##   1 46 17
##   2  5  9
\end{verbatim}

We can manipulate the output directly but it is easier if we instruct
\texttt{R} to save the output of the \texttt{table()} function in an
object:

\begin{Shaded}
\begin{Highlighting}[]
\NormalTok{tab <-}\StringTok{ }\KeywordTok{table}\NormalTok{(HAM, ILL)}
\end{Highlighting}
\end{Shaded}

The \texttt{tab} object contains the output of the \texttt{table()}
function:

\begin{Shaded}
\begin{Highlighting}[]
\NormalTok{tab}
\end{Highlighting}
\end{Shaded}

\begin{verbatim}
##    ILL
## HAM  1  2
##   1 46 17
##   2  5  9
\end{verbatim}

As it is stored in an object we can examine its contents on an item by
item basis.

The \texttt{tab} object is an object of class \texttt{table}:

\begin{Shaded}
\begin{Highlighting}[]
\KeywordTok{class}\NormalTok{(tab)}
\end{Highlighting}
\end{Shaded}

\begin{verbatim}
## [1] "table"
\end{verbatim}

We can extract data from a table object by using indices or row and
column co-ordinates:

\begin{Shaded}
\begin{Highlighting}[]
\NormalTok{tab[}\DecValTok{1}\NormalTok{,}\DecValTok{1}\NormalTok{]}
\end{Highlighting}
\end{Shaded}

\begin{verbatim}
## [1] 46
\end{verbatim}

\begin{Shaded}
\begin{Highlighting}[]
\NormalTok{tab[}\DecValTok{1}\NormalTok{,}\DecValTok{2}\NormalTok{]}
\end{Highlighting}
\end{Shaded}

\begin{verbatim}
## [1] 17
\end{verbatim}

\begin{Shaded}
\begin{Highlighting}[]
\NormalTok{tab[}\DecValTok{2}\NormalTok{,}\DecValTok{1}\NormalTok{]}
\end{Highlighting}
\end{Shaded}

\begin{verbatim}
## [1] 5
\end{verbatim}

The numbers in the square brackets refer to the \textbf{\emph{position}}
(as row and column co-ordinates) of the data item in the table
\textbf{\emph{not}} the \textbf{\emph{values}} of the variables. We can
extract data using the values of the row and column variables by
enclosing the index values in double quotes (``). For example:

\begin{Shaded}
\begin{Highlighting}[]
\NormalTok{tab[}\StringTok{"1"}\NormalTok{,}\StringTok{"1"}\NormalTok{]}
\end{Highlighting}
\end{Shaded}

\begin{verbatim}
## [1] 46
\end{verbatim}

The two methods of extracting data may be combined. For example:

\begin{Shaded}
\begin{Highlighting}[]
\NormalTok{tab[}\DecValTok{1}\NormalTok{,}\StringTok{"1"}\NormalTok{]}
\end{Highlighting}
\end{Shaded}

\begin{verbatim}
## [1] 46
\end{verbatim}

We can calculate a risk ratio using the extracted data:

\begin{Shaded}
\begin{Highlighting}[]
\NormalTok{(tab[}\DecValTok{1}\NormalTok{,}\DecValTok{1}\NormalTok{]}\OperatorTok{/}\NormalTok{(tab[}\DecValTok{1}\NormalTok{,}\DecValTok{1}\NormalTok{]}\OperatorTok{+}\NormalTok{tab[}\DecValTok{1}\NormalTok{,}\DecValTok{2}\NormalTok{]))}\OperatorTok{/}\NormalTok{(tab[}\DecValTok{2}\NormalTok{,}\DecValTok{1}\NormalTok{]}\OperatorTok{/}\NormalTok{(tab[}\DecValTok{2}\NormalTok{,}\DecValTok{1}\NormalTok{]}\OperatorTok{+}\NormalTok{tab[}\DecValTok{2}\NormalTok{,}\DecValTok{2}\NormalTok{]))}
\end{Highlighting}
\end{Shaded}

Which returns a risk ratio of

\begin{verbatim}
## [1] 2.044444
\end{verbatim}

This is a tedious calculation to have to type in every time you need to
calculate a risk ratio from a two-by-two table. It would be better to
have a function that calculates and displays the risk ratio
automatically. Fortunately, \texttt{R} allows us to do just that.

The \texttt{function()} function allows us to create new functions in
\texttt{R}:

\begin{Shaded}
\begin{Highlighting}[]
\NormalTok{tab2by2 <-}\StringTok{ }\ControlFlowTok{function}\NormalTok{(exposure, outcome) \{\}}
\end{Highlighting}
\end{Shaded}

This creates an empty function called \texttt{tab2by2} that expects two
parameters called \texttt{exposure} and \texttt{outcome}. We could type
the whole function in at the \texttt{R} command prompt but it is easier
to use a text editor:

\begin{Shaded}
\begin{Highlighting}[]
\KeywordTok{fix}\NormalTok{(tab2by2)}
\end{Highlighting}
\end{Shaded}

This will start an editor with the empty \texttt{tab2by2()} function
already loaded. We can now edit this function to make it do something
useful:

\begin{Shaded}
\begin{Highlighting}[]
\ControlFlowTok{function}\NormalTok{(exposure, outcome)}
\NormalTok{  \{}
\NormalTok{  tab <-}\StringTok{ }\KeywordTok{table}\NormalTok{(exposure, outcome)}
\NormalTok{  a <-}\StringTok{ }\NormalTok{tab[}\DecValTok{1}\NormalTok{,}\DecValTok{1}\NormalTok{]}
\NormalTok{  b <-}\StringTok{ }\NormalTok{tab[}\DecValTok{1}\NormalTok{,}\DecValTok{2}\NormalTok{]}
\NormalTok{  c <-}\StringTok{ }\NormalTok{tab[}\DecValTok{2}\NormalTok{,}\DecValTok{1}\NormalTok{]}
\NormalTok{  d <-}\StringTok{ }\NormalTok{tab[}\DecValTok{2}\NormalTok{,}\DecValTok{2}\NormalTok{]}
\NormalTok{  rr <-}\StringTok{ }\NormalTok{(a }\OperatorTok{/}\StringTok{ }\NormalTok{(a }\OperatorTok{+}\StringTok{ }\NormalTok{b)) }\OperatorTok{/}\StringTok{ }\NormalTok{(c }\OperatorTok{/}\StringTok{ }\NormalTok{(c }\OperatorTok{+}\StringTok{ }\NormalTok{d))}
  \KeywordTok{print}\NormalTok{(tab)}
  \KeywordTok{print}\NormalTok{(rr) }
\NormalTok{  \}}
\end{Highlighting}
\end{Shaded}

Once you have made the changes shown above, check your work, save the
file, and quit the editor. Before proceeding we should examine the
\texttt{tab2by2()} function to make sure we understand what the function
will do:

\begin{itemize}
\item
  The first line defines \texttt{tab2by2} as a function that expects to
  be given two parameters which are called \texttt{exposure} and
  \texttt{outcome}.
\item
  The body of the function (i.e.~the work of the function) is enclosed
  within curly brackets (\texttt{\{\}}).
\item
  The first line of the body of the function creates a table object
  (\texttt{tab}) using the variables specified when the
  \texttt{tab2by2()} function is called (these are the parameters
  \texttt{exposure} and \texttt{outcome}).
\item
  The next line creates four new objects (called \texttt{a}, \texttt{b},
  \texttt{c}, and \texttt{d}) which contain the values of the four cells
  in the two-by-two table.
\item
  The following line calculates the risk ratio using the objects
  \texttt{a}, \texttt{b}, \texttt{c}, and \texttt{d} and stores the
  result of the calculation in an object called \texttt{rr}.
\item
  The final two lines print the contents of the \texttt{tab} and
  \texttt{rr} objects.
\end{itemize}

Let's try the \texttt{tab2by2()} function with our test data:

\begin{Shaded}
\begin{Highlighting}[]
\KeywordTok{tab2by2}\NormalTok{(HAM, ILL)}
\end{Highlighting}
\end{Shaded}

\begin{verbatim}
##         outcome
## exposure  1  2
##        1 46 17
##        2  5  9
## [1] 2.044444
\end{verbatim}

The \texttt{tab2by2()} function displays a table of \texttt{HAM} by
\texttt{ILL} followed by the risk ratio calculated from the data in the
table.

Try producing another table:

\begin{Shaded}
\begin{Highlighting}[]
\KeywordTok{tab2by2}\NormalTok{(PASTA, ILL)}
\end{Highlighting}
\end{Shaded}

\begin{verbatim}
##         outcome
## exposure  1  2
##        1 25  3
##        2 26 23
## [1] 1.682692
\end{verbatim}

Have a look at the \texttt{R} objects available to you:

\begin{Shaded}
\begin{Highlighting}[]
\KeywordTok{ls}\NormalTok{()}
\end{Highlighting}
\end{Shaded}

\begin{verbatim}
## [1] "fem"     "salex"   "tab"     "tab2by2"
\end{verbatim}

Note that there are no \texttt{a}, \texttt{b}, \texttt{c}, \texttt{d},
or \texttt{rr} objects.

Examine the \texttt{tab} object:

\begin{Shaded}
\begin{Highlighting}[]
\NormalTok{tab}
\end{Highlighting}
\end{Shaded}

\begin{verbatim}
##    ILL
## HAM  1  2
##   1 46 17
##   2  5  9
\end{verbatim}

This is the table of \texttt{HAM} by \texttt{ILL} that you created
earlier \textbf{\emph{not}} the table of \texttt{PASTA} by \texttt{ILL}
that was created by the \texttt{tab2by2()} function.

The \texttt{tab}, \texttt{a}, \texttt{b}, \texttt{c}, \texttt{d}, and
\texttt{rr} objects in the \texttt{tab2by2()} function are local to that
function and do not change anything outside of that function. This means
that the \texttt{tab} object inside the function is independent of any
object of the same name outside of the function.

When a function completes its work, all of the objects that are local to
that function are automatically removed. This is useful as it means that
you can use object names inside functions that will not interfere with
objects of the same name that are stored elsewhere. It also means that
you do not clutter up the \texttt{R} workspace with temporary objects.

Just to prove that \texttt{tab} in the \texttt{tab2by2()} function
exists only in the \texttt{tab2by2()} function we can delete the tab
object from the \texttt{R} workspace:

\begin{Shaded}
\begin{Highlighting}[]
\KeywordTok{rm}\NormalTok{(tab)}
\end{Highlighting}
\end{Shaded}

Now try another call to the \texttt{tab2by2()} function:

\begin{Shaded}
\begin{Highlighting}[]
\KeywordTok{tab2by2}\NormalTok{(FRUIT, ILL)}
\end{Highlighting}
\end{Shaded}

\begin{verbatim}
##         outcome
## exposure  1  2
##        1  1  4
##        2 49 22
## [1] 0.2897959
\end{verbatim}

Now list the \texttt{R} objects available to you:

\begin{Shaded}
\begin{Highlighting}[]
\KeywordTok{ls}\NormalTok{()}
\end{Highlighting}
\end{Shaded}

\begin{verbatim}
## [1] "fem"     "salex"   "tab2by2"
\end{verbatim}

Note that there are no \texttt{tab}, \texttt{a}, \texttt{b}, \texttt{c},
\texttt{d}, or \texttt{rr} objects.

The \texttt{tab2by2()} function is very limited. It only displays a
table and calculates and displays a simple ratio. A more useful function
would also calculate and display a confidence interval for the risk
ratio. This is what we will do now. Use the \texttt{fix()} function to
edit the \texttt{tab2by2()} function:

\begin{Shaded}
\begin{Highlighting}[]
\KeywordTok{fix}\NormalTok{(tab2by2)}
\end{Highlighting}
\end{Shaded}

We can now edit this function to calculate and display a 95\% confidence
interval for the risk ratio.

\begin{Shaded}
\begin{Highlighting}[]
\ControlFlowTok{function}\NormalTok{(exposure, outcome) \{}
\NormalTok{  tab <-}\StringTok{ }\KeywordTok{table}\NormalTok{(exposure, outcome)}
\NormalTok{  a <-}\StringTok{ }\NormalTok{tab[}\DecValTok{1}\NormalTok{,}\DecValTok{1}\NormalTok{]}
\NormalTok{  b <-}\StringTok{ }\NormalTok{tab[}\DecValTok{1}\NormalTok{,}\DecValTok{2}\NormalTok{]}
\NormalTok{  c <-}\StringTok{ }\NormalTok{tab[}\DecValTok{2}\NormalTok{,}\DecValTok{1}\NormalTok{]}
\NormalTok{  d <-}\StringTok{ }\NormalTok{tab[}\DecValTok{2}\NormalTok{,}\DecValTok{2}\NormalTok{]}
\NormalTok{  rr <-}\StringTok{ }\NormalTok{(a }\OperatorTok{/}\StringTok{ }\NormalTok{(a }\OperatorTok{+}\StringTok{ }\NormalTok{b)) }\OperatorTok{/}\StringTok{ }\NormalTok{(c }\OperatorTok{/}\StringTok{ }\NormalTok{(c }\OperatorTok{+}\StringTok{ }\NormalTok{d))}
\NormalTok{  se.log.rr <-}\StringTok{ }\KeywordTok{sqrt}\NormalTok{((b }\OperatorTok{/}\StringTok{ }\NormalTok{a) }\OperatorTok{/}\StringTok{ }\NormalTok{(a }\OperatorTok{+}\StringTok{ }\NormalTok{b) }\OperatorTok{+}\StringTok{ }\NormalTok{(d }\OperatorTok{/}\StringTok{ }\NormalTok{c) }\OperatorTok{/}\StringTok{ }\NormalTok{(c }\OperatorTok{+}\StringTok{ }\NormalTok{d)) }
\NormalTok{  lci.rr <-}\StringTok{ }\KeywordTok{exp}\NormalTok{(}\KeywordTok{log}\NormalTok{(rr) }\OperatorTok{-}\StringTok{ }\FloatTok{1.96} \OperatorTok{*}\StringTok{ }\NormalTok{se.log.rr)}
\NormalTok{  uci.rr <-}\StringTok{ }\KeywordTok{exp}\NormalTok{(}\KeywordTok{log}\NormalTok{(rr) }\OperatorTok{+}\StringTok{ }\FloatTok{1.96} \OperatorTok{*}\StringTok{ }\NormalTok{se.log.rr)}
  \KeywordTok{print}\NormalTok{(tab)}
  \KeywordTok{print}\NormalTok{(rr)}
  \KeywordTok{print}\NormalTok{(lci.rr)}
  \KeywordTok{print}\NormalTok{(uci.rr)}
\NormalTok{\}}
\end{Highlighting}
\end{Shaded}

Once you have made the changes shown above, check your work, save the
file, and quit the editor. We should test our revised function:

\begin{Shaded}
\begin{Highlighting}[]
\KeywordTok{tab2by2}\NormalTok{(EGGS, ILL)}
\end{Highlighting}
\end{Shaded}

which produces the following output:

\begin{verbatim}
##         outcome
## exposure  1  2
##        1 40  6
##        2 10 20
## [1] 2.608696
## [1] 1.553564
## [1] 4.38044
\end{verbatim}

The function works but the output could be improved. Use the
\texttt{fix()} function to edit the \texttt{tab2by2()} function:

\begin{Shaded}
\begin{Highlighting}[]
\ControlFlowTok{function}\NormalTok{(exposure, outcome) \{}
\NormalTok{  tab <-}\StringTok{ }\KeywordTok{table}\NormalTok{(exposure, outcome)}
\NormalTok{  a <-}\StringTok{ }\NormalTok{tab[}\DecValTok{1}\NormalTok{,}\DecValTok{1}\NormalTok{]}
\NormalTok{  b <-}\StringTok{ }\NormalTok{tab[}\DecValTok{1}\NormalTok{,}\DecValTok{2}\NormalTok{]}
\NormalTok{  c <-}\StringTok{ }\NormalTok{tab[}\DecValTok{2}\NormalTok{,}\DecValTok{1}\NormalTok{]}
\NormalTok{  d <-}\StringTok{ }\NormalTok{tab[}\DecValTok{2}\NormalTok{,}\DecValTok{2}\NormalTok{]}
\NormalTok{  rr <-}\StringTok{ }\NormalTok{(a }\OperatorTok{/}\StringTok{ }\NormalTok{(a }\OperatorTok{+}\StringTok{ }\NormalTok{b)) }\OperatorTok{/}\StringTok{ }\NormalTok{(c }\OperatorTok{/}\StringTok{ }\NormalTok{(c }\OperatorTok{+}\StringTok{ }\NormalTok{d))}
\NormalTok{  se.log.rr <-}\StringTok{ }\KeywordTok{sqrt}\NormalTok{((b }\OperatorTok{/}\StringTok{ }\NormalTok{a) }\OperatorTok{/}\StringTok{ }\NormalTok{(a }\OperatorTok{+}\StringTok{ }\NormalTok{b) }\OperatorTok{+}\StringTok{ }\NormalTok{(d }\OperatorTok{/}\StringTok{ }\NormalTok{c) }\OperatorTok{/}\StringTok{ }\NormalTok{(c }\OperatorTok{+}\StringTok{ }\NormalTok{d)) }
\NormalTok{  lci.rr <-}\StringTok{ }\KeywordTok{exp}\NormalTok{(}\KeywordTok{log}\NormalTok{(rr) }\OperatorTok{-}\StringTok{ }\FloatTok{1.96} \OperatorTok{*}\StringTok{ }\NormalTok{se.log.rr)}
\NormalTok{  uci.rr <-}\StringTok{ }\KeywordTok{exp}\NormalTok{(}\KeywordTok{log}\NormalTok{(rr) }\OperatorTok{+}\StringTok{ }\FloatTok{1.96} \OperatorTok{*}\StringTok{ }\NormalTok{se.log.rr)}
  \KeywordTok{print}\NormalTok{(tab)}
  \KeywordTok{cat}\NormalTok{(}\StringTok{"}\CharTok{\textbackslash{}n}\StringTok{RR :"}\NormalTok{, rr,}
      \StringTok{"}\CharTok{\textbackslash{}n}\StringTok{95% CI :"}\NormalTok{, lci.rr, uci.rr, }\StringTok{"}\CharTok{\textbackslash{}n}\StringTok{"}\NormalTok{)}
\NormalTok{\}}
\end{Highlighting}
\end{Shaded}

Once you have made the changes shown above, save the file and quit the
editor.

Now we can test our function again:

\begin{Shaded}
\begin{Highlighting}[]
\KeywordTok{tab2by2}\NormalTok{(EGGS, ILL)}
\end{Highlighting}
\end{Shaded}

Which produces the following output:

\begin{verbatim}
##         outcome
## exposure  1  2
##        1 40  6
##        2 10 20
## 
## RR : 2.608696 
## 95% CI : 1.553564 4.38044
\end{verbatim}

The \texttt{tab2by2()} function displays output but does not behave like
a standard \texttt{R} function in the sense that you cannot save the
results of the \texttt{tab2by2()} function into an object:

\begin{Shaded}
\begin{Highlighting}[]
\NormalTok{test2by2 <-}\StringTok{ }\KeywordTok{tab2by2}\NormalTok{(EGGS, ILL)}
\end{Highlighting}
\end{Shaded}

\begin{verbatim}
##         outcome
## exposure  1  2
##        1 40  6
##        2 10 20
## 
## RR : 2.608696 
## 95% CI : 1.553564 4.38044
\end{verbatim}

displays output but does not save anything in the \texttt{test2by2}
object:

\begin{Shaded}
\begin{Highlighting}[]
\NormalTok{test2by2}
\end{Highlighting}
\end{Shaded}

\begin{verbatim}
## NULL
\end{verbatim}

The returned value (\texttt{NULL}) means that \texttt{test2by2} is an
empty object. We will not worry about this at the moment as the
\texttt{tab2by2()} function is good-enough for our current purposes. In
Exercise 6 we will explore how to make our own functions behave like
standard \texttt{R} functions.

We will now add the calculation of the odds ratio and its 95\%
confidence interval to the \texttt{tab2by2()} function using the
\texttt{fix()} function.

There are two ways of doing this. We could either calculate the odds
ratio from the table and use (e.g.) the method of Woolf to calculate the
confidence interval:

\begin{Shaded}
\begin{Highlighting}[]
\NormalTok{or <-}\StringTok{ }\NormalTok{(a }\OperatorTok{/}\StringTok{ }\NormalTok{b) }\OperatorTok{/}\StringTok{ }\NormalTok{(c }\OperatorTok{/}\StringTok{ }\NormalTok{d)}
\NormalTok{se.log.or <-}\StringTok{ }\KeywordTok{sqrt}\NormalTok{(}\DecValTok{1} \OperatorTok{/}\StringTok{ }\NormalTok{a }\OperatorTok{+}\StringTok{ }\DecValTok{1} \OperatorTok{/}\StringTok{ }\NormalTok{b }\OperatorTok{+}\StringTok{ }\DecValTok{1} \OperatorTok{/}\StringTok{ }\NormalTok{c }\OperatorTok{+}\StringTok{ }\DecValTok{1} \OperatorTok{/}\StringTok{ }\NormalTok{d)}
\NormalTok{lci.or <-}\StringTok{ }\KeywordTok{exp}\NormalTok{(}\KeywordTok{log}\NormalTok{(or) }\OperatorTok{-}\StringTok{ }\FloatTok{1.96} \OperatorTok{*}\StringTok{ }\NormalTok{se.log.or)}
\NormalTok{uci.or <-}\StringTok{ }\KeywordTok{exp}\NormalTok{(}\KeywordTok{log}\NormalTok{(or) }\OperatorTok{+}\StringTok{ }\FloatTok{1.96} \OperatorTok{*}\StringTok{ }\NormalTok{se.log.or)}
\KeywordTok{cat}\NormalTok{(}\StringTok{"}\CharTok{\textbackslash{}n}\StringTok{OR     :"}\NormalTok{, or,}
    \StringTok{"}\CharTok{\textbackslash{}n}\StringTok{95% CI :"}\NormalTok{, lci.or, uci.or, }\StringTok{"}\CharTok{\textbackslash{}n}\StringTok{"}\NormalTok{)}
\end{Highlighting}
\end{Shaded}

or use the output of the \texttt{fisher.test()} function:

\begin{Shaded}
\begin{Highlighting}[]
\NormalTok{ft <-}\StringTok{ }\KeywordTok{fisher.test}\NormalTok{(tab)}
\KeywordTok{cat}\NormalTok{(}\StringTok{"}\CharTok{\textbackslash{}n}\StringTok{OR     :"}\NormalTok{, ft}\OperatorTok{$}\NormalTok{estimate,}
    \StringTok{"}\CharTok{\textbackslash{}n}\StringTok{95% CI :"}\NormalTok{, ft}\OperatorTok{$}\NormalTok{conf.int, }\StringTok{"}\CharTok{\textbackslash{}n}\StringTok{"}\NormalTok{)}
\end{Highlighting}
\end{Shaded}

Note that we can refer to components of a function's output using the
same syntax as when we refer to columns in a data.frame (e.g.
\texttt{ft\$estimate} to examine the estimate of the odds ratio from the
\texttt{fisher.test()} function stored in the object \texttt{ft}).

The names of elements in the output of a standard function such as
\texttt{fisher.test()} can be found in the documentation or the help
system. For example:

\begin{Shaded}
\begin{Highlighting}[]
\KeywordTok{help}\NormalTok{(fisher.test)}
\end{Highlighting}
\end{Shaded}

Output elements are listed under the \texttt{Value} heading.

Revise the \texttt{tab2by2()} function to include the calculation of the
odds ratio and the 95\% confidence interval. The revised function will
look something like this:

\begin{Shaded}
\begin{Highlighting}[]
\ControlFlowTok{function}\NormalTok{(exposure, outcome) \{}
\NormalTok{  tab <-}\StringTok{ }\KeywordTok{table}\NormalTok{(exposure, outcome)}
\NormalTok{  a <-}\StringTok{ }\NormalTok{tab[}\DecValTok{1}\NormalTok{,}\DecValTok{1}\NormalTok{]}
\NormalTok{  b <-}\StringTok{ }\NormalTok{tab[}\DecValTok{1}\NormalTok{,}\DecValTok{2}\NormalTok{]}
\NormalTok{  c <-}\StringTok{ }\NormalTok{tab[}\DecValTok{2}\NormalTok{,}\DecValTok{1}\NormalTok{]}
\NormalTok{  d <-}\StringTok{ }\NormalTok{tab[}\DecValTok{2}\NormalTok{,}\DecValTok{2}\NormalTok{]}
\NormalTok{  rr <-}\StringTok{ }\NormalTok{(a }\OperatorTok{/}\StringTok{ }\NormalTok{(a }\OperatorTok{+}\StringTok{ }\NormalTok{b)) }\OperatorTok{/}\StringTok{ }\NormalTok{(c }\OperatorTok{/}\StringTok{ }\NormalTok{(c }\OperatorTok{+}\StringTok{ }\NormalTok{d))}
\NormalTok{  se.log.rr <-}\StringTok{ }\KeywordTok{sqrt}\NormalTok{((b }\OperatorTok{/}\StringTok{ }\NormalTok{a) }\OperatorTok{/}\StringTok{ }\NormalTok{(a }\OperatorTok{+}\StringTok{ }\NormalTok{b) }\OperatorTok{+}\StringTok{ }\NormalTok{(d }\OperatorTok{/}\StringTok{ }\NormalTok{c) }\OperatorTok{/}\StringTok{ }\NormalTok{(c }\OperatorTok{+}\StringTok{ }\NormalTok{d)) }
\NormalTok{  lci.rr <-}\StringTok{ }\KeywordTok{exp}\NormalTok{(}\KeywordTok{log}\NormalTok{(rr) }\OperatorTok{-}\StringTok{ }\FloatTok{1.96} \OperatorTok{*}\StringTok{ }\NormalTok{se.log.rr)}
\NormalTok{  uci.rr <-}\StringTok{ }\KeywordTok{exp}\NormalTok{(}\KeywordTok{log}\NormalTok{(rr) }\OperatorTok{+}\StringTok{ }\FloatTok{1.96} \OperatorTok{*}\StringTok{ }\NormalTok{se.log.rr)}
\NormalTok{  or <-}\StringTok{ }\NormalTok{(a }\OperatorTok{/}\StringTok{ }\NormalTok{b) }\OperatorTok{/}\StringTok{ }\NormalTok{(c }\OperatorTok{/}\StringTok{ }\NormalTok{d)}
\NormalTok{  se.log.or <-}\StringTok{ }\KeywordTok{sqrt}\NormalTok{(}\DecValTok{1} \OperatorTok{/}\StringTok{ }\NormalTok{a }\OperatorTok{+}\StringTok{ }\DecValTok{1} \OperatorTok{/}\StringTok{ }\NormalTok{b }\OperatorTok{+}\StringTok{ }\DecValTok{1} \OperatorTok{/}\StringTok{ }\NormalTok{c }\OperatorTok{+}\StringTok{ }\DecValTok{1} \OperatorTok{/}\StringTok{ }\NormalTok{d)}
\NormalTok{  lci.or <-}\StringTok{ }\KeywordTok{exp}\NormalTok{(}\KeywordTok{log}\NormalTok{(or) }\OperatorTok{-}\StringTok{ }\FloatTok{1.96} \OperatorTok{*}\StringTok{ }\NormalTok{se.log.or)}
\NormalTok{  uci.or <-}\StringTok{ }\KeywordTok{exp}\NormalTok{(}\KeywordTok{log}\NormalTok{(or) }\OperatorTok{+}\StringTok{ }\FloatTok{1.96} \OperatorTok{*}\StringTok{ }\NormalTok{se.log.or)}
\NormalTok{  ft <-}\StringTok{ }\KeywordTok{fisher.test}\NormalTok{(tab)}
  \KeywordTok{cat}\NormalTok{(}\StringTok{"}\CharTok{\textbackslash{}n}\StringTok{"}\NormalTok{)}
  \KeywordTok{print}\NormalTok{(tab)}
  
  \KeywordTok{cat}\NormalTok{(}\StringTok{"}\CharTok{\textbackslash{}n}\StringTok{Relative Risk     :"}\NormalTok{, rr,}
      \StringTok{"}\CharTok{\textbackslash{}n}\StringTok{95% CI            :"}\NormalTok{, lci.rr, uci.rr, }\StringTok{"}\CharTok{\textbackslash{}n}\StringTok{"}\NormalTok{)}
  
  \KeywordTok{cat}\NormalTok{(}\StringTok{"}\CharTok{\textbackslash{}n}\StringTok{Sample Odds Ratio :"}\NormalTok{, or,}
      \StringTok{"}\CharTok{\textbackslash{}n}\StringTok{95% CI            :"}\NormalTok{, lci.or, uci.or, }\StringTok{"}\CharTok{\textbackslash{}n}\StringTok{"}\NormalTok{)}

  \KeywordTok{cat}\NormalTok{(}\StringTok{"}\CharTok{\textbackslash{}n}\StringTok{MLE Odds Ratio    :"}\NormalTok{, ft}\OperatorTok{$}\NormalTok{estimate,}
      \StringTok{"}\CharTok{\textbackslash{}n}\StringTok{95% CI             :"}\NormalTok{,  ft}\OperatorTok{$}\NormalTok{conf.int, }\StringTok{"}\CharTok{\textbackslash{}n\textbackslash{}n}\StringTok{"}\NormalTok{)}
\NormalTok{\}}
\end{Highlighting}
\end{Shaded}

Once you have made the changes shown above, check your work, save the
file, and quit the editor.

Test the \texttt{tab2by2()} function when you have added the calculation
of the odds ratio and its 95\% confidence interval.

Now that we have a function that will calculate risk ratios and odds
ratios with confidence intervals from a two- by-two table we can use it
to analyse the \texttt{salex} data:

\begin{Shaded}
\begin{Highlighting}[]
\KeywordTok{tab2by2}\NormalTok{(HAM, ILL)}
\end{Highlighting}
\end{Shaded}

\begin{verbatim}
## 
##         outcome
## exposure  1  2
##        1 46 17
##        2  5  9
## 
## Relative Risk     : 2.044444 
## 95% CI            : 0.9964841 4.194501 
## 
## Sample Odds Ratio : 4.870588 
## 95% CI            : 1.428423 16.60756 
## 
## MLE Odds Ratio    : 4.75649 
## 95% CI             : 1.22777 20.82921
\end{verbatim}

\begin{Shaded}
\begin{Highlighting}[]
\KeywordTok{tab2by2}\NormalTok{(BEEF, ILL)}
\end{Highlighting}
\end{Shaded}

\begin{verbatim}
## 
##         outcome
## exposure  1  2
##        1 45 22
##        2  6  4
## 
## Relative Risk     : 1.119403 
## 95% CI            : 0.6568821 1.907592 
## 
## Sample Odds Ratio : 1.363636 
## 95% CI            : 0.3485746 5.334594 
## 
## MLE Odds Ratio    : 1.357903 
## 95% CI             : 0.2547114 6.428414
\end{verbatim}

\begin{Shaded}
\begin{Highlighting}[]
\KeywordTok{tab2by2}\NormalTok{(EGGS, ILL)}
\end{Highlighting}
\end{Shaded}

\begin{verbatim}
## 
##         outcome
## exposure  1  2
##        1 40  6
##        2 10 20
## 
## Relative Risk     : 2.608696 
## 95% CI            : 1.553564 4.38044 
## 
## Sample Odds Ratio : 13.33333 
## 95% CI            : 4.240168 41.92706 
## 
## MLE Odds Ratio    : 12.74512 
## 95% CI             : 3.762787 50.05419
\end{verbatim}

\begin{Shaded}
\begin{Highlighting}[]
\KeywordTok{tab2by2}\NormalTok{(MUSHROOM, ILL)}
\end{Highlighting}
\end{Shaded}

\begin{verbatim}
## 
##         outcome
## exposure  1  2
##        1 24  6
##        2 25 19
## 
## Relative Risk     : 1.408 
## 95% CI            : 1.028944 1.926697 
## 
## Sample Odds Ratio : 3.04 
## 95% CI            : 1.037274 8.909506 
## 
## MLE Odds Ratio    : 2.995207 
## 95% CI             : 0.9421008 10.7953
\end{verbatim}

\begin{Shaded}
\begin{Highlighting}[]
\KeywordTok{tab2by2}\NormalTok{(PEPPER, ILL)}
\end{Highlighting}
\end{Shaded}

\begin{verbatim}
## 
##         outcome
## exposure  1  2
##        1 24  3
##        2 23 22
## 
## Relative Risk     : 1.73913 
## 95% CI            : 1.26876 2.383882 
## 
## Sample Odds Ratio : 7.652174 
## 95% CI            : 2.013718 29.07844 
## 
## MLE Odds Ratio    : 7.448216 
## 95% CI             : 1.861728 44.12015
\end{verbatim}

\begin{Shaded}
\begin{Highlighting}[]
\KeywordTok{tab2by2}\NormalTok{(PORKPIE, ILL)}
\end{Highlighting}
\end{Shaded}

\begin{verbatim}
## 
##         outcome
## exposure  1  2
##        1 21  9
##        2 29 17
## 
## Relative Risk     : 1.110345 
## 95% CI            : 0.8044752 1.532509 
## 
## Sample Odds Ratio : 1.367816 
## 95% CI            : 0.5113158 3.659032 
## 
## MLE Odds Ratio    : 1.362228 
## 95% CI             : 0.4636016 4.190667
\end{verbatim}

\begin{Shaded}
\begin{Highlighting}[]
\KeywordTok{tab2by2}\NormalTok{(PASTA, ILL)}
\end{Highlighting}
\end{Shaded}

\begin{verbatim}
## 
##         outcome
## exposure  1  2
##        1 25  3
##        2 26 23
## 
## Relative Risk     : 1.682692 
## 95% CI            : 1.255392 2.255433 
## 
## Sample Odds Ratio : 7.371795 
## 95% CI            : 1.964371 27.66451 
## 
## MLE Odds Ratio    : 7.195422 
## 95% CI             : 1.829867 42.07488
\end{verbatim}

\begin{Shaded}
\begin{Highlighting}[]
\KeywordTok{tab2by2}\NormalTok{(RICE, ILL)}
\end{Highlighting}
\end{Shaded}

\begin{verbatim}
## 
##         outcome
## exposure  1  2
##        1 28  4
##        2 23 22
## 
## Relative Risk     : 1.711957 
## 95% CI            : 1.250197 2.344268 
## 
## Sample Odds Ratio : 6.695652 
## 95% CI            : 2.017327 22.22335 
## 
## MLE Odds Ratio    : 6.532868 
## 95% CI             : 1.852297 29.84928
\end{verbatim}

\begin{Shaded}
\begin{Highlighting}[]
\KeywordTok{tab2by2}\NormalTok{(LETTUCE, ILL)}
\end{Highlighting}
\end{Shaded}

\begin{verbatim}
## 
##         outcome
## exposure  1  2
##        1 28  1
##        2 23 25
## 
## Relative Risk     : 2.014993 
## 95% CI            : 1.488481 2.727744 
## 
## Sample Odds Ratio : 30.43478 
## 95% CI            : 3.826938 242.041 
## 
## MLE Odds Ratio    : 29.32825 
## 95% CI             : 4.161299 1284.306
\end{verbatim}

\begin{Shaded}
\begin{Highlighting}[]
\KeywordTok{tab2by2}\NormalTok{(TOMATO, ILL)}
\end{Highlighting}
\end{Shaded}

\begin{verbatim}
## 
##         outcome
## exposure  1  2
##        1 29  9
##        2 22 17
## 
## Relative Risk     : 1.352871 
## 95% CI            : 0.974698 1.877771 
## 
## Sample Odds Ratio : 2.489899 
## 95% CI            : 0.9347213 6.632562 
## 
## MLE Odds Ratio    : 2.459981 
## 95% CI             : 0.8467562 7.558026
\end{verbatim}

\begin{Shaded}
\begin{Highlighting}[]
\KeywordTok{tab2by2}\NormalTok{(COLESLAW, ILL)}
\end{Highlighting}
\end{Shaded}

\begin{verbatim}
## 
##         outcome
## exposure  1  2
##        1 29  3
##        2 21 23
## 
## Relative Risk     : 1.89881 
## 95% CI            : 1.366876 2.63775 
## 
## Sample Odds Ratio : 10.5873 
## 95% CI            : 2.806364 39.9417 
## 
## MLE Odds Ratio    : 10.26269 
## 95% CI             : 2.600771 60.35431
\end{verbatim}

\begin{Shaded}
\begin{Highlighting}[]
\KeywordTok{tab2by2}\NormalTok{(CRISPS, ILL)}
\end{Highlighting}
\end{Shaded}

\begin{verbatim}
## 
##         outcome
## exposure  1  2
##        1 21 10
##        2 30 16
## 
## Relative Risk     : 1.03871 
## 95% CI            : 0.7529065 1.433004 
## 
## Sample Odds Ratio : 1.12 
## 95% CI            : 0.4258139 2.945888 
## 
## MLE Odds Ratio    : 1.118358 
## 95% CI             : 0.3858206 3.340535
\end{verbatim}

\begin{Shaded}
\begin{Highlighting}[]
\KeywordTok{tab2by2}\NormalTok{(PEACHCAKE, ILL)}
\end{Highlighting}
\end{Shaded}

\begin{verbatim}
## 
##         outcome
## exposure  1  2
##        1  2  2
##        2 49 24
## 
## Relative Risk     : 0.744898 
## 95% CI            : 0.27594 2.010846 
## 
## Sample Odds Ratio : 0.4897959 
## 95% CI            : 0.06497947 3.691936 
## 
## MLE Odds Ratio    : 0.4947099 
## 95% CI             : 0.03393887 7.209143
\end{verbatim}

\begin{Shaded}
\begin{Highlighting}[]
\KeywordTok{tab2by2}\NormalTok{(CHOCOLATE, ILL)}
\end{Highlighting}
\end{Shaded}

\begin{verbatim}
## 
##         outcome
## exposure  1  2
##        1 12  2
##        2 38 24
## 
## Relative Risk     : 1.398496 
## 95% CI            : 1.045064 1.871456 
## 
## Sample Odds Ratio : 3.789474 
## 95% CI            : 0.7791326 18.43089 
## 
## MLE Odds Ratio    : 3.733535 
## 95% CI             : 0.7318646 37.28268
\end{verbatim}

\begin{Shaded}
\begin{Highlighting}[]
\KeywordTok{tab2by2}\NormalTok{(FRUIT, ILL)}
\end{Highlighting}
\end{Shaded}

\begin{verbatim}
## 
##         outcome
## exposure  1  2
##        1  1  4
##        2 49 22
## 
## Relative Risk     : 0.2897959 
## 95% CI            : 0.04985828 1.684408 
## 
## Sample Odds Ratio : 0.1122449 
## 95% CI            : 0.01185022 1.06318 
## 
## MLE Odds Ratio    : 0.1157141 
## 95% CI             : 0.002240848 1.256134
\end{verbatim}

\begin{Shaded}
\begin{Highlighting}[]
\KeywordTok{tab2by2}\NormalTok{(TRIFLE, ILL)}
\end{Highlighting}
\end{Shaded}

\begin{verbatim}
## 
##         outcome
## exposure  1  2
##        1 19  5
##        2 32 21
## 
## Relative Risk     : 1.311198 
## 95% CI            : 0.9718621 1.769016 
## 
## Sample Odds Ratio : 2.49375 
## 95% CI            : 0.8067804 7.708156 
## 
## MLE Odds Ratio    : 2.465794 
## 95% CI             : 0.7363311 9.778463
\end{verbatim}

\begin{Shaded}
\begin{Highlighting}[]
\KeywordTok{tab2by2}\NormalTok{(ALMONDS, ILL)}
\end{Highlighting}
\end{Shaded}

\begin{verbatim}
## 
##         outcome
## exposure  1  2
##        1  3  3
##        2 38 19
## 
## Relative Risk     : 0.75 
## 95% CI            : 0.3300089 1.7045 
## 
## Sample Odds Ratio : 0.5 
## 95% CI            : 0.09203498 2.716358 
## 
## MLE Odds Ratio    : 0.505905 
## 95% CI             : 0.06170211 4.141891
\end{verbatim}

Make a note of any positive associations (i.e.~with a risk ratio
\textgreater{} 1 with a 95\% confidence intervals that does not include
one). We will use these for the next exercise when we will use logistic
regression to analyse this data.

Save the \texttt{tab2by2()} function:

\begin{Shaded}
\begin{Highlighting}[]
\KeywordTok{save}\NormalTok{(tab2by2, }\DataTypeTok{file =} \StringTok{"tab2by2.r"}\NormalTok{)}
\end{Highlighting}
\end{Shaded}

We can now quit \texttt{R}:

\begin{Shaded}
\begin{Highlighting}[]
\KeywordTok{q}\NormalTok{()}
\end{Highlighting}
\end{Shaded}

For this exercise there is no need to save the workspace image so click
the \textbf{No} or \textbf{Don't Save} button (GUI) or enter \texttt{n}
when prompted to save the workspace image (terminal).

\hypertarget{summary-1}{%
\section{Summary}\label{summary-1}}

\begin{itemize}
\item
  \texttt{R} objects contain information that can be examined and
  manipulated.
\item
  \texttt{R} can be extended by writing new functions.
\item
  New functions can perform simple or complex data analysis.
\item
  New functions can be composed of parts of existing function.
\item
  New functions can be saved and used in subsequent \texttt{R} sessions.
\item
  Objects defined within functions are local to that function and only
  exist while that function is being used. This means that you can
  re-use meaningful names within functions without them interfering with
  each other.
\end{itemize}

\hypertarget{exercise3}{%
\chapter{Logistic regression and stratified analysis}\label{exercise3}}

In this exercise we will explore how \texttt{R} handles generalised
linear models using the example of logistic regression. We will continue
using the \texttt{salex} dataset. Start \texttt{R} and retrieve the
\texttt{salex} dataset:

\begin{Shaded}
\begin{Highlighting}[]
\NormalTok{salex <-}\StringTok{ }\KeywordTok{read.table}\NormalTok{(}\StringTok{"salex.dat"}\NormalTok{, }\DataTypeTok{header =} \OtherTok{TRUE}\NormalTok{, }\DataTypeTok{na.strings =} \StringTok{"9"}\NormalTok{)}
\end{Highlighting}
\end{Shaded}

When we analysed this data using two-by-two tables and examining the
risk ratio and 95\% confidence interval associated with each exposure we
found many significant positive associations:

\begin{longtable}[]{@{}lll@{}}
\toprule
\begin{minipage}[b]{0.20\columnwidth}\raggedright
\textbf{Variable}\strut
\end{minipage} & \begin{minipage}[b]{0.14\columnwidth}\raggedright
\textbf{RR}\strut
\end{minipage} & \begin{minipage}[b]{0.27\columnwidth}\raggedright
\textbf{95\% CI}\strut
\end{minipage}\tabularnewline
\midrule
\endhead
\begin{minipage}[t]{0.20\columnwidth}\raggedright
EGGS\strut
\end{minipage} & \begin{minipage}[t]{0.14\columnwidth}\raggedright
2.61\strut
\end{minipage} & \begin{minipage}[t]{0.27\columnwidth}\raggedright
1.55, 4.38\strut
\end{minipage}\tabularnewline
\begin{minipage}[t]{0.20\columnwidth}\raggedright
MUSHROOM\strut
\end{minipage} & \begin{minipage}[t]{0.14\columnwidth}\raggedright
1.41\strut
\end{minipage} & \begin{minipage}[t]{0.27\columnwidth}\raggedright
1.03, 1.93\strut
\end{minipage}\tabularnewline
\begin{minipage}[t]{0.20\columnwidth}\raggedright
PEPPER\strut
\end{minipage} & \begin{minipage}[t]{0.14\columnwidth}\raggedright
1.74\strut
\end{minipage} & \begin{minipage}[t]{0.27\columnwidth}\raggedright
1.27, 2.38\strut
\end{minipage}\tabularnewline
\begin{minipage}[t]{0.20\columnwidth}\raggedright
PASTA\strut
\end{minipage} & \begin{minipage}[t]{0.14\columnwidth}\raggedright
1.68\strut
\end{minipage} & \begin{minipage}[t]{0.27\columnwidth}\raggedright
1.26, 2.26\strut
\end{minipage}\tabularnewline
\begin{minipage}[t]{0.20\columnwidth}\raggedright
RICE\strut
\end{minipage} & \begin{minipage}[t]{0.14\columnwidth}\raggedright
1.72\strut
\end{minipage} & \begin{minipage}[t]{0.27\columnwidth}\raggedright
1.25, 2.34\strut
\end{minipage}\tabularnewline
\begin{minipage}[t]{0.20\columnwidth}\raggedright
LETTUCE\strut
\end{minipage} & \begin{minipage}[t]{0.14\columnwidth}\raggedright
2.01\strut
\end{minipage} & \begin{minipage}[t]{0.27\columnwidth}\raggedright
1.49, 2.73\strut
\end{minipage}\tabularnewline
\begin{minipage}[t]{0.20\columnwidth}\raggedright
COLESLAW\strut
\end{minipage} & \begin{minipage}[t]{0.14\columnwidth}\raggedright
1.89\strut
\end{minipage} & \begin{minipage}[t]{0.27\columnwidth}\raggedright
1.37, 2.64\strut
\end{minipage}\tabularnewline
\begin{minipage}[t]{0.20\columnwidth}\raggedright
CHOCOLATE\strut
\end{minipage} & \begin{minipage}[t]{0.14\columnwidth}\raggedright
1.39\strut
\end{minipage} & \begin{minipage}[t]{0.27\columnwidth}\raggedright
1.05, 1.87\strut
\end{minipage}\tabularnewline
\bottomrule
\end{longtable}

Some of these associations may be due to \emph{confounding} in the data.
We can use logistic regression to help us identify independent
associations.

Logistic regression requires the dependent variable to be either 0 or 1.
In order to perform a logistic regression we must first recode the
\texttt{ILL} variable so that 0=no and 1=yes:

\begin{Shaded}
\begin{Highlighting}[]
\KeywordTok{table}\NormalTok{(salex}\OperatorTok{$}\NormalTok{ILL)}
\end{Highlighting}
\end{Shaded}

\begin{verbatim}
## 
##  1  2 
## 51 26
\end{verbatim}

\begin{Shaded}
\begin{Highlighting}[]
\NormalTok{salex}\OperatorTok{$}\NormalTok{ILL[salex}\OperatorTok{$}\NormalTok{ILL }\OperatorTok{==}\StringTok{ }\DecValTok{2}\NormalTok{] <-}\StringTok{ }\DecValTok{0}
\KeywordTok{table}\NormalTok{(salex}\OperatorTok{$}\NormalTok{ILL)}
\end{Highlighting}
\end{Shaded}

\begin{verbatim}
## 
##  0  1 
## 26 51
\end{verbatim}

We could work with our data as it is but if we wanted to calculate odds
ratios and confidence intervals we would calculate their reciprocals
(i.e.~odds ratios for non-exposure rather than for exposure). This is
because of the way the data has been coded (1=yes, 2=no).

In order to calculate meaningful odds ratios the exposure variables
should also be coded 0=no, 1=yes. The actual codes used are not
important as long as the value used for `exposed' is one greater than
the value used for `not exposed'.

We could issue a series of commands similar to the one we have just used
to recode the \texttt{ILL} variable. This is both tedious and
unnecessary as the structure of the dataset (i.e.~all variables are
coded identically) allows us to recode all variables with a single
command:

\begin{Shaded}
\begin{Highlighting}[]
\NormalTok{salex <-}\StringTok{ }\KeywordTok{read.table}\NormalTok{(}\StringTok{"salex.dat"}\NormalTok{, }\DataTypeTok{header =} \OtherTok{TRUE}\NormalTok{, }\DataTypeTok{na.strings =} \StringTok{"9"}\NormalTok{)}
\NormalTok{salex[}\DecValTok{1}\OperatorTok{:}\DecValTok{5}\NormalTok{, ]}
\end{Highlighting}
\end{Shaded}

\begin{verbatim}
##   ILL HAM BEEF EGGS MUSHROOM PEPPER PORKPIE PASTA RICE LETTUCE TOMATO
## 1   1   1    1    1        1      1       2     2    2       2      2
## 2   1   1    1    1        2      2       1     2    2       2      1
## 3   1   1    1    1        1      1       1     1    1       1      2
## 4   1   1    1    1        2      2       2     2    2       1      1
## 5   1   1    1    1        1      1       1     1    1       1      1
##   COLESLAW CRISPS PEACHCAKE CHOCOLATE FRUIT TRIFLE ALMONDS
## 1        2      2         2         2     2      2       2
## 2        2      2         2         2     2      2       2
## 3        2      1         2         1     2      2       2
## 4        2      2         2         1     2      2       2
## 5        1      2         2         1     2      1       2
\end{verbatim}

\begin{Shaded}
\begin{Highlighting}[]
\NormalTok{salex <-}\StringTok{ }\DecValTok{2} \OperatorTok{-}\StringTok{ }\NormalTok{salex}
\NormalTok{salex[}\DecValTok{1}\OperatorTok{:}\DecValTok{5}\NormalTok{, ]}
\end{Highlighting}
\end{Shaded}

\begin{verbatim}
##   ILL HAM BEEF EGGS MUSHROOM PEPPER PORKPIE PASTA RICE LETTUCE TOMATO
## 1   1   1    1    1        1      1       0     0    0       0      0
## 2   1   1    1    1        0      0       1     0    0       0      1
## 3   1   1    1    1        1      1       1     1    1       1      0
## 4   1   1    1    1        0      0       0     0    0       1      1
## 5   1   1    1    1        1      1       1     1    1       1      1
##   COLESLAW CRISPS PEACHCAKE CHOCOLATE FRUIT TRIFLE ALMONDS
## 1        0      0         0         0     0      0       0
## 2        0      0         0         0     0      0       0
## 3        0      1         0         1     0      0       0
## 4        0      0         0         1     0      0       0
## 5        1      0         0         1     0      1       0
\end{verbatim}

\textbf{\emph{WARNING}} : The \texttt{attach()} function works with a
copy of the data.frame rather than the original data.frame. Commands
that manipulate variables in a data.frame may not work as expected if
the data.frame has been attached using the \texttt{attach()} function.

It is better to manipulate data \textbf{\emph{before}} attaching a
data.frame. The \texttt{detach()} function may be used to remove an
attachment prior to any data manipulation.

Many \texttt{R} users avoid using the \texttt{attach()} function
altogether.

We can now use the generalised linear model \texttt{glm()} function to
specify the logistic regression model:

\begin{Shaded}
\begin{Highlighting}[]
\NormalTok{salex.lreg <-}\StringTok{ }\KeywordTok{glm}\NormalTok{(}\DataTypeTok{formula =}\NormalTok{ ILL }\OperatorTok{~}\StringTok{ }\NormalTok{EGGS }\OperatorTok{+}\StringTok{ }\NormalTok{MUSHROOM }\OperatorTok{+}\StringTok{ }\NormalTok{PEPPER }\OperatorTok{+}\StringTok{ }\NormalTok{PASTA }\OperatorTok{+}
\StringTok{                  }\NormalTok{RICE }\OperatorTok{+}\StringTok{ }\NormalTok{LETTUCE }\OperatorTok{+}\StringTok{ }\NormalTok{COLESLAW }\OperatorTok{+}\StringTok{ }\NormalTok{CHOCOLATE,}
                  \DataTypeTok{family =} \KeywordTok{binomial}\NormalTok{(logit), }\DataTypeTok{data =}\NormalTok{ salex)}
\end{Highlighting}
\end{Shaded}

The method used by the \texttt{glm()} function is defined by the
\texttt{family} parameter. Here we specify \texttt{binomial} errors and
a \texttt{logit} (logistic) linking function.

We have saved the output of the \texttt{glm()} function in the
\texttt{salex.lreg} object. We can examine some basic information about
the specified model using the \texttt{summary()} function:

\begin{Shaded}
\begin{Highlighting}[]
\KeywordTok{summary}\NormalTok{(salex.lreg)}
\end{Highlighting}
\end{Shaded}

\begin{verbatim}
## 
## Call:
## glm(formula = ILL ~ EGGS + MUSHROOM + PEPPER + PASTA + RICE + 
##     LETTUCE + COLESLAW + CHOCOLATE, family = binomial(logit), 
##     data = salex)
## 
## Deviance Residuals: 
##      Min        1Q    Median        3Q       Max  
## -1.92036  -0.49869   0.06877   0.40906   2.07182  
## 
## Coefficients:
##              Estimate Std. Error z value Pr(>|z|)   
## (Intercept) -2.021864   0.676606  -2.988  0.00281 **
## EGGS         3.579366   1.267870   2.823  0.00476 **
## MUSHROOM    -3.584345   1.728999  -2.073  0.03817 * 
## PEPPER       2.348074   1.428177   1.644  0.10015   
## PASTA        1.774818   1.162762   1.526  0.12692   
## RICE         0.114180   1.193840   0.096  0.92381   
## LETTUCE      3.401828   1.234060   2.757  0.00584 **
## COLESLAW     0.763857   1.024373   0.746  0.45586   
## CHOCOLATE    0.009782   1.314683   0.007  0.99406   
## ---
## Signif. codes:  0 '***' 0.001 '**' 0.01 '*' 0.05 '.' 0.1 ' ' 1
## 
## (Dispersion parameter for binomial family taken to be 1)
## 
##     Null deviance: 91.246  on 69  degrees of freedom
## Residual deviance: 41.260  on 61  degrees of freedom
##   (7 observations deleted due to missingness)
## AIC: 59.26
## 
## Number of Fisher Scoring iterations: 7
\end{verbatim}

We will use \emph{backwards elimination} to remove non-significant
variables from the logistic regression model. Remember that previous
commands can be recalled and edited using the up and down arrow keys --
they do not need to be typed out in full each time.

\texttt{CHOCOLATE} is the least significant variable in the model so we
will remove this variable from the model. Storing the output of the
\texttt{glm()} function is useful as it allows us to use the
\texttt{update()} function to add, remove, or modify variables without
having to describe the model in full:

\begin{Shaded}
\begin{Highlighting}[]
\NormalTok{salex.lreg <-}\StringTok{ }\KeywordTok{update}\NormalTok{(salex.lreg, . }\OperatorTok{~}\StringTok{ }\NormalTok{. }\OperatorTok{-}\StringTok{ }\NormalTok{CHOCOLATE)}
\KeywordTok{summary}\NormalTok{(salex.lreg)}
\end{Highlighting}
\end{Shaded}

\begin{verbatim}
## 
## Call:
## glm(formula = ILL ~ EGGS + MUSHROOM + PEPPER + PASTA + RICE + 
##     LETTUCE + COLESLAW, family = binomial(logit), data = salex)
## 
## Deviance Residuals: 
##      Min        1Q    Median        3Q       Max  
## -1.92561  -0.49859   0.07555   0.38723   2.07200  
## 
## Coefficients:
##             Estimate Std. Error z value Pr(>|z|)   
## (Intercept)  -2.0223     0.6623  -3.053  0.00226 **
## EGGS          3.5890     1.2188   2.945  0.00323 **
## MUSHROOM     -3.5992     1.6885  -2.132  0.03305 * 
## PEPPER        2.3544     1.4275   1.649  0.09910 . 
## PASTA         1.7770     1.1215   1.585  0.11308   
## RICE          0.1170     1.1388   0.103  0.91819   
## LETTUCE       3.4109     1.2316   2.770  0.00561 **
## COLESLAW      0.7630     1.0224   0.746  0.45547   
## ---
## Signif. codes:  0 '***' 0.001 '**' 0.01 '*' 0.05 '.' 0.1 ' ' 1
## 
## (Dispersion parameter for binomial family taken to be 1)
## 
##     Null deviance: 92.122  on 70  degrees of freedom
## Residual deviance: 41.273  on 63  degrees of freedom
##   (6 observations deleted due to missingness)
## AIC: 57.273
## 
## Number of Fisher Scoring iterations: 7
\end{verbatim}

\texttt{RICE} is now the least significant variable in the model so we
will remove this variable from the model:

\begin{Shaded}
\begin{Highlighting}[]
\NormalTok{salex.lreg <-}\StringTok{ }\KeywordTok{update}\NormalTok{(salex.lreg, . }\OperatorTok{~}\StringTok{ }\NormalTok{. }\OperatorTok{-}\StringTok{ }\NormalTok{RICE)}
\KeywordTok{summary}\NormalTok{(salex.lreg)}
\end{Highlighting}
\end{Shaded}

\begin{verbatim}
## 
## Call:
## glm(formula = ILL ~ EGGS + MUSHROOM + PEPPER + PASTA + LETTUCE + 
##     COLESLAW, family = binomial(logit), data = salex)
## 
## Deviance Residuals: 
##     Min       1Q   Median       3Q      Max  
## -1.8877  -0.4999   0.0786   0.3897   2.0697  
## 
## Coefficients:
##             Estimate Std. Error z value Pr(>|z|)   
## (Intercept)  -2.0169     0.6600  -3.056  0.00224 **
## EGGS          3.6142     1.1944   3.026  0.00248 **
## MUSHROOM     -3.5508     1.6134  -2.201  0.02774 * 
## PEPPER        2.3002     1.3200   1.743  0.08141 . 
## PASTA         1.8230     1.0280   1.773  0.07617 . 
## LETTUCE       3.4199     1.2273   2.787  0.00533 **
## COLESLAW      0.7611     1.0203   0.746  0.45571   
## ---
## Signif. codes:  0 '***' 0.001 '**' 0.01 '*' 0.05 '.' 0.1 ' ' 1
## 
## (Dispersion parameter for binomial family taken to be 1)
## 
##     Null deviance: 92.122  on 70  degrees of freedom
## Residual deviance: 41.283  on 64  degrees of freedom
##   (6 observations deleted due to missingness)
## AIC: 55.283
## 
## Number of Fisher Scoring iterations: 6
\end{verbatim}

\texttt{COLESLAW} is now the least significant variable in the model so
we will remove this variable from the model:

\begin{Shaded}
\begin{Highlighting}[]
\NormalTok{salex.lreg <-}\StringTok{ }\KeywordTok{update}\NormalTok{(salex.lreg, . }\OperatorTok{~}\StringTok{ }\NormalTok{. }\OperatorTok{-}\StringTok{ }\NormalTok{COLESLAW)}
\KeywordTok{summary}\NormalTok{(salex.lreg)}
\end{Highlighting}
\end{Shaded}

\begin{verbatim}
## 
## Call:
## glm(formula = ILL ~ EGGS + MUSHROOM + PEPPER + PASTA + LETTUCE, 
##     family = binomial(logit), data = salex)
## 
## Deviance Residuals: 
##      Min        1Q    Median        3Q       Max  
## -1.98481  -0.50486   0.08871   0.36910   2.06065  
## 
## Coefficients:
##             Estimate Std. Error z value Pr(>|z|)   
## (Intercept)  -1.9957     0.6545  -3.049  0.00230 **
## EGGS          3.8152     1.1640   3.278  0.00105 **
## MUSHROOM     -3.4008     1.5922  -2.136  0.03269 * 
## PEPPER        2.3520     1.3269   1.773  0.07631 . 
## PASTA         1.9706     0.9922   1.986  0.04701 * 
## LETTUCE       3.4786     1.2246   2.841  0.00450 **
## ---
## Signif. codes:  0 '***' 0.001 '**' 0.01 '*' 0.05 '.' 0.1 ' ' 1
## 
## (Dispersion parameter for binomial family taken to be 1)
## 
##     Null deviance: 92.982  on 71  degrees of freedom
## Residual deviance: 41.895  on 66  degrees of freedom
##   (5 observations deleted due to missingness)
## AIC: 53.895
## 
## Number of Fisher Scoring iterations: 6
\end{verbatim}

\texttt{PEPPER} is now the least significant variable in the model so we
will remove this variable from the model:

\begin{Shaded}
\begin{Highlighting}[]
\NormalTok{salex.lreg <-}\StringTok{ }\KeywordTok{update}\NormalTok{(salex.lreg, . }\OperatorTok{~}\StringTok{ }\NormalTok{. }\OperatorTok{-}\StringTok{ }\NormalTok{PEPPER)}
\KeywordTok{summary}\NormalTok{(salex.lreg)}
\end{Highlighting}
\end{Shaded}

\begin{verbatim}
## 
## Call:
## glm(formula = ILL ~ EGGS + MUSHROOM + PASTA + LETTUCE, family = binomial(logit), 
##     data = salex)
## 
## Deviance Residuals: 
##     Min       1Q   Median       3Q      Max  
## -2.0920  -0.5360   0.1109   0.4876   2.0056  
## 
## Coefficients:
##             Estimate Std. Error z value Pr(>|z|)    
## (Intercept)  -1.8676     0.6128  -3.048 0.002306 ** 
## EGGS          3.7094     1.0682   3.473 0.000515 ***
## MUSHROOM     -1.6165     1.0829  -1.493 0.135524    
## PASTA         1.8440     0.9193   2.006 0.044864 *  
## LETTUCE       3.2458     1.1698   2.775 0.005527 ** 
## ---
## Signif. codes:  0 '***' 0.001 '**' 0.01 '*' 0.05 '.' 0.1 ' ' 1
## 
## (Dispersion parameter for binomial family taken to be 1)
## 
##     Null deviance: 94.659  on 73  degrees of freedom
## Residual deviance: 45.578  on 69  degrees of freedom
##   (3 observations deleted due to missingness)
## AIC: 55.578
## 
## Number of Fisher Scoring iterations: 6
\end{verbatim}

\texttt{MUSHROOM} is now the least significant variable in the model so
we will remove this variable from the model:

\begin{Shaded}
\begin{Highlighting}[]
\NormalTok{salex.lreg <-}\StringTok{ }\KeywordTok{update}\NormalTok{(salex.lreg, . }\OperatorTok{~}\StringTok{ }\NormalTok{. }\OperatorTok{-}\StringTok{ }\NormalTok{MUSHROOM)}
\KeywordTok{summary}\NormalTok{(salex.lreg)}
\end{Highlighting}
\end{Shaded}

\begin{verbatim}
## 
## Call:
## glm(formula = ILL ~ EGGS + PASTA + LETTUCE, family = binomial(logit), 
##     data = salex)
## 
## Deviance Residuals: 
##     Min       1Q   Median       3Q      Max  
## -2.2024  -0.5108   0.2038   0.4304   2.0501  
## 
## Coefficients:
##             Estimate Std. Error z value Pr(>|z|)    
## (Intercept)  -1.9710     0.6146  -3.207  0.00134 ** 
## EGGS          2.6391     0.7334   3.599  0.00032 ***
## PASTA         1.6646     0.8376   1.987  0.04689 *  
## LETTUCE       3.1956     1.1516   2.775  0.00552 ** 
## ---
## Signif. codes:  0 '***' 0.001 '**' 0.01 '*' 0.05 '.' 0.1 ' ' 1
## 
## (Dispersion parameter for binomial family taken to be 1)
## 
##     Null deviance: 97.648  on 75  degrees of freedom
## Residual deviance: 50.529  on 72  degrees of freedom
##   (1 observation deleted due to missingness)
## AIC: 58.529
## 
## Number of Fisher Scoring iterations: 6
\end{verbatim}

There are now no non-significant variables in the model.

Unfortunately \texttt{R} does not present information on the model
coefficients in terms of odds ratios and confidence intervals but we can
write a function to calculate them for us.

The first step in doing this is to realise that the \texttt{salex.lreg}
object contains essential information about the fitted model. To
calculate odds ratios and confidence intervals we need the regression
coefficients and their standard errors. Both:

\begin{Shaded}
\begin{Highlighting}[]
\KeywordTok{summary}\NormalTok{(salex.lreg)}\OperatorTok{$}\NormalTok{coefficients}
\end{Highlighting}
\end{Shaded}

\begin{verbatim}
##              Estimate Std. Error   z value     Pr(>|z|)
## (Intercept) -1.970967  0.6145691 -3.207071 0.0013409398
## EGGS         2.639115  0.7333899  3.598515 0.0003200388
## PASTA        1.664581  0.8375970  1.987330 0.0468858898
## LETTUCE      3.195594  1.1516159  2.774879 0.0055222320
\end{verbatim}

and:

\begin{Shaded}
\begin{Highlighting}[]
\KeywordTok{coef}\NormalTok{(}\KeywordTok{summary}\NormalTok{(salex.lreg))}
\end{Highlighting}
\end{Shaded}

\begin{verbatim}
##              Estimate Std. Error   z value     Pr(>|z|)
## (Intercept) -1.970967  0.6145691 -3.207071 0.0013409398
## EGGS         2.639115  0.7333899  3.598515 0.0003200388
## PASTA        1.664581  0.8375970  1.987330 0.0468858898
## LETTUCE      3.195594  1.1516159  2.774879 0.0055222320
\end{verbatim}

extract the data that we require. The preferred method is to use the
\texttt{coef()} function. This is because some fitted models may return
coefficients in a more complicated manner than (e.g.) those created by
the \texttt{glm()} function. The \texttt{coef()} function provides a
standard way of extracting this data from all classes of fitted objects.

We can store the \texttt{coefficients} data in a separate object to make
it easier to work with:

\begin{Shaded}
\begin{Highlighting}[]
\NormalTok{salex.lreg.coeffs <-}\StringTok{ }\KeywordTok{coef}\NormalTok{(}\KeywordTok{summary}\NormalTok{(salex.lreg))}
\NormalTok{salex.lreg.coeffs}
\end{Highlighting}
\end{Shaded}

\begin{verbatim}
##              Estimate Std. Error   z value     Pr(>|z|)
## (Intercept) -1.970967  0.6145691 -3.207071 0.0013409398
## EGGS         2.639115  0.7333899  3.598515 0.0003200388
## PASTA        1.664581  0.8375970  1.987330 0.0468858898
## LETTUCE      3.195594  1.1516159  2.774879 0.0055222320
\end{verbatim}

We can extract information from this object by addressing each piece of
information by its row and column position in the object. For example:

\begin{Shaded}
\begin{Highlighting}[]
\NormalTok{salex.lreg.coeffs[}\DecValTok{2}\NormalTok{,}\DecValTok{1}\NormalTok{]}
\end{Highlighting}
\end{Shaded}

\begin{verbatim}
## [1] 2.639115
\end{verbatim}

Is the regression coefficient for \texttt{EGGS}, and:

\begin{Shaded}
\begin{Highlighting}[]
\NormalTok{salex.lreg.coeffs[}\DecValTok{3}\NormalTok{,}\DecValTok{2}\NormalTok{]}
\end{Highlighting}
\end{Shaded}

\begin{verbatim}
## [1] 0.837597
\end{verbatim}

is the standard error of the regression coefficient for \texttt{PASTA}.
Similarly:

\begin{Shaded}
\begin{Highlighting}[]
\NormalTok{salex.lreg.coeffs[ ,}\DecValTok{1}\NormalTok{]}
\end{Highlighting}
\end{Shaded}

\begin{verbatim}
## (Intercept)        EGGS       PASTA     LETTUCE 
##   -1.970967    2.639115    1.664581    3.195594
\end{verbatim}

Returns the regression coefficients for all of the variables in the
model, and:

\begin{Shaded}
\begin{Highlighting}[]
\NormalTok{salex.lreg.coeffs[ ,}\DecValTok{2}\NormalTok{]}
\end{Highlighting}
\end{Shaded}

\begin{verbatim}
## (Intercept)        EGGS       PASTA     LETTUCE 
##   0.6145691   0.7333899   0.8375970   1.1516159
\end{verbatim}

Returns the standard errors of the regression coefficients.

The table below shows the indices that address each cell in the table of
regression coefficients:

\begin{Shaded}
\begin{Highlighting}[]
\KeywordTok{matrix}\NormalTok{(salex.lreg.coeffs, }\DataTypeTok{nrow =} \DecValTok{4}\NormalTok{, }\DataTypeTok{ncol =} \DecValTok{4}\NormalTok{)}
\end{Highlighting}
\end{Shaded}

\begin{verbatim}
##           [,1]      [,2]      [,3]         [,4]
## [1,] -1.970967 0.6145691 -3.207071 0.0013409398
## [2,]  2.639115 0.7333899  3.598515 0.0003200388
## [3,]  1.664581 0.8375970  1.987330 0.0468858898
## [4,]  3.195594 1.1516159  2.774879 0.0055222320
\end{verbatim}

We can use this information to calculate odds ratio sand 95\% confidence
intervals:

\begin{Shaded}
\begin{Highlighting}[]
\NormalTok{or <-}\StringTok{ }\KeywordTok{exp}\NormalTok{(salex.lreg.coeffs[ ,}\DecValTok{1}\NormalTok{])}
\NormalTok{lci <-}\StringTok{ }\KeywordTok{exp}\NormalTok{(salex.lreg.coeffs[ ,}\DecValTok{1}\NormalTok{] }\OperatorTok{-}\StringTok{ }\FloatTok{1.96} \OperatorTok{*}\StringTok{ }\NormalTok{salex.lreg.coeffs[ ,}\DecValTok{2}\NormalTok{])}
\NormalTok{uci <-}\StringTok{ }\KeywordTok{exp}\NormalTok{(salex.lreg.coeffs[ ,}\DecValTok{1}\NormalTok{] }\OperatorTok{+}\StringTok{ }\FloatTok{1.96} \OperatorTok{*}\StringTok{ }\NormalTok{salex.lreg.coeffs[ ,}\DecValTok{2}\NormalTok{])}
\end{Highlighting}
\end{Shaded}

and make a single object that contains all of the required information:

\begin{Shaded}
\begin{Highlighting}[]
\NormalTok{lreg.or <-}\StringTok{ }\KeywordTok{cbind}\NormalTok{(or, lci, uci)}
\NormalTok{lreg.or}
\end{Highlighting}
\end{Shaded}

\begin{verbatim}
##                     or       lci         uci
## (Intercept)  0.1393221 0.0417723   0.4646777
## EGGS        14.0008053 3.3256684  58.9423019
## PASTA        5.2834608 1.0231552  27.2832114
## LETTUCE     24.4246856 2.5559581 233.4018193
\end{verbatim}

We seldom need to report estimates and confidence intervals to more than
two decimal places. We can use the \texttt{round()} function to remove
the excess digits:

\begin{Shaded}
\begin{Highlighting}[]
\KeywordTok{round}\NormalTok{(lreg.or, }\DataTypeTok{digits =} \DecValTok{2}\NormalTok{)}
\end{Highlighting}
\end{Shaded}

\begin{verbatim}
##                or  lci    uci
## (Intercept)  0.14 0.04   0.46
## EGGS        14.00 3.33  58.94
## PASTA        5.28 1.02  27.28
## LETTUCE     24.42 2.56 233.40
\end{verbatim}

We have now gone through all the necessary calculations step-by-step but
it would be nice to have a function that did it all for us that we could
use whenever we needed to.

First we will create a template for the function:

\begin{Shaded}
\begin{Highlighting}[]
\NormalTok{lreg.or <-}\StringTok{ }\ControlFlowTok{function}\NormalTok{(model, }\DataTypeTok{digits =} \DecValTok{2}\NormalTok{) \{\}}
\end{Highlighting}
\end{Shaded}

and then use the \texttt{fix()} function to edit the \texttt{lreg.or()}
function:

\begin{Shaded}
\begin{Highlighting}[]
\KeywordTok{fix}\NormalTok{(lreg.or)}
\end{Highlighting}
\end{Shaded}

We can now edit this function to add a calculation of odds ratios and
95\% confidence intervals:

\begin{Shaded}
\begin{Highlighting}[]
\ControlFlowTok{function}\NormalTok{(model, }\DataTypeTok{digits =} \DecValTok{2}\NormalTok{) \{}
\NormalTok{  lreg.coeffs <-}\StringTok{ }\KeywordTok{coef}\NormalTok{(}\KeywordTok{summary}\NormalTok{(model))}
\NormalTok{  OR <-}\StringTok{ }\KeywordTok{exp}\NormalTok{(lreg.coeffs[ ,}\DecValTok{1}\NormalTok{])}
\NormalTok{  LCI <-}\StringTok{ }\KeywordTok{exp}\NormalTok{(lreg.coeffs[ ,}\DecValTok{1}\NormalTok{] }\OperatorTok{-}\StringTok{ }\FloatTok{1.96} \OperatorTok{*}\StringTok{ }\NormalTok{lreg.coeffs[ ,}\DecValTok{2}\NormalTok{])}
\NormalTok{  UCI <-}\StringTok{ }\KeywordTok{exp}\NormalTok{(lreg.coeffs[ ,}\DecValTok{1}\NormalTok{] }\OperatorTok{+}\StringTok{ }\FloatTok{1.96} \OperatorTok{*}\StringTok{ }\NormalTok{lreg.coeffs[ ,}\DecValTok{2}\NormalTok{])}
\NormalTok{  lreg.or <-}\StringTok{ }\KeywordTok{round}\NormalTok{(}\KeywordTok{cbind}\NormalTok{(OR, LCI, UCI), }\DataTypeTok{digits =}\NormalTok{ digits)}
\NormalTok{  lreg.or}
\NormalTok{\}}
\end{Highlighting}
\end{Shaded}

\begin{Shaded}
\begin{Highlighting}[]
\NormalTok{lreg.or <-}\StringTok{ }\ControlFlowTok{function}\NormalTok{(model, }\DataTypeTok{digits =} \DecValTok{2}\NormalTok{) \{}
\NormalTok{  lreg.coeffs <-}\StringTok{ }\KeywordTok{coef}\NormalTok{(}\KeywordTok{summary}\NormalTok{(model))}
\NormalTok{  OR <-}\StringTok{ }\KeywordTok{exp}\NormalTok{(lreg.coeffs[ ,}\DecValTok{1}\NormalTok{])}
\NormalTok{  LCI <-}\StringTok{ }\KeywordTok{exp}\NormalTok{(lreg.coeffs[ ,}\DecValTok{1}\NormalTok{] }\OperatorTok{-}\StringTok{ }\FloatTok{1.96} \OperatorTok{*}\StringTok{ }\NormalTok{lreg.coeffs[ ,}\DecValTok{2}\NormalTok{])}
\NormalTok{  UCI <-}\StringTok{ }\KeywordTok{exp}\NormalTok{(lreg.coeffs[ ,}\DecValTok{1}\NormalTok{] }\OperatorTok{+}\StringTok{ }\FloatTok{1.96} \OperatorTok{*}\StringTok{ }\NormalTok{lreg.coeffs[ ,}\DecValTok{2}\NormalTok{])}
\NormalTok{  lreg.or <-}\StringTok{ }\KeywordTok{round}\NormalTok{(}\KeywordTok{cbind}\NormalTok{(OR, LCI, UCI), }\DataTypeTok{digits =}\NormalTok{ digits)}
\NormalTok{  lreg.or}
\NormalTok{\}}
\end{Highlighting}
\end{Shaded}

Once you have made the changes shown above, check your work, save the
file, and quit the editor.

We can test our function:

\begin{Shaded}
\begin{Highlighting}[]
\KeywordTok{lreg.or}\NormalTok{(salex.lreg)}
\end{Highlighting}
\end{Shaded}

Which produces the following output:

\begin{Shaded}
\begin{Highlighting}[]
\KeywordTok{lreg.or}\NormalTok{(salex.lreg)}
\end{Highlighting}
\end{Shaded}

\begin{verbatim}
##                OR  LCI    UCI
## (Intercept)  0.14 0.04   0.46
## EGGS        14.00 3.33  58.94
## PASTA        5.28 1.02  27.28
## LETTUCE     24.42 2.56 233.40
\end{verbatim}

The \texttt{digits} parameter of the \texttt{lreg.or()} function, which
has \texttt{digits\ =\ 2} as its default value, allows us to specify the
precision with which the estimates and their confidence intervals are
reported:

\begin{Shaded}
\begin{Highlighting}[]
\KeywordTok{lreg.or}\NormalTok{(salex.lreg, }\DataTypeTok{digits =} \DecValTok{4}\NormalTok{)}
\end{Highlighting}
\end{Shaded}

\begin{verbatim}
##                  OR    LCI      UCI
## (Intercept)  0.1393 0.0418   0.4647
## EGGS        14.0008 3.3257  58.9423
## PASTA        5.2835 1.0232  27.2832
## LETTUCE     24.4247 2.5560 233.4018
\end{verbatim}

Before we continue, it is probably a good idea to save this function for
later use:

\begin{Shaded}
\begin{Highlighting}[]
\KeywordTok{save}\NormalTok{(lreg.or, }\DataTypeTok{file =} \StringTok{"lregor.r"}\NormalTok{)}
\end{Highlighting}
\end{Shaded}

Which can be reloaded whenever it is needed:

\begin{Shaded}
\begin{Highlighting}[]
\KeywordTok{load}\NormalTok{(}\StringTok{"lregor.r"}\NormalTok{)}
\end{Highlighting}
\end{Shaded}

An alternative to using logistic regression with data that contains
associations that may be due to confounding is to use stratified
analysis (i.e. \emph{Mantel-Haenszel} techniques). With several
potential confounders, a stratified analysis results in the analysis of
many tables which can be difficult to interpret. For example, four
potential confounders, each with two levels would produce sixteen
tables. In such situations, logistic regression might be a better
approach. In order to illustrate Mantel-Haenszel techniques in
\texttt{R} we will work with a simpler dataset.

On Saturday, 21st April 1990, a luncheon was held in the home of Jean
Bateman. There was a total of forty-five guests which included
thirty-five members of the Department of Epidemiology and Population
Sciences at the London School of Hygiene and Tropical Medicine. On
Sunday morning, 22nd April 1990, Jean awoke with symptoms of
gastrointestinal illness; her husband awoke with similar symptoms. The
possibility of an outbreak related to the luncheon was strengthened when
several of the guests telephoned Jean on Sunday and reported illness. On
Monday, 23rd April 1990, there was an unusually large number of
department members absent from work and reporting illness. Data from
this outbreak is stored in the file \texttt{bateman.dat}.

The variables in the file \texttt{bateman.dat} are:

\begin{longtable}[]{@{}ll@{}}
\toprule
\begin{minipage}[t]{0.21\columnwidth}\raggedright
\textbf{ILL}\strut
\end{minipage} & \begin{minipage}[t]{0.44\columnwidth}\raggedright
Ill?\strut
\end{minipage}\tabularnewline
\begin{minipage}[t]{0.21\columnwidth}\raggedright
\textbf{CHEESE}\strut
\end{minipage} & \begin{minipage}[t]{0.44\columnwidth}\raggedright
Cheddar cheese\strut
\end{minipage}\tabularnewline
\begin{minipage}[t]{0.21\columnwidth}\raggedright
\textbf{CRABDIP}\strut
\end{minipage} & \begin{minipage}[t]{0.44\columnwidth}\raggedright
Crab dip\strut
\end{minipage}\tabularnewline
\begin{minipage}[t]{0.21\columnwidth}\raggedright
\textbf{CRISPS}\strut
\end{minipage} & \begin{minipage}[t]{0.44\columnwidth}\raggedright
Crisps\strut
\end{minipage}\tabularnewline
\begin{minipage}[t]{0.21\columnwidth}\raggedright
\textbf{BREAD}\strut
\end{minipage} & \begin{minipage}[t]{0.44\columnwidth}\raggedright
French bread\strut
\end{minipage}\tabularnewline
\begin{minipage}[t]{0.21\columnwidth}\raggedright
\textbf{CHICKEN}\strut
\end{minipage} & \begin{minipage}[t]{0.44\columnwidth}\raggedright
Chicken (roasted, served warm)\strut
\end{minipage}\tabularnewline
\begin{minipage}[t]{0.21\columnwidth}\raggedright
\textbf{RICE}\strut
\end{minipage} & \begin{minipage}[t]{0.44\columnwidth}\raggedright
Rice (boiled, served warm)\strut
\end{minipage}\tabularnewline
\begin{minipage}[t]{0.21\columnwidth}\raggedright
\textbf{CAESAR}\strut
\end{minipage} & \begin{minipage}[t]{0.44\columnwidth}\raggedright
Caesar salad\strut
\end{minipage}\tabularnewline
\begin{minipage}[t]{0.21\columnwidth}\raggedright
\textbf{TOMATO}\strut
\end{minipage} & \begin{minipage}[t]{0.44\columnwidth}\raggedright
Tomato salad\strut
\end{minipage}\tabularnewline
\begin{minipage}[t]{0.21\columnwidth}\raggedright
\textbf{ICECREAM}\strut
\end{minipage} & \begin{minipage}[t]{0.44\columnwidth}\raggedright
Vanilla ice-cream\strut
\end{minipage}\tabularnewline
\begin{minipage}[t]{0.21\columnwidth}\raggedright
\textbf{CAKE}\strut
\end{minipage} & \begin{minipage}[t]{0.44\columnwidth}\raggedright
Chocolate cake\strut
\end{minipage}\tabularnewline
\begin{minipage}[t]{0.21\columnwidth}\raggedright
\textbf{JUICE}\strut
\end{minipage} & \begin{minipage}[t]{0.44\columnwidth}\raggedright
Orange juice\strut
\end{minipage}\tabularnewline
\begin{minipage}[t]{0.21\columnwidth}\raggedright
\textbf{WINE}\strut
\end{minipage} & \begin{minipage}[t]{0.44\columnwidth}\raggedright
White wine\strut
\end{minipage}\tabularnewline
\begin{minipage}[t]{0.21\columnwidth}\raggedright
\textbf{COFFEE}\strut
\end{minipage} & \begin{minipage}[t]{0.44\columnwidth}\raggedright
Coffee\strut
\end{minipage}\tabularnewline
\bottomrule
\end{longtable}

Data is available for all forty-five guests at the luncheon. All of the
variables are coded 1=yes, 2=no. Retrieve and attach the
\texttt{bateman} dataset in \texttt{R}:

\begin{Shaded}
\begin{Highlighting}[]
\NormalTok{bateman <-}\StringTok{ }\KeywordTok{read.table}\NormalTok{(}\StringTok{"bateman.dat"}\NormalTok{, }\DataTypeTok{header =} \OtherTok{TRUE}\NormalTok{)}
\NormalTok{bateman}
\end{Highlighting}
\end{Shaded}

\begin{verbatim}
##    ILL CHEESE CRABDIP CRISPS BREAD CHICKEN RICE CAESAR TOMATO ICECREAM
## 1    1      1       1      1     2       1    1      1      1        1
## 2    2      1       1      1     2       1    2      2      2        1
## 3    1      2       2      1     2       1    2      1      2        1
## 4    1      1       2      1     1       1    2      1      2        1
## 5    1      1       1      1     2       1    1      1      1        2
## 6    1      1       1      1     1       1    2      1      1        2
## 7    1      2       1      1     2       1    1      1      1        1
## 8    2      1       1      1     2       1    1      2      1        1
## 9    2      1       1      1     2       1    1      2      1        1
## 10   2      2       1      1     2       1    2      2      2        1
## 11   1      1       2      1     1       1    1      1      1        1
## 12   1      1       1      1     1       1    1      1      1        1
## 13   2      2       1      1     2       1    1      2      2        1
## 14   1      2       1      1     1       1    1      1      1        1
## 15   1      1       1      1     2       2    1      1      1        2
## 16   1      2       2      2     2       1    1      1      1        1
## 17   2      1       2      1     1       1    1      2      2        1
## 18   1      2       1      1     2       1    1      1      1        1
## 19   1      1       2      2     1       1    1      2      1        1
## 20   2      2       2      2     2       2    2      2      2        2
## 21   2      1       2      2     1       2    1      1      2        2
## 22   2      2       2      2     2       2    2      2      2        1
## 23   2      2       2      2     2       2    2      2      2        1
## 24   1      2       1      1     2       1    1      1      2        1
## 25   1      1       2      2     1       1    1      1      1        1
## 26   2      2       1      1     1       1    1      2      2        2
## 27   2      2       1      1     1       1    1      2      2        2
## 28   1      2       1      2     2       1    1      2      2        1
## 29   1      1       2      2     1       1    1      2      2        1
## 30   1      2       1      1     2       1    1      1      1        1
## 31   1      2       1      1     2       1    1      1      1        1
## 32   1      1       2      2     2       1    1      1      1        2
## 33   2      1       2      1     1       1    1      1      1        1
## 34   1      2       1      1     2       1    1      1      1        2
## 35   1      1       2      1     2       1    1      1      1        2
## 36   2      1       2      1     1       2    1      1      1        2
## 37   1      2       1      1     2       1    1      1      1        1
## 38   1      1       2      2     2       1    2      1      1        1
## 39   2      2       1      1     1       1    1      1      1        2
## 40   1      1       1      1     2       1    2      1      1        2
## 41   2      2       1      1     1       1    2      1      1        2
## 42   1      1       2      2     1       2    2      1      1        1
## 43   1      2       1      1     2       1    2      1      1        2
## 44   1      2       1      1     2       2    1      1      1        2
## 45   1      2       1      1     2       2    1      1      1        1
##    CAKE JUICE WINE COFFEE
## 1     1     1    1      1
## 2     1     1    1      2
## 3     1     2    1      2
## 4     1     2    1      2
## 5     1     1    1      1
## 6     1     1    2      2
## 7     1     2    1      1
## 8     1     2    1      1
## 9     1     2    1      1
## 10    2     1    2      1
## 11    1     1    1      2
## 12    1     2    1      1
## 13    1     2    1      1
## 14    2     2    1      1
## 15    1     1    1      1
## 16    1     2    1      2
## 17    2     2    1      2
## 18    1     2    1      1
## 19    1     2    1      1
## 20    1     2    2      1
## 21    2     2    1      1
## 22    1     2    1      2
## 23    1     2    1      2
## 24    2     2    1      1
## 25    2     2    1      1
## 26    1     1    1      1
## 27    1     1    1      1
## 28    2     1    2      1
## 29    1     1    2      1
## 30    1     2    2      1
## 31    2     2    2      1
## 32    1     2    2      1
## 33    1     2    2      1
## 34    1     2    1      2
## 35    1     2    1      1
## 36    2     2    1      1
## 37    2     2    1      2
## 38    1     2    1      2
## 39    1     1    2      2
## 40    1     2    1      1
## 41    2     2    1      1
## 42    2     2    1      2
## 43    1     2    1      2
## 44    1     2    1      2
## 45    1     1    2      2
\end{verbatim}

\begin{Shaded}
\begin{Highlighting}[]
\KeywordTok{attach}\NormalTok{(bateman)}
\end{Highlighting}
\end{Shaded}

\begin{verbatim}
## The following objects are masked from salex (pos = 3):
## 
##     CRISPS, ILL, RICE, TOMATO
\end{verbatim}

\begin{verbatim}
## The following objects are masked from salex (pos = 12):
## 
##     CRISPS, ILL, RICE, TOMATO
\end{verbatim}

We will use our \texttt{tab2by2()} function to analyse this data.
Retrieve this function:

\begin{Shaded}
\begin{Highlighting}[]
\KeywordTok{load}\NormalTok{(}\StringTok{"tab2by2.r"}\NormalTok{)}
\end{Highlighting}
\end{Shaded}

Use the \texttt{tab2by2()} function to analyse the data:

\begin{Shaded}
\begin{Highlighting}[]
\KeywordTok{tab2by2}\NormalTok{(CHEESE, ILL)}
\end{Highlighting}
\end{Shaded}

\begin{verbatim}
## 
##         outcome
## exposure  1  2
##        1 15  7
##        2 14  9
## 
## Relative Risk     : 1.12013 
## 95% CI            : 0.7253229 1.729838 
## 
## Sample Odds Ratio : 1.377551 
## 95% CI            : 0.4037553 4.699992 
## 
## MLE Odds Ratio    : 1.367743 
## 95% CI             : 0.3427732 5.649399
\end{verbatim}

\begin{Shaded}
\begin{Highlighting}[]
\KeywordTok{tab2by2}\NormalTok{(CRABDIP, ILL)}
\end{Highlighting}
\end{Shaded}

\begin{verbatim}
## 
##         outcome
## exposure  1  2
##        1 18  9
##        2 11  7
## 
## Relative Risk     : 1.090909 
## 95% CI            : 0.6921784 1.719329 
## 
## Sample Odds Ratio : 1.272727 
## 95% CI            : 0.3682028 4.3993 
## 
## MLE Odds Ratio    : 1.265848 
## 95% CI             : 0.3042941 5.188297
\end{verbatim}

\begin{Shaded}
\begin{Highlighting}[]
\KeywordTok{tab2by2}\NormalTok{(CRISPS, ILL)}
\end{Highlighting}
\end{Shaded}

\begin{verbatim}
## 
##         outcome
## exposure  1  2
##        1 21 12
##        2  8  4
## 
## Relative Risk     : 0.9545455 
## 95% CI            : 0.5930168 1.536478 
## 
## Sample Odds Ratio : 0.875 
## 95% CI            : 0.2170373 3.527619 
## 
## MLE Odds Ratio    : 0.8775841 
## 95% CI             : 0.1587568 4.184763
\end{verbatim}

\begin{Shaded}
\begin{Highlighting}[]
\KeywordTok{tab2by2}\NormalTok{(BREAD, ILL)}
\end{Highlighting}
\end{Shaded}

\begin{verbatim}
## 
##         outcome
## exposure  1  2
##        1  9  8
##        2 20  8
## 
## Relative Risk     : 0.7411765 
## 95% CI            : 0.4469843 1.228997 
## 
## Sample Odds Ratio : 0.45 
## 95% CI            : 0.1280647 1.581232 
## 
## MLE Odds Ratio    : 0.4584416 
## 95% CI             : 0.1072622 1.897017
\end{verbatim}

\begin{Shaded}
\begin{Highlighting}[]
\KeywordTok{tab2by2}\NormalTok{(CHICKEN, ILL)}
\end{Highlighting}
\end{Shaded}

\begin{verbatim}
## 
##         outcome
## exposure  1  2
##        1 25 11
##        2  4  5
## 
## Relative Risk     : 1.5625 
## 95% CI            : 0.7293337 3.347448 
## 
## Sample Odds Ratio : 2.840909 
## 95% CI            : 0.637796 12.65415 
## 
## MLE Odds Ratio    : 2.76979 
## 95% CI             : 0.4912167 16.93409
\end{verbatim}

\begin{Shaded}
\begin{Highlighting}[]
\KeywordTok{tab2by2}\NormalTok{(RICE, ILL)}
\end{Highlighting}
\end{Shaded}

\begin{verbatim}
## 
##         outcome
## exposure  1  2
##        1 22 10
##        2  7  6
## 
## Relative Risk     : 1.276786 
## 95% CI            : 0.7330759 2.223756 
## 
## Sample Odds Ratio : 1.885714 
## 95% CI            : 0.5027038 7.073586 
## 
## MLE Odds Ratio    : 1.85813 
## 95% CI             : 0.4026256 8.531602
\end{verbatim}

\begin{Shaded}
\begin{Highlighting}[]
\KeywordTok{tab2by2}\NormalTok{(CAESAR, ILL)}
\end{Highlighting}
\end{Shaded}

\begin{verbatim}
## 
##         outcome
## exposure  1  2
##        1 26  5
##        2  3 11
## 
## Relative Risk     : 3.913978 
## 95% CI            : 1.418617 10.7987 
## 
## Sample Odds Ratio : 19.06667 
## 95% CI            : 3.866585 94.02038 
## 
## MLE Odds Ratio    : 17.33517 
## 95% CI             : 3.179027 133.7994
\end{verbatim}

\begin{Shaded}
\begin{Highlighting}[]
\KeywordTok{tab2by2}\NormalTok{(TOMATO, ILL)}
\end{Highlighting}
\end{Shaded}

\begin{verbatim}
## 
##         outcome
## exposure  1  2
##        1 24  6
##        2  5 10
## 
## Relative Risk     : 2.4 
## 95% CI            : 1.14769 5.018775 
## 
## Sample Odds Ratio : 8 
## 95% CI            : 1.97785 32.35836 
## 
## MLE Odds Ratio    : 7.553116 
## 95% CI             : 1.642249 41.02567
\end{verbatim}

\begin{Shaded}
\begin{Highlighting}[]
\KeywordTok{tab2by2}\NormalTok{(ICECREAM, ILL)}
\end{Highlighting}
\end{Shaded}

\begin{verbatim}
## 
##         outcome
## exposure  1  2
##        1 20  9
##        2  9  7
## 
## Relative Risk     : 1.226054 
## 95% CI            : 0.7463643 2.01404 
## 
## Sample Odds Ratio : 1.728395 
## 95% CI            : 0.4889138 6.110177 
## 
## MLE Odds Ratio    : 1.7069 
## 95% CI             : 0.4021245 7.255001
\end{verbatim}

\begin{Shaded}
\begin{Highlighting}[]
\KeywordTok{tab2by2}\NormalTok{(CAKE, ILL)}
\end{Highlighting}
\end{Shaded}

\begin{verbatim}
## 
##         outcome
## exposure  1  2
##        1 22 11
##        2  7  5
## 
## Relative Risk     : 1.142857 
## 95% CI            : 0.6689315 1.95255 
## 
## Sample Odds Ratio : 1.428571 
## 95% CI            : 0.3678242 5.548347 
## 
## MLE Odds Ratio    : 1.416945 
## 95% CI             : 0.2847257 6.685098
\end{verbatim}

\begin{Shaded}
\begin{Highlighting}[]
\KeywordTok{tab2by2}\NormalTok{(JUICE, ILL)}
\end{Highlighting}
\end{Shaded}

\begin{verbatim}
## 
##         outcome
## exposure  1  2
##        1  8  5
##        2 21 11
## 
## Relative Risk     : 0.9377289 
## 95% CI            : 0.5701453 1.542301 
## 
## Sample Odds Ratio : 0.8380952 
## 95% CI            : 0.2206785 3.182927 
## 
## MLE Odds Ratio    : 0.8414367 
## 95% CI             : 0.185464 4.101313
\end{verbatim}

\begin{Shaded}
\begin{Highlighting}[]
\KeywordTok{tab2by2}\NormalTok{(WINE, ILL)}
\end{Highlighting}
\end{Shaded}

\begin{verbatim}
## 
##         outcome
## exposure  1  2
##        1 22 12
##        2  7  4
## 
## Relative Risk     : 1.016807 
## 95% CI            : 0.6099343 1.695094 
## 
## Sample Odds Ratio : 1.047619 
## 95% CI            : 0.2543383 4.315141 
## 
## MLE Odds Ratio    : 1.046515 
## 95% CI             : 0.1855742 5.186546
\end{verbatim}

\begin{Shaded}
\begin{Highlighting}[]
\KeywordTok{tab2by2}\NormalTok{(COFFEE, ILL)}
\end{Highlighting}
\end{Shaded}

\begin{verbatim}
## 
##         outcome
## exposure  1  2
##        1 17 11
##        2 12  5
## 
## Relative Risk     : 0.860119 
## 95% CI            : 0.5607997 1.319196 
## 
## Sample Odds Ratio : 0.6439394 
## 95% CI            : 0.1772875 2.338901 
## 
## MLE Odds Ratio    : 0.6502015 
## 95% CI             : 0.1388979 2.729586
\end{verbatim}

Two variables (\texttt{CAESAR} and \texttt{TOMATO}) are associated with
\texttt{ILL}.

These two variables are also associated with each other:

\begin{Shaded}
\begin{Highlighting}[]
\KeywordTok{tab2by2}\NormalTok{(CAESAR, TOMATO)}
\end{Highlighting}
\end{Shaded}

\begin{verbatim}
## 
##         outcome
## exposure  1  2
##        1 27  4
##        2  3 11
## 
## Relative Risk     : 4.064516 
## 95% CI            : 1.477162 11.1838 
## 
## Sample Odds Ratio : 24.75 
## 95% CI            : 4.738936 129.2616 
## 
## MLE Odds Ratio    : 22.10962 
## 95% CI             : 3.850174 183.4671
\end{verbatim}

\begin{Shaded}
\begin{Highlighting}[]
\KeywordTok{chisq.test}\NormalTok{(}\KeywordTok{table}\NormalTok{(CAESAR, TOMATO))}
\end{Highlighting}
\end{Shaded}

\begin{verbatim}
## Warning in chisq.test(table(CAESAR, TOMATO)): Chi-squared approximation may
## be incorrect
\end{verbatim}

\begin{verbatim}
## 
##  Pearson's Chi-squared test with Yates' continuity correction
## 
## data:  table(CAESAR, TOMATO)
## X-squared = 15.877, df = 1, p-value = 6.759e-05
\end{verbatim}

\begin{Shaded}
\begin{Highlighting}[]
\KeywordTok{fisher.test}\NormalTok{(}\KeywordTok{table}\NormalTok{(CAESAR, TOMATO))}
\end{Highlighting}
\end{Shaded}

\begin{verbatim}
## 
##  Fisher's Exact Test for Count Data
## 
## data:  table(CAESAR, TOMATO)
## p-value = 3.442e-05
## alternative hypothesis: true odds ratio is not equal to 1
## 95 percent confidence interval:
##    3.850174 183.467108
## sample estimates:
## odds ratio 
##   22.10962
\end{verbatim}

This suggests the potential for one of these associations to be due to
confounding. We can perform a simple stratified analysis using the
\texttt{table()} function:

\begin{Shaded}
\begin{Highlighting}[]
\KeywordTok{table}\NormalTok{(CAESAR, ILL, TOMATO)}
\end{Highlighting}
\end{Shaded}

\begin{verbatim}
## , , TOMATO = 1
## 
##       ILL
## CAESAR  1  2
##      1 23  4
##      2  1  2
## 
## , , TOMATO = 2
## 
##       ILL
## CAESAR  1  2
##      1  3  1
##      2  2  9
\end{verbatim}

\begin{Shaded}
\begin{Highlighting}[]
\KeywordTok{table}\NormalTok{(TOMATO, ILL, CAESAR)}
\end{Highlighting}
\end{Shaded}

\begin{verbatim}
## , , CAESAR = 1
## 
##       ILL
## TOMATO  1  2
##      1 23  4
##      2  3  1
## 
## , , CAESAR = 2
## 
##       ILL
## TOMATO  1  2
##      1  1  2
##      2  2  9
\end{verbatim}

It would be useful to calculate odds ratios for each stratum. We can
define a simple function to calculate an odds ratio from a two-by-two
table:

\begin{Shaded}
\begin{Highlighting}[]
\NormalTok{or <-}\StringTok{ }\ControlFlowTok{function}\NormalTok{(x) \{(x[}\DecValTok{1}\NormalTok{,}\DecValTok{1}\NormalTok{] }\OperatorTok{/}\StringTok{ }\NormalTok{x[}\DecValTok{1}\NormalTok{,}\DecValTok{2}\NormalTok{]) }\OperatorTok{/}\StringTok{ }\NormalTok{(x[}\DecValTok{2}\NormalTok{,}\DecValTok{1}\NormalTok{] }\OperatorTok{/}\StringTok{ }\NormalTok{x[}\DecValTok{2}\NormalTok{,}\DecValTok{2}\NormalTok{])\}}
\end{Highlighting}
\end{Shaded}

We can use \texttt{apply()} to apply the \texttt{or()} function to the
two-by-two table in each stratum:

\begin{Shaded}
\begin{Highlighting}[]
\NormalTok{tabC <-}\StringTok{ }\KeywordTok{table}\NormalTok{(CAESAR, ILL, TOMATO)}
\KeywordTok{apply}\NormalTok{(tabC, }\DecValTok{3}\NormalTok{, or)}
\end{Highlighting}
\end{Shaded}

\begin{verbatim}
##    1    2 
## 11.5 13.5
\end{verbatim}

\begin{Shaded}
\begin{Highlighting}[]
\NormalTok{tabT <-}\StringTok{ }\KeywordTok{table}\NormalTok{(TOMATO, ILL, CAESAR)}
\KeywordTok{apply}\NormalTok{(tabT, }\DecValTok{3}\NormalTok{, or)}
\end{Highlighting}
\end{Shaded}

\begin{verbatim}
##        1        2 
## 1.916667 2.250000
\end{verbatim}

The 3 instructs the \texttt{apply()} function to apply the \texttt{or()}
function to the third dimension of the table objects (i.e.~levels of the
potential confounder in \texttt{tabC} and \texttt{tabT}).

The \texttt{mantelhaen.test()} function performs the stratified
analysis:

\begin{Shaded}
\begin{Highlighting}[]
\KeywordTok{mantelhaen.test}\NormalTok{(tabC)}
\end{Highlighting}
\end{Shaded}

\begin{verbatim}
## 
##  Mantel-Haenszel chi-squared test with continuity correction
## 
## data:  tabC
## Mantel-Haenszel X-squared = 5.752, df = 1, p-value = 0.01647
## alternative hypothesis: true common odds ratio is not equal to 1
## 95 percent confidence interval:
##   1.878994 83.156212
## sample estimates:
## common odds ratio 
##              12.5
\end{verbatim}

\begin{Shaded}
\begin{Highlighting}[]
\KeywordTok{mantelhaen.test}\NormalTok{(tabT)}
\end{Highlighting}
\end{Shaded}

\begin{verbatim}
## 
##  Mantel-Haenszel chi-squared test with continuity correction
## 
## data:  tabT
## Mantel-Haenszel X-squared = 0.049144, df = 1, p-value = 0.8246
## alternative hypothesis: true common odds ratio is not equal to 1
## 95 percent confidence interval:
##   0.3156862 13.4192331
## sample estimates:
## common odds ratio 
##          2.058219
\end{verbatim}

It is likely that \texttt{CAESAR} salad was a vehicle of food-poisoning,
and that \texttt{TOMATO} salad was not a vehicle of food-poisoning. Many
of those at the luncheon ate both \texttt{CAESAR} salad and
\texttt{TOMATO} salad. \texttt{CAESAR} confounded the relationship
between \texttt{TOMATO} and \texttt{ILL}. This resulted in a spurious
association between \texttt{TOMATO} and \texttt{ILL}.

It only makes sense to calculate a common odds ratio in the absence of
interaction. We can check for interaction `by eye' by examining and
comparing the odds ratios for each stratum as we did above.

There does appear to be an interaction between \texttt{CAESAR},
\texttt{WINE}, and \texttt{ILL}:

\begin{Shaded}
\begin{Highlighting}[]
\NormalTok{tabW <-}\StringTok{ }\KeywordTok{table}\NormalTok{(CAESAR, ILL, WINE)}
\KeywordTok{apply}\NormalTok{(tabW, }\DecValTok{3}\NormalTok{, or)}
\end{Highlighting}
\end{Shaded}

\begin{verbatim}
##    1    2 
## 63.0  2.5
\end{verbatim}

\emph{Woolf's test} for interaction (also known as \emph{Woolf's test
for the homogeneity of odds ratios}) provides a formal test for
interaction.

\texttt{R} does not provide a function to perform \emph{Woolf's test}
for the homogeneity of odds ratios but it is possible to write a
function to perform this test.

First we will create a template for the function:

\begin{Shaded}
\begin{Highlighting}[]
\NormalTok{woolf.test <-}\StringTok{ }\ControlFlowTok{function}\NormalTok{(x) \{\}}
\end{Highlighting}
\end{Shaded}

And then use the \texttt{fix()} function to edit the
\texttt{woolf.test()} function:

\begin{Shaded}
\begin{Highlighting}[]
\KeywordTok{fix}\NormalTok{(woolf.test)}
\end{Highlighting}
\end{Shaded}

We can now edit this function to make it do something useful:

\begin{Shaded}
\begin{Highlighting}[]
\ControlFlowTok{function}\NormalTok{(x) \{}
\NormalTok{  x <-}\StringTok{ }\NormalTok{x }\OperatorTok{+}\StringTok{ }\FloatTok{0.5}
\NormalTok{  k <-}\StringTok{ }\KeywordTok{dim}\NormalTok{(x)[}\DecValTok{3}\NormalTok{]}
\NormalTok{  or <-}\StringTok{ }\KeywordTok{apply}\NormalTok{(x, }\DecValTok{3}\NormalTok{, }\ControlFlowTok{function}\NormalTok{(x)}
\NormalTok{              \{(x[}\DecValTok{1}\NormalTok{, }\DecValTok{1}\NormalTok{] }\OperatorTok{/}\StringTok{ }\NormalTok{x[}\DecValTok{1}\NormalTok{, }\DecValTok{2}\NormalTok{]) }\OperatorTok{/}\StringTok{ }\NormalTok{(x[}\DecValTok{2}\NormalTok{, }\DecValTok{1}\NormalTok{] }\OperatorTok{/}\StringTok{ }\NormalTok{x[}\DecValTok{2}\NormalTok{, }\DecValTok{2}\NormalTok{])\})}
\NormalTok{  w <-}\StringTok{ }\KeywordTok{apply}\NormalTok{(x, }\DecValTok{3}\NormalTok{, }\ControlFlowTok{function}\NormalTok{(x) \{}\DecValTok{1} \OperatorTok{/}\StringTok{ }\KeywordTok{sum}\NormalTok{(}\DecValTok{1} \OperatorTok{/}\StringTok{ }\NormalTok{x)\})}
\NormalTok{  chi.sq <-}\StringTok{ }\KeywordTok{sum}\NormalTok{(w }\OperatorTok{*}\StringTok{ }\NormalTok{(}\KeywordTok{log}\NormalTok{(or) }\OperatorTok{-}\StringTok{ }\KeywordTok{weighted.mean}\NormalTok{(}\KeywordTok{log}\NormalTok{(or), w))}\OperatorTok{^}\DecValTok{2}\NormalTok{)}
\NormalTok{  p <-}\StringTok{ }\KeywordTok{pchisq}\NormalTok{(chi.sq, }\DataTypeTok{df =}\NormalTok{ k }\OperatorTok{-}\StringTok{ }\DecValTok{1}\NormalTok{, }\DataTypeTok{lower.tail =} \OtherTok{FALSE}\NormalTok{)}
  \KeywordTok{cat}\NormalTok{(}\StringTok{"}\CharTok{\textbackslash{}n}\StringTok{Woolf's X2 :"}\NormalTok{, chi.sq,}
      \StringTok{"}\CharTok{\textbackslash{}n}\StringTok{p-value    :"}\NormalTok{, p, }\StringTok{"}\CharTok{\textbackslash{}n}\StringTok{"}\NormalTok{)}
\NormalTok{\}}
\end{Highlighting}
\end{Shaded}

Once you have made the changes shown above, check your work, save the
file, and quit the editor. We can use the \texttt{woolf.test()} function
to test for a three-way interaction between \texttt{CAESAR},
\texttt{WINE}, and \texttt{ILL}:

\begin{Shaded}
\begin{Highlighting}[]
\KeywordTok{woolf.test}\NormalTok{(tabW)}
\end{Highlighting}
\end{Shaded}

Which returns:

\begin{Shaded}
\begin{Highlighting}[]
\KeywordTok{woolf.test}\NormalTok{(tabW)}
\end{Highlighting}
\end{Shaded}

\begin{verbatim}
## 
## Woolf's X2 : 3.319492 
## p-value    : 0.06846297
\end{verbatim}

Which is weak evidence of an interaction.

We should test for interaction between \texttt{CAESAR}, \texttt{TOMATO},
and \texttt{ILL} before accepting the results reported by the
\texttt{mantelhaen.test()} function:

\begin{Shaded}
\begin{Highlighting}[]
\KeywordTok{woolf.test}\NormalTok{(tabC)}
\end{Highlighting}
\end{Shaded}

\begin{verbatim}
## 
## Woolf's X2 : 0.0001233783 
## p-value    : 0.9911376
\end{verbatim}

We can repeat this analysis using logistic regression.

We need to change the coding of the variables to 0 and 1 before
specifying the model:

\begin{Shaded}
\begin{Highlighting}[]
\KeywordTok{detach}\NormalTok{(bateman)}
\NormalTok{bateman <-}\StringTok{ }\DecValTok{2} \OperatorTok{-}\StringTok{ }\NormalTok{bateman}
\NormalTok{bateman}
\NormalTok{bateman.lreg <-}\StringTok{ }\KeywordTok{glm}\NormalTok{(}\DataTypeTok{formula =}\NormalTok{ ILL }\OperatorTok{~}\StringTok{ }\NormalTok{CAESAR }\OperatorTok{+}\StringTok{ }\NormalTok{TOMATO,}
                    \DataTypeTok{family =} \KeywordTok{binomial}\NormalTok{(logit), }\DataTypeTok{data =}\NormalTok{ bateman)}
\KeywordTok{summary}\NormalTok{(bateman.lreg)}
\NormalTok{bateman.lreg <-}\StringTok{ }\KeywordTok{update}\NormalTok{(bateman.lreg, . }\OperatorTok{~}\StringTok{ }\NormalTok{. }\OperatorTok{-}\StringTok{ }\NormalTok{TOMATO)}
\KeywordTok{summary}\NormalTok{(bateman.lreg)}
\end{Highlighting}
\end{Shaded}

\begin{verbatim}
##    ILL CHEESE CRABDIP CRISPS BREAD CHICKEN RICE CAESAR TOMATO ICECREAM
## 1    1      1       1      1     0       1    1      1      1        1
## 2    0      1       1      1     0       1    0      0      0        1
## 3    1      0       0      1     0       1    0      1      0        1
## 4    1      1       0      1     1       1    0      1      0        1
## 5    1      1       1      1     0       1    1      1      1        0
## 6    1      1       1      1     1       1    0      1      1        0
## 7    1      0       1      1     0       1    1      1      1        1
## 8    0      1       1      1     0       1    1      0      1        1
## 9    0      1       1      1     0       1    1      0      1        1
## 10   0      0       1      1     0       1    0      0      0        1
## 11   1      1       0      1     1       1    1      1      1        1
## 12   1      1       1      1     1       1    1      1      1        1
## 13   0      0       1      1     0       1    1      0      0        1
## 14   1      0       1      1     1       1    1      1      1        1
## 15   1      1       1      1     0       0    1      1      1        0
## 16   1      0       0      0     0       1    1      1      1        1
## 17   0      1       0      1     1       1    1      0      0        1
## 18   1      0       1      1     0       1    1      1      1        1
## 19   1      1       0      0     1       1    1      0      1        1
## 20   0      0       0      0     0       0    0      0      0        0
## 21   0      1       0      0     1       0    1      1      0        0
## 22   0      0       0      0     0       0    0      0      0        1
## 23   0      0       0      0     0       0    0      0      0        1
## 24   1      0       1      1     0       1    1      1      0        1
## 25   1      1       0      0     1       1    1      1      1        1
## 26   0      0       1      1     1       1    1      0      0        0
## 27   0      0       1      1     1       1    1      0      0        0
## 28   1      0       1      0     0       1    1      0      0        1
## 29   1      1       0      0     1       1    1      0      0        1
## 30   1      0       1      1     0       1    1      1      1        1
## 31   1      0       1      1     0       1    1      1      1        1
## 32   1      1       0      0     0       1    1      1      1        0
## 33   0      1       0      1     1       1    1      1      1        1
## 34   1      0       1      1     0       1    1      1      1        0
## 35   1      1       0      1     0       1    1      1      1        0
## 36   0      1       0      1     1       0    1      1      1        0
## 37   1      0       1      1     0       1    1      1      1        1
## 38   1      1       0      0     0       1    0      1      1        1
## 39   0      0       1      1     1       1    1      1      1        0
## 40   1      1       1      1     0       1    0      1      1        0
## 41   0      0       1      1     1       1    0      1      1        0
## 42   1      1       0      0     1       0    0      1      1        1
## 43   1      0       1      1     0       1    0      1      1        0
## 44   1      0       1      1     0       0    1      1      1        0
## 45   1      0       1      1     0       0    1      1      1        1
##    CAKE JUICE WINE COFFEE
## 1     1     1    1      1
## 2     1     1    1      0
## 3     1     0    1      0
## 4     1     0    1      0
## 5     1     1    1      1
## 6     1     1    0      0
## 7     1     0    1      1
## 8     1     0    1      1
## 9     1     0    1      1
## 10    0     1    0      1
## 11    1     1    1      0
## 12    1     0    1      1
## 13    1     0    1      1
## 14    0     0    1      1
## 15    1     1    1      1
## 16    1     0    1      0
## 17    0     0    1      0
## 18    1     0    1      1
## 19    1     0    1      1
## 20    1     0    0      1
## 21    0     0    1      1
## 22    1     0    1      0
## 23    1     0    1      0
## 24    0     0    1      1
## 25    0     0    1      1
## 26    1     1    1      1
## 27    1     1    1      1
## 28    0     1    0      1
## 29    1     1    0      1
## 30    1     0    0      1
## 31    0     0    0      1
## 32    1     0    0      1
## 33    1     0    0      1
## 34    1     0    1      0
## 35    1     0    1      1
## 36    0     0    1      1
## 37    0     0    1      0
## 38    1     0    1      0
## 39    1     1    0      0
## 40    1     0    1      1
## 41    0     0    1      1
## 42    0     0    1      0
## 43    1     0    1      0
## 44    1     0    1      0
## 45    1     1    0      0
\end{verbatim}

\begin{verbatim}
## 
## Call:
## glm(formula = ILL ~ CAESAR + TOMATO, family = binomial(logit), 
##     data = bateman)
## 
## Deviance Residuals: 
##    Min      1Q  Median      3Q     Max  
## -1.960  -0.641   0.563   0.563   1.835  
## 
## Coefficients:
##             Estimate Std. Error z value Pr(>|z|)   
## (Intercept)  -1.4780     0.7101  -2.082  0.03739 * 
## CAESAR        2.5202     0.9653   2.611  0.00904 **
## TOMATO        0.7197     0.9552   0.753  0.45116   
## ---
## Signif. codes:  0 '***' 0.001 '**' 0.01 '*' 0.05 '.' 0.1 ' ' 1
## 
## (Dispersion parameter for binomial family taken to be 1)
## 
##     Null deviance: 58.574  on 44  degrees of freedom
## Residual deviance: 41.408  on 42  degrees of freedom
## AIC: 47.408
## 
## Number of Fisher Scoring iterations: 4
\end{verbatim}

\begin{verbatim}
## 
## Call:
## glm(formula = ILL ~ CAESAR, family = binomial(logit), data = bateman)
## 
## Deviance Residuals: 
##     Min       1Q   Median       3Q      Max  
## -1.9103  -0.6945   0.5931   0.5931   1.7552  
## 
## Coefficients:
##             Estimate Std. Error z value Pr(>|z|)    
## (Intercept)  -1.2993     0.6513  -1.995 0.046066 *  
## CAESAR        2.9479     0.8141   3.621 0.000293 ***
## ---
## Signif. codes:  0 '***' 0.001 '**' 0.01 '*' 0.05 '.' 0.1 ' ' 1
## 
## (Dispersion parameter for binomial family taken to be 1)
## 
##     Null deviance: 58.574  on 44  degrees of freedom
## Residual deviance: 41.940  on 43  degrees of freedom
## AIC: 45.94
## 
## Number of Fisher Scoring iterations: 4
\end{verbatim}

Interactions are specified using the multiply (\_\_*\_\_) symbol in the
model formula:

\begin{Shaded}
\begin{Highlighting}[]
\NormalTok{bateman.lreg <-}\StringTok{ }\KeywordTok{glm}\NormalTok{(}\DataTypeTok{formula =}\NormalTok{ ILL }\OperatorTok{~}\StringTok{ }\NormalTok{CAESAR }\OperatorTok{+}\StringTok{ }\NormalTok{WINE }\OperatorTok{+}\StringTok{ }\NormalTok{CAESAR }\OperatorTok{*}\StringTok{ }\NormalTok{WINE,}
                    \DataTypeTok{family =} \KeywordTok{binomial}\NormalTok{(logit), }\DataTypeTok{data =}\NormalTok{ bateman)}
\KeywordTok{summary}\NormalTok{(bateman.lreg)}
\end{Highlighting}
\end{Shaded}

\begin{verbatim}
## 
## Call:
## glm(formula = ILL ~ CAESAR + WINE + CAESAR * WINE, family = binomial(logit), 
##     data = bateman)
## 
## Deviance Residuals: 
##     Min       1Q   Median       3Q      Max  
## -2.0393  -0.4590   0.5168   0.5168   2.1460  
## 
## Coefficients:
##               Estimate Std. Error z value Pr(>|z|)  
## (Intercept) -1.092e-15  1.000e+00   0.000   1.0000  
## CAESAR       9.163e-01  1.304e+00   0.703   0.4822  
## WINE        -2.197e+00  1.453e+00  -1.512   0.1305  
## CAESAR:WINE  3.227e+00  1.787e+00   1.806   0.0709 .
## ---
## Signif. codes:  0 '***' 0.001 '**' 0.01 '*' 0.05 '.' 0.1 ' ' 1
## 
## (Dispersion parameter for binomial family taken to be 1)
## 
##     Null deviance: 58.574  on 44  degrees of freedom
## Residual deviance: 38.508  on 41  degrees of freedom
## AIC: 46.508
## 
## Number of Fisher Scoring iterations: 4
\end{verbatim}

Before we continue, it is probably a good idea to save the
\texttt{woolf.test()} function for later use:

\begin{Shaded}
\begin{Highlighting}[]
\KeywordTok{save}\NormalTok{(woolf.test, }\DataTypeTok{file =} \StringTok{"woolf.r"}\NormalTok{)}
\end{Highlighting}
\end{Shaded}

\hypertarget{matched-data}{%
\section{Matched data}\label{matched-data}}

\emph{Matching} is another way to control for the effects of potential
confounding variables. Matching is usually performed during
data-collection as part of the design of a study.

In a matched case-control studies, each case is matched with one or more
controls which are chosen to have the same values over a set of
potential confounding variables. In order to illustrate how matched data
may be analysed using tabulation and stratification in \texttt{R} we
will start with the simple case of one-to-one matching (i.e.~each case
has a single matched control):

\begin{Shaded}
\begin{Highlighting}[]
\NormalTok{octe <-}\StringTok{ }\KeywordTok{read.table}\NormalTok{(}\StringTok{"octe.dat"}\NormalTok{, }\DataTypeTok{header =} \OtherTok{TRUE}\NormalTok{)}
\NormalTok{octe[}\DecValTok{1}\OperatorTok{:}\DecValTok{10}\NormalTok{, ]}
\end{Highlighting}
\end{Shaded}

\begin{verbatim}
##    ID CASE OC
## 1   1    1  1
## 2   1    2  1
## 3   2    1  1
## 4   2    2  1
## 5   3    1  1
## 6   3    2  1
## 7   4    1  1
## 8   4    2  1
## 9   5    1  1
## 10  5    2  1
\end{verbatim}

This data is from a matched case-control study investigating the
association between oral contraceptive use and thromboembolism. The
cases are 175 women aged between 15 and 44 years admitted to hospital
for thromboembolism and discharged alive. The controls are female
patients admitted for conditions believed to be unrelated to oral
contraceptive use. Cases and controls were matched on age, ethnic group,
marital status, parity, income, place of residence, and date of
hospitalisation. The variables in the dataset are:

\begin{longtable}[]{@{}ll@{}}
\toprule
\begin{minipage}[t]{0.19\columnwidth}\raggedright
\textbf{ID}\strut
\end{minipage} & \begin{minipage}[t]{0.75\columnwidth}\raggedright
Identifier for the matched sets of cases and controls\strut
\end{minipage}\tabularnewline
\begin{minipage}[t]{0.19\columnwidth}\raggedright
\textbf{CASE}\strut
\end{minipage} & \begin{minipage}[t]{0.75\columnwidth}\raggedright
Case (1) or control (2)\strut
\end{minipage}\tabularnewline
\begin{minipage}[t]{0.19\columnwidth}\raggedright
\textbf{OC}\strut
\end{minipage} & \begin{minipage}[t]{0.75\columnwidth}\raggedright
Used oral contraceptives in the previous month (1=yes, 2=no)\strut
\end{minipage}\tabularnewline
\bottomrule
\end{longtable}

The dataset consists of 350 records:

\begin{Shaded}
\begin{Highlighting}[]
\KeywordTok{nrow}\NormalTok{(octe)}
\end{Highlighting}
\end{Shaded}

\begin{verbatim}
## [1] 350
\end{verbatim}

There are 175 matched sets of cases and controls:

\begin{Shaded}
\begin{Highlighting}[]
\KeywordTok{length}\NormalTok{(}\KeywordTok{unique}\NormalTok{(octe}\OperatorTok{$}\NormalTok{ID))}
\end{Highlighting}
\end{Shaded}

\begin{verbatim}
## [1] 175
\end{verbatim}

In each matched set of cases and controls there is one case and one
control:

\begin{Shaded}
\begin{Highlighting}[]
\KeywordTok{table}\NormalTok{(octe}\OperatorTok{$}\NormalTok{ID, octe}\OperatorTok{$}\NormalTok{CASE)}
\end{Highlighting}
\end{Shaded}

\begin{verbatim}
##      
##       1 2
##   1   1 1
##   2   1 1
##   3   1 1
##   4   1 1
##   5   1 1
##   6   1 1
##   7   1 1
##   8   1 1
##   9   1 1
##   10  1 1
##   11  1 1
##   12  1 1
##   13  1 1
##   14  1 1
##   15  1 1
##   16  1 1
##   17  1 1
##   18  1 1
##   19  1 1
##   20  1 1
##   21  1 1
##   22  1 1
##   23  1 1
##   24  1 1
##   25  1 1
##   26  1 1
##   27  1 1
##   28  1 1
##   29  1 1
##   30  1 1
##   31  1 1
##   32  1 1
##   33  1 1
##   34  1 1
##   35  1 1
##   36  1 1
##   37  1 1
##   38  1 1
##   39  1 1
##   40  1 1
##   41  1 1
##   42  1 1
##   43  1 1
##   44  1 1
##   45  1 1
##   46  1 1
##   47  1 1
##   48  1 1
##   49  1 1
##   50  1 1
##   51  1 1
##   52  1 1
##   53  1 1
##   54  1 1
##   55  1 1
##   56  1 1
##   57  1 1
##   58  1 1
##   59  1 1
##   60  1 1
##   61  1 1
##   62  1 1
##   63  1 1
##   64  1 1
##   65  1 1
##   66  1 1
##   67  1 1
##   68  1 1
##   69  1 1
##   70  1 1
##   71  1 1
##   72  1 1
##   73  1 1
##   74  1 1
##   75  1 1
##   76  1 1
##   77  1 1
##   78  1 1
##   79  1 1
##   80  1 1
##   81  1 1
##   82  1 1
##   83  1 1
##   84  1 1
##   85  1 1
##   86  1 1
##   87  1 1
##   88  1 1
##   89  1 1
##   90  1 1
##   91  1 1
##   92  1 1
##   93  1 1
##   94  1 1
##   95  1 1
##   96  1 1
##   97  1 1
##   98  1 1
##   99  1 1
##   100 1 1
##   101 1 1
##   102 1 1
##   103 1 1
##   104 1 1
##   105 1 1
##   106 1 1
##   107 1 1
##   108 1 1
##   109 1 1
##   110 1 1
##   111 1 1
##   112 1 1
##   113 1 1
##   114 1 1
##   115 1 1
##   116 1 1
##   117 1 1
##   118 1 1
##   119 1 1
##   120 1 1
##   121 1 1
##   122 1 1
##   123 1 1
##   124 1 1
##   125 1 1
##   126 1 1
##   127 1 1
##   128 1 1
##   129 1 1
##   130 1 1
##   131 1 1
##   132 1 1
##   133 1 1
##   134 1 1
##   135 1 1
##   136 1 1
##   137 1 1
##   138 1 1
##   139 1 1
##   140 1 1
##   141 1 1
##   142 1 1
##   143 1 1
##   144 1 1
##   145 1 1
##   146 1 1
##   147 1 1
##   148 1 1
##   149 1 1
##   150 1 1
##   151 1 1
##   152 1 1
##   153 1 1
##   154 1 1
##   155 1 1
##   156 1 1
##   157 1 1
##   158 1 1
##   159 1 1
##   160 1 1
##   161 1 1
##   162 1 1
##   163 1 1
##   164 1 1
##   165 1 1
##   166 1 1
##   167 1 1
##   168 1 1
##   169 1 1
##   170 1 1
##   171 1 1
##   172 1 1
##   173 1 1
##   174 1 1
##   175 1 1
\end{verbatim}

This data may be analysed using \emph{McNemar's chi-squared test} which
use the number of discordant (i.e.~relative to exposure) pairs of
matched cases and controls.

To find the number of discordant pairs we need to split the dataset into
cases and controls:

\begin{Shaded}
\begin{Highlighting}[]
\NormalTok{octe.cases <-}\StringTok{ }\KeywordTok{subset}\NormalTok{(octe, CASE }\OperatorTok{==}\StringTok{ }\DecValTok{1}\NormalTok{)}
\NormalTok{octe.controls <-}\StringTok{ }\KeywordTok{subset}\NormalTok{(octe, CASE }\OperatorTok{==}\StringTok{ }\DecValTok{2}\NormalTok{)}
\end{Highlighting}
\end{Shaded}

Sorting these two datasets (i.e. \texttt{octe.cases} and
\texttt{octe.controls}) by the \texttt{ID} variable simplifies the
analysis:

\begin{Shaded}
\begin{Highlighting}[]
\NormalTok{octe.cases <-}\StringTok{ }\NormalTok{octe.cases[}\KeywordTok{order}\NormalTok{(octe.cases}\OperatorTok{$}\NormalTok{ID), ]}
\NormalTok{octe.controls <-}\StringTok{ }\NormalTok{octe.controls[}\KeywordTok{order}\NormalTok{(octe.controls}\OperatorTok{$}\NormalTok{ID), ]}
\end{Highlighting}
\end{Shaded}

Since the two datasets (i.e. \texttt{octe.cases} and
\texttt{octe.controls}) are now sorted by the \texttt{ID} variable we
can use the \texttt{table()} function to retrieve the number if
concordant and discordant pairs and store them in a table object:

\begin{Shaded}
\begin{Highlighting}[]
\NormalTok{tab <-}\StringTok{ }\KeywordTok{table}\NormalTok{(octe.cases}\OperatorTok{$}\NormalTok{OC, octe.controls}\OperatorTok{$}\NormalTok{OC)}
\NormalTok{tab}
\end{Highlighting}
\end{Shaded}

\begin{verbatim}
##    
##      1  2
##   1 10 57
##   2 13 95
\end{verbatim}

This table object (i.e. \texttt{tab}) can then be passed to the
\texttt{mcnemar.test()} function:

\begin{Shaded}
\begin{Highlighting}[]
\KeywordTok{mcnemar.test}\NormalTok{(tab)}
\end{Highlighting}
\end{Shaded}

\begin{verbatim}
## 
##  McNemar's Chi-squared test with continuity correction
## 
## data:  tab
## McNemar's chi-squared = 26.414, df = 1, p-value = 2.755e-07
\end{verbatim}

The \texttt{mcnemar.test()} function does not provide an estimate of the
odds ratio. This is the ratio of the discordant pairs:

\begin{Shaded}
\begin{Highlighting}[]
\NormalTok{r <-}\StringTok{ }\NormalTok{tab[}\DecValTok{1}\NormalTok{,}\DecValTok{2}\NormalTok{]}
\NormalTok{s <-}\StringTok{ }\NormalTok{tab[}\DecValTok{2}\NormalTok{,}\DecValTok{1}\NormalTok{]}
\NormalTok{rdp <-}\StringTok{ }\NormalTok{r }\OperatorTok{/}\StringTok{ }\NormalTok{s}
\NormalTok{rdp}
\end{Highlighting}
\end{Shaded}

\begin{verbatim}
## [1] 4.384615
\end{verbatim}

A confidence interval can also be calculated:

\begin{Shaded}
\begin{Highlighting}[]
\NormalTok{ci.p <-}\StringTok{ }\KeywordTok{binom.test}\NormalTok{(r, r }\OperatorTok{+}\StringTok{ }\NormalTok{s)}\OperatorTok{$}\NormalTok{conf.int}
\NormalTok{ci.rdp <-}\StringTok{ }\NormalTok{ci.p }\OperatorTok{/}\StringTok{ }\NormalTok{(}\DecValTok{1} \OperatorTok{-}\StringTok{ }\NormalTok{ci.p)}
\NormalTok{ci.rdp}
\end{Highlighting}
\end{Shaded}

\begin{verbatim}
## [1] 2.371377 8.731311
## attr(,"conf.level")
## [1] 0.95
\end{verbatim}

This provides a 95\% confidence interval. Other (e.g.~99\%) confidence
intervals can be produced by specifying appropriate values for the
\texttt{conf.level} parameter of the \texttt{binom.test()} function:

\begin{Shaded}
\begin{Highlighting}[]
\NormalTok{ci.p <-}\StringTok{ }\KeywordTok{binom.test}\NormalTok{(r, r }\OperatorTok{+}\StringTok{ }\NormalTok{s, }\DataTypeTok{conf.level =} \FloatTok{0.99}\NormalTok{)}\OperatorTok{$}\NormalTok{conf.int}
\NormalTok{ci.rdp <-}\StringTok{ }\NormalTok{ci.p }\OperatorTok{/}\StringTok{ }\NormalTok{(}\DecValTok{1} \OperatorTok{-}\StringTok{ }\NormalTok{ci.p)}
\NormalTok{ci.rdp}
\end{Highlighting}
\end{Shaded}

\begin{verbatim}
## [1]  2.010478 10.949095
## attr(,"conf.level")
## [1] 0.99
\end{verbatim}

An alternative way of analysing this data is to use the
\texttt{mantelhaen.test()} function:

\begin{Shaded}
\begin{Highlighting}[]
\NormalTok{tab <-}\StringTok{ }\KeywordTok{table}\NormalTok{(octe}\OperatorTok{$}\NormalTok{OC, octe}\OperatorTok{$}\NormalTok{CASE, octe}\OperatorTok{$}\NormalTok{ID)}
\KeywordTok{mantelhaen.test}\NormalTok{(tab)}
\end{Highlighting}
\end{Shaded}

\begin{verbatim}
## 
##  Mantel-Haenszel chi-squared test with continuity correction
## 
## data:  tab
## Mantel-Haenszel X-squared = 26.414, df = 1, p-value = 2.755e-07
## alternative hypothesis: true common odds ratio is not equal to 1
## 95 percent confidence interval:
##  2.400550 8.008521
## sample estimates:
## common odds ratio 
##          4.384615
\end{verbatim}

The Mantel-Haenszel approach is preferred because it can be used with
data from matched case-control studies that match more than one control
to each case. Multiple matching is useful when the condition being
studied is rare or at the early stages of an outbreak (i.e.~when cases
are hard to find and controls are easy to find).

We will now work with some data where each case has one or more
controls:

\begin{Shaded}
\begin{Highlighting}[]
\NormalTok{tsstamp <-}\StringTok{ }\KeywordTok{read.table}\NormalTok{(}\StringTok{"tsstamp.dat"}\NormalTok{, }\DataTypeTok{header =} \OtherTok{TRUE}\NormalTok{)}
\NormalTok{tsstamp}
\end{Highlighting}
\end{Shaded}

\begin{verbatim}
##    ID CASE RBTAMP
## 1   1    1      1
## 2   1    2      1
## 3   1    2      1
## 4   2    1      1
## 5   2    2      2
## 6   2    2      1
## 7   2    2      1
## 8   3    1      2
## 9   3    2      2
## 10  3    2      1
## 11  4    1      1
## 12  4    2      2
## 13  5    1      1
## 14  5    2      1
## 15  5    2      2
## 16  6    1      1
## 17  6    2      1
## 18  7    1      1
## 19  7    2      2
## 20  7    2      2
## 21  8    1      1
## 22  8    2      1
## 23  8    2      2
## 24  9    1      2
## 25  9    2      1
## 26  9    2      2
## 27  9    2      2
## 28 10    1      1
## 29 10    2      2
## 30 10    2      2
## 31 11    1      1
## 32 11    2      2
## 33 11    2      2
## 34 12    1      1
## 35 12    2      2
## 36 12    2      2
## 37 12    2      2
## 38 13    1      1
## 39 13    2      2
## 40 13    2      2
## 41 14    1      2
## 42 14    2      2
## 43 14    2      2
\end{verbatim}

This data is from a matched case-control study investigating the
association between the use of different brands of tampon and toxic
shock syndrome undertaken during an outbreak. Only a subset of the
original dataset is used here. The variables in the dataset are:

\begin{longtable}[]{@{}ll@{}}
\toprule
\begin{minipage}[t]{0.20\columnwidth}\raggedright
\textbf{ID}\strut
\end{minipage} & \begin{minipage}[t]{0.75\columnwidth}\raggedright
Identifier for the matched sets of cases and controls\strut
\end{minipage}\tabularnewline
\begin{minipage}[t]{0.20\columnwidth}\raggedright
\textbf{CASE}\strut
\end{minipage} & \begin{minipage}[t]{0.75\columnwidth}\raggedright
Case (1) or control (2)\strut
\end{minipage}\tabularnewline
\begin{minipage}[t]{0.20\columnwidth}\raggedright
\textbf{RBTAMP}\strut
\end{minipage} & \begin{minipage}[t]{0.75\columnwidth}\raggedright
Used Rely brand tampons (1=yes, 2=no)\strut
\end{minipage}\tabularnewline
\bottomrule
\end{longtable}

The dataset consists of forty-three (43) records:

\begin{Shaded}
\begin{Highlighting}[]
\KeywordTok{nrow}\NormalTok{(tsstamp)}
\end{Highlighting}
\end{Shaded}

\begin{verbatim}
## [1] 43
\end{verbatim}

There are fourteen (14) matched sets of cases and controls:

\begin{Shaded}
\begin{Highlighting}[]
\KeywordTok{length}\NormalTok{(}\KeywordTok{unique}\NormalTok{(tsstamp}\OperatorTok{$}\NormalTok{ID))}
\end{Highlighting}
\end{Shaded}

\begin{verbatim}
## [1] 14
\end{verbatim}

Each matched set of cases and controls consists of one case and one or
more controls:

\begin{Shaded}
\begin{Highlighting}[]
\KeywordTok{table}\NormalTok{(tsstamp}\OperatorTok{$}\NormalTok{ID, tsstamp}\OperatorTok{$}\NormalTok{CASE)}
\end{Highlighting}
\end{Shaded}

\begin{verbatim}
##     
##      1 2
##   1  1 2
##   2  1 3
##   3  1 2
##   4  1 1
##   5  1 2
##   6  1 1
##   7  1 2
##   8  1 2
##   9  1 3
##   10 1 2
##   11 1 2
##   12 1 3
##   13 1 2
##   14 1 2
\end{verbatim}

The \emph{McNemar's chi-squared test} is not useful for this data as it
is limited to the special case of one-to-one matching.

Analysing this data using a simple tabulation such as:

\begin{Shaded}
\begin{Highlighting}[]
\KeywordTok{fisher.test}\NormalTok{(}\KeywordTok{table}\NormalTok{(tsstamp}\OperatorTok{$}\NormalTok{RBTAMP, tsstamp}\OperatorTok{$}\NormalTok{CASE))}
\end{Highlighting}
\end{Shaded}

\begin{verbatim}
## 
##  Fisher's Exact Test for Count Data
## 
## data:  table(tsstamp$RBTAMP, tsstamp$CASE)
## p-value = 0.007805
## alternative hypothesis: true odds ratio is not equal to 1
## 95 percent confidence interval:
##   1.542686 53.734756
## sample estimates:
## odds ratio 
##   7.709932
\end{verbatim}

ignores the matched nature of the data and is, therefore, also not
useful for this data.

The matched nature of the data may be accounted by stratifying on the
variable that identifies the matched sets of cases and controls
(i.e.~the \texttt{ID} variable) using the \texttt{mantelhaen.test()}
function:

\begin{Shaded}
\begin{Highlighting}[]
\KeywordTok{mantelhaen.test}\NormalTok{(}\KeywordTok{table}\NormalTok{(tsstamp}\OperatorTok{$}\NormalTok{RBTAMP, tsstamp}\OperatorTok{$}\NormalTok{CASE, tsstamp}\OperatorTok{$}\NormalTok{ID))}
\end{Highlighting}
\end{Shaded}

\begin{verbatim}
## 
##  Mantel-Haenszel chi-squared test with continuity correction
## 
## data:  table(tsstamp$RBTAMP, tsstamp$CASE, tsstamp$ID)
## Mantel-Haenszel X-squared = 5.9384, df = 1, p-value = 0.01481
## alternative hypothesis: true common odds ratio is not equal to 1
## 95 percent confidence interval:
##   1.589505 43.191463
## sample estimates:
## common odds ratio 
##          8.285714
\end{verbatim}

Analysis of several risk factors or adjustment for confounding variables
not matched for in the design of a matched case-control study cannot be
performed using tabulation-based procedures such as the
\texttt{McNemar\textquotesingle{}s\ chi-\ squared\ test} and
Mantel-Haenszel procedures. In these situations a special form of
logistic regression, called \texttt{conditional\ logistic\ regression},
should be used.

We can now quit \texttt{R}:

\begin{Shaded}
\begin{Highlighting}[]
\KeywordTok{q}\NormalTok{()}
\end{Highlighting}
\end{Shaded}

For this exercise there is no need to save the workspace image so click
the \textbf{No} or \textbf{Don't Save} button (GUI) or enter \texttt{n}
when prompted to save the workspace image (terminal).

\hypertarget{summary-2}{%
\section{Summary}\label{summary-2}}

\begin{itemize}
\item
  \texttt{R} provides functions for many kinds of complex statistical
  analysis. We have looked at using the generalised linear model
  \texttt{glm()} function to perform logistic regression. We have looked
  ar the \texttt{mantelhaen.test()} function to perform stratified
  analyses and the \texttt{mantelhaen.test()} and
  \texttt{mcnemar.test()} functions to analyse data from matched
  case-control studies.
\item
  \texttt{R} can be extended by writing new functions. New functions can
  perform simple or complex data analysis. New functions can be composed
  of parts of existing function. New functions can be saved and used in
  subsequent \texttt{R} sessions. By building your own functions you can
  use \texttt{R} to build your own statistical analysis system.
\end{itemize}

\hypertarget{exercise4}{%
\chapter{Analysing some data with R}\label{exercise4}}

In this exercise we will use the \texttt{R} functions we have already
used and the functions we have added to \texttt{R} to analyse a small
dataset. First we will start \texttt{R} and retrieve our functions:

\begin{Shaded}
\begin{Highlighting}[]
\KeywordTok{load}\NormalTok{(}\StringTok{"tab2by2.r"}\NormalTok{)}
\KeywordTok{load}\NormalTok{(}\StringTok{"lregor.r"}\NormalTok{)}
\end{Highlighting}
\end{Shaded}

And then retrieve and attach the sample dataset:

\begin{Shaded}
\begin{Highlighting}[]
\NormalTok{gudhiv <-}\StringTok{ }\KeywordTok{read.table}\NormalTok{(}\StringTok{"gudhiv.dat"}\NormalTok{, }\DataTypeTok{header =} \OtherTok{TRUE}\NormalTok{, }\DataTypeTok{na.strings =} \StringTok{"X"}\NormalTok{)}
\KeywordTok{attach}\NormalTok{(gudhiv)}
\end{Highlighting}
\end{Shaded}

\begin{verbatim}
## The following objects are masked from gudhiv (pos = 11):
## 
##     CIR, GAMBIAN, GUD, HIV, INJ12M, MARRIED, PARTNERS, SEXPRO,
##     TRAVOUT, UTIGC
\end{verbatim}

This data is from a cross-sectional study of 435 male patients who
presented with sexually transmitted infections at an outpatient clinic
in The Gambia between August 1988 and June 1990. Several studies have
documented an association between genital ulcer disease (GUD) and HIV
infection. A study of Gambian prostitutes documented an association
between seropositivity for HIV-2 and antibodies against
\texttt{Treponema\ pallidum} (a serological test for syphilis).
Prostitutes are not the ideal population for such studies as they may
have experienced multiple sexually transmitted infections and it is
difficult to quantify the number of times they may have had sex with
HIV-2 seropositive customers. A sample of males with sexually
transmitted infections is easier to study as they have probably had
fewer sexual partners than prostitutes and much less contact with
sexually transmitted infection pathogens. In such a sample it is also
easier to find subjects and collect data. The variables in the dataset
are:

\begin{longtable}[]{@{}ll@{}}
\toprule
\begin{minipage}[t]{0.20\columnwidth}\raggedright
\textbf{MARRIED}\strut
\end{minipage} & \begin{minipage}[t]{0.75\columnwidth}\raggedright
Married (1=yes, 0=no)\strut
\end{minipage}\tabularnewline
\begin{minipage}[t]{0.20\columnwidth}\raggedright
\textbf{GAMBIAN}\strut
\end{minipage} & \begin{minipage}[t]{0.75\columnwidth}\raggedright
Gambian Citizen (1=yes, 0=no)\strut
\end{minipage}\tabularnewline
\begin{minipage}[t]{0.20\columnwidth}\raggedright
\textbf{GUD}\strut
\end{minipage} & \begin{minipage}[t]{0.75\columnwidth}\raggedright
History of GUD or syphilis (1=yes, 0=no)\strut
\end{minipage}\tabularnewline
\begin{minipage}[t]{0.20\columnwidth}\raggedright
\textbf{UTIGC}\strut
\end{minipage} & \begin{minipage}[t]{0.75\columnwidth}\raggedright
History of urethral discharge (1=yes, 0=no)\strut
\end{minipage}\tabularnewline
\begin{minipage}[t]{0.20\columnwidth}\raggedright
\textbf{CIR}\strut
\end{minipage} & \begin{minipage}[t]{0.75\columnwidth}\raggedright
Circumcised (1=yes, 0=no)\strut
\end{minipage}\tabularnewline
\begin{minipage}[t]{0.20\columnwidth}\raggedright
\textbf{TRAVOUT}\strut
\end{minipage} & \begin{minipage}[t]{0.75\columnwidth}\raggedright
Travelled outside of Gambia and Senegal (1=yes, 0=no)\strut
\end{minipage}\tabularnewline
\begin{minipage}[t]{0.20\columnwidth}\raggedright
\textbf{SEXPRO}\strut
\end{minipage} & \begin{minipage}[t]{0.75\columnwidth}\raggedright
Ever had sex with a prostitute (1=yes, 0=no)\strut
\end{minipage}\tabularnewline
\begin{minipage}[t]{0.20\columnwidth}\raggedright
\textbf{INJ12M}\strut
\end{minipage} & \begin{minipage}[t]{0.75\columnwidth}\raggedright
Injection in previous 12 months (1=yes, 0=no)\strut
\end{minipage}\tabularnewline
\begin{minipage}[t]{0.20\columnwidth}\raggedright
\textbf{PARTNERS}\strut
\end{minipage} & \begin{minipage}[t]{0.75\columnwidth}\raggedright
Sexual partners in previous 12 months (number)\strut
\end{minipage}\tabularnewline
\begin{minipage}[t]{0.20\columnwidth}\raggedright
\textbf{HIV HIV-2}\strut
\end{minipage} & \begin{minipage}[t]{0.75\columnwidth}\raggedright
positive serology (1=yes, 0=no)\strut
\end{minipage}\tabularnewline
\bottomrule
\end{longtable}

Data is available for all 435 patients enrolled in the study.

We will start our analysis by examining pairwise associations between
the binary exposure variables and the HIV variable using the
\texttt{tab2by2()} function that we wrote earlier:

\begin{Shaded}
\begin{Highlighting}[]
\KeywordTok{tab2by2}\NormalTok{(MARRIED, HIV)}
\KeywordTok{tab2by2}\NormalTok{(GAMBIAN, HIV)}
\KeywordTok{tab2by2}\NormalTok{(GUD, HIV)}
\KeywordTok{tab2by2}\NormalTok{(UTIGC, HIV)}
\KeywordTok{tab2by2}\NormalTok{(CIR, HIV)}
\KeywordTok{tab2by2}\NormalTok{(TRAVOUT, HIV)}
\KeywordTok{tab2by2}\NormalTok{(SEXPRO, HIV)}
\KeywordTok{tab2by2}\NormalTok{(INJ12M, HIV)}
\end{Highlighting}
\end{Shaded}

\begin{Shaded}
\begin{Highlighting}[]
\KeywordTok{tab2by2}\NormalTok{(MARRIED, HIV)}
\end{Highlighting}
\end{Shaded}

\begin{verbatim}
## 
##         outcome
## exposure   0   1
##        0 321  13
##        1  93   8
## 
## Relative Risk     : 1.043751 
## 95% CI            : 0.9818512 1.109554 
## 
## Sample Odds Ratio : 2.124069 
## 95% CI            : 0.8545749 5.279433 
## 
## MLE Odds Ratio    : 2.119801 
## 95% CI             : 0.7380371 5.714354
\end{verbatim}

\begin{Shaded}
\begin{Highlighting}[]
\KeywordTok{tab2by2}\NormalTok{(GAMBIAN, HIV)}
\end{Highlighting}
\end{Shaded}

\begin{verbatim}
## 
##         outcome
## exposure   0   1
##        0  73   4
##        1 341  17
## 
## Relative Risk     : 0.9953155 
## 95% CI            : 0.9400068 1.053879 
## 
## Sample Odds Ratio : 0.909824 
## 95% CI            : 0.2974059 2.783333 
## 
## MLE Odds Ratio    : 0.9100104 
## 95% CI             : 0.2853202 3.826485
\end{verbatim}

\begin{Shaded}
\begin{Highlighting}[]
\KeywordTok{tab2by2}\NormalTok{(GUD, HIV)}
\end{Highlighting}
\end{Shaded}

\begin{verbatim}
## 
##         outcome
## exposure   0   1
##        0 339  12
##        1  72   9
## 
## Relative Risk     : 1.086538 
## 95% CI            : 1.003531 1.176412 
## 
## Sample Odds Ratio : 3.53125 
## 95% CI            : 1.434372 8.693509 
## 
## MLE Odds Ratio    : 3.517408 
## 95% CI             : 1.258556 9.491924
\end{verbatim}

\begin{Shaded}
\begin{Highlighting}[]
\KeywordTok{tab2by2}\NormalTok{(UTIGC, HIV)}
\end{Highlighting}
\end{Shaded}

\begin{verbatim}
## 
##         outcome
## exposure   0   1
##        0 261  12
##        1 151   9
## 
## Relative Risk     : 1.013027 
## 95% CI            : 0.9678841 1.060275 
## 
## Sample Odds Ratio : 1.296358 
## 95% CI            : 0.5338453 3.147997 
## 
## MLE Odds Ratio    : 1.295532 
## 95% CI             : 0.4703496 3.438842
\end{verbatim}

\begin{Shaded}
\begin{Highlighting}[]
\KeywordTok{tab2by2}\NormalTok{(CIR, HIV)}
\end{Highlighting}
\end{Shaded}

\begin{verbatim}
## 
##         outcome
## exposure   0   1
##        0  10   3
##        1 392  17
## 
## Relative Risk     : 0.8025903 
## 95% CI            : 0.5955085 1.081682 
## 
## Sample Odds Ratio : 0.1445578 
## 95% CI            : 0.0364195 0.5737851 
## 
## MLE Odds Ratio    : 0.1460183 
## 95% CI             : 0.03322189 0.899754
\end{verbatim}

\begin{Shaded}
\begin{Highlighting}[]
\KeywordTok{tab2by2}\NormalTok{(TRAVOUT, HIV)}
\end{Highlighting}
\end{Shaded}

\begin{verbatim}
## 
##         outcome
## exposure   0   1
##        0 152   2
##        1 256  19
## 
## Relative Risk     : 1.060268 
## 95% CI            : 1.02181 1.100173 
## 
## Sample Odds Ratio : 5.640625 
## 95% CI            : 1.295879 24.55218 
## 
## MLE Odds Ratio    : 5.624226 
## 95% CI             : 1.32716 50.45859
\end{verbatim}

\begin{Shaded}
\begin{Highlighting}[]
\KeywordTok{tab2by2}\NormalTok{(SEXPRO, HIV)}
\end{Highlighting}
\end{Shaded}

\begin{verbatim}
## 
##         outcome
## exposure   0   1
##        0 268  13
##        1 143   8
## 
## Relative Risk     : 1.007093 
## 95% CI            : 0.9621259 1.054161 
## 
## Sample Odds Ratio : 1.153308 
## 95% CI            : 0.4671083 2.847562 
## 
## MLE Odds Ratio    : 1.152912 
## 95% CI             : 0.4042323 3.083152
\end{verbatim}

\begin{Shaded}
\begin{Highlighting}[]
\KeywordTok{tab2by2}\NormalTok{(INJ12M, HIV)}
\end{Highlighting}
\end{Shaded}

\begin{verbatim}
## 
##         outcome
## exposure   0   1
##        0 146   7
##        1 268  14
## 
## Relative Risk     : 1.004097 
## 95% CI            : 0.9610996 1.049018 
## 
## Sample Odds Ratio : 1.089552 
## 95% CI            : 0.4301305 2.759916 
## 
## MLE Odds Ratio    : 1.089351 
## 95% CI             : 0.4006202 3.263814
\end{verbatim}

Note that our \texttt{tab2by2()} function returns misleading risk ratio
estimates and confidence intervals for this dataset. This is because the
function expects the \texttt{exposure} and \texttt{outcome} variables to
be ordered with exposure-present and outcome-present as the first
category (e.g.~1 = present, 2 = absent). This coding is reversed (i.e.~0
= absent, 1 = present) in the \texttt{gudhiv} dataset.

We can produce risk ratio estimates for variables in the \texttt{gudhiv}
data using the \texttt{tab2by2()} function and a simple transformation
of the \texttt{exposure} and \texttt{outcome} variables. For example:

\begin{Shaded}
\begin{Highlighting}[]
\KeywordTok{tab2by2}\NormalTok{(}\DecValTok{2} \OperatorTok{-}\StringTok{ }\NormalTok{GUD, }\DecValTok{2} \OperatorTok{-}\StringTok{ }\NormalTok{HIV)}
\end{Highlighting}
\end{Shaded}

\begin{verbatim}
## 
##         outcome
## exposure   1   2
##        1   9  72
##        2  12 339
## 
## Relative Risk     : 3.25 
## 95% CI            : 1.417411 7.451965 
## 
## Sample Odds Ratio : 3.53125 
## 95% CI            : 1.434372 8.693509 
## 
## MLE Odds Ratio    : 3.517408 
## 95% CI             : 1.258556 9.491924
\end{verbatim}

The odds ratio estimates returned by the \texttt{tab2by2()} function,
with or without this transformation, are correct. The \texttt{GUD} and
\texttt{TRAVOUT} variables are associated with \texttt{HIV}.

\texttt{PARTNERS} is a continuous variable and we should examine its
distribution before doing anything with it:

\begin{Shaded}
\begin{Highlighting}[]
\KeywordTok{table}\NormalTok{(PARTNERS)}
\KeywordTok{hist}\NormalTok{(PARTNERS)}
\end{Highlighting}
\end{Shaded}

\begin{verbatim}
## PARTNERS
##   1   2   3   4   5   6   7   8   9 
##  61 129 133  71  25   6   6   2   2
\end{verbatim}

\includegraphics{prfe_files/figure-latex/unnamed-chunk-207-1.pdf}

The distribution of \texttt{PARTNERS} is severely non-normal. Instead of
attempting to transform the variable we will produce summary statistics
for each level of the \texttt{HIV} variable and perform a non-parametric
test:

\begin{Shaded}
\begin{Highlighting}[]
\KeywordTok{by}\NormalTok{(PARTNERS, HIV, summary)}
\KeywordTok{kruskal.test}\NormalTok{(PARTNERS }\OperatorTok{~}\StringTok{ }\NormalTok{HIV)}
\end{Highlighting}
\end{Shaded}

\begin{verbatim}
## HIV: 0
##    Min. 1st Qu.  Median    Mean 3rd Qu.    Max. 
##    1.00    2.00    3.00    2.72    3.00    8.00 
## -------------------------------------------------------- 
## HIV: 1
##    Min. 1st Qu.  Median    Mean 3rd Qu.    Max. 
##   1.000   4.000   5.000   5.381   7.000   9.000
\end{verbatim}

\begin{verbatim}
## 
##  Kruskal-Wallis rank sum test
## 
## data:  PARTNERS by HIV
## Kruskal-Wallis chi-squared = 32.036, df = 1, p-value = 1.514e-08
\end{verbatim}

An alternative way of looking at the data is as a tabulation:

\begin{Shaded}
\begin{Highlighting}[]
\KeywordTok{table}\NormalTok{(PARTNERS, HIV)}
\end{Highlighting}
\end{Shaded}

\begin{verbatim}
##         HIV
## PARTNERS   0   1
##        1  60   1
##        2 128   1
##        3 131   2
##        4  68   3
##        5  21   4
##        6   3   3
##        7   2   4
##        8   1   1
##        9   0   2
\end{verbatim}

You can use the \texttt{plot()} function to represent this table
graphically:

\begin{Shaded}
\begin{Highlighting}[]
\KeywordTok{plot}\NormalTok{(}\KeywordTok{table}\NormalTok{(PARTNERS, HIV), }\DataTypeTok{color =} \KeywordTok{c}\NormalTok{(}\StringTok{"lightgreen"}\NormalTok{, }\StringTok{"red"}\NormalTok{))}
\end{Highlighting}
\end{Shaded}

\includegraphics{prfe_files/figure-latex/unnamed-chunk-211-1.pdf}

There appears to be an association between the number of sexual
\texttt{PARTNERS} in the previous twelve months and positive
\texttt{HIV} serology. The proportion with positive \texttt{HIV}
serology increases as the number of sexual partners increases:

\begin{Shaded}
\begin{Highlighting}[]
\KeywordTok{prop.table}\NormalTok{(}\KeywordTok{table}\NormalTok{(PARTNERS, HIV), }\DecValTok{1}\NormalTok{) }\OperatorTok{*}\StringTok{ }\DecValTok{100}
\end{Highlighting}
\end{Shaded}

\begin{verbatim}
##         HIV
## PARTNERS           0           1
##        1  98.3606557   1.6393443
##        2  99.2248062   0.7751938
##        3  98.4962406   1.5037594
##        4  95.7746479   4.2253521
##        5  84.0000000  16.0000000
##        6  50.0000000  50.0000000
##        7  33.3333333  66.6666667
##        8  50.0000000  50.0000000
##        9   0.0000000 100.0000000
\end{verbatim}

The \textbf{`1'} instructs the \texttt{prop.table()} function to
calculate row proportions. You can also use the \texttt{plot()} function
to represent this table graphically:

\begin{Shaded}
\begin{Highlighting}[]
\KeywordTok{plot}\NormalTok{(}\KeywordTok{prop.table}\NormalTok{(}\KeywordTok{table}\NormalTok{(PARTNERS, HIV), }\DecValTok{1}\NormalTok{) }\OperatorTok{*}\StringTok{ }\DecValTok{100}\NormalTok{, }\DataTypeTok{color =} \KeywordTok{c}\NormalTok{(}\StringTok{"lightgreen"}\NormalTok{, }\StringTok{"red"}\NormalTok{))}
\end{Highlighting}
\end{Shaded}

\includegraphics{prfe_files/figure-latex/unnamed-chunk-213-1.pdf}

The \emph{chi-square test for trend} is an appropriate test to perform
on this data. The \texttt{prop.trend.test()} function that performs the
\emph{chi-square test for trend} requires you to specify the
\emph{number of events} and the \emph{number of trials}. In this table:

\begin{Shaded}
\begin{Highlighting}[]
\KeywordTok{table}\NormalTok{(PARTNERS, HIV)}
\end{Highlighting}
\end{Shaded}

\begin{verbatim}
##         HIV
## PARTNERS   0   1
##        1  60   1
##        2 128   1
##        3 131   2
##        4  68   3
##        5  21   4
##        6   3   3
##        7   2   4
##        8   1   1
##        9   0   2
\end{verbatim}

The \emph{number of events} in each row is in the second column
(labelled \textbf{1}) and the \emph{number of trials} is the total
number of cases in each row of the table.

We can extract this data from a table object:

\begin{Shaded}
\begin{Highlighting}[]
\NormalTok{tab <-}\StringTok{ }\KeywordTok{table}\NormalTok{(PARTNERS, HIV)}
\NormalTok{events <-}\StringTok{ }\NormalTok{tab[ ,}\DecValTok{2}\NormalTok{]}
\NormalTok{trials <-}\StringTok{ }\NormalTok{tab[ ,}\DecValTok{1}\NormalTok{] }\OperatorTok{+}\StringTok{ }\NormalTok{tab[ ,}\DecValTok{2}\NormalTok{]}
\end{Highlighting}
\end{Shaded}

\begin{Shaded}
\begin{Highlighting}[]
\NormalTok{tab <-}\StringTok{ }\KeywordTok{table}\NormalTok{(PARTNERS, HIV)}
\NormalTok{events <-}\StringTok{ }\NormalTok{tab[ ,}\DecValTok{2}\NormalTok{]}
\NormalTok{trials <-}\StringTok{ }\NormalTok{tab[ ,}\DecValTok{1}\NormalTok{] }\OperatorTok{+}\StringTok{ }\NormalTok{tab[ ,}\DecValTok{2}\NormalTok{]}
\end{Highlighting}
\end{Shaded}

Another way of creating the \texttt{trials} object would be to use the
\texttt{apply()} function to sum the rows of the tab object:

\begin{Shaded}
\begin{Highlighting}[]
\NormalTok{trials <-}\StringTok{ }\KeywordTok{apply}\NormalTok{(tab, }\DecValTok{1}\NormalTok{, sum)}
\end{Highlighting}
\end{Shaded}

Pass this data to the \texttt{prop.trend.test()} function:

\begin{Shaded}
\begin{Highlighting}[]
\KeywordTok{prop.trend.test}\NormalTok{(events, trials)}
\end{Highlighting}
\end{Shaded}

\begin{verbatim}
## 
##  Chi-squared Test for Trend in Proportions
## 
## data:  events out of trials ,
##  using scores: 1 2 3 4 5 6 7 8 9
## X-squared = 76.389, df = 1, p-value < 2.2e-16
\end{verbatim}

With a linear trend such as this we can use \texttt{PARTNERS} in a
logistic model without recoding or creating indicator variables. We can
now specify and fit the logistic regression model:

\begin{Shaded}
\begin{Highlighting}[]
\NormalTok{gudhiv.lreg <-}\StringTok{ }\KeywordTok{glm}\NormalTok{(}\DataTypeTok{formula =}\NormalTok{ HIV }\OperatorTok{~}\StringTok{ }\NormalTok{GUD }\OperatorTok{+}\StringTok{ }\NormalTok{TRAVOUT }\OperatorTok{+}\StringTok{ }\NormalTok{PARTNERS,}
                   \DataTypeTok{family =} \KeywordTok{binomial}\NormalTok{(logit))}
\KeywordTok{summary}\NormalTok{(gudhiv.lreg)}
\end{Highlighting}
\end{Shaded}

\begin{verbatim}
## 
## Call:
## glm(formula = HIV ~ GUD + TRAVOUT + PARTNERS, family = binomial(logit))
## 
## Deviance Residuals: 
##      Min        1Q    Median        3Q       Max  
## -1.70415  -0.19849  -0.11148  -0.06247   3.11742  
## 
## Coefficients:
##             Estimate Std. Error z value Pr(>|z|)    
## (Intercept)  -9.4854     1.4663  -6.469 9.86e-11 ***
## GUD           1.3869     0.5937   2.336   0.0195 *  
## TRAVOUT       2.0867     0.9547   2.186   0.0288 *  
## PARTNERS      1.1605     0.2050   5.662 1.50e-08 ***
## ---
## Signif. codes:  0 '***' 0.001 '**' 0.01 '*' 0.05 '.' 0.1 ' ' 1
## 
## (Dispersion parameter for binomial family taken to be 1)
## 
##     Null deviance: 167.364  on 425  degrees of freedom
## Residual deviance:  99.377  on 422  degrees of freedom
##   (9 observations deleted due to missingness)
## AIC: 107.38
## 
## Number of Fisher Scoring iterations: 8
\end{verbatim}

We can use the \texttt{lreg.or()} function that we wrote earlier to
calculate and display odds ratios and confidence intervals:

\begin{Shaded}
\begin{Highlighting}[]
\KeywordTok{lreg.or}\NormalTok{(gudhiv.lreg)}
\end{Highlighting}
\end{Shaded}

\begin{verbatim}
##               OR  LCI   UCI
## (Intercept) 0.00 0.00  0.00
## GUD         4.00 1.25 12.81
## TRAVOUT     8.06 1.24 52.35
## PARTNERS    3.19 2.14  4.77
\end{verbatim}

\texttt{PARTNERS} is incorporated into the logistic model as a
continuous variable.

The odds ratio reported for \texttt{PARTNERS} is the odds ratio
associated with a unit increase in the number of sexual
\texttt{PARTNERS}. A man reporting five sexual partners, for example,
was over three times as likely (odds ratio = 3.19) to have a positive
HIV-2 serology than a man reporting four sexual partners.

An alternative approach would be to have created an \emph{indicator}
variables:

\begin{Shaded}
\begin{Highlighting}[]
\NormalTok{part.gt}\FloatTok{.5}\NormalTok{ <-}\StringTok{ }\KeywordTok{ifelse}\NormalTok{(PARTNERS }\OperatorTok{>}\StringTok{ }\DecValTok{5}\NormalTok{, }\DecValTok{1}\NormalTok{, }\DecValTok{0}\NormalTok{)}
\end{Highlighting}
\end{Shaded}

This creates a new variable (\texttt{part.gt.5}) that indicates whether
or not an individual subject reported having more than five sexual
partners in the previous twelve months:

\begin{Shaded}
\begin{Highlighting}[]
\KeywordTok{table}\NormalTok{(PARTNERS, part.gt}\FloatTok{.5}\NormalTok{)}
\end{Highlighting}
\end{Shaded}

\begin{verbatim}
##         part.gt.5
## PARTNERS   0   1
##        1  61   0
##        2 129   0
##        3 133   0
##        4  71   0
##        5  25   0
##        6   0   6
##        7   0   6
##        8   0   2
##        9   0   2
\end{verbatim}

You can also inspect this on a case-by-case basis:

\begin{Shaded}
\begin{Highlighting}[]
\KeywordTok{cbind}\NormalTok{(PARTNERS, part.gt}\FloatTok{.5}\NormalTok{)}
\end{Highlighting}
\end{Shaded}

\begin{verbatim}
##        PARTNERS part.gt.5
##   [1,]        2         0
##   [2,]        2         0
##   [3,]        1         0
##   [4,]        2         0
##   [5,]        3         0
##   [6,]        2         0
##   [7,]        4         0
##   [8,]        5         0
##   [9,]        2         0
##  [10,]        3         0
##  [11,]        4         0
##  [12,]        3         0
##  [13,]        1         0
##  [14,]        2         0
##  [15,]        5         0
##  [16,]        8         1
##  [17,]        5         0
##  [18,]        3         0
##  [19,]        2         0
##  [20,]        1         0
##  [21,]        2         0
##  [22,]        3         0
##  [23,]        2         0
##  [24,]        3         0
##  [25,]        4         0
##  [26,]        3         0
##  [27,]        4         0
##  [28,]        3         0
##  [29,]        4         0
##  [30,]        5         0
##  [31,]        4         0
##  [32,]        3         0
##  [33,]        4         0
##  [34,]        5         0
##  [35,]        2         0
##  [36,]        4         0
##  [37,]        3         0
##  [38,]        2         0
##  [39,]        1         0
##  [40,]        2         0
##  [41,]        3         0
##  [42,]        4         0
##  [43,]        3         0
##  [44,]        3         0
##  [45,]        2         0
##  [46,]        4         0
##  [47,]        5         0
##  [48,]        4         0
##  [49,]        3         0
##  [50,]        4         0
##  [51,]        1         0
##  [52,]        1         0
##  [53,]        2         0
##  [54,]        3         0
##  [55,]        3         0
##  [56,]        3         0
##  [57,]        3         0
##  [58,]        4         0
##  [59,]        5         0
##  [60,]        4         0
##  [61,]        3         0
##  [62,]        3         0
##  [63,]        5         0
##  [64,]        2         0
##  [65,]        2         0
##  [66,]        3         0
##  [67,]        2         0
##  [68,]        1         0
##  [69,]        2         0
##  [70,]        3         0
##  [71,]        2         0
##  [72,]        3         0
##  [73,]        4         0
##  [74,]        3         0
##  [75,]        3         0
##  [76,]        2         0
##  [77,]        5         0
##  [78,]        4         0
##  [79,]        3         0
##  [80,]        1         0
##  [81,]        2         0
##  [82,]        5         0
##  [83,]        3         0
##  [84,]        7         1
##  [85,]        6         1
##  [86,]        5         0
##  [87,]        5         0
##  [88,]        5         0
##  [89,]        4         0
##  [90,]        3         0
##  [91,]        2         0
##  [92,]        5         0
##  [93,]        1         0
##  [94,]        1         0
##  [95,]        1         0
##  [96,]        1         0
##  [97,]        2         0
##  [98,]        3         0
##  [99,]        4         0
## [100,]        3         0
## [101,]        3         0
## [102,]        2         0
## [103,]        3         0
## [104,]        4         0
## [105,]        3         0
## [106,]        2         0
## [107,]        3         0
## [108,]        4         0
## [109,]        3         0
## [110,]        3         0
## [111,]        4         0
## [112,]        3         0
## [113,]        6         1
## [114,]        3         0
## [115,]        4         0
## [116,]        3         0
## [117,]        3         0
## [118,]        2         0
## [119,]        3         0
## [120,]        4         0
## [121,]        7         1
## [122,]        3         0
## [123,]        2         0
## [124,]        3         0
## [125,]        4         0
## [126,]        3         0
## [127,]        2         0
## [128,]        3         0
## [129,]        4         0
## [130,]        8         1
## [131,]        5         0
## [132,]        6         1
## [133,]        5         0
## [134,]        4         0
## [135,]        4         0
## [136,]        4         0
## [137,]        3         0
## [138,]        4         0
## [139,]        3         0
## [140,]        3         0
## [141,]        2         0
## [142,]        2         0
## [143,]        1         0
## [144,]        2         0
## [145,]        1         0
## [146,]        2         0
## [147,]        3         0
## [148,]        1         0
## [149,]        2         0
## [150,]        3         0
## [151,]        2         0
## [152,]        2         0
## [153,]        2         0
## [154,]        1         0
## [155,]        1         0
## [156,]        2         0
## [157,]        3         0
## [158,]        3         0
## [159,]        2         0
## [160,]        3         0
## [161,]        4         0
## [162,]        2         0
## [163,]        5         0
## [164,]        4         0
## [165,]        2         0
## [166,]        3         0
## [167,]        2         0
## [168,]        2         0
## [169,]        1         0
## [170,]        4         0
## [171,]        3         0
## [172,]        3         0
## [173,]        2         0
## [174,]        3         0
## [175,]        2         0
## [176,]        4         0
## [177,]        3         0
## [178,]        2         0
## [179,]        3         0
## [180,]        4         0
## [181,]        2         0
## [182,]        3         0
## [183,]        3         0
## [184,]        4         0
## [185,]        2         0
## [186,]        3         0
## [187,]        2         0
## [188,]        2         0
## [189,]        3         0
## [190,]        3         0
## [191,]        2         0
## [192,]        3         0
## [193,]        2         0
## [194,]        4         0
## [195,]        3         0
## [196,]        2         0
## [197,]        2         0
## [198,]        3         0
## [199,]        2         0
## [200,]        3         0
## [201,]        2         0
## [202,]        3         0
## [203,]        3         0
## [204,]        2         0
## [205,]        3         0
## [206,]        2         0
## [207,]        3         0
## [208,]        2         0
## [209,]        1         0
## [210,]        6         1
## [211,]        9         1
## [212,]        1         0
## [213,]        2         0
## [214,]        3         0
## [215,]        4         0
## [216,]        5         0
## [217,]        4         0
## [218,]        5         0
## [219,]        5         0
## [220,]        5         0
## [221,]        4         0
## [222,]        3         0
## [223,]        4         0
## [224,]        3         0
## [225,]        2         0
## [226,]        1         0
## [227,]        2         0
## [228,]        3         0
## [229,]        2         0
## [230,]        1         0
## [231,]        4         0
## [232,]        3         0
## [233,]        4         0
## [234,]        3         0
## [235,]        3         0
## [236,]        2         0
## [237,]        2         0
## [238,]        1         0
## [239,]        2         0
## [240,]        3         0
## [241,]        2         0
## [242,]        1         0
## [243,]        2         0
## [244,]        4         0
## [245,]        3         0
## [246,]        2         0
## [247,]        3         0
## [248,]        2         0
## [249,]        2         0
## [250,]        1         0
## [251,]        2         0
## [252,]        3         0
## [253,]        2         0
## [254,]        3         0
## [255,]        1         0
## [256,]        2         0
## [257,]        3         0
## [258,]        2         0
## [259,]        4         0
## [260,]        3         0
## [261,]        3         0
## [262,]        2         0
## [263,]        2         0
## [264,]        1         0
## [265,]        1         0
## [266,]        1         0
## [267,]        1         0
## [268,]        1         0
## [269,]        1         0
## [270,]        1         0
## [271,]        1         0
## [272,]        1         0
## [273,]        1         0
## [274,]        2         0
## [275,]        3         0
## [276,]        2         0
## [277,]        3         0
## [278,]        2         0
## [279,]        1         0
## [280,]        2         0
## [281,]        3         0
## [282,]        4         0
## [283,]        3         0
## [284,]        2         0
## [285,]        3         0
## [286,]        2         0
## [287,]        1         0
## [288,]        2         0
## [289,]        4         0
## [290,]        7         1
## [291,]        1         0
## [292,]        4         0
## [293,]        1         0
## [294,]        3         0
## [295,]        3         0
## [296,]        4         0
## [297,]        3         0
## [298,]        2         0
## [299,]        2         0
## [300,]        1         0
## [301,]        2         0
## [302,]        3         0
## [303,]        3         0
## [304,]        3         0
## [305,]        2         0
## [306,]        4         0
## [307,]        4         0
## [308,]        5         0
## [309,]        4         0
## [310,]        4         0
## [311,]        9         1
## [312,]        3         0
## [313,]        3         0
## [314,]        2         0
## [315,]        2         0
## [316,]        1         0
## [317,]        1         0
## [318,]        2         0
## [319,]        7         1
## [320,]        3         0
## [321,]        2         0
## [322,]        1         0
## [323,]        2         0
## [324,]        4         0
## [325,]        6         1
## [326,]        5         0
## [327,]        3         0
## [328,]        2         0
## [329,]        3         0
## [330,]        4         0
## [331,]        3         0
## [332,]        2         0
## [333,]        3         0
## [334,]        4         0
## [335,]        3         0
## [336,]        3         0
## [337,]        2         0
## [338,]        3         0
## [339,]        4         0
## [340,]        3         0
## [341,]        2         0
## [342,]        3         0
## [343,]        1         0
## [344,]        1         0
## [345,]        2         0
## [346,]        3         0
## [347,]        4         0
## [348,]        3         0
## [349,]        3         0
## [350,]        2         0
## [351,]        4         0
## [352,]        5         0
## [353,]        4         0
## [354,]        3         0
## [355,]        3         0
## [356,]        2         0
## [357,]        2         0
## [358,]        1         0
## [359,]        4         0
## [360,]        1         0
## [361,]        1         0
## [362,]        4         0
## [363,]        3         0
## [364,]        2         0
## [365,]        1         0
## [366,]        4         0
## [367,]        1         0
## [368,]        2         0
## [369,]        3         0
## [370,]        1         0
## [371,]        5         0
## [372,]        4         0
## [373,]        3         0
## [374,]        2         0
## [375,]        1         0
## [376,]        2         0
## [377,]        3         0
## [378,]        2         0
## [379,]        4         0
## [380,]        2         0
## [381,]        3         0
## [382,]        4         0
## [383,]        7         1
## [384,]        3         0
## [385,]        2         0
## [386,]        4         0
## [387,]        4         0
## [388,]        3         0
## [389,]        2         0
## [390,]        2         0
## [391,]        1         0
## [392,]        2         0
## [393,]        6         1
## [394,]        7         1
## [395,]        2         0
## [396,]        1         0
## [397,]        2         0
## [398,]        3         0
## [399,]        1         0
## [400,]        2         0
## [401,]        3         0
## [402,]        2         0
## [403,]        1         0
## [404,]        2         0
## [405,]        3         0
## [406,]        2         0
## [407,]        3         0
## [408,]        2         0
## [409,]        3         0
## [410,]        2         0
## [411,]        4         0
## [412,]        2         0
## [413,]        2         0
## [414,]        1         0
## [415,]        2         0
## [416,]        3         0
## [417,]        2         0
## [418,]        3         0
## [419,]        2         0
## [420,]        3         0
## [421,]        2         0
## [422,]        3         0
## [423,]        4         0
## [424,]        2         0
## [425,]        2         0
## [426,]        3         0
## [427,]        4         0
## [428,]        4         0
## [429,]        1         0
## [430,]        2         0
## [431,]        3         0
## [432,]        2         0
## [433,]        1         0
## [434,]        1         0
## [435,]        2         0
\end{verbatim}

We can now specify and fit the logistic regression model using our
indicator variable:

\begin{Shaded}
\begin{Highlighting}[]
\NormalTok{gudhiv.lreg <-}\StringTok{ }\KeywordTok{glm}\NormalTok{(}\DataTypeTok{formula =}\NormalTok{ HIV }\OperatorTok{~}\StringTok{ }\NormalTok{GUD }\OperatorTok{+}\StringTok{ }\NormalTok{TRAVOUT }\OperatorTok{+}\StringTok{ }\NormalTok{part.gt}\FloatTok{.5}\NormalTok{,}
                   \DataTypeTok{family =} \KeywordTok{binomial}\NormalTok{(logit))}
\KeywordTok{summary}\NormalTok{(gudhiv.lreg)}
\KeywordTok{lreg.or}\NormalTok{(gudhiv.lreg)}
\end{Highlighting}
\end{Shaded}

\begin{verbatim}
## 
## Call:
## glm(formula = HIV ~ GUD + TRAVOUT + part.gt.5, family = binomial(logit))
## 
## Deviance Residuals: 
##     Min       1Q   Median       3Q      Max  
## -1.6092  -0.2205  -0.2205  -0.0719   3.4521  
## 
## Coefficients:
##             Estimate Std. Error z value Pr(>|z|)    
## (Intercept)  -5.9559     0.9850  -6.046 1.48e-09 ***
## GUD           1.4930     0.5805   2.572   0.0101 *  
## TRAVOUT       2.2514     0.9319   2.416   0.0157 *  
## part.gt.5     4.6791     0.7560   6.189 6.05e-10 ***
## ---
## Signif. codes:  0 '***' 0.001 '**' 0.01 '*' 0.05 '.' 0.1 ' ' 1
## 
## (Dispersion parameter for binomial family taken to be 1)
## 
##     Null deviance: 167.36  on 425  degrees of freedom
## Residual deviance: 106.43  on 422  degrees of freedom
##   (9 observations deleted due to missingness)
## AIC: 114.43
## 
## Number of Fisher Scoring iterations: 7
\end{verbatim}

\begin{verbatim}
##                 OR   LCI    UCI
## (Intercept)   0.00  0.00   0.02
## GUD           4.45  1.43  13.89
## TRAVOUT       9.50  1.53  59.02
## part.gt.5   107.67 24.47 473.84
\end{verbatim}

We can now quit R:

\begin{Shaded}
\begin{Highlighting}[]
\KeywordTok{q}\NormalTok{()}
\end{Highlighting}
\end{Shaded}

For this exercise there is no need to save the workspace image so click
the \textbf{No} or \textbf{Don't Save} button (GUI) or enter \texttt{n}
when prompted to save the workspace image (terminal).

\hypertarget{summary-3}{%
\section{Summary}\label{summary-3}}

\begin{itemize}
\item
  Using built-in functions and our own functions we can use \texttt{R}
  to analyse epidemiological data.
\item
  The power of \texttt{R} is that it can be easily extended. Many
  user-contributed functions (usually packages of related functions) are
  available for download over the Internet. We will use one of these
  packages in the next exercise.
\end{itemize}

\hypertarget{exercise5}{%
\chapter{Extending R with packages}\label{exercise5}}

\texttt{R} has no built-in functions for survival analysis but, because
it is an extensible system, survival analysis is available as an add-in
package. You can find a list of add-in packages at the \texttt{R}
website.

\url{http://www.r-project.org/}

Add-in packages are installed from the Internet. There are a series of
\texttt{R} functions that enable you to download and install add-in
packages.

The \texttt{survival} package adds functions to \texttt{R} that enable
it to analyse survival data. This package may be downloaded and
installed using \texttt{install.packages("survival")} or from the
\texttt{Packages} or \texttt{Packages\ \&\ Data} menu if you are using a
GUI version of \texttt{R}.

Packages are loaded into \texttt{R} as they are needed using the
\texttt{library()} function. Start \texttt{R} and load the
\texttt{survival} package:

\begin{Shaded}
\begin{Highlighting}[]
\KeywordTok{library}\NormalTok{(survival)}
\end{Highlighting}
\end{Shaded}

Before we go any further we should retrieve a dataset:

\begin{Shaded}
\begin{Highlighting}[]
\NormalTok{ca <-}\StringTok{ }\KeywordTok{read.table}\NormalTok{(}\StringTok{"ca.dat"}\NormalTok{, }\DataTypeTok{header =} \OtherTok{TRUE}\NormalTok{)}
\KeywordTok{attach}\NormalTok{(ca)}
\end{Highlighting}
\end{Shaded}

\begin{verbatim}
## The following objects are masked from ca (pos = 10):
## 
##     group, status, time
\end{verbatim}

The columns in this dataset on the survival of cancer patients in two
different treatment groups are as follows:

\begin{longtable}[]{@{}ll@{}}
\toprule
\begin{minipage}[t]{0.21\columnwidth}\raggedright
\textbf{time}\strut
\end{minipage} & \begin{minipage}[t]{0.54\columnwidth}\raggedright
Survival or censoring time (months)\strut
\end{minipage}\tabularnewline
\begin{minipage}[t]{0.21\columnwidth}\raggedright
\textbf{status}\strut
\end{minipage} & \begin{minipage}[t]{0.54\columnwidth}\raggedright
Censoring status (1=dead, 0=censored)\strut
\end{minipage}\tabularnewline
\begin{minipage}[t]{0.21\columnwidth}\raggedright
\textbf{group}\strut
\end{minipage} & \begin{minipage}[t]{0.54\columnwidth}\raggedright
Treatment group (1 / 2)\strut
\end{minipage}\tabularnewline
\bottomrule
\end{longtable}

We next need to create a \texttt{survival} object from the \texttt{time}
and \texttt{status} variables using the \texttt{Surv()} function:

\begin{Shaded}
\begin{Highlighting}[]
\NormalTok{response <-}\StringTok{ }\KeywordTok{Surv}\NormalTok{(time, status)}
\end{Highlighting}
\end{Shaded}

We can then specify the model for the survival analysis. In this case we
state that survival (\texttt{response}) is dependent upon the treatment
\texttt{group}:

\begin{Shaded}
\begin{Highlighting}[]
\NormalTok{ca.surv <-}\StringTok{ }\KeywordTok{survfit}\NormalTok{(response }\OperatorTok{~}\StringTok{ }\NormalTok{group)}
\end{Highlighting}
\end{Shaded}

The \texttt{summary()} function applied to a \texttt{survfit} object
lists the survival probabilities at each time point with 95\% confidence
intervals:

\begin{Shaded}
\begin{Highlighting}[]
\KeywordTok{summary}\NormalTok{(ca.surv)}
\end{Highlighting}
\end{Shaded}

\begin{verbatim}
## Call: survfit(formula = response ~ group)
## 
##                 group=1 
##  time n.risk n.event survival std.err lower 95% CI upper 95% CI
##     8     22       1    0.955  0.0444       0.8714        1.000
##     9     21       1    0.909  0.0613       0.7966        1.000
##    13     19       1    0.861  0.0744       0.7270        1.000
##    14     17       1    0.811  0.0856       0.6591        0.997
##    18     16       1    0.760  0.0940       0.5963        0.968
##    19     15       1    0.709  0.1005       0.5373        0.936
##    21     14       1    0.659  0.1053       0.4814        0.901
##    23     13       1    0.608  0.1087       0.4282        0.863
##    30     10       1    0.547  0.1136       0.3643        0.822
##    31      9       1    0.486  0.1161       0.3046        0.776
##    32      8       1    0.426  0.1164       0.2489        0.727
##    34      7       1    0.365  0.1146       0.1971        0.675
##    48      5       1    0.292  0.1125       0.1371        0.621
##    56      3       1    0.195  0.1092       0.0647        0.585
## 
##                 group=2 
##  time n.risk n.event survival std.err lower 95% CI upper 95% CI
##     4     24       1   0.9583  0.0408      0.88163        1.000
##     5     23       2   0.8750  0.0675      0.75221        1.000
##     6     21       1   0.8333  0.0761      0.69681        0.997
##     7     20       1   0.7917  0.0829      0.64478        0.972
##     8     19       2   0.7083  0.0928      0.54795        0.916
##     9     17       1   0.6667  0.0962      0.50240        0.885
##    11     16       1   0.6250  0.0988      0.45845        0.852
##    12     15       1   0.5833  0.1006      0.41598        0.818
##    21     12       1   0.5347  0.1033      0.36614        0.781
##    23     11       1   0.4861  0.1047      0.31866        0.742
##    27     10       1   0.4375  0.1049      0.27340        0.700
##    28      9       1   0.3889  0.1039      0.23032        0.657
##    30      8       1   0.3403  0.1017      0.18945        0.611
##    32      7       1   0.2917  0.0981      0.15088        0.564
##    33      6       1   0.2431  0.0930      0.11481        0.515
##    37      5       1   0.1944  0.0862      0.08157        0.464
##    41      4       2   0.0972  0.0650      0.02624        0.360
##    43      2       1   0.0486  0.0473      0.00722        0.327
##    45      1       1   0.0000     NaN           NA           NA
\end{verbatim}

Printing the \texttt{ca.surv} object provides another view of the
results:

\begin{Shaded}
\begin{Highlighting}[]
\NormalTok{ca.surv}
\end{Highlighting}
\end{Shaded}

\begin{verbatim}
## Call: survfit(formula = response ~ group)
## 
##          n events median 0.95LCL 0.95UCL
## group=1 22     14     31      21      NA
## group=2 24     22     23      11      37
\end{verbatim}

The \texttt{plot()} function with a \texttt{survfit} object displays the
survival curves:

\begin{Shaded}
\begin{Highlighting}[]
\KeywordTok{plot}\NormalTok{(ca.surv, }\DataTypeTok{xlab =} \StringTok{"Months"}\NormalTok{, }\DataTypeTok{ylab =} \StringTok{"Survival"}\NormalTok{)}
\end{Highlighting}
\end{Shaded}

\includegraphics{prfe_files/figure-latex/unnamed-chunk-233-1.pdf}

We can make it easier to distinguish between the two lines by specifying
a width for each line using thelwd parameter of the \texttt{plot()}
function:

\begin{Shaded}
\begin{Highlighting}[]
\KeywordTok{plot}\NormalTok{(ca.surv, }\DataTypeTok{xlab =} \StringTok{"Months"}\NormalTok{, }\DataTypeTok{ylab =} \StringTok{"Survival"}\NormalTok{, }\DataTypeTok{lwd =} \KeywordTok{c}\NormalTok{(}\DecValTok{1}\NormalTok{, }\DecValTok{2}\NormalTok{))}
\end{Highlighting}
\end{Shaded}

\includegraphics{prfe_files/figure-latex/unnamed-chunk-234-1.pdf}

It would also be useful to add a legend:

\begin{Shaded}
\begin{Highlighting}[]
\KeywordTok{legend}\NormalTok{(}\DecValTok{125}\NormalTok{, }\DecValTok{1}\NormalTok{, }\KeywordTok{names}\NormalTok{(ca.surv}\OperatorTok{$}\NormalTok{strata), }\DataTypeTok{lwd =} \KeywordTok{c}\NormalTok{(}\DecValTok{1}\NormalTok{, }\DecValTok{2}\NormalTok{))}
\end{Highlighting}
\end{Shaded}

\includegraphics{prfe_files/figure-latex/unnamed-chunk-236-1.pdf}

If there is only one survival curve to plot then plotting a
\texttt{survfit} object will plot the survival curve with 95\%
confidence limits. You can specify that confidence limits should be
plotted when there is more than one survival curve but the results can
be disappointing:

\begin{Shaded}
\begin{Highlighting}[]
\KeywordTok{plot}\NormalTok{(ca.surv, }\DataTypeTok{conf.int =} \OtherTok{TRUE}\NormalTok{)}
\end{Highlighting}
\end{Shaded}

\includegraphics{prfe_files/figure-latex/unnamed-chunk-237-1.pdf}

Plots can be improved by specifying different colours for each curve:

\begin{Shaded}
\begin{Highlighting}[]
\KeywordTok{plot}\NormalTok{(ca.surv, }\DataTypeTok{conf.int =} \OtherTok{TRUE}\NormalTok{, }\DataTypeTok{col =} \KeywordTok{c}\NormalTok{(}\StringTok{"red"}\NormalTok{, }\StringTok{"darkgreen"}\NormalTok{))}
\end{Highlighting}
\end{Shaded}

\includegraphics{prfe_files/figure-latex/unnamed-chunk-238-1.pdf}

We can perform a formal test of the two survival times using the
\texttt{survdiff()} function:

\begin{Shaded}
\begin{Highlighting}[]
\KeywordTok{survdiff}\NormalTok{(response }\OperatorTok{~}\StringTok{ }\NormalTok{group)}
\end{Highlighting}
\end{Shaded}

\begin{verbatim}
## Call:
## survdiff(formula = response ~ group)
## 
##          N Observed Expected (O-E)^2/E (O-E)^2/V
## group=1 22       14     21.1      2.38      6.26
## group=2 24       22     14.9      3.36      6.26
## 
##  Chisq= 6.3  on 1 degrees of freedom, p= 0.0123
\end{verbatim}

We can now quit \texttt{R}:

\begin{Shaded}
\begin{Highlighting}[]
\KeywordTok{q}\NormalTok{()}
\end{Highlighting}
\end{Shaded}

For this exercise there is no need to save the workspace image so click
the \textbf{No} or \textbf{Don't Save} button (GUI) or enter \texttt{n}
when prompted to save the workspace image (terminal).

\hypertarget{summary-4}{%
\section{Summary}\label{summary-4}}

\begin{itemize}
\item
  \texttt{R} can be extended by adding additional packages. Some
  packages are included with the standard \texttt{R} installation but
  many others are available and may be downloaded from the Internet.
\item
  You can find a list of add-in packages at the \texttt{R} website:
  \url{http://www.r-project.org/}
\item
  Packages may also be downloaded and installed from this site using the
  \texttt{install.packages()} function or from the \textbf{Packages} or
  \textbf{Packages \& Data} menu if you are using a GUI version of
  \texttt{R}.
\item
  Packages are loaded into \texttt{R} as they are needed using the
  \texttt{library()} function. You can use the \texttt{search()}
  function to display a list of loaded packages and attached
  data.frames.
\end{itemize}

\hypertarget{exercise6}{%
\chapter{Making your own objects behave like R
objects}\label{exercise6}}

In the previous exercises we concentrated on writing functions that take
some input data, analyse it, and display the results of the analysis.
The standard \texttt{R} functions we have used all do this. The
\texttt{fisher.test()} function, for example, takes a \texttt{table}
object (or the names of two variables) as input and calculates and
displays the p- value for \emph{Fisher's exact test} and the odds ratio
and associated confidence interval for two-by-two tables:

\begin{Shaded}
\begin{Highlighting}[]
\NormalTok{fem <-}\StringTok{ }\KeywordTok{read.table}\NormalTok{(}\StringTok{"fem.dat"}\NormalTok{, }\DataTypeTok{header =} \OtherTok{TRUE}\NormalTok{)}
\KeywordTok{attach}\NormalTok{(fem)}
\end{Highlighting}
\end{Shaded}

\begin{verbatim}
## The following objects are masked from fem (pos = 6):
## 
##     AGE, ANX, DEP, ID, IQ, LIFE, SEX, SLP, WT
\end{verbatim}

\begin{verbatim}
## The following objects are masked from fem (pos = 7):
## 
##     AGE, ANX, DEP, ID, IQ, LIFE, SEX, SLP, WT
\end{verbatim}

\begin{verbatim}
## The following objects are masked from fem (pos = 9):
## 
##     AGE, ANX, DEP, ID, IQ, LIFE, SEX, SLP, WT
\end{verbatim}

\begin{verbatim}
## The following objects are masked from fem (pos = 10):
## 
##     AGE, ANX, DEP, ID, IQ, LIFE, SEX, SLP, WT
\end{verbatim}

\begin{verbatim}
## The following objects are masked from fem (pos = 15):
## 
##     AGE, ANX, DEP, ID, IQ, LIFE, SEX, SLP, WT
\end{verbatim}

\begin{Shaded}
\begin{Highlighting}[]
\KeywordTok{fisher.test}\NormalTok{(SEX, LIFE)}
\end{Highlighting}
\end{Shaded}

\begin{verbatim}
## 
##  Fisher's Exact Test for Count Data
## 
## data:  SEX and LIFE
## p-value = 0.03175
## alternative hypothesis: true odds ratio is not equal to 1
## 95 percent confidence interval:
##   1.080298 14.214482
## sample estimates:
## odds ratio 
##   3.620646
\end{verbatim}

The results of the \texttt{fisher.test()} function may also be saved for
later use:

\begin{Shaded}
\begin{Highlighting}[]
\NormalTok{ft <-}\StringTok{ }\KeywordTok{fisher.test}\NormalTok{(SEX, LIFE)}
\NormalTok{ft}
\end{Highlighting}
\end{Shaded}

\begin{verbatim}
## 
##  Fisher's Exact Test for Count Data
## 
## data:  SEX and LIFE
## p-value = 0.03175
## alternative hypothesis: true odds ratio is not equal to 1
## 95 percent confidence interval:
##   1.080298 14.214482
## sample estimates:
## odds ratio 
##   3.620646
\end{verbatim}

The \texttt{fisher.test()} function returns an object of the class
\texttt{htest}:

\begin{Shaded}
\begin{Highlighting}[]
\KeywordTok{class}\NormalTok{(ft)}
\end{Highlighting}
\end{Shaded}

\begin{verbatim}
## [1] "htest"
\end{verbatim}

which is a list containing the output of the \texttt{fisher.test()}
function. Each item of output is stored as a different named item in the
list:

\begin{Shaded}
\begin{Highlighting}[]
\KeywordTok{names}\NormalTok{(ft)}
\KeywordTok{str}\NormalTok{(ft)}
\end{Highlighting}
\end{Shaded}

\begin{verbatim}
## [1] "p.value"     "conf.int"    "estimate"    "null.value"  "alternative"
## [6] "method"      "data.name"
\end{verbatim}

\begin{verbatim}
## List of 7
##  $ p.value    : num 0.0318
##  $ conf.int   : num [1:2] 1.08 14.21
##   ..- attr(*, "conf.level")= num 0.95
##  $ estimate   : Named num 3.62
##   ..- attr(*, "names")= chr "odds ratio"
##  $ null.value : Named num 1
##   ..- attr(*, "names")= chr "odds ratio"
##  $ alternative: chr "two.sided"
##  $ method     : chr "Fisher's Exact Test for Count Data"
##  $ data.name  : chr "SEX and LIFE"
##  - attr(*, "class")= chr "htest"
\end{verbatim}

Each of these items can be referred to by name:

\begin{Shaded}
\begin{Highlighting}[]
\NormalTok{ft}\OperatorTok{$}\NormalTok{estimate}
\NormalTok{ft}\OperatorTok{$}\NormalTok{conf.int}
\end{Highlighting}
\end{Shaded}

\begin{verbatim}
## odds ratio 
##   3.620646
\end{verbatim}

\begin{verbatim}
## [1]  1.080298 14.214482
## attr(,"conf.level")
## [1] 0.95
\end{verbatim}

When you display the output of the \texttt{fisher.test()} function
either by calling the function directly:

\begin{Shaded}
\begin{Highlighting}[]
\KeywordTok{fisher.test}\NormalTok{(SEX, LIFE)}
\end{Highlighting}
\end{Shaded}

\begin{verbatim}
## 
##  Fisher's Exact Test for Count Data
## 
## data:  SEX and LIFE
## p-value = 0.03175
## alternative hypothesis: true odds ratio is not equal to 1
## 95 percent confidence interval:
##   1.080298 14.214482
## sample estimates:
## odds ratio 
##   3.620646
\end{verbatim}

or by typing the name of an object created using the
\texttt{fisher.test()} function:

\begin{Shaded}
\begin{Highlighting}[]
\NormalTok{ft}
\end{Highlighting}
\end{Shaded}

\begin{verbatim}
## 
##  Fisher's Exact Test for Count Data
## 
## data:  SEX and LIFE
## p-value = 0.03175
## alternative hypothesis: true odds ratio is not equal to 1
## 95 percent confidence interval:
##   1.080298 14.214482
## sample estimates:
## odds ratio 
##   3.620646
\end{verbatim}

The \texttt{print()} function takes over and formatted output is
produced. The \texttt{print()} function knows about \texttt{htest} class
objects and produces output of the correct format for that class of
object. This means that any function that produces an \texttt{htest}
object (or any other standard \texttt{R} object) does not need to
include \texttt{R} commands to produce formatted output.

All hypothesis testing functions supplied with \texttt{R} produce
objects of the htest class and use the \texttt{print()} function to
produce formatted output. For example:

\begin{Shaded}
\begin{Highlighting}[]
\NormalTok{tt <-}\StringTok{ }\KeywordTok{t.test}\NormalTok{(WT }\OperatorTok{~}\StringTok{ }\NormalTok{LIFE)}
\KeywordTok{class}\NormalTok{(tt)}
\NormalTok{tt}
\end{Highlighting}
\end{Shaded}

\begin{verbatim}
## [1] "htest"
\end{verbatim}

\begin{verbatim}
## 
##  Welch Two Sample t-test
## 
## data:  WT by LIFE
## t = 0.60608, df = 98.866, p-value = 0.5459
## alternative hypothesis: true difference in means is not equal to 0
## 95 percent confidence interval:
##  -0.3326225  0.6251763
## sample estimates:
## mean in group 1 mean in group 2 
##       0.7867213       0.6404444
\end{verbatim}

You can use this feature of \texttt{R} in your own functions. We will
explore this by writing a function to test the null hypothesis that the
\emph{variance to mean ratio} of a vector of numbers is equal to one.
Such a test might be used to investigate the spatial distribution
(e.g.~over natural sampling units such as households) of cases of a
disease.

Create a new function using the \texttt{function()} function:

\begin{Shaded}
\begin{Highlighting}[]
\NormalTok{v2m.test <-}\StringTok{ }\ControlFlowTok{function}\NormalTok{(data) \{\}}
\end{Highlighting}
\end{Shaded}

And start the function editor:

\begin{Shaded}
\begin{Highlighting}[]
\KeywordTok{fix}\NormalTok{(v2m.test)}
\end{Highlighting}
\end{Shaded}

Now edit this function to make it do something useful:

\begin{Shaded}
\begin{Highlighting}[]
\ControlFlowTok{function}\NormalTok{(data) \{}
\NormalTok{  nsu <-}\StringTok{ }\KeywordTok{length}\NormalTok{(data)}
\NormalTok{  obs <-}\StringTok{ }\KeywordTok{sum}\NormalTok{(data)}
\NormalTok{  m <-}\StringTok{ }\NormalTok{obs }\OperatorTok{/}\StringTok{ }\NormalTok{nsu}
\NormalTok{  v <-}\StringTok{ }\KeywordTok{var}\NormalTok{(data)}
\NormalTok{  vmr <-}\StringTok{ }\NormalTok{v }\OperatorTok{/}\StringTok{ }\NormalTok{m}
\NormalTok{  chi2 <-}\StringTok{ }\KeywordTok{sum}\NormalTok{((data }\OperatorTok{-}\StringTok{ }\NormalTok{m)}\OperatorTok{^}\DecValTok{2}\NormalTok{) }\OperatorTok{/}\StringTok{ }\NormalTok{m}
\NormalTok{  df <-}\StringTok{ }\NormalTok{nsu }\OperatorTok{-}\StringTok{ }\DecValTok{1}
\NormalTok{  p <-}\StringTok{ }\DecValTok{1} \OperatorTok{-}\StringTok{ }\KeywordTok{pchisq}\NormalTok{(chi2, df)}
  \KeywordTok{names}\NormalTok{(chi2) <-}\StringTok{ "Chi-square"}
  \KeywordTok{names}\NormalTok{(df) <-}\StringTok{ "df"}
  \KeywordTok{names}\NormalTok{(vmr) <-}\StringTok{ "Variance : mean ratio"}
\NormalTok{  v2m <-}\StringTok{ }\KeywordTok{list}\NormalTok{(}\DataTypeTok{method =} \StringTok{"Variance to mean test"}\NormalTok{,}
              \DataTypeTok{data.name =} \KeywordTok{deparse}\NormalTok{(}\KeywordTok{substitute}\NormalTok{(data)),}
              \DataTypeTok{statistic =}\NormalTok{ chi2,}
              \DataTypeTok{parameter =}\NormalTok{ df,}
              \DataTypeTok{p.value =}\NormalTok{ p,}
              \DataTypeTok{estimate =}\NormalTok{ vmr)}
  \KeywordTok{class}\NormalTok{(v2m) <-}\StringTok{ "htest"}
  \KeywordTok{return}\NormalTok{(v2m)}
\NormalTok{\}}
\end{Highlighting}
\end{Shaded}

\begin{Shaded}
\begin{Highlighting}[]
\NormalTok{v2m.test <-}\StringTok{ }\ControlFlowTok{function}\NormalTok{(data) \{}
\NormalTok{  nsu <-}\StringTok{ }\KeywordTok{length}\NormalTok{(data)}
\NormalTok{  obs <-}\StringTok{ }\KeywordTok{sum}\NormalTok{(data)}
\NormalTok{  m <-}\StringTok{ }\NormalTok{obs }\OperatorTok{/}\StringTok{ }\NormalTok{nsu}
\NormalTok{  v <-}\StringTok{ }\KeywordTok{var}\NormalTok{(data)}
\NormalTok{  vmr <-}\StringTok{ }\NormalTok{v }\OperatorTok{/}\StringTok{ }\NormalTok{m}
\NormalTok{  chi2 <-}\StringTok{ }\KeywordTok{sum}\NormalTok{((data }\OperatorTok{-}\StringTok{ }\NormalTok{m)}\OperatorTok{^}\DecValTok{2}\NormalTok{) }\OperatorTok{/}\StringTok{ }\NormalTok{m}
\NormalTok{  df <-}\StringTok{ }\NormalTok{nsu }\OperatorTok{-}\StringTok{ }\DecValTok{1}
\NormalTok{  p <-}\StringTok{ }\DecValTok{1} \OperatorTok{-}\StringTok{ }\KeywordTok{pchisq}\NormalTok{(chi2, df)}
  \KeywordTok{names}\NormalTok{(chi2) <-}\StringTok{ "Chi-square"}
  \KeywordTok{names}\NormalTok{(df) <-}\StringTok{ "df"}
  \KeywordTok{names}\NormalTok{(vmr) <-}\StringTok{ "Variance : mean ratio"}
\NormalTok{  v2m <-}\StringTok{ }\KeywordTok{list}\NormalTok{(}\DataTypeTok{method =} \StringTok{"Variance to mean test"}\NormalTok{,}
              \DataTypeTok{data.name =} \KeywordTok{deparse}\NormalTok{(}\KeywordTok{substitute}\NormalTok{(data)),}
              \DataTypeTok{statistic =}\NormalTok{ chi2,}
              \DataTypeTok{parameter =}\NormalTok{ df,}
              \DataTypeTok{p.value =}\NormalTok{ p,}
              \DataTypeTok{estimate =}\NormalTok{ vmr)}
  \KeywordTok{class}\NormalTok{(v2m) <-}\StringTok{ "htest"}
  \KeywordTok{return}\NormalTok{(v2m)}
\NormalTok{\}}
\end{Highlighting}
\end{Shaded}

Once you have made the changes shown above, check your work, save the
file, and quit the editor.

Before proceeding we should examine the \texttt{v2m.test()} function to
make sure we understand what is happening:

\begin{enumerate}
\def\labelenumi{\arabic{enumi}.}
\item
  The first eight lines after the opening curly bracket (\texttt{\{})
  contain the required calculations.
\item
  The next three lines use the \texttt{names()} function to give our
  variables names that will make sense in formatted output.
\item
  The next line creates a list of items that the function returns using
  some of the names used by \texttt{htest} class objects.
\item
  The next line tells \texttt{R} that the list object called
  \texttt{v2m} is of the class \texttt{htest}.
\item
  The next line causes the function to return the \texttt{v2m} object
  (i.e.~a list of class \texttt{htest} containing the named items
  \texttt{method}, \texttt{data.name}, \texttt{statistic},
  \texttt{parameter}, \texttt{p.value}, and \texttt{estimate}).
\item
  The final line ends the function definition.
\end{enumerate}

Note that objects of class htest may contain items with the following
names:

\begin{longtable}[]{@{}ll@{}}
\toprule
\begin{minipage}[b]{0.21\columnwidth}\raggedright
\textbf{Item}\strut
\end{minipage} & \begin{minipage}[b]{0.73\columnwidth}\raggedright
\textbf{Usage}\strut
\end{minipage}\tabularnewline
\midrule
\endhead
\begin{minipage}[t]{0.21\columnwidth}\raggedright
\textbf{method}\strut
\end{minipage} & \begin{minipage}[t]{0.73\columnwidth}\raggedright
Text description of the test used to title output\strut
\end{minipage}\tabularnewline
\begin{minipage}[t]{0.21\columnwidth}\raggedright
\textbf{data.name}\strut
\end{minipage} & \begin{minipage}[t]{0.73\columnwidth}\raggedright
Name(s) of data or variables used for the test\strut
\end{minipage}\tabularnewline
\begin{minipage}[t]{0.21\columnwidth}\raggedright
\textbf{null.value}\strut
\end{minipage} & \begin{minipage}[t]{0.73\columnwidth}\raggedright
The null value\strut
\end{minipage}\tabularnewline
\begin{minipage}[t]{0.21\columnwidth}\raggedright
\textbf{statistic}\strut
\end{minipage} & \begin{minipage}[t]{0.73\columnwidth}\raggedright
Value of test statistic\strut
\end{minipage}\tabularnewline
\begin{minipage}[t]{0.21\columnwidth}\raggedright
\textbf{parameter}\strut
\end{minipage} & \begin{minipage}[t]{0.73\columnwidth}\raggedright
A test parameter such as the degrees of freedom of the test
statistic\strut
\end{minipage}\tabularnewline
\begin{minipage}[t]{0.21\columnwidth}\raggedright
\textbf{p.value}\strut
\end{minipage} & \begin{minipage}[t]{0.73\columnwidth}\raggedright
The p-value of the test\strut
\end{minipage}\tabularnewline
\begin{minipage}[t]{0.21\columnwidth}\raggedright
\textbf{estimate}\strut
\end{minipage} & \begin{minipage}[t]{0.73\columnwidth}\raggedright
An estimate (e.g.~the mean)\strut
\end{minipage}\tabularnewline
\begin{minipage}[t]{0.21\columnwidth}\raggedright
\textbf{conf.int}\strut
\end{minipage} & \begin{minipage}[t]{0.73\columnwidth}\raggedright
Confidence interval of estimate\strut
\end{minipage}\tabularnewline
\begin{minipage}[t]{0.21\columnwidth}\raggedright
\textbf{alternative}\strut
\end{minipage} & \begin{minipage}[t]{0.73\columnwidth}\raggedright
Text describing the alternative hypothesis\strut
\end{minipage}\tabularnewline
\begin{minipage}[t]{0.21\columnwidth}\raggedright
\textbf{note}\strut
\end{minipage} & \begin{minipage}[t]{0.73\columnwidth}\raggedright
Text note\strut
\end{minipage}\tabularnewline
\bottomrule
\end{longtable}

We are now ready to test the \texttt{v2m.test()} function. This table:

\begin{verbatim}
Number of cases :       0  1  2  3  4  6
Number of households : 24 29 26 14  5  2
\end{verbatim}

shows the number of cases of chronic (stunting) undernutrition found in
a random sample of 100 households.

We can reproduce the data behind this table using a combination of the
\texttt{c()} and \texttt{rep()} functions:

\begin{Shaded}
\begin{Highlighting}[]
\NormalTok{stunt <-}\StringTok{ }\KeywordTok{c}\NormalTok{(}\KeywordTok{rep}\NormalTok{(}\DecValTok{0}\NormalTok{,}\DecValTok{24}\NormalTok{), }\KeywordTok{rep}\NormalTok{(}\DecValTok{1}\NormalTok{,}\DecValTok{29}\NormalTok{), }\KeywordTok{rep}\NormalTok{(}\DecValTok{2}\NormalTok{,}\DecValTok{26}\NormalTok{), }\KeywordTok{rep}\NormalTok{(}\DecValTok{3}\NormalTok{,}\DecValTok{14}\NormalTok{), }\KeywordTok{rep}\NormalTok{(}\DecValTok{4}\NormalTok{,}\DecValTok{5}\NormalTok{),}
           \KeywordTok{rep}\NormalTok{(}\DecValTok{5}\NormalTok{,}\DecValTok{0}\NormalTok{), }\KeywordTok{rep}\NormalTok{(}\DecValTok{6}\NormalTok{,}\DecValTok{2}\NormalTok{))}
\KeywordTok{table}\NormalTok{(stunt)}
\end{Highlighting}
\end{Shaded}

\begin{verbatim}
## stunt
##  0  1  2  3  4  6 
## 24 29 26 14  5  2
\end{verbatim}

And use it to test our new \texttt{v2m.test()} function:

\begin{Shaded}
\begin{Highlighting}[]
\KeywordTok{v2m.test}\NormalTok{(stunt)}
\end{Highlighting}
\end{Shaded}

Which should produce the following output:

\begin{Shaded}
\begin{Highlighting}[]
\KeywordTok{v2m.test}\NormalTok{(stunt)}
\end{Highlighting}
\end{Shaded}

\begin{verbatim}
## 
##  Variance to mean test
## 
## data:  stunt
## Chi-square = 110.16, df = 99, p-value = 0.2083
## sample estimates:
## Variance : mean ratio 
##               1.11274
\end{verbatim}

If your \texttt{vm2.test()} function does not produce this output then
use the \texttt{fix()} function:

\begin{Shaded}
\begin{Highlighting}[]
\KeywordTok{fix}\NormalTok{(v2m.test)}
\end{Highlighting}
\end{Shaded}

to check and edit the \texttt{vm2.test()} function and try again.

The important thing to note from this exercise is that \texttt{R} allows
us to specify a class for the output of our functions. This means that
we can use standard \texttt{R} classes and functions to (e.g.) produce
formatted output without us having to write commands to format the
output ourselves.

More importantly, it also means that we can write functions that return
values when we need them to return values but can also produce formatted
output when we need them to produce formatted output.

Our \texttt{v2m.test()} function can produce values for later use:

\begin{Shaded}
\begin{Highlighting}[]
\NormalTok{vm <-}\StringTok{ }\KeywordTok{v2m.test}\NormalTok{(stunt)}
\NormalTok{vm}\OperatorTok{$}\NormalTok{p.value}
\end{Highlighting}
\end{Shaded}

\begin{verbatim}
## [1] 0.2083442
\end{verbatim}

or produce formatted output:

\begin{Shaded}
\begin{Highlighting}[]
\KeywordTok{v2m.test}\NormalTok{(stunt)}
\end{Highlighting}
\end{Shaded}

\begin{verbatim}
## 
##  Variance to mean test
## 
## data:  stunt
## Chi-square = 110.16, df = 99, p-value = 0.2083
## sample estimates:
## Variance : mean ratio 
##               1.11274
\end{verbatim}

This way of working is not limited to using standard \texttt{R} classes
and functions.

\texttt{R} also allows us to define our own classes. We will explore
this by defining functions and a new class to deal with two-by-two
tables.

We need to create two functions:

\begin{enumerate}
\def\labelenumi{\arabic{enumi}.}
\item
  One function will handle the calculations.
\item
  A second function function will produce formatted output when
  required.
\end{enumerate}

Create a new function using the \texttt{function()} function:

\begin{Shaded}
\begin{Highlighting}[]
\NormalTok{rr22 <-}\StringTok{ }\ControlFlowTok{function}\NormalTok{(exposure, outcome) \{\}}
\end{Highlighting}
\end{Shaded}

And start the function editor:

\begin{Shaded}
\begin{Highlighting}[]
\KeywordTok{fix}\NormalTok{(rr22)}
\end{Highlighting}
\end{Shaded}

Now edit this function to make it do something useful:

\begin{Shaded}
\begin{Highlighting}[]
\ControlFlowTok{function}\NormalTok{(exposure, outcome) \{}
\NormalTok{  tab <-}\StringTok{ }\KeywordTok{table}\NormalTok{(exposure, outcome)}
\NormalTok{  a <-}\StringTok{ }\NormalTok{tab[}\DecValTok{1}\NormalTok{,}\DecValTok{1}\NormalTok{]}
\NormalTok{  b <-}\StringTok{ }\NormalTok{tab[}\DecValTok{1}\NormalTok{,}\DecValTok{2}\NormalTok{]}
\NormalTok{  c <-}\StringTok{ }\NormalTok{tab[}\DecValTok{2}\NormalTok{,}\DecValTok{1}\NormalTok{]}
\NormalTok{  d <-}\StringTok{ }\NormalTok{tab[}\DecValTok{2}\NormalTok{,}\DecValTok{2}\NormalTok{]}
\NormalTok{  rr <-}\StringTok{ }\NormalTok{(a }\OperatorTok{/}\StringTok{ }\NormalTok{(a }\OperatorTok{+}\StringTok{ }\NormalTok{b)) }\OperatorTok{/}\StringTok{ }\NormalTok{(c }\OperatorTok{/}\StringTok{ }\NormalTok{(c }\OperatorTok{+}\StringTok{ }\NormalTok{d))}
\NormalTok{  se.log.rr <-}\StringTok{ }\KeywordTok{sqrt}\NormalTok{((b }\OperatorTok{/}\StringTok{ }\NormalTok{a) }\OperatorTok{/}\StringTok{ }\NormalTok{(a }\OperatorTok{+}\StringTok{ }\NormalTok{b) }\OperatorTok{+}\StringTok{ }\NormalTok{(d }\OperatorTok{/}\StringTok{ }\NormalTok{c) }\OperatorTok{/}\StringTok{ }\NormalTok{(c }\OperatorTok{+}\StringTok{ }\NormalTok{d))}
\NormalTok{  lci <-}\StringTok{ }\KeywordTok{exp}\NormalTok{(}\KeywordTok{log}\NormalTok{(rr) }\OperatorTok{-}\StringTok{ }\FloatTok{1.96} \OperatorTok{*}\StringTok{ }\NormalTok{se.log.rr)}
\NormalTok{  uci <-}\StringTok{ }\KeywordTok{exp}\NormalTok{(}\KeywordTok{log}\NormalTok{(rr) }\OperatorTok{+}\StringTok{ }\FloatTok{1.96} \OperatorTok{*}\StringTok{ }\NormalTok{se.log.rr)}
\NormalTok{  rr22.output <-}\StringTok{ }\KeywordTok{list}\NormalTok{(}\DataTypeTok{estimate =}\NormalTok{ rr, }\DataTypeTok{ci =} \KeywordTok{c}\NormalTok{(lci, uci))}
  \KeywordTok{class}\NormalTok{(rr22.output) <-}\StringTok{ "rr22"}
  \KeywordTok{return}\NormalTok{(rr22.output)}
\NormalTok{\}}
\end{Highlighting}
\end{Shaded}

Once you have made the changes shown above, save the file and quit the
editor.

The \texttt{rr22()} function is similar to the \texttt{tab2by2()}
function that you created in the second exercise of this tutorial except
that the function now returns a list of values instead of formatted
output:

\begin{Shaded}
\begin{Highlighting}[]
\NormalTok{fem <-}\StringTok{ }\KeywordTok{read.table}\NormalTok{(}\StringTok{"fem.dat"}\NormalTok{, }\DataTypeTok{header =} \OtherTok{TRUE}\NormalTok{)}
\KeywordTok{attach}\NormalTok{(fem)}
\NormalTok{rr22.test <-}\StringTok{ }\KeywordTok{rr22}\NormalTok{(SEX, LIFE)}
\KeywordTok{names}\NormalTok{(rr22.test)}
\NormalTok{rr22.test}\OperatorTok{$}\NormalTok{estimate}
\NormalTok{rr22.test}\OperatorTok{$}\NormalTok{conf.int}
\NormalTok{rr22.test}\OperatorTok{$}\NormalTok{conf.int[}\DecValTok{1}\NormalTok{]}
\NormalTok{rr22.test}\OperatorTok{$}\NormalTok{conf.int[}\DecValTok{2}\NormalTok{]}
\end{Highlighting}
\end{Shaded}

\begin{verbatim}
## The following objects are masked from fem (pos = 3):
## 
##     AGE, ANX, DEP, ID, IQ, LIFE, SEX, SLP, WT
\end{verbatim}

\begin{verbatim}
## The following objects are masked from fem (pos = 7):
## 
##     AGE, ANX, DEP, ID, IQ, LIFE, SEX, SLP, WT
\end{verbatim}

\begin{verbatim}
## The following objects are masked from fem (pos = 8):
## 
##     AGE, ANX, DEP, ID, IQ, LIFE, SEX, SLP, WT
\end{verbatim}

\begin{verbatim}
## The following objects are masked from fem (pos = 10):
## 
##     AGE, ANX, DEP, ID, IQ, LIFE, SEX, SLP, WT
\end{verbatim}

\begin{verbatim}
## The following objects are masked from fem (pos = 11):
## 
##     AGE, ANX, DEP, ID, IQ, LIFE, SEX, SLP, WT
\end{verbatim}

\begin{verbatim}
## The following objects are masked from fem (pos = 16):
## 
##     AGE, ANX, DEP, ID, IQ, LIFE, SEX, SLP, WT
\end{verbatim}

\begin{verbatim}
## [1] "estimate" "ci"
\end{verbatim}

\begin{verbatim}
## [1] 2.054167
\end{verbatim}

\begin{verbatim}
## NULL
\end{verbatim}

\begin{verbatim}
## NULL
\end{verbatim}

\begin{verbatim}
## NULL
\end{verbatim}

The function returns a list of class \texttt{rr22}:

\begin{Shaded}
\begin{Highlighting}[]
\KeywordTok{class}\NormalTok{(rr22.test)}
\end{Highlighting}
\end{Shaded}

\begin{verbatim}
## [1] "rr22"
\end{verbatim}

The displayed output from the \texttt{rr22()} function is, however, not
pretty:

\begin{Shaded}
\begin{Highlighting}[]
\KeywordTok{print}\NormalTok{(rr22.test)}
\KeywordTok{rr22}\NormalTok{(SEX, LIFE)}
\end{Highlighting}
\end{Shaded}

\begin{verbatim}
## $estimate
## [1] 2.054167
## 
## $ci
## [1] 0.966417 4.366232
## 
## attr(,"class")
## [1] "rr22"
\end{verbatim}

\begin{verbatim}
## $estimate
## [1] 2.054167
## 
## $ci
## [1] 0.966417 4.366232
## 
## attr(,"class")
## [1] "rr22"
\end{verbatim}

This can be fixed by creating a new function:

\begin{Shaded}
\begin{Highlighting}[]
\NormalTok{print.rr22 <-}\StringTok{ }\ControlFlowTok{function}\NormalTok{(x) \{\}}
\end{Highlighting}
\end{Shaded}

And start the function editor:

\begin{Shaded}
\begin{Highlighting}[]
\KeywordTok{fix}\NormalTok{(print.rr22)}
\end{Highlighting}
\end{Shaded}

Now edit this function to make it do something useful:

\begin{Shaded}
\begin{Highlighting}[]
\ControlFlowTok{function}\NormalTok{(x) \{}
  \KeywordTok{cat}\NormalTok{(}\StringTok{"RR     : "}\NormalTok{, x}\OperatorTok{$}\NormalTok{estimate, }\StringTok{"}\CharTok{\textbackslash{}n}\StringTok{"}\NormalTok{,}
      \StringTok{"95% CI : "}\NormalTok{, x}\OperatorTok{$}\NormalTok{ci[}\DecValTok{1}\NormalTok{], }\StringTok{"; "}\NormalTok{, x}\OperatorTok{$}\NormalTok{ci[}\DecValTok{2}\NormalTok{], }\StringTok{"}\CharTok{\textbackslash{}n}\StringTok{"}\NormalTok{, }\DataTypeTok{sep =} \StringTok{""}\NormalTok{)}
\NormalTok{\}}
\end{Highlighting}
\end{Shaded}

Once you have made the changes shown above, check your work, save the
file, and quit the editor.

The function name \texttt{print.rr22()} indicates that this function
contains the print method for objects of class \texttt{rr22}. All
objects of class \texttt{rr22} will use the function
\texttt{print.rr22()} instead of the standard \texttt{R}
\texttt{print()} function to produce formatted output:

\begin{Shaded}
\begin{Highlighting}[]
\KeywordTok{rr22}\NormalTok{(SEX, LIFE)}
\NormalTok{rr22.test <-}\StringTok{ }\KeywordTok{rr22}\NormalTok{(SEX, LIFE)}
\NormalTok{rr22.test}
\KeywordTok{print}\NormalTok{(rr22.test)}
\end{Highlighting}
\end{Shaded}

\begin{verbatim}
## RR     : 2.054167
## 95% CI : 0.966417; 4.366232
\end{verbatim}

\begin{verbatim}
## RR     : 2.054167
## 95% CI : 0.966417; 4.366232
\end{verbatim}

\begin{verbatim}
## RR     : 2.054167
## 95% CI : 0.966417; 4.366232
\end{verbatim}

Note that we can still extract returned values from an \texttt{rr22}
class object:

\begin{Shaded}
\begin{Highlighting}[]
\NormalTok{rr22.test}\OperatorTok{$}\NormalTok{estimate}
\end{Highlighting}
\end{Shaded}

The \texttt{print.rr22()} function only controls the way an entire
\texttt{rr22} object is displayed.

You might like to use the \texttt{save()} function to save the
\texttt{v2m.test()}, \texttt{rr22()}, and \texttt{print.rr22()}
functions before quitting \texttt{R}. We can now quit \texttt{R}:

\begin{Shaded}
\begin{Highlighting}[]
\KeywordTok{q}\NormalTok{()}
\end{Highlighting}
\end{Shaded}

For this exercise there is no need to save the workspace image so click
the \textbf{No} or \textbf{Don't Save} button (GUI) or enter \texttt{n}
when prompted to save the workspace image (terminal).

\hypertarget{summary-5}{%
\section{Summary}\label{summary-5}}

\begin{itemize}
\item
  \texttt{R} objects can be assigned a class or type.
\item
  Objects of a specific class or type may share functions that extract
  and manipulate data common to members of that class. This allows you
  to write functions that handle data that is common to all members of
  that class (e.g.~to produce formatted output for hypothesis testing
  functions).
\item
  \texttt{R} provides a set of ready-made classes (e.g. \texttt{htest})
  which can be used by standard R functions such as the \texttt{print()}
  and \texttt{summary()} functions.
\item
  \texttt{R} allows you to create new classes and class-specific
  functions that can extract and manipulate data common to the new
  classes.
\item
  Classes allows you to create versatile functions that return values
  when you need them to return values but can also produce formatted
  output when you need them to produce formatted output.
\item
  Classes allow you to write functions that can be chained together so
  that the output of one function is the input of another function.
\end{itemize}

\hypertarget{exercise7}{%
\chapter{Writing your own graphical functions}\label{exercise7}}

\texttt{R} provides a pretty full set of graphical functions for
plotting data as well as \texttt{plot()} methods for a wide variety of
statistical functions. There will be times, however, when you will need
to write you own graphical functions to present and analyse data in a
specific way. In this exercise we will create a function that produces a
plot that may be used for assessing agreement between two methods of
clinical measurement as described in:

\begin{verbatim}
Bland JM, Altman DG. Statistical Methods for Assessing Agreement Between Two Methods of Clinical Measurement. Lancet. 1986;1: 307–310. 
\end{verbatim}

Which involves plotting the difference of two measurements against the
mean of the two measurements and calculating and displaying limits of
agreement.

Start \texttt{R} and retrieve and attach the sample dataset:

\begin{Shaded}
\begin{Highlighting}[]
\NormalTok{ba <-}\StringTok{ }\KeywordTok{read.table}\NormalTok{(}\StringTok{"ba.dat"}\NormalTok{, }\DataTypeTok{header =} \OtherTok{TRUE}\NormalTok{)}
\KeywordTok{attach}\NormalTok{(ba)}
\end{Highlighting}
\end{Shaded}

\begin{verbatim}
## The following objects are masked from ba (pos = 10):
## 
##     Mini, Wright
\end{verbatim}

The \texttt{ba} data.frame contains measurements (in litres per minute)
taken with a \emph{Wright peak flow meter} and a \emph{Mini-Wright peak
flow meter}. This is the same data that is presented in the referenced
Lancet article:

\begin{verbatim}
##    Wright Mini
## 1     494  512
## 2     395  430
## 3     516  520
## 4     434  428
## 5     476  500
## 6     557  600
## 7     413  364
## 8     442  380
## 9     650  658
## 10    433  445
## 11    417  432
## 12    656  626
## 13    267  260
## 14    478  477
## 15    178  259
## 16    423  350
## 17    427  451
\end{verbatim}

You can examine the \texttt{ba} data.frame using the \texttt{print()}
and \texttt{summary()} functions:

\begin{Shaded}
\begin{Highlighting}[]
\KeywordTok{print}\NormalTok{(ba)}
\NormalTok{ba}
\KeywordTok{summary}\NormalTok{(ba)}
\end{Highlighting}
\end{Shaded}

\begin{verbatim}
##    Wright Mini
## 1     494  512
## 2     395  430
## 3     516  520
## 4     434  428
## 5     476  500
## 6     557  600
## 7     413  364
## 8     442  380
## 9     650  658
## 10    433  445
## 11    417  432
## 12    656  626
## 13    267  260
## 14    478  477
## 15    178  259
## 16    423  350
## 17    427  451
\end{verbatim}

\begin{verbatim}
##    Wright Mini
## 1     494  512
## 2     395  430
## 3     516  520
## 4     434  428
## 5     476  500
## 6     557  600
## 7     413  364
## 8     442  380
## 9     650  658
## 10    433  445
## 11    417  432
## 12    656  626
## 13    267  260
## 14    478  477
## 15    178  259
## 16    423  350
## 17    427  451
\end{verbatim}

\begin{verbatim}
##      Wright           Mini      
##  Min.   :178.0   Min.   :259.0  
##  1st Qu.:417.0   1st Qu.:380.0  
##  Median :434.0   Median :445.0  
##  Mean   :450.4   Mean   :452.5  
##  3rd Qu.:494.0   3rd Qu.:512.0  
##  Max.   :656.0   Max.   :658.0
\end{verbatim}

The \texttt{function()} function allows us to create new functions in
\texttt{R}:

\begin{Shaded}
\begin{Highlighting}[]
\NormalTok{ba.plot <-}\StringTok{ }\ControlFlowTok{function}\NormalTok{(a, b) \{\}}
\end{Highlighting}
\end{Shaded}

This creates an empty function called \texttt{ba.plot()} that expects
two parameters called \texttt{a} and \texttt{b}. We could type the whole
function in at the \texttt{R} command prompt but it is easier to use a
text editor:

\begin{Shaded}
\begin{Highlighting}[]
\KeywordTok{fix}\NormalTok{(ba.plot)}
\end{Highlighting}
\end{Shaded}

We will start be writing a basic function which we will gradually
improve throughout this exercise.

Edit the \texttt{ba.plot()} function to read:

\begin{Shaded}
\begin{Highlighting}[]
\ControlFlowTok{function}\NormalTok{(a, b) \{}
\NormalTok{  mean.two <-}\StringTok{ }\NormalTok{(a }\OperatorTok{+}\StringTok{ }\NormalTok{b) }\OperatorTok{/}\StringTok{ }\DecValTok{2}
\NormalTok{  diff.two <-}\StringTok{ }\NormalTok{a }\OperatorTok{-}\StringTok{ }\NormalTok{b}
  \KeywordTok{plot}\NormalTok{(mean.two, diff.two)}
\NormalTok{\}}
\end{Highlighting}
\end{Shaded}

Once you have made the changes shown above, check your work, save the
file, and quit the editor.

The function calculates the mean and the difference of the two measures
and then plots the results. Lets try the \texttt{ba.plot()} function
with the test data:

\begin{Shaded}
\begin{Highlighting}[]
\KeywordTok{ba.plot}\NormalTok{(Wright, Mini)}
\end{Highlighting}
\end{Shaded}

\includegraphics{prfe_files/figure-latex/unnamed-chunk-287-1.pdf}

The resulting plot is rather plain and lacks meaningful titles and axis
labels. Use the \texttt{fix()} function to edit the \texttt{ba.plot()}
function:

\begin{Shaded}
\begin{Highlighting}[]
\KeywordTok{fix}\NormalTok{(ba.plot)}
\end{Highlighting}
\end{Shaded}

Edit the function to read:

\begin{Shaded}
\begin{Highlighting}[]
\ControlFlowTok{function}\NormalTok{(a, b, }\DataTypeTok{title =} \StringTok{"Bland and Altman Plot"}\NormalTok{) \{}
\NormalTok{  a.txt <-}\StringTok{ }\KeywordTok{deparse}\NormalTok{(}\KeywordTok{substitute}\NormalTok{(a))}
\NormalTok{  b.txt <-}\StringTok{ }\KeywordTok{deparse}\NormalTok{(}\KeywordTok{substitute}\NormalTok{(b))}
\NormalTok{  x.lab <-}\StringTok{ }\KeywordTok{paste}\NormalTok{(}\StringTok{"Mean of"}\NormalTok{, a.txt, }\StringTok{"and"}\NormalTok{, b.txt)}
\NormalTok{  y.lab <-}\StringTok{ }\KeywordTok{paste}\NormalTok{(a.txt, }\StringTok{"-"}\NormalTok{, b.txt)}
\NormalTok{  mean.two <-}\StringTok{ }\NormalTok{(a }\OperatorTok{+}\StringTok{ }\NormalTok{b) }\OperatorTok{/}\StringTok{ }\DecValTok{2}
\NormalTok{  diff.two <-}\StringTok{ }\NormalTok{a }\OperatorTok{-}\StringTok{ }\NormalTok{b}
  \KeywordTok{plot}\NormalTok{(mean.two, diff.two, }\DataTypeTok{xlab =}\NormalTok{ x.lab, }\DataTypeTok{ylab =}\NormalTok{ y.lab, }\DataTypeTok{main =}\NormalTok{ title)}
\NormalTok{\}}
\end{Highlighting}
\end{Shaded}

Once you have made the changes shown above, check your work, save the
file, and quit the editor.

We have added a new parameter (\texttt{title}) to the function and given
this a default value of \texttt{Bland\ and\ Altman\ Plot}. Adding
\texttt{title} as a parameter means that we will be able to specify a
title for the plot when we call the function. We have also used the
function combination \texttt{deparse(substitute())} to retrieve the
names of the vectors passed to parameters \texttt{a} and \texttt{b}. The
\texttt{paste()} function pastes pieces of text together. It is used
here to create the text for the axis labels used with the
\texttt{plot()} function.

Lets try the \texttt{ba.plot()} function with the test data:

\begin{Shaded}
\begin{Highlighting}[]
\KeywordTok{ba.plot}\NormalTok{(Wright, Mini)}
\end{Highlighting}
\end{Shaded}

\includegraphics{prfe_files/figure-latex/unnamed-chunk-291-1.pdf}

We may also specify a title for the plot using the title parameter:

\begin{Shaded}
\begin{Highlighting}[]
\KeywordTok{ba.plot}\NormalTok{(Wright, Mini, }\DataTypeTok{title =} \StringTok{"PEFR data"}\NormalTok{)}
\end{Highlighting}
\end{Shaded}

\includegraphics{prfe_files/figure-latex/unnamed-chunk-292-1.pdf}

We can now edit the function to calculate and plot mean, difference, and
the limits of agreement. Use the \texttt{fix()} function to edit the
\texttt{ba.plot()} function:

\begin{Shaded}
\begin{Highlighting}[]
\KeywordTok{fix}\NormalTok{(ba.plot)}
\end{Highlighting}
\end{Shaded}

Edit the function to read:

\begin{Shaded}
\begin{Highlighting}[]
\ControlFlowTok{function}\NormalTok{(a, b, }\DataTypeTok{title =} \StringTok{"Bland and Altman Plot"}\NormalTok{) \{}
\NormalTok{  a.txt <-}\StringTok{ }\KeywordTok{deparse}\NormalTok{(}\KeywordTok{substitute}\NormalTok{(a))}
\NormalTok{  b.txt <-}\StringTok{ }\KeywordTok{deparse}\NormalTok{(}\KeywordTok{substitute}\NormalTok{(b))}
\NormalTok{  x.lab <-}\StringTok{ }\KeywordTok{paste}\NormalTok{(}\StringTok{"Mean of"}\NormalTok{, a.txt, }\StringTok{"and"}\NormalTok{, b.txt)}
\NormalTok{  y.lab <-}\StringTok{ }\KeywordTok{paste}\NormalTok{(a.txt, }\StringTok{"-"}\NormalTok{, b.txt)}
\NormalTok{  mean.two <-}\StringTok{ }\NormalTok{(a }\OperatorTok{+}\StringTok{ }\NormalTok{b) }\OperatorTok{/}\StringTok{ }\DecValTok{2}
\NormalTok{  diff.two <-}\StringTok{ }\NormalTok{a }\OperatorTok{-}\StringTok{ }\NormalTok{b}
  \KeywordTok{plot}\NormalTok{(mean.two, diff.two, }\DataTypeTok{xlab =}\NormalTok{ x.lab, }\DataTypeTok{ylab =}\NormalTok{ y.lab, }\DataTypeTok{main =}\NormalTok{ title) }
\NormalTok{  mean.diff <-}\StringTok{ }\KeywordTok{mean}\NormalTok{(diff.two)}
\NormalTok{  sd.diff <-}\StringTok{ }\KeywordTok{sd}\NormalTok{(diff.two)}
\NormalTok{  upper <-}\StringTok{ }\NormalTok{mean.diff }\OperatorTok{+}\StringTok{ }\FloatTok{1.96} \OperatorTok{*}\StringTok{ }\NormalTok{sd.diff}
\NormalTok{  lower <-}\StringTok{ }\NormalTok{mean.diff }\OperatorTok{-}\StringTok{ }\FloatTok{1.96} \OperatorTok{*}\StringTok{ }\NormalTok{sd.diff}
  \KeywordTok{lines}\NormalTok{(}\DataTypeTok{x =} \KeywordTok{range}\NormalTok{(mean.two), }\DataTypeTok{y =} \KeywordTok{c}\NormalTok{(mean.diff, mean.diff), }\DataTypeTok{lty =} \DecValTok{3}\NormalTok{) }
  \KeywordTok{lines}\NormalTok{(}\DataTypeTok{x =} \KeywordTok{range}\NormalTok{(mean.two), }\DataTypeTok{y =} \KeywordTok{c}\NormalTok{(upper, upper), }\DataTypeTok{lty =} \DecValTok{3}\NormalTok{)}
  \KeywordTok{lines}\NormalTok{(}\DataTypeTok{x =} \KeywordTok{range}\NormalTok{(mean.two), }\DataTypeTok{y =} \KeywordTok{c}\NormalTok{(lower, lower), }\DataTypeTok{lty =} \DecValTok{3}\NormalTok{)}
\NormalTok{\}}
\end{Highlighting}
\end{Shaded}

Once you have made the changes shown above, check your work, save the
file and quit the editor.

We have used the \texttt{mean()} and \texttt{sd()} functions to
calculate the mean and standard deviation of the difference between the
two measures and calculated the limits of agreement (\texttt{upper} and
\texttt{lower}) assuming that the differences are \emph{Normally}
distributed.

The \texttt{lines()} function is then used to plot the mean and the
limits of agreement on top of the existing scatter plot.

The parameter \texttt{lty\ =\ 3} used with the \texttt{lines()} function
specifies dotted lines.

\texttt{R} provided a great number of graphical parameters that can be
used to customise plots. You can see a list of these parameters using:

\begin{Shaded}
\begin{Highlighting}[]
\KeywordTok{help}\NormalTok{(par)}
\end{Highlighting}
\end{Shaded}

These parameters can be specified for almost all graphical functions.

Lets try the \texttt{ba.plot()} function with the test data:

\begin{Shaded}
\begin{Highlighting}[]
\KeywordTok{ba.plot}\NormalTok{(Wright, Mini, }\DataTypeTok{title =}  \StringTok{"Difference vs. mean for PEFR data"}\NormalTok{)}
\end{Highlighting}
\end{Shaded}

\includegraphics{prfe_files/figure-latex/unnamed-chunk-297-1.pdf}

The function is almost complete. All that remains to do is to label the
lines with the values of the mean difference and the limits of
agreement.

Use the \texttt{fix()} function to edit the \texttt{ba.plot()} function:

\begin{Shaded}
\begin{Highlighting}[]
\KeywordTok{fix}\NormalTok{(ba.plot)}
\end{Highlighting}
\end{Shaded}

Edit the function to read:

\begin{Shaded}
\begin{Highlighting}[]
\ControlFlowTok{function}\NormalTok{(a, b, }\DataTypeTok{title =} \StringTok{"Bland and Altman Plot"}\NormalTok{) \{}
\NormalTok{  a.txt <-}\StringTok{ }\KeywordTok{deparse}\NormalTok{(}\KeywordTok{substitute}\NormalTok{(a))}
\NormalTok{  b.txt <-}\StringTok{ }\KeywordTok{deparse}\NormalTok{(}\KeywordTok{substitute}\NormalTok{(b))}
\NormalTok{  x.lab <-}\StringTok{ }\KeywordTok{paste}\NormalTok{(}\StringTok{"Mean of"}\NormalTok{, a.txt, }\StringTok{"and"}\NormalTok{, b.txt)}
\NormalTok{  y.lab <-}\StringTok{ }\KeywordTok{paste}\NormalTok{(a.txt, }\StringTok{"-"}\NormalTok{, b.txt)}
\NormalTok{  mean.two <-}\StringTok{ }\NormalTok{(a }\OperatorTok{+}\StringTok{ }\NormalTok{b) }\OperatorTok{/}\StringTok{ }\DecValTok{2}
\NormalTok{  diff.two <-}\StringTok{ }\NormalTok{a }\OperatorTok{-}\StringTok{ }\NormalTok{b}
  \KeywordTok{plot}\NormalTok{(mean.two, diff.two, }\DataTypeTok{xlab =}\NormalTok{ x.lab, }\DataTypeTok{ylab =}\NormalTok{ y.lab, }\DataTypeTok{main =}\NormalTok{ title) }
\NormalTok{  mean.diff <-}\StringTok{ }\KeywordTok{mean}\NormalTok{(diff.two)}
\NormalTok{  sd.diff <-}\StringTok{ }\KeywordTok{sd}\NormalTok{(diff.two)}
\NormalTok{  upper <-}\StringTok{ }\NormalTok{mean.diff }\OperatorTok{+}\StringTok{ }\FloatTok{1.96} \OperatorTok{*}\StringTok{ }\NormalTok{sd.diff}
\NormalTok{  lower <-}\StringTok{ }\NormalTok{mean.diff }\OperatorTok{-}\StringTok{ }\FloatTok{1.96} \OperatorTok{*}\StringTok{ }\NormalTok{sd.diff}
  \KeywordTok{lines}\NormalTok{(}\DataTypeTok{x =} \KeywordTok{range}\NormalTok{(mean.two), }\DataTypeTok{y =} \KeywordTok{c}\NormalTok{(mean.diff, mean.diff), }\DataTypeTok{lty =} \DecValTok{3}\NormalTok{) }
  \KeywordTok{lines}\NormalTok{(}\DataTypeTok{x =} \KeywordTok{range}\NormalTok{(mean.two), }\DataTypeTok{y =} \KeywordTok{c}\NormalTok{(upper, upper), }\DataTypeTok{lty =} \DecValTok{3}\NormalTok{)}
  \KeywordTok{lines}\NormalTok{(}\DataTypeTok{x =} \KeywordTok{range}\NormalTok{(mean.two), }\DataTypeTok{y =} \KeywordTok{c}\NormalTok{(lower, lower), }\DataTypeTok{lty =} \DecValTok{3}\NormalTok{)}
\NormalTok{  m.text <-}\StringTok{ }\KeywordTok{round}\NormalTok{(mean.diff, }\DataTypeTok{digits =} \DecValTok{1}\NormalTok{)}
\NormalTok{  u.text <-}\StringTok{ }\KeywordTok{round}\NormalTok{(upper , }\DataTypeTok{digits =} \DecValTok{1}\NormalTok{)}
\NormalTok{  l.text <-}\StringTok{ }\KeywordTok{round}\NormalTok{(lower, }\DataTypeTok{digits =} \DecValTok{1}\NormalTok{)}
  \KeywordTok{text}\NormalTok{(}\KeywordTok{max}\NormalTok{(mean.two), mean.diff, m.text, }\DataTypeTok{adj =} \KeywordTok{c}\NormalTok{(}\DecValTok{1}\NormalTok{,}\DecValTok{1}\NormalTok{)) }
  \KeywordTok{text}\NormalTok{(}\KeywordTok{max}\NormalTok{(mean.two), upper, u.text, }\DataTypeTok{adj =} \KeywordTok{c}\NormalTok{(}\DecValTok{1}\NormalTok{,}\DecValTok{1}\NormalTok{)) }
  \KeywordTok{text}\NormalTok{(}\KeywordTok{max}\NormalTok{(mean.two), lower, l.text, }\DataTypeTok{adj =} \KeywordTok{c}\NormalTok{(}\DecValTok{1}\NormalTok{,}\DecValTok{1}\NormalTok{))}
\NormalTok{\}}
\end{Highlighting}
\end{Shaded}

Once you have made the changes shown above, check your work, save the
file, and quit the editor. We have used the \texttt{round()} function to
limit the display of the mean difference and the limits of agreement to
one decimal place and used the \texttt{text()} function to display these
(rounded) values. The \texttt{adj} parameter to the \texttt{text()}
function controls the position and justification of text.

Let's try the \texttt{ba.plot()} function with the test data:

\begin{Shaded}
\begin{Highlighting}[]
\KeywordTok{ba.plot}\NormalTok{(Wright, Mini, }\DataTypeTok{title =} \StringTok{"PEFR data"}\NormalTok{)}
\end{Highlighting}
\end{Shaded}

\includegraphics{prfe_files/figure-latex/unnamed-chunk-301-1.pdf}

The graphical function is now complete.

One improvement that we could make is for the function to produce a
chart and return the values of the mean difference and the limits of
agreement.

We would do this in exactly the same way as we would with a
non-graphical function. We would return the mean difference and the
limits of agreement as members of a list.

We could also specify a class for the returned list and create a class
specific \texttt{print()} function (or \emph{method}) to produce nicely
formatted output.

Use the \texttt{fix()} function to edit the \texttt{ba.plot()} function:

\begin{Shaded}
\begin{Highlighting}[]
\KeywordTok{fix}\NormalTok{(ba.plot)}
\end{Highlighting}
\end{Shaded}

Edit the function to read:

\begin{Shaded}
\begin{Highlighting}[]
\ControlFlowTok{function}\NormalTok{(a, b, }\DataTypeTok{title =} \StringTok{"Bland and Altman Plot"}\NormalTok{) \{}
\NormalTok{  a.txt <-}\StringTok{ }\KeywordTok{deparse}\NormalTok{(}\KeywordTok{substitute}\NormalTok{(a))}
\NormalTok{  b.txt <-}\StringTok{ }\KeywordTok{deparse}\NormalTok{(}\KeywordTok{substitute}\NormalTok{(b))}
\NormalTok{  x.lab <-}\StringTok{ }\KeywordTok{paste}\NormalTok{(}\StringTok{"Mean of"}\NormalTok{, a.txt, }\StringTok{"and"}\NormalTok{, b.txt)}
\NormalTok{  y.lab <-}\StringTok{ }\KeywordTok{paste}\NormalTok{(a.txt, }\StringTok{"-"}\NormalTok{, b.txt)}
\NormalTok{  mean.two <-}\StringTok{ }\NormalTok{(a }\OperatorTok{+}\StringTok{ }\NormalTok{b) }\OperatorTok{/}\StringTok{ }\DecValTok{2}
\NormalTok{  diff.two <-}\StringTok{ }\NormalTok{a }\OperatorTok{-}\StringTok{ }\NormalTok{b}
  \KeywordTok{plot}\NormalTok{(mean.two, diff.two, }\DataTypeTok{xlab =}\NormalTok{ x.lab, }\DataTypeTok{ylab =}\NormalTok{ y.lab, }\DataTypeTok{main =}\NormalTok{ title) }
\NormalTok{  mean.diff <-}\StringTok{ }\KeywordTok{mean}\NormalTok{(diff.two)}
\NormalTok{  sd.diff <-}\StringTok{ }\KeywordTok{sd}\NormalTok{(diff.two)}
\NormalTok{  upper <-}\StringTok{ }\NormalTok{mean.diff }\OperatorTok{+}\StringTok{ }\FloatTok{1.96} \OperatorTok{*}\StringTok{ }\NormalTok{sd.diff}
\NormalTok{  lower <-}\StringTok{ }\NormalTok{mean.diff }\OperatorTok{-}\StringTok{ }\FloatTok{1.96} \OperatorTok{*}\StringTok{ }\NormalTok{sd.diff}
  \KeywordTok{lines}\NormalTok{(}\DataTypeTok{x =} \KeywordTok{range}\NormalTok{(mean.two), }\DataTypeTok{y =} \KeywordTok{c}\NormalTok{(mean.diff, mean.diff), }\DataTypeTok{lty =} \DecValTok{3}\NormalTok{) }
  \KeywordTok{lines}\NormalTok{(}\DataTypeTok{x =} \KeywordTok{range}\NormalTok{(mean.two), }\DataTypeTok{y =} \KeywordTok{c}\NormalTok{(upper, upper), }\DataTypeTok{lty =} \DecValTok{3}\NormalTok{)}
  \KeywordTok{lines}\NormalTok{(}\DataTypeTok{x =} \KeywordTok{range}\NormalTok{(mean.two), }\DataTypeTok{y =} \KeywordTok{c}\NormalTok{(lower, lower), }\DataTypeTok{lty =} \DecValTok{3}\NormalTok{)}
\NormalTok{  m.text <-}\StringTok{ }\KeywordTok{round}\NormalTok{(mean.diff, }\DataTypeTok{digits =} \DecValTok{1}\NormalTok{)}
\NormalTok{  u.text <-}\StringTok{ }\KeywordTok{round}\NormalTok{(upper , }\DataTypeTok{digits =} \DecValTok{1}\NormalTok{)}
\NormalTok{  l.text <-}\StringTok{ }\KeywordTok{round}\NormalTok{(lower, }\DataTypeTok{digits =} \DecValTok{1}\NormalTok{)}
  \KeywordTok{text}\NormalTok{(}\KeywordTok{max}\NormalTok{(mean.two), mean.diff, m.text, }\DataTypeTok{adj =} \KeywordTok{c}\NormalTok{(}\DecValTok{1}\NormalTok{,}\DecValTok{1}\NormalTok{)) }
  \KeywordTok{text}\NormalTok{(}\KeywordTok{max}\NormalTok{(mean.two), upper, u.text, }\DataTypeTok{adj =} \KeywordTok{c}\NormalTok{(}\DecValTok{1}\NormalTok{,}\DecValTok{1}\NormalTok{)) }
  \KeywordTok{text}\NormalTok{(}\KeywordTok{max}\NormalTok{(mean.two), lower, l.text, }\DataTypeTok{adj =} \KeywordTok{c}\NormalTok{(}\DecValTok{1}\NormalTok{,}\DecValTok{1}\NormalTok{))}
\NormalTok{  ba <-}\StringTok{ }\KeywordTok{list}\NormalTok{(}\DataTypeTok{mean =}\NormalTok{ mean.diff, }\DataTypeTok{limits =} \KeywordTok{c}\NormalTok{(lower, upper))}
  \KeywordTok{class}\NormalTok{(ba) <-}\StringTok{ "ba"}
  \KeywordTok{return}\NormalTok{(ba)}
\NormalTok{\}}
\end{Highlighting}
\end{Shaded}

Once you have made the changes shown above, save the file and quit the
editor.

Create a \texttt{print()} function for objects of the \texttt{ba} class:

\begin{Shaded}
\begin{Highlighting}[]
\NormalTok{print.ba <-}\StringTok{ }\ControlFlowTok{function}\NormalTok{(x) \{\}}
\end{Highlighting}
\end{Shaded}

Use the \texttt{fix()} function to edit the new function:

\begin{Shaded}
\begin{Highlighting}[]
\KeywordTok{fix}\NormalTok{(print.ba)}
\end{Highlighting}
\end{Shaded}

Edit the function to read:

\begin{Shaded}
\begin{Highlighting}[]
\ControlFlowTok{function}\NormalTok{(x) \{}
  \KeywordTok{cat}\NormalTok{(}\StringTok{"Mean difference     : "}\NormalTok{, x}\OperatorTok{$}\NormalTok{mean, }\StringTok{"}\CharTok{\textbackslash{}n}\StringTok{"}\NormalTok{,}
      \StringTok{"Limits of agreement : "}\NormalTok{, x}\OperatorTok{$}\NormalTok{limits[}\DecValTok{1}\NormalTok{], }\StringTok{"; "}\NormalTok{, x}\OperatorTok{$}\NormalTok{limits[}\DecValTok{2}\NormalTok{], }\StringTok{"}\CharTok{\textbackslash{}n}\StringTok{"}\NormalTok{,}
      \DataTypeTok{sep =} \StringTok{""}\NormalTok{)}
\NormalTok{\}}
\end{Highlighting}
\end{Shaded}

Once you have made the changes shown above, check your work, save the
file, and quit the editor.

Let's try the \texttt{ba.plot()} function with the test data:

\begin{Shaded}
\begin{Highlighting}[]
\KeywordTok{ba.plot}\NormalTok{(Wright, Mini, }\DataTypeTok{title =} \StringTok{"PEFR data"}\NormalTok{)}
\end{Highlighting}
\end{Shaded}

\includegraphics{prfe_files/figure-latex/unnamed-chunk-309-1.pdf}

\begin{verbatim}
## Mean difference     : -2.117647
## Limits of agreement : -78.0973; 73.86201
\end{verbatim}

The function produces the plot and returns the mean difference and
limits of agreement as a list of class \texttt{ba} which is formatted
and printed by the \texttt{print.ba()} function.

We can manipulate the returned values just as we would with any other
function:

\begin{Shaded}
\begin{Highlighting}[]
\NormalTok{ba.test <-}\StringTok{ }\KeywordTok{ba.plot}\NormalTok{(Wright, Mini)}
\KeywordTok{print}\NormalTok{(ba.test)}
\NormalTok{ba.test}
\NormalTok{ba.test}\OperatorTok{$}\NormalTok{mean}
\NormalTok{ba.test}\OperatorTok{$}\NormalTok{limits}
\NormalTok{ba.test}\OperatorTok{$}\NormalTok{limits[}\DecValTok{1}\NormalTok{]}
\NormalTok{ba.test}\OperatorTok{$}\NormalTok{limits[}\DecValTok{2}\NormalTok{]}
\end{Highlighting}
\end{Shaded}

\includegraphics{prfe_files/figure-latex/unnamed-chunk-311-1.pdf}

\begin{verbatim}
## Mean difference     : -2.117647
## Limits of agreement : -78.0973; 73.86201
\end{verbatim}

\begin{verbatim}
## Mean difference     : -2.117647
## Limits of agreement : -78.0973; 73.86201
\end{verbatim}

\begin{verbatim}
## [1] -2.117647
\end{verbatim}

\begin{verbatim}
## [1] -78.09730  73.86201
\end{verbatim}

\begin{verbatim}
## [1] -78.0973
\end{verbatim}

\begin{verbatim}
## [1] 73.86201
\end{verbatim}

You might like to use the \texttt{save()} function to save the
\texttt{ba.plot()} and \texttt{print.ba()} functions before quitting
\texttt{R}.

We can now quit \texttt{R}:

\begin{Shaded}
\begin{Highlighting}[]
\KeywordTok{q}\NormalTok{()}
\end{Highlighting}
\end{Shaded}

For this exercise there is no need to save the workspace image so click
the \textbf{No} or \textbf{Don't Save} button (GUI) or enter \texttt{n}
when prompted to save the workspace image (terminal).

\hypertarget{summary-6}{%
\section{Summary}\label{summary-6}}

\begin{itemize}
\item
  \texttt{R} allows you to create functions that produce graphical
  output.
\item
  \texttt{R} allows you to create functions that produce graphical
  output and return values.
\item
  \texttt{R} objects can be assigned a class or type.
\item
  \texttt{R} allows you to create new classes and class-specific
  functions that can extract and manipulate data common to the new
  classes.
\item
  Classes allows you to create versatile functions that return values
  when we need them to return values but can also produce formatted
  output when we need them to produce formatted output.
\item
  Classes allow you to write functions that can be chained together so
  that the output of one function is the input of another function.
\end{itemize}

\hypertarget{exercise8}{%
\chapter{More graphical functions}\label{exercise8}}

Graphical functions in \texttt{R} are just like any other function in
\texttt{R} in the sense that \texttt{R} provides you with a set of
functions which can be altered or added to. In this exercise we will
experiment with some of the graphical functions provided by R to
demonstrate the flexibility of graphical functions in \texttt{R}. We
will then use the graphical functions that we experiment with to create
some useful graphical functions of our own.

The first function that we will develop will be a function that is
capable of plotting two data series on a single graph. We will take this
exercise slowly in order to introduce some further graphical functions.
Before we go any further we should start \texttt{R} and retrieve a
dataset:

\begin{Shaded}
\begin{Highlighting}[]
\NormalTok{mal <-}\StringTok{ }\KeywordTok{read.table}\NormalTok{(}\StringTok{"malaria.dat"}\NormalTok{, }\DataTypeTok{header =} \OtherTok{TRUE}\NormalTok{)}
\KeywordTok{attach}\NormalTok{(mal)}
\end{Highlighting}
\end{Shaded}

The file \texttt{malaria.dat} contains data on rainfall (in mm) and the
number of cases of malaria reported from health centres in an
administrative district of Ethiopia between July 1997 and July 1999. The
columns in this dataset are as follows:

\begin{longtable}[]{@{}ll@{}}
\toprule
\begin{minipage}[t]{0.14\columnwidth}\raggedright
\textbf{Time}\strut
\end{minipage} & \begin{minipage}[t]{0.54\columnwidth}\raggedright
Month and year (as text)\strut
\end{minipage}\tabularnewline
\begin{minipage}[t]{0.14\columnwidth}\raggedright
\textbf{Cases}\strut
\end{minipage} & \begin{minipage}[t]{0.54\columnwidth}\raggedright
Number of cases of malaria reported\strut
\end{minipage}\tabularnewline
\begin{minipage}[t]{0.14\columnwidth}\raggedright
\textbf{Rain}\strut
\end{minipage} & \begin{minipage}[t]{0.54\columnwidth}\raggedright
Rainfall in mm\strut
\end{minipage}\tabularnewline
\bottomrule
\end{longtable}

Examine the dataset:

\begin{Shaded}
\begin{Highlighting}[]
\NormalTok{mal}
\end{Highlighting}
\end{Shaded}

\begin{verbatim}
##      Time Cases  Rain
## 1  Jul-97   997  68.5
## 2  Aug-97   824 162.1
## 3  Sep-97   573 138.8
## 4  Oct-97   586 222.2
## 5  Nov-97   523 115.5
## 6  Dec-97   968  37.2
## 7  Jan-98   985  96.6
## 8  Feb-98   745  99.1
## 9  Mar-98   520  51.2
## 10 Apr-98   406  80.0
## 11 May-98   523 112.4
## 12 Jun-98   560 183.7
## 13 Jul-98   671 101.0
## 14 Aug-98   667 252.5
## 15 Sep-98   768  40.8
## 16 Oct-98   990 193.7
## 17 Nov-98   775  17.5
## 18 Dec-98   833   0.0
## 19 Jan-99   672  33.5
## 20 Feb-99   505   0.0
## 21 Mar-99   320 106.8
## 22 Apr-99   274 117.4
## 23 May-99   263 175.8
## 24 Jun-99   264 187.5
## 25 Jul-99   179 283.5
\end{verbatim}

First we will plot the number of cases of malaria seen over time using
the \texttt{plot()} function:

\begin{Shaded}
\begin{Highlighting}[]
\KeywordTok{plot}\NormalTok{(Cases, }\DataTypeTok{type =} \StringTok{"l"}\NormalTok{)}
\end{Highlighting}
\end{Shaded}

\includegraphics{prfe_files/figure-latex/unnamed-chunk-315-1.pdf}

The problem with this plot is that it does not treat the data as a time
series. Adding the \texttt{Time} variable to the plot does not solve the
problem:

\begin{Shaded}
\begin{Highlighting}[]
\KeywordTok{plot}\NormalTok{(Time, Cases, }\DataTypeTok{type =} \StringTok{"l"}\NormalTok{)}
\end{Highlighting}
\end{Shaded}

\includegraphics{prfe_files/figure-latex/unnamed-chunk-316-1.pdf}

Because \texttt{Time} is a factor variable. If you convert \texttt{Time}
to a character variable using \texttt{as.character()} or prevent
\texttt{R} from converting \texttt{Time} to a factor using the
\texttt{as.is} parameter to the \texttt{read.table()} function the
\texttt{plot()} function will return an error because it expects a
numeric x-axis variable. We should, instead, specify a time series
(\texttt{ts}) class object. Rather than change the original data, we
will create a new object using the \texttt{ts()} function:

\begin{Shaded}
\begin{Highlighting}[]
\NormalTok{cases.ts <-}\StringTok{ }\KeywordTok{ts}\NormalTok{(Cases, }\DataTypeTok{start =} \KeywordTok{c}\NormalTok{(}\DecValTok{1997}\NormalTok{, }\DecValTok{7}\NormalTok{), }\DataTypeTok{frequency =} \DecValTok{12}\NormalTok{)}
\end{Highlighting}
\end{Shaded}

Examine the cases.ts object:

\begin{Shaded}
\begin{Highlighting}[]
\NormalTok{cases.ts}
\end{Highlighting}
\end{Shaded}

\begin{verbatim}
##      Jan Feb Mar Apr May Jun Jul Aug Sep Oct Nov Dec
## 1997                         997 824 573 586 523 968
## 1998 985 745 520 406 523 560 671 667 768 990 775 833
## 1999 672 505 320 274 263 264 179
\end{verbatim}

We can now plot \texttt{cases.ts} as a time series:

\begin{Shaded}
\begin{Highlighting}[]
\KeywordTok{plot}\NormalTok{(cases.ts)}
\end{Highlighting}
\end{Shaded}

\includegraphics{prfe_files/figure-latex/unnamed-chunk-319-1.pdf}

We might want to explore the association between the \texttt{Rain} and
\texttt{Cases} variables. A simple scatter plot is not particularly
informative:

\begin{Shaded}
\begin{Highlighting}[]
\KeywordTok{plot}\NormalTok{(Rain, Cases)}
\end{Highlighting}
\end{Shaded}

\includegraphics{prfe_files/figure-latex/unnamed-chunk-320-1.pdf}

It is better to treat both variables as time series (which they are) and
use the built-in \texttt{plot()} methods for objects of class
\texttt{ts}:

\begin{Shaded}
\begin{Highlighting}[]
\NormalTok{rain.cases.ts <-}\StringTok{ }\KeywordTok{ts}\NormalTok{(}\KeywordTok{cbind}\NormalTok{(Rain, Cases), }\DataTypeTok{start =} \KeywordTok{c}\NormalTok{(}\DecValTok{1997}\NormalTok{,}\DecValTok{7}\NormalTok{), }\DataTypeTok{frequency =} \DecValTok{12}\NormalTok{)}
\KeywordTok{plot}\NormalTok{(rain.cases.ts)}
\end{Highlighting}
\end{Shaded}

\includegraphics{prfe_files/figure-latex/unnamed-chunk-321-1.pdf}

The association between the \texttt{Rain} and \texttt{Cases} variables
is now clearer with the number of malaria cases peaking shortly after
peaks in rainfall.

The \texttt{plot()} function when used with objects of class \texttt{ts}
produces useful output but it is not particularly flexible and the
output is, sometimes, not particularly pretty. We can however use basic
graphical functions to produce multiple plots. First we will set the
\texttt{mfrow} graphical parameter using the \texttt{par()} function:

\begin{Shaded}
\begin{Highlighting}[]
\KeywordTok{par}\NormalTok{(}\DataTypeTok{mfrow =} \KeywordTok{c}\NormalTok{(}\DecValTok{2}\NormalTok{, }\DecValTok{1}\NormalTok{))}
\end{Highlighting}
\end{Shaded}

The \texttt{par()} function sets a graphical parameter. The
\texttt{mfrow} parameter is used to set the number of charts that will
appear on a page in rows and columns. We have specified two rows with
one chart per row. Test this by plotting two charts:

\begin{Shaded}
\begin{Highlighting}[]
\KeywordTok{plot}\NormalTok{(Rain, }\DataTypeTok{type =} \StringTok{"l"}\NormalTok{)}
\end{Highlighting}
\end{Shaded}

\includegraphics{prfe_files/figure-latex/unnamed-chunk-323-1.pdf}

\begin{Shaded}
\begin{Highlighting}[]
\KeywordTok{plot}\NormalTok{(Cases, }\DataTypeTok{type =} \StringTok{"l"}\NormalTok{)}
\end{Highlighting}
\end{Shaded}

\includegraphics{prfe_files/figure-latex/unnamed-chunk-323-2.pdf}

We will want to have tick-marks on the x-axis of each for each record.
We can set the number of tick-marks on axes by setting the \texttt{lab}
graphical parameter using the \texttt{par()} function:

\begin{Shaded}
\begin{Highlighting}[]
\KeywordTok{par}\NormalTok{(}\DataTypeTok{lab =} \KeywordTok{c}\NormalTok{(}\KeywordTok{length}\NormalTok{(Time), }\DecValTok{10}\NormalTok{, }\DecValTok{7}\NormalTok{))}
\end{Highlighting}
\end{Shaded}

The \texttt{par()} function sets a graphical parameter. The \texttt{lab}
parameter is used to set the number tick-marks on the x and y axes and
the label size. We have specified a tick-mark on the x-axis for each
record (i.e.~using \texttt{length(Time)}), ten tick-marks on the y-axis,
and a label length of seven. Test this by plotting two charts:

\begin{Shaded}
\begin{Highlighting}[]
\KeywordTok{plot}\NormalTok{(Rain, }\DataTypeTok{type =} \StringTok{"l"}\NormalTok{)}
\end{Highlighting}
\end{Shaded}

\includegraphics{prfe_files/figure-latex/unnamed-chunk-325-1.pdf}

\begin{Shaded}
\begin{Highlighting}[]
\KeywordTok{plot}\NormalTok{(Cases, }\DataTypeTok{type =} \StringTok{"l"}\NormalTok{)}
\end{Highlighting}
\end{Shaded}

\includegraphics{prfe_files/figure-latex/unnamed-chunk-325-2.pdf}

The problem with these charts is that the month and year are not
displayed on the x-axis. We can get round this by plotting a chart
without axes and then specifying the axes and labels directly:

\begin{Shaded}
\begin{Highlighting}[]
\KeywordTok{plot}\NormalTok{(Rain, }\DataTypeTok{type =} \StringTok{"l"}\NormalTok{, }\DataTypeTok{axes =} \OtherTok{FALSE}\NormalTok{, }\DataTypeTok{xlab =} \StringTok{"Time"}\NormalTok{, }\DataTypeTok{ylab =} \StringTok{"mm"}\NormalTok{, }\DataTypeTok{main =} \StringTok{"Rainfall"}\NormalTok{)}
\KeywordTok{axis}\NormalTok{(}\DataTypeTok{side =} \DecValTok{1}\NormalTok{, }\DataTypeTok{labels =} \KeywordTok{as.character}\NormalTok{(Time), }\DataTypeTok{at =} \DecValTok{1}\OperatorTok{:}\KeywordTok{length}\NormalTok{(Time))}
\KeywordTok{axis}\NormalTok{(}\DataTypeTok{side =} \DecValTok{2}\NormalTok{)}
\end{Highlighting}
\end{Shaded}

\includegraphics{prfe_files/figure-latex/unnamed-chunk-326-1.pdf}

\begin{Shaded}
\begin{Highlighting}[]
\KeywordTok{plot}\NormalTok{(Cases, }\DataTypeTok{type =} \StringTok{"l"}\NormalTok{, }\DataTypeTok{axes =} \OtherTok{FALSE}\NormalTok{, }\DataTypeTok{xlab =} \StringTok{"Time"}\NormalTok{, }\DataTypeTok{ylab =} \StringTok{"n"}\NormalTok{, }\DataTypeTok{main =} \StringTok{"Cases"}\NormalTok{)}
\KeywordTok{axis}\NormalTok{(}\DataTypeTok{side =} \DecValTok{1}\NormalTok{, }\DataTypeTok{labels =} \KeywordTok{as.character}\NormalTok{(Time), }\DataTypeTok{at =} \DecValTok{1}\OperatorTok{:}\KeywordTok{length}\NormalTok{(Time))}
\KeywordTok{axis}\NormalTok{(}\DataTypeTok{side =} \DecValTok{2}\NormalTok{)}
\end{Highlighting}
\end{Shaded}

\includegraphics{prfe_files/figure-latex/unnamed-chunk-326-2.pdf}

The resulting charts now look much better (you may need to resize the
plot to display the x-axis labels correctly) but it would be nice to be
able draw the two lines on a single chart.

Before proceeding we will use the \texttt{par()} function to specify one
plot per window (using the \texttt{mfrow} parameter) and set the default
number of tick-marks on the axes (using the \texttt{lab} parameter):

\begin{Shaded}
\begin{Highlighting}[]
\KeywordTok{par}\NormalTok{(}\DataTypeTok{mfrow =} \KeywordTok{c}\NormalTok{(}\DecValTok{1}\NormalTok{, }\DecValTok{1}\NormalTok{))}
\KeywordTok{par}\NormalTok{(}\DataTypeTok{lab =} \KeywordTok{c}\NormalTok{(}\DecValTok{5}\NormalTok{, }\DecValTok{5}\NormalTok{, }\DecValTok{7}\NormalTok{))}
\end{Highlighting}
\end{Shaded}

And then use the \texttt{plot()} and \texttt{lines()} function to draw
the two lines on the same graph:

\begin{Shaded}
\begin{Highlighting}[]
\KeywordTok{plot}\NormalTok{(Cases, }\DataTypeTok{type =} \StringTok{"l"}\NormalTok{)}
\KeywordTok{lines}\NormalTok{(Rain, }\DataTypeTok{lty =} \DecValTok{2}\NormalTok{)}
\end{Highlighting}
\end{Shaded}

\includegraphics{prfe_files/figure-latex/unnamed-chunk-328-1.pdf}

The problem with this is that the ranges of the two variables are
different and the \texttt{plot()} function automatically sets the y-axis
to the range of the specified variable. To fix this problem we need to
set the limits of the y-axis to the minimum and maximum value of both of
variables using the \texttt{ylim} parameter of the \texttt{plot()}
function:

\begin{Shaded}
\begin{Highlighting}[]
\KeywordTok{plot}\NormalTok{(Cases, }\DataTypeTok{type =} \StringTok{"l"}\NormalTok{, }\DataTypeTok{ylim =} \KeywordTok{c}\NormalTok{(}\KeywordTok{min}\NormalTok{(Cases, Rain), }\KeywordTok{max}\NormalTok{(Cases, Rain)))}
\KeywordTok{lines}\NormalTok{(Rain, }\DataTypeTok{lty =} \DecValTok{2}\NormalTok{)}
\end{Highlighting}
\end{Shaded}

\includegraphics{prfe_files/figure-latex/unnamed-chunk-329-1.pdf}

We can improve the chart by adding a legend:

\begin{Shaded}
\begin{Highlighting}[]
\KeywordTok{legend}\NormalTok{(}\DecValTok{18}\NormalTok{, }\DecValTok{1000}\NormalTok{, }\DataTypeTok{legend =} \KeywordTok{c}\NormalTok{(}\StringTok{"Cases"}\NormalTok{, }\StringTok{"Rainfall (mm)"}\NormalTok{), }\DataTypeTok{lty =} \KeywordTok{c}\NormalTok{(}\DecValTok{1}\NormalTok{,}\DecValTok{2}\NormalTok{))}
\end{Highlighting}
\end{Shaded}

\includegraphics{prfe_files/figure-latex/unnamed-chunk-331-1.pdf}

We could continue to improve the chart (e.g.~by adding labels for the
x-axis tick-marks taken from the \texttt{Time} variable, specifying more
meaningful axis labels, and specifying a title) but the chart would be
more useful if each variable made full use of the plotting area. We can
do this by plotting one chart on top of another by using the
\texttt{new} graphical parameter:

\begin{Shaded}
\begin{Highlighting}[]
\KeywordTok{par}\NormalTok{(}\DataTypeTok{lab =} \KeywordTok{c}\NormalTok{(}\KeywordTok{length}\NormalTok{(Time), }\DecValTok{5}\NormalTok{, }\DecValTok{7}\NormalTok{))}
\KeywordTok{plot}\NormalTok{(Cases, }\DataTypeTok{type =} \StringTok{"l"}\NormalTok{, }\DataTypeTok{lty =} \DecValTok{1}\NormalTok{, }\DataTypeTok{axes =} \OtherTok{FALSE}\NormalTok{)}
\KeywordTok{axis}\NormalTok{(}\DataTypeTok{side =} \DecValTok{2}\NormalTok{)}
\KeywordTok{par}\NormalTok{(}\DataTypeTok{new =} \OtherTok{TRUE}\NormalTok{)}
\KeywordTok{plot}\NormalTok{(Rain, }\DataTypeTok{type =} \StringTok{"l"}\NormalTok{, }\DataTypeTok{lty =} \DecValTok{2}\NormalTok{, }\DataTypeTok{axes =} \OtherTok{FALSE}\NormalTok{)}
\KeywordTok{axis}\NormalTok{(}\DataTypeTok{side =} \DecValTok{4}\NormalTok{)}
\KeywordTok{axis}\NormalTok{(}\DataTypeTok{side =} \DecValTok{1}\NormalTok{)}
\end{Highlighting}
\end{Shaded}

\includegraphics{prfe_files/figure-latex/unnamed-chunk-332-1.pdf}

This chart is much clearer but there are still some improvements that
could be made:

\begin{itemize}
\item
  The chart should have a title. We can do this using the \texttt{main}
  parameter of either of the \texttt{plot()} functions.
\item
  The y-axis labels are displayed on top of each other beside the
  left-hand y-axis. We can solve this problem by preventing the second
  \texttt{plot()} function from displaying a y-axis label (i.e.~by
  specifying an empty character string for the \texttt{ylab} parameter).
\item
  We will need to make room on the right-hand side of the chart for an
  axis label (i.e.~by setting the \texttt{mar} (margin) graphical
  parameter) and place the label there ourselves (using the
  \texttt{mtext()} function).
\item
  The x-axis should display the month and year which are held as
  character strings in the \texttt{Time} variable. We can do this using
  the labels parameter of the \texttt{axis()} function after setting the
  appropriate number of tick-marks using the \texttt{lab} graphical
  parameter.
\end{itemize}

The x-axis should be properly labelled. We can do this using the
\texttt{xlab} parameters of the \texttt{plot()} functions. An empty
string must be specified for one of the \texttt{plot()} functions in
order to prevent the default label from being displayed.

Try this now:

\begin{Shaded}
\begin{Highlighting}[]
\KeywordTok{par}\NormalTok{(}\DataTypeTok{mar =} \KeywordTok{c}\NormalTok{(}\DecValTok{5}\NormalTok{, }\DecValTok{5}\NormalTok{, }\DecValTok{4}\NormalTok{, }\DecValTok{5}\NormalTok{))}
\KeywordTok{par}\NormalTok{(}\DataTypeTok{lab =} \KeywordTok{c}\NormalTok{(}\KeywordTok{length}\NormalTok{(Time), }\DecValTok{5}\NormalTok{, }\DecValTok{7}\NormalTok{))}
\KeywordTok{plot}\NormalTok{(Cases, }\DataTypeTok{type =} \StringTok{"l"}\NormalTok{, }\DataTypeTok{lty =} \DecValTok{1}\NormalTok{, }\DataTypeTok{axes =} \OtherTok{FALSE}\NormalTok{,}
     \DataTypeTok{xlab =} \StringTok{""}\NormalTok{, }\DataTypeTok{ylab =} \StringTok{""}\NormalTok{, }\DataTypeTok{main =} \StringTok{"Malaria cases and rainfall"}\NormalTok{)}
\KeywordTok{axis}\NormalTok{(}\DataTypeTok{side =} \DecValTok{2}\NormalTok{)}
\KeywordTok{mtext}\NormalTok{(}\DataTypeTok{text =} \StringTok{"Malaria cases"}\NormalTok{, }\DataTypeTok{side =} \DecValTok{2}\NormalTok{, }\DataTypeTok{line =} \DecValTok{3}\NormalTok{)}
\KeywordTok{par}\NormalTok{(}\DataTypeTok{new =} \OtherTok{TRUE}\NormalTok{)}
\KeywordTok{plot}\NormalTok{(Rain, }\DataTypeTok{type =} \StringTok{"l"}\NormalTok{, }\DataTypeTok{lty =} \DecValTok{2}\NormalTok{, }\DataTypeTok{axes =} \OtherTok{FALSE}\NormalTok{,}
     \DataTypeTok{xlab =} \StringTok{"Month & Year"}\NormalTok{, }\DataTypeTok{ylab =} \StringTok{""}\NormalTok{)}
\KeywordTok{axis}\NormalTok{(}\DataTypeTok{side =} \DecValTok{4}\NormalTok{)}
\KeywordTok{mtext}\NormalTok{(}\DataTypeTok{text =} \StringTok{"Rainfall (mm)"}\NormalTok{, }\DataTypeTok{side =} \DecValTok{4}\NormalTok{, }\DataTypeTok{line =} \DecValTok{3}\NormalTok{)}
\KeywordTok{axis}\NormalTok{(}\DataTypeTok{side =} \DecValTok{1}\NormalTok{, }\DataTypeTok{labels =} \KeywordTok{as.character}\NormalTok{(Time), }\DataTypeTok{at =} \DecValTok{1}\OperatorTok{:}\KeywordTok{length}\NormalTok{(Time))}
\end{Highlighting}
\end{Shaded}

\includegraphics{prfe_files/figure-latex/unnamed-chunk-333-1.pdf}

Now that we know how to create a two-axis chart, we can write a function
that we will be able to use whenever we need to plot two variables on
the same chart. Create a new function called \texttt{plot2var()}:

\begin{Shaded}
\begin{Highlighting}[]
\NormalTok{plot2var <-}\StringTok{ }\ControlFlowTok{function}\NormalTok{() \{\}}
\end{Highlighting}
\end{Shaded}

This creates an empty function called \texttt{plot2var()}.

Use the \texttt{fix()} function to edit the \texttt{plot2var()}
function:

\begin{Shaded}
\begin{Highlighting}[]
\KeywordTok{fix}\NormalTok{(plot2var)}
\end{Highlighting}
\end{Shaded}

Edit the function to read:

\begin{Shaded}
\begin{Highlighting}[]
\ControlFlowTok{function}\NormalTok{(y1,}
\NormalTok{         y2,}
\NormalTok{         x.ticks,}
         \DataTypeTok{x.lab =} \KeywordTok{deparse}\NormalTok{(}\KeywordTok{substitute}\NormalTok{(x.ticks)),}
         \DataTypeTok{y1.lab =} \KeywordTok{deparse}\NormalTok{(}\KeywordTok{substitute}\NormalTok{(y1)),}
         \DataTypeTok{y2.lab =} \KeywordTok{deparse}\NormalTok{(}\KeywordTok{substitute}\NormalTok{(y2)),}
         \DataTypeTok{main =} \KeywordTok{paste}\NormalTok{(y1.lab, }\StringTok{"&"}\NormalTok{, y2.lab)) \{}
\NormalTok{  old.par.mar <-}\StringTok{ }\KeywordTok{par}\NormalTok{(}\StringTok{"mar"}\NormalTok{)}
\NormalTok{  old.par.lab <-}\StringTok{ }\KeywordTok{par}\NormalTok{(}\StringTok{"lab"}\NormalTok{)}
  \KeywordTok{par}\NormalTok{(}\DataTypeTok{mar =} \KeywordTok{c}\NormalTok{(}\DecValTok{5}\NormalTok{, }\DecValTok{5}\NormalTok{, }\DecValTok{4}\NormalTok{, }\DecValTok{5}\NormalTok{))}
  \ControlFlowTok{if}\NormalTok{(}\OperatorTok{!}\KeywordTok{missing}\NormalTok{(x.ticks)) \{}\KeywordTok{par}\NormalTok{(}\DataTypeTok{lab =} \KeywordTok{c}\NormalTok{(}\KeywordTok{length}\NormalTok{(x.ticks), }\DecValTok{5}\NormalTok{, }\DecValTok{7}\NormalTok{))\}}
  \KeywordTok{plot}\NormalTok{(y1, }\DataTypeTok{type =} \StringTok{"l"}\NormalTok{, }\DataTypeTok{lty =} \DecValTok{1}\NormalTok{, }\DataTypeTok{axes =} \OtherTok{FALSE}\NormalTok{,}
       \DataTypeTok{xlab =} \StringTok{""}\NormalTok{, }\DataTypeTok{ylab =} \StringTok{""}\NormalTok{, }\DataTypeTok{main =}\NormalTok{ main)}
  \KeywordTok{axis}\NormalTok{(}\DataTypeTok{side =} \DecValTok{2}\NormalTok{)}
  \KeywordTok{mtext}\NormalTok{(}\DataTypeTok{text =}\NormalTok{ y1.lab, }\DataTypeTok{side =} \DecValTok{2}\NormalTok{, }\DataTypeTok{line =} \DecValTok{3}\NormalTok{)}
  \KeywordTok{par}\NormalTok{(}\DataTypeTok{new =} \OtherTok{TRUE}\NormalTok{)}
  \KeywordTok{plot}\NormalTok{(y2, }\DataTypeTok{type =} \StringTok{"l"}\NormalTok{, }\DataTypeTok{lty =} \DecValTok{2}\NormalTok{, }\DataTypeTok{axes =} \OtherTok{FALSE}\NormalTok{,}
       \DataTypeTok{ylab =} \StringTok{""}\NormalTok{, }\DataTypeTok{xlab =}\NormalTok{ x.lab)}
  \KeywordTok{axis}\NormalTok{(}\DataTypeTok{side =} \DecValTok{4}\NormalTok{)}
  \KeywordTok{mtext}\NormalTok{(}\DataTypeTok{text =}\NormalTok{ y2.lab, }\DataTypeTok{side =} \DecValTok{4}\NormalTok{, }\DataTypeTok{line =} \DecValTok{3}\NormalTok{)}
  \ControlFlowTok{if}\NormalTok{(}\OperatorTok{!}\KeywordTok{missing}\NormalTok{(x.ticks)) \{}
    \KeywordTok{axis}\NormalTok{(}\DataTypeTok{side =} \DecValTok{1}\NormalTok{, }\DataTypeTok{labels =} \KeywordTok{as.character}\NormalTok{(x.ticks),}
         \DataTypeTok{at =} \DecValTok{1}\OperatorTok{:}\KeywordTok{length}\NormalTok{(x.ticks))}
\NormalTok{  \} }\ControlFlowTok{else}\NormalTok{ \{}\KeywordTok{axis}\NormalTok{(}\DataTypeTok{side =} \DecValTok{1}\NormalTok{)\}}
  \KeywordTok{par}\NormalTok{(}\DataTypeTok{mar =}\NormalTok{ old.par.mar)}
  \KeywordTok{par}\NormalTok{(}\DataTypeTok{lab =}\NormalTok{ old.par.lab)}
\NormalTok{\}}
\end{Highlighting}
\end{Shaded}

Once you have made the changes shown above, save the file and quit the
editor.

Note that with this function we have given some of the parameters
default values in the function definition and we have also used the
\texttt{if()} function to check whether the user specified a value for
the \texttt{x.ticks} parameter. We also save and restore the graphical
parameters \texttt{mar} and \texttt{lab} so as to prevent changes to
these parameters in the \texttt{plot2var()} function affecting other
graphical functions.

Let's try the \texttt{plot2var()} function with the test data:

\begin{Shaded}
\begin{Highlighting}[]
\KeywordTok{plot2var}\NormalTok{(Rain, Cases)}
\KeywordTok{plot2var}\NormalTok{(Rain, Cases, Time)}
\end{Highlighting}
\end{Shaded}

\includegraphics{prfe_files/figure-latex/unnamed-chunk-339-1.pdf}
\includegraphics{prfe_files/figure-latex/unnamed-chunk-339-2.pdf}

Note how the function has used default values for the axis labels and
chart title. We can override these default values if we want to:

\begin{Shaded}
\begin{Highlighting}[]
\KeywordTok{plot2var}\NormalTok{(Rain, Cases, Time, }\DataTypeTok{x.lab =} \StringTok{"Month and Year"}\NormalTok{,}
         \DataTypeTok{y1.lab =} \StringTok{"Rainfall (mm)"}\NormalTok{, }\DataTypeTok{y2.lab =} \StringTok{"Cases of malaria"}\NormalTok{)}
\end{Highlighting}
\end{Shaded}

\includegraphics{prfe_files/figure-latex/unnamed-chunk-340-1.pdf}

You might like to use the \texttt{save()} function to save the
\texttt{plot2var()} function.

As an exercise you might want to edit the \texttt{plot2var()} function
to automatically add a legend to the two-axis chart using the
\texttt{legend()} function with \texttt{y1.lab} and \texttt{y2.lab}.
Before continuing we should detach the \texttt{mal} data.frame:

\begin{Shaded}
\begin{Highlighting}[]
\KeywordTok{detach}\NormalTok{(mal)}
\end{Highlighting}
\end{Shaded}

\hypertarget{population-pyramid}{%
\section{Population pyramid}\label{population-pyramid}}

A common chart type that is not available in many statistical
applications and in \texttt{R} is the \emph{population pyramid}.

Before we go any further we should retrieve a dataset:

\begin{Shaded}
\begin{Highlighting}[]
\NormalTok{pop <-}\StringTok{ }\KeywordTok{read.table}\NormalTok{(}\StringTok{"pop.dat"}\NormalTok{, }\DataTypeTok{header =} \OtherTok{TRUE}\NormalTok{)}
\KeywordTok{attach}\NormalTok{(pop)}
\end{Highlighting}
\end{Shaded}

\begin{verbatim}
## The following objects are masked from fem (pos = 4):
## 
##     AGE, SEX
\end{verbatim}

\begin{verbatim}
## The following objects are masked from fem (pos = 5):
## 
##     AGE, SEX
\end{verbatim}

\begin{verbatim}
## The following objects are masked from fem (pos = 9):
## 
##     AGE, SEX
\end{verbatim}

\begin{verbatim}
## The following objects are masked from fem (pos = 10):
## 
##     AGE, SEX
\end{verbatim}

\begin{verbatim}
## The following objects are masked from fem (pos = 12):
## 
##     AGE, SEX
\end{verbatim}

\begin{verbatim}
## The following objects are masked from fem (pos = 13):
## 
##     AGE, SEX
\end{verbatim}

\begin{verbatim}
## The following objects are masked from fem (pos = 18):
## 
##     AGE, SEX
\end{verbatim}

The file \texttt{pop.dat} contains data on the age (in months) and sex
of 438 children aged between six and sixty months collected as part of a
nutritional anthropometry survey of the Khosh Valley in Northeast
Afghanistan.

The columns in this dataset are as follows:

\begin{longtable}[]{@{}ll@{}}
\toprule
\begin{minipage}[t]{0.14\columnwidth}\raggedright
\textbf{AGE}\strut
\end{minipage} & \begin{minipage}[t]{0.38\columnwidth}\raggedright
Age of the child in months\strut
\end{minipage}\tabularnewline
\begin{minipage}[t]{0.14\columnwidth}\raggedright
\textbf{SEX}\strut
\end{minipage} & \begin{minipage}[t]{0.38\columnwidth}\raggedright
Sex of the child (M/F)\strut
\end{minipage}\tabularnewline
\bottomrule
\end{longtable}

Examine the first twenty records of the dataset:

\begin{Shaded}
\begin{Highlighting}[]
\NormalTok{pop[}\DecValTok{1}\OperatorTok{:}\DecValTok{20}\NormalTok{, ]}
\end{Highlighting}
\end{Shaded}

\begin{verbatim}
##    AGE SEX
## 1    7   M
## 2   42   M
## 3   60   M
## 4   60   F
## 5   48   M
## 6   60   F
## 7   18   M
## 8   48   M
## 9   60   F
## 10  36   M
## 11  24   F
## 12  60   M
## 13  60   M
## 14  48   F
## 15  18   M
## 16  60   M
## 17   6   M
## 18   7   M
## 19  12   M
## 20  60   M
\end{verbatim}

The first step is to make groups from the \texttt{AGE} variable since
many ages are biased towards full years:

\begin{Shaded}
\begin{Highlighting}[]
\KeywordTok{table}\NormalTok{(AGE)}
\KeywordTok{barplot}\NormalTok{(}\KeywordTok{table}\NormalTok{(AGE), }\DataTypeTok{col =} \StringTok{"white"}\NormalTok{)}
\end{Highlighting}
\end{Shaded}

\begin{verbatim}
## AGE
##  6  7  8  9 10 12 13 14 15 17 18 22 23 24 25 26 30 34 36 38 40 42 48 54 60 
##  7  5  8 15  3 21  1  3  5  1 45  1  1 48  1  2 24  1 80  2  1  9 67  4 83
\end{verbatim}

\includegraphics{prfe_files/figure-latex/unnamed-chunk-345-1.pdf}

So we will centre the age-groups around the months representing full
years:

\begin{Shaded}
\begin{Highlighting}[]
\NormalTok{age.group <-}\StringTok{ }\KeywordTok{cut}\NormalTok{(AGE, }\KeywordTok{c}\NormalTok{(}\DecValTok{0}\NormalTok{, }\DecValTok{17}\NormalTok{, }\DecValTok{29}\NormalTok{, }\DecValTok{41}\NormalTok{, }\DecValTok{53}\NormalTok{, }\DecValTok{99}\NormalTok{))}
\end{Highlighting}
\end{Shaded}

We can check that the grouping operation has worked as expected by
tabulating \texttt{AGE} and \texttt{age.group}:

\begin{Shaded}
\begin{Highlighting}[]
\KeywordTok{table}\NormalTok{(AGE, age.group)}
\end{Highlighting}
\end{Shaded}

\begin{verbatim}
##     age.group
## AGE  (0,17] (17,29] (29,41] (41,53] (53,99]
##   6       7       0       0       0       0
##   7       5       0       0       0       0
##   8       8       0       0       0       0
##   9      15       0       0       0       0
##   10      3       0       0       0       0
##   12     21       0       0       0       0
##   13      1       0       0       0       0
##   14      3       0       0       0       0
##   15      5       0       0       0       0
##   17      1       0       0       0       0
##   18      0      45       0       0       0
##   22      0       1       0       0       0
##   23      0       1       0       0       0
##   24      0      48       0       0       0
##   25      0       1       0       0       0
##   26      0       2       0       0       0
##   30      0       0      24       0       0
##   34      0       0       1       0       0
##   36      0       0      80       0       0
##   38      0       0       2       0       0
##   40      0       0       1       0       0
##   42      0       0       0       9       0
##   48      0       0       0      67       0
##   54      0       0       0       0       4
##   60      0       0       0       0      83
\end{verbatim}

We now use the \texttt{table()} function to produce the summary data for
the population pyramid:

\begin{Shaded}
\begin{Highlighting}[]
\KeywordTok{table}\NormalTok{(age.group, SEX)}
\end{Highlighting}
\end{Shaded}

\begin{verbatim}
##          SEX
## age.group  F  M
##   (0,17]  34 35
##   (17,29] 54 44
##   (29,41] 49 59
##   (41,53] 39 37
##   (53,99] 41 46
\end{verbatim}

We will construct our population pyramid using the \texttt{barplot()}
function:

\begin{Shaded}
\begin{Highlighting}[]
\KeywordTok{barplot}\NormalTok{(}\KeywordTok{table}\NormalTok{(age.group, SEX))}
\end{Highlighting}
\end{Shaded}

\includegraphics{prfe_files/figure-latex/unnamed-chunk-349-1.pdf}

The default behaviour of the \texttt{barplot()} function is to produce
stacked bars. We can set the \texttt{beside} parameter to display the
bars side-by-side:

\begin{Shaded}
\begin{Highlighting}[]
\KeywordTok{barplot}\NormalTok{(}\KeywordTok{table}\NormalTok{(age.group, SEX), }\DataTypeTok{beside =} \OtherTok{TRUE}\NormalTok{)}
\end{Highlighting}
\end{Shaded}

\includegraphics{prfe_files/figure-latex/unnamed-chunk-350-1.pdf}

We can also use the \texttt{horiz} parameter to present the data as
horizontal bars:

\begin{Shaded}
\begin{Highlighting}[]
\KeywordTok{barplot}\NormalTok{(}\KeywordTok{table}\NormalTok{(age.group, SEX), }\DataTypeTok{beside =} \OtherTok{TRUE}\NormalTok{, }\DataTypeTok{horiz =} \OtherTok{TRUE}\NormalTok{)}
\end{Highlighting}
\end{Shaded}

\includegraphics{prfe_files/figure-latex/unnamed-chunk-351-1.pdf}

In order to centre the bars around zero we need to make one column of
the summary data table contain negative numbers:

\begin{Shaded}
\begin{Highlighting}[]
\NormalTok{tab <-}\StringTok{ }\KeywordTok{table}\NormalTok{(age.group, SEX)}
\NormalTok{tab}
\end{Highlighting}
\end{Shaded}

\begin{verbatim}
##          SEX
## age.group  F  M
##   (0,17]  34 35
##   (17,29] 54 44
##   (29,41] 49 59
##   (41,53] 39 37
##   (53,99] 41 46
\end{verbatim}

\begin{Shaded}
\begin{Highlighting}[]
\NormalTok{tab[ ,}\DecValTok{1}\NormalTok{] <-}\StringTok{ }\OperatorTok{-}\NormalTok{tab[ ,}\DecValTok{1}\NormalTok{]}
\NormalTok{tab}
\end{Highlighting}
\end{Shaded}

\begin{verbatim}
##          SEX
## age.group   F   M
##   (0,17]  -34  35
##   (17,29] -54  44
##   (29,41] -49  59
##   (41,53] -39  37
##   (53,99] -41  46
\end{verbatim}

\begin{Shaded}
\begin{Highlighting}[]
\KeywordTok{barplot}\NormalTok{(tab, }\DataTypeTok{beside =} \OtherTok{TRUE}\NormalTok{, }\DataTypeTok{horiz =} \OtherTok{TRUE}\NormalTok{)}
\end{Highlighting}
\end{Shaded}

\includegraphics{prfe_files/figure-latex/unnamed-chunk-352-1.pdf}

This is looking better.

We still need to shift the second set of bars down beside the first set
of bars using thespace parameter:

\begin{Shaded}
\begin{Highlighting}[]
\KeywordTok{barplot}\NormalTok{(tab, }\DataTypeTok{beside =} \OtherTok{TRUE}\NormalTok{, }\DataTypeTok{horiz =} \OtherTok{TRUE}\NormalTok{, }\DataTypeTok{space =} \KeywordTok{c}\NormalTok{(}\DecValTok{0}\NormalTok{, }\OperatorTok{-}\KeywordTok{nrow}\NormalTok{(tab)))}
\end{Highlighting}
\end{Shaded}

\includegraphics{prfe_files/figure-latex/unnamed-chunk-353-1.pdf}

The axis labels are wrong but we can fix that using the
\texttt{names.arg} parameter:

\begin{Shaded}
\begin{Highlighting}[]
\NormalTok{bar.names <-}\StringTok{ }\KeywordTok{c}\NormalTok{(}\KeywordTok{dimnames}\NormalTok{(tab)}\OperatorTok{$}\NormalTok{age.group, }\KeywordTok{dimnames}\NormalTok{(tab)}\OperatorTok{$}\NormalTok{age.group)}
\KeywordTok{barplot}\NormalTok{(tab, }\DataTypeTok{beside =} \OtherTok{TRUE}\NormalTok{, }\DataTypeTok{horiz =} \OtherTok{TRUE}\NormalTok{, }\DataTypeTok{space =} \KeywordTok{c}\NormalTok{(}\DecValTok{0}\NormalTok{, }\OperatorTok{-}\KeywordTok{nrow}\NormalTok{(tab)),}
        \DataTypeTok{names.arg =}\NormalTok{ bar.names)}
\end{Highlighting}
\end{Shaded}

\includegraphics{prfe_files/figure-latex/unnamed-chunk-354-1.pdf}

The chart can still be improved upon by making the fill-colour of each
bar white and by expanding the x-axis slightly:

\begin{Shaded}
\begin{Highlighting}[]
\KeywordTok{barplot}\NormalTok{(tab, }\DataTypeTok{beside =} \OtherTok{TRUE}\NormalTok{, }\DataTypeTok{horiz =} \OtherTok{TRUE}\NormalTok{, }\DataTypeTok{space =} \KeywordTok{c}\NormalTok{(}\DecValTok{0}\NormalTok{, }\OperatorTok{-}\KeywordTok{nrow}\NormalTok{(tab)),}
        \DataTypeTok{col =} \StringTok{"white"}\NormalTok{, }\DataTypeTok{xlim =} \KeywordTok{c}\NormalTok{(}\KeywordTok{min}\NormalTok{(tab) }\OperatorTok{*}\StringTok{ }\FloatTok{1.2}\NormalTok{, }\KeywordTok{max}\NormalTok{(tab) }\OperatorTok{*}\StringTok{ }\FloatTok{1.2}\NormalTok{),}
        \DataTypeTok{names.arg =}\NormalTok{ bar.names)}
\end{Highlighting}
\end{Shaded}

\includegraphics{prfe_files/figure-latex/unnamed-chunk-355-1.pdf}

The chart would be better if the x-axis displayed only positive numbers:

\begin{Shaded}
\begin{Highlighting}[]
\KeywordTok{barplot}\NormalTok{(tab, }\DataTypeTok{beside =} \OtherTok{TRUE}\NormalTok{, }\DataTypeTok{horiz =} \OtherTok{TRUE}\NormalTok{, }\DataTypeTok{space =} \KeywordTok{c}\NormalTok{(}\DecValTok{0}\NormalTok{, }\OperatorTok{-}\KeywordTok{nrow}\NormalTok{(tab)),}
        \DataTypeTok{col =} \StringTok{"white"}\NormalTok{, }\DataTypeTok{xlim =} \KeywordTok{c}\NormalTok{(}\KeywordTok{min}\NormalTok{(tab) }\OperatorTok{*}\StringTok{ }\FloatTok{1.2}\NormalTok{, }\KeywordTok{max}\NormalTok{(tab) }\OperatorTok{*}\StringTok{ }\FloatTok{1.2}\NormalTok{),}
        \DataTypeTok{names.arg =}\NormalTok{ bar.names, }\DataTypeTok{axes =} \OtherTok{FALSE}\NormalTok{)}
\KeywordTok{axis}\NormalTok{(}\DataTypeTok{side =} \DecValTok{1}\NormalTok{,}
     \DataTypeTok{labels =} \KeywordTok{abs}\NormalTok{(}\KeywordTok{axTicks}\NormalTok{(}\DataTypeTok{side =} \DecValTok{1}\NormalTok{)),}
     \DataTypeTok{at =}\NormalTok{ (}\KeywordTok{axTicks}\NormalTok{(}\DataTypeTok{side =} \DecValTok{1}\NormalTok{)))}
\end{Highlighting}
\end{Shaded}

\includegraphics{prfe_files/figure-latex/unnamed-chunk-356-1.pdf}

Now we know how to create a population pyramid, we can write a function
that we will be able to use whenever we need to plot a population
pyramid.

Create a new function called \texttt{pyramid.plot()}:

\begin{Shaded}
\begin{Highlighting}[]
\NormalTok{pyramid.plot <-}\StringTok{ }\ControlFlowTok{function}\NormalTok{() \{\}}
\end{Highlighting}
\end{Shaded}

This creates an empty function called \texttt{pyramid.plot()}.

Use the \texttt{fix()} function to edit the \texttt{pyramid.plot()}
function:

\begin{Shaded}
\begin{Highlighting}[]
\KeywordTok{fix}\NormalTok{(pyramid.plot)}
\end{Highlighting}
\end{Shaded}

Edit the function to read:

\begin{Shaded}
\begin{Highlighting}[]
\ControlFlowTok{function}\NormalTok{(x,}
\NormalTok{         g,}
         \DataTypeTok{main =} \KeywordTok{paste}\NormalTok{(}\StringTok{"Pyramid plot of"}\NormalTok{, }\KeywordTok{deparse}\NormalTok{(}\KeywordTok{substitute}\NormalTok{(x)),}
                      \StringTok{"by"}\NormalTok{, }\KeywordTok{deparse}\NormalTok{(}\KeywordTok{substitute}\NormalTok{(g))),}
         \DataTypeTok{xlab =} \KeywordTok{paste}\NormalTok{(}\KeywordTok{deparse}\NormalTok{(}\KeywordTok{substitute}\NormalTok{(g)),}
                      \StringTok{"("}\NormalTok{, }\KeywordTok{levels}\NormalTok{(g)[}\DecValTok{1}\NormalTok{], }\StringTok{"/"}\NormalTok{,}\KeywordTok{levels}\NormalTok{(g)[}\DecValTok{2}\NormalTok{],}\StringTok{")"}\NormalTok{),}
         \DataTypeTok{ylab =} \KeywordTok{deparse}\NormalTok{(}\KeywordTok{substitute}\NormalTok{(x))) \{}
\NormalTok{  tab <-}\StringTok{ }\KeywordTok{table}\NormalTok{(x, g)}
\NormalTok{  tab[ ,}\DecValTok{1}\NormalTok{] <-}\StringTok{ }\OperatorTok{-}\NormalTok{tab[ ,}\DecValTok{1}\NormalTok{]}
  \KeywordTok{barplot}\NormalTok{(tab,}
          \DataTypeTok{horiz =} \OtherTok{TRUE}\NormalTok{,}
          \DataTypeTok{beside =} \OtherTok{TRUE}\NormalTok{,}
          \DataTypeTok{space =} \KeywordTok{c}\NormalTok{(}\DecValTok{0}\NormalTok{, }\OperatorTok{-}\KeywordTok{nrow}\NormalTok{(tab)),}
          \DataTypeTok{names.arg =} \KeywordTok{c}\NormalTok{(}\KeywordTok{dimnames}\NormalTok{(tab)}\OperatorTok{$}\NormalTok{x, }\KeywordTok{dimnames}\NormalTok{(tab)}\OperatorTok{$}\NormalTok{x),}
          \DataTypeTok{xlim =} \KeywordTok{c}\NormalTok{(}\KeywordTok{min}\NormalTok{(tab) }\OperatorTok{*}\StringTok{ }\FloatTok{1.2}\NormalTok{,}
          \KeywordTok{max}\NormalTok{(tab) }\OperatorTok{*}\StringTok{ }\FloatTok{1.2}\NormalTok{),}
          \DataTypeTok{col =} \StringTok{"white"}\NormalTok{,}
          \DataTypeTok{main =}\NormalTok{ main,}
          \DataTypeTok{xlab =}\NormalTok{ xlab,}
          \DataTypeTok{ylab =}\NormalTok{ ylab,}
          \DataTypeTok{axes =} \OtherTok{FALSE}\NormalTok{)}
  \KeywordTok{axis}\NormalTok{(}\DataTypeTok{side =} \DecValTok{1}\NormalTok{,}
       \DataTypeTok{labels =} \KeywordTok{abs}\NormalTok{(}\KeywordTok{axTicks}\NormalTok{(}\DataTypeTok{side =} \DecValTok{1}\NormalTok{)),}
       \DataTypeTok{at =}\NormalTok{ (}\KeywordTok{axTicks}\NormalTok{(}\DataTypeTok{side =} \DecValTok{1}\NormalTok{)))}
\NormalTok{\}}
\end{Highlighting}
\end{Shaded}

Note that with this function we have given some of the parameters
default values in the function definition.

Giving default values to parameters is useful because it means that you
do not need to specify parameters such as titles and axis labels unless
you want to. Many \texttt{R} functions use default parameters which are
usually set to the most frequently used values.

Once you have made the changes shown above, check your work, save the
file, and quit the editor. Let's try the \texttt{pyramid.plot()}
function with the test data:

\begin{Shaded}
\begin{Highlighting}[]
\KeywordTok{pyramid.plot}\NormalTok{(age.group, SEX)}
\end{Highlighting}
\end{Shaded}

\includegraphics{prfe_files/figure-latex/unnamed-chunk-361-1.pdf}

Note how the function has used default values for the axis labels and
chart titles. We can override these default values if we want to:

\begin{Shaded}
\begin{Highlighting}[]
\KeywordTok{pyramid.plot}\NormalTok{(age.group, SEX, }\DataTypeTok{ylab =} \StringTok{"Months"}\NormalTok{, }\DataTypeTok{xlab =} \StringTok{"Sex F / M"}\NormalTok{,}
             \DataTypeTok{main =} \StringTok{"Children by age and sex"}\NormalTok{)}
\end{Highlighting}
\end{Shaded}

\includegraphics{prfe_files/figure-latex/unnamed-chunk-362-1.pdf}

You might like to use the \texttt{save()} function to save the
\texttt{pyramid.plot()} function.

\hypertarget{pareto-chart}{%
\section{Pareto chart}\label{pareto-chart}}

Another type of chart that is missing from many statistical applications
is the \emph{Pareto} chart which is a bar chart where the bars are
sorted by the bar value with the largest bar drawn first. Such a chart
is easier to interpret than a pie chart, particularly when there are
more than a few categories being plotted.

Before we go any further we should detach the \texttt{pop} data.frame
and retrieve a new dataset:

\begin{Shaded}
\begin{Highlighting}[]
\KeywordTok{detach}\NormalTok{(pop)}
\NormalTok{sssw <-}\StringTok{ }\KeywordTok{read.table}\NormalTok{(}\StringTok{"sssw.dat"}\NormalTok{, }\DataTypeTok{header =} \OtherTok{TRUE}\NormalTok{)}
\KeywordTok{attach}\NormalTok{(sssw)}
\end{Highlighting}
\end{Shaded}

The file \texttt{sssw.dat} contains data on the marital status, home
circumstances, and ethnic group of 152 persons recruited into a study
into the levels of stress experienced by student social workers in the
United Kingdom. The columns in this dataset are as follows:

+---------------+----------------------------------------+ \textbar{}
\textbf{marital} \textbar{} Marital status coded as: \textbar{}
\textbar{} \textbar{} \textbar{} \textbar{} \textbar{} 1 = Married
\textbar{} \textbar{} \textbar{} \textbar{} \textbar{} \textbar{} 2 =
Single \textbar{} \textbar{} \textbar{} \textbar{} \textbar{} \textbar{}
3 = Divorced \textbar{} \textbar{} \textbar{} \textbar{} \textbar{}
\textbar{} 4 = Separated \textbar{} \textbar{} \textbar{} \textbar{}
\textbar{} \textbar{} 5 = Cohabiting \textbar{} \textbar{} \textbar{}
\textbar{} \textbar{} \textbar{} 6 = Widowed \textbar{}
+---------------+----------------------------------------+
+---------------+----------------------------------------+ \textbar{}
\textbf{living} \textbar{} Living with \ldots{} coded as: \textbar{}
\textbar{} \textbar{} \textbar{} \textbar{} \textbar{} 1 = Alone
\textbar{} \textbar{} \textbar{} \textbar{} \textbar{} \textbar{} 2 =
Parents or siblings \textbar{} \textbar{} \textbar{} \textbar{}
\textbar{} \textbar{} 3 = Partner \textbar{} \textbar{} \textbar{}
\textbar{} \textbar{} \textbar{} 4 = Partner and children \textbar{}
\textbar{} \textbar{} \textbar{} \textbar{} \textbar{} 5 = Children
\textbar{} \textbar{} \textbar{} \textbar{} \textbar{} \textbar{} 6 =
Friends or colleagues \textbar{}
+---------------+----------------------------------------+
+---------------+----------------------------------------+ \textbar{}
\textbf{ethnic} \textbar{} Ethnic group coded as: \textbar{} \textbar{}
\textbar{} \textbar{} \textbar{} \textbar{} 1 = African \textbar{}
\textbar{} \textbar{} \textbar{} \textbar{} \textbar{} 2 = West-Indian
\textbar{} \textbar{} \textbar{} \textbar{} \textbar{} \textbar{} 3 =
Indian \textbar{} \textbar{} \textbar{} \textbar{} \textbar{} \textbar{}
4 = Pakistani \textbar{} \textbar{} \textbar{} \textbar{} \textbar{}
\textbar{} 5 = Bangladeshi \textbar{} \textbar{} \textbar{} \textbar{}
\textbar{} \textbar{} 6 = East African Asian \textbar{} \textbar{}
\textbar{} \textbar{} \textbar{} \textbar{} 7 = Chinese \textbar{}
\textbar{} \textbar{} \textbar{} \textbar{} \textbar{} 8 = Cypriot
\textbar{} \textbar{} \textbar{} \textbar{} \textbar{} \textbar{} 9 =
Black European \textbar{} \textbar{} \textbar{} \textbar{} \textbar{}
\textbar{} 10 = White European \textbar{} \textbar{} \textbar{}
\textbar{} \textbar{} \textbar{} 11 = Other \textbar{}
+---------------+----------------------------------------+

Examine the dataset:

\begin{Shaded}
\begin{Highlighting}[]
\NormalTok{sssw[}\DecValTok{1}\OperatorTok{:}\DecValTok{20}\NormalTok{, ]}
\end{Highlighting}
\end{Shaded}

\begin{verbatim}
##    marital living ethnic
## 1        2      5      2
## 2        1      4      2
## 3        1      1     11
## 4        1      4     10
## 5        2      3     10
## 6        2      6     10
## 7        1      4     10
## 8        1      4     10
## 9        2      2      4
## 10       2      6     10
## 11       2      5     10
## 12       1      3      1
## 13       5      3     10
## 14       1      4     10
## 15       2      6      3
## 16       1      4     10
## 17       1      3      3
## 18       5      3     10
## 19       3      5     10
## 20       5      3     10
\end{verbatim}

Producing a bar chart from this data is simple as long as we remember to
pass summary data (i.e.~created using the \texttt{table()} function) to
the \texttt{barplot()} function instead of the variable name:

\begin{Shaded}
\begin{Highlighting}[]
\KeywordTok{barplot}\NormalTok{(}\KeywordTok{table}\NormalTok{(marital))}
\KeywordTok{barplot}\NormalTok{(}\KeywordTok{table}\NormalTok{(living))}
\KeywordTok{barplot}\NormalTok{(}\KeywordTok{table}\NormalTok{(ethnic))}
\end{Highlighting}
\end{Shaded}

\includegraphics{prfe_files/figure-latex/unnamed-chunk-366-1.pdf}
\includegraphics{prfe_files/figure-latex/unnamed-chunk-366-2.pdf}
\includegraphics{prfe_files/figure-latex/unnamed-chunk-366-3.pdf}

Creating a Pareto chart only requires us to sort the summary data. We do
this using the \texttt{rev()} and \texttt{sort()} functions:

\begin{Shaded}
\begin{Highlighting}[]
\KeywordTok{barplot}\NormalTok{(}\KeywordTok{rev}\NormalTok{(}\KeywordTok{sort}\NormalTok{(}\KeywordTok{table}\NormalTok{(marital))))}
\end{Highlighting}
\end{Shaded}

\includegraphics{prfe_files/figure-latex/unnamed-chunk-367-1.pdf}

Having to specify \texttt{rev(sort(table(variable)))} each time we want
to produce a \emph{Pareto} plot is rather tedious but now that we know
how to create a \emph{Pareto} chart, we can write a function that we
will be able to use whenever we need to plot a Pareto chart. Create a
new function called \texttt{pareto()}:

\begin{Shaded}
\begin{Highlighting}[]
\NormalTok{pareto <-}\StringTok{ }\ControlFlowTok{function}\NormalTok{() \{\}}
\end{Highlighting}
\end{Shaded}

This creates an empty function called \texttt{pareto()}. Use the
\texttt{fix()} function to edit the \texttt{pareto()} function:

\begin{Shaded}
\begin{Highlighting}[]
\KeywordTok{fix}\NormalTok{(pareto)}
\end{Highlighting}
\end{Shaded}

Edit the function to read:

\begin{Shaded}
\begin{Highlighting}[]
\ControlFlowTok{function}\NormalTok{(x,}
         \DataTypeTok{xlab =} \KeywordTok{deparse}\NormalTok{(}\KeywordTok{substitute}\NormalTok{(x)),}
         \DataTypeTok{ylab =} \StringTok{"Count"}\NormalTok{,}
         \DataTypeTok{main =} \KeywordTok{paste}\NormalTok{(}\StringTok{"Pareto Chart of"}\NormalTok{, }\KeywordTok{deparse}\NormalTok{(}\KeywordTok{substitute}\NormalTok{(x)))) \{}
  \KeywordTok{barplot}\NormalTok{(}\KeywordTok{rev}\NormalTok{(}\KeywordTok{sort}\NormalTok{(}\KeywordTok{table}\NormalTok{(x))),}
          \DataTypeTok{xlab =}\NormalTok{ xlab,}
          \DataTypeTok{ylab =}\NormalTok{ ylab,}
          \DataTypeTok{main =}\NormalTok{ main,}
          \DataTypeTok{col =} \StringTok{"white"}\NormalTok{)}
\NormalTok{\}}
\end{Highlighting}
\end{Shaded}

Once you have made the changes shown above, check your work, save the
file, and quit the editor.

Let's try the \texttt{pareto()} function with the test data:

\begin{Shaded}
\begin{Highlighting}[]
\KeywordTok{pareto}\NormalTok{(marital)}
\end{Highlighting}
\end{Shaded}

\includegraphics{prfe_files/figure-latex/unnamed-chunk-372-1.pdf}

Note how the function has used default values for the axis labels and
chart titles. We can override these default values if we want to:

\begin{Shaded}
\begin{Highlighting}[]
\KeywordTok{pareto}\NormalTok{(marital, }\DataTypeTok{ylab =} \StringTok{"n"}\NormalTok{, }\DataTypeTok{xlab =} \StringTok{"Marital Status"}\NormalTok{,}
       \DataTypeTok{main =} \StringTok{"Marital Status"}\NormalTok{)}
\end{Highlighting}
\end{Shaded}

\includegraphics{prfe_files/figure-latex/unnamed-chunk-373-1.pdf}

Note that we can use value labels if the variable we plot is a factor
with value labels as levels rather than a simple numeric vector:

\begin{Shaded}
\begin{Highlighting}[]
\NormalTok{ms <-}\StringTok{ }\KeywordTok{as.factor}\NormalTok{(marital)}
\KeywordTok{levels}\NormalTok{(ms) <-}\StringTok{ }\KeywordTok{c}\NormalTok{(}\StringTok{"Married"}\NormalTok{, }\StringTok{"Single"}\NormalTok{, }\StringTok{"Divorced"}\NormalTok{, }\StringTok{"Separated"}\NormalTok{,}
                \StringTok{"Cohabiting"}\NormalTok{, }\StringTok{"Widowed"}\NormalTok{)}
\KeywordTok{table}\NormalTok{(ms)}
\KeywordTok{pareto}\NormalTok{(ms, }\DataTypeTok{ylab =} \StringTok{"n"}\NormalTok{, }\DataTypeTok{xlab =} \StringTok{"Marital Status"}\NormalTok{,}
       \DataTypeTok{main =} \StringTok{"Marital Status"}\NormalTok{)}
\end{Highlighting}
\end{Shaded}

\begin{verbatim}
## ms
##    Married     Single   Divorced  Separated Cohabiting    Widowed 
##         32         90          5          4         21          0
\end{verbatim}

\includegraphics{prfe_files/figure-latex/unnamed-chunk-375-1.pdf}

You may need to resize the plot to display the x-axis labels correctly.

You might like to use the \texttt{save()} function to save the
\texttt{pareto()} function.

\hypertarget{adding-confidence-intervals-or-error-bars-on-plots}{%
\section{Adding confidence intervals or error bars on
plots}\label{adding-confidence-intervals-or-error-bars-on-plots}}

You may want to plot your data with confidence intervals or error bars.
\texttt{R} does not have a function to do this but it is a relatively
simple matter to write a function to do so. On the way we will use some
of \texttt{R}s data management functions as well.

Before we go any further we should detach the \texttt{sssw} data.frame
and retrieve a new dataset:

\begin{Shaded}
\begin{Highlighting}[]
\KeywordTok{detach}\NormalTok{(sssw)}
\NormalTok{diets <-}\StringTok{ }\KeywordTok{read.table}\NormalTok{(}\StringTok{"diets.dat"}\NormalTok{, }\DataTypeTok{header =} \OtherTok{TRUE}\NormalTok{)}
\end{Highlighting}
\end{Shaded}

The file \texttt{diets.dat} contains data from a trial of two different
diets undertaken at an adult therapeutic feeding centre in Somalia. The
columns in this dataset are as follows:

\begin{longtable}[]{@{}ll@{}}
\toprule
\begin{minipage}[t]{0.16\columnwidth}\raggedright
\textbf{day}\strut
\end{minipage} & \begin{minipage}[t]{0.79\columnwidth}\raggedright
The day after start of diet that measurements were taken\strut
\end{minipage}\tabularnewline
\begin{minipage}[t]{0.16\columnwidth}\raggedright
\textbf{oedema}\strut
\end{minipage} & \begin{minipage}[t]{0.79\columnwidth}\raggedright
Type of undernutrition coded as: 1 = Oedematous; 2 = Marasmic\strut
\end{minipage}\tabularnewline
\begin{minipage}[t]{0.16\columnwidth}\raggedright
\textbf{diet}\strut
\end{minipage} & \begin{minipage}[t]{0.79\columnwidth}\raggedright
The trial diets coded as: LP = Low protein; HP = High protein\strut
\end{minipage}\tabularnewline
\begin{minipage}[t]{0.16\columnwidth}\raggedright
\textbf{wt}\strut
\end{minipage} & \begin{minipage}[t]{0.79\columnwidth}\raggedright
Mean weight change weight velocity) in g/kg/day since the previous
measurement\strut
\end{minipage}\tabularnewline
\begin{minipage}[t]{0.16\columnwidth}\raggedright
\textbf{sd}\strut
\end{minipage} & \begin{minipage}[t]{0.79\columnwidth}\raggedright
Standard deviation of weight change in g/kg/day\strut
\end{minipage}\tabularnewline
\begin{minipage}[t]{0.16\columnwidth}\raggedright
\textbf{n}\strut
\end{minipage} & \begin{minipage}[t]{0.79\columnwidth}\raggedright
Number of subjects at each observation\strut
\end{minipage}\tabularnewline
\bottomrule
\end{longtable}

Examine the dataset:

\begin{Shaded}
\begin{Highlighting}[]
\NormalTok{diets}
\end{Highlighting}
\end{Shaded}

\begin{verbatim}
##    day oedema diet   wt   sd   n
## 1    3      1   HP  1.1 11.5  37
## 2    6      1   HP  1.0 11.0  37
## 3    9      1   HP  1.2  8.4  37
## 4   12      1   HP -1.0  7.4  37
## 5   15      1   HP -0.9  3.2  37
## 6   18      1   HP -1.7  5.8  37
## 7   21      1   HP -1.7  4.8  37
## 8   24      1   HP -1.7  4.8  37
## 9   27      1   HP -1.7  5.3  37
## 10  30      1   HP -2.5  6.8  37
## 11  33      1   HP -1.3  8.4  37
## 12   3      2   HP  5.1 15.1 291
## 13   6      2   HP  5.1 13.4 291
## 14   9      2   HP  4.5  6.5 291
## 15  12      2   HP  5.2  9.9 291
## 16  15      2   HP  4.6  8.2 291
## 17  18      2   HP  5.1 13.4 291
## 18  21      2   HP  5.0 13.4 291
## 19  24      2   HP  5.2 15.1 291
## 20  27      2   HP  5.1 11.6 291
## 21  30      2   HP  5.0  9.2 291
## 22  33      2   HP  5.2  8.2 291
## 23   3      1   LP -3.0 15.7  65
## 24   6      1   LP -2.5 11.9  65
## 25   9      1   LP -2.0 13.7  65
## 26  12      1   LP  2.0 11.0  65
## 27  15      1   LP  2.1 13.2  65
## 28  18      1   LP  1.7 16.2  65
## 29  21      1   LP  4.0 11.0  65
## 30  24      1   LP  6.5 14.7  65
## 31  27      1   LP  6.4 10.2  65
## 32  30      1   LP  6.6  8.0  65
## 33  33      1   LP  6.5 11.0  65
## 34   3      2   LP  5.1 11.1  86
## 35   6      2   LP  6.0  8.2  86
## 36   9      2   LP  5.1  8.2  86
## 37  12      2   LP  6.5 11.9  86
## 38  15      2   LP  6.4  9.1  86
## 39  18      2   LP  5.9 13.8  86
## 40  21      2   LP  6.1 16.5  86
## 41  24      2   LP  4.0  9.1  86
## 42  27      2   LP  3.1 12.8  86
## 43  30      2   LP  4.0  7.3  86
## 44  33      2   LP  5.0  9.1  86
\end{verbatim}

Note that the dataset contains a summary of the results from the four
arms of the trial:

\begin{longtable}[]{@{}ccc@{}}
\toprule
\begin{minipage}[b]{0.14\columnwidth}\centering
\textbf{Arm}\strut
\end{minipage} & \begin{minipage}[b]{0.20\columnwidth}\centering
\textbf{Oedema}\strut
\end{minipage} & \begin{minipage}[b]{0.33\columnwidth}\centering
\textbf{Therapeutic diet}\strut
\end{minipage}\tabularnewline
\midrule
\endhead
\begin{minipage}[t]{0.14\columnwidth}\centering
1\strut
\end{minipage} & \begin{minipage}[t]{0.20\columnwidth}\centering
Present\strut
\end{minipage} & \begin{minipage}[t]{0.33\columnwidth}\centering
High protein\strut
\end{minipage}\tabularnewline
\begin{minipage}[t]{0.14\columnwidth}\centering
2\strut
\end{minipage} & \begin{minipage}[t]{0.20\columnwidth}\centering
Present\strut
\end{minipage} & \begin{minipage}[t]{0.33\columnwidth}\centering
Low protein\strut
\end{minipage}\tabularnewline
\begin{minipage}[t]{0.14\columnwidth}\centering
3\strut
\end{minipage} & \begin{minipage}[t]{0.20\columnwidth}\centering
Absent\strut
\end{minipage} & \begin{minipage}[t]{0.33\columnwidth}\centering
High protein\strut
\end{minipage}\tabularnewline
\begin{minipage}[t]{0.14\columnwidth}\centering
4\strut
\end{minipage} & \begin{minipage}[t]{0.20\columnwidth}\centering
Absent\strut
\end{minipage} & \begin{minipage}[t]{0.33\columnwidth}\centering
Low protein\strut
\end{minipage}\tabularnewline
\bottomrule
\end{longtable}

with observations at 3, 6, 9, 12, 15, 18, 21, 24, 27, 30, and 33 days
after admission.

We can calculate a confidence interval for the mean weight velocity
(\texttt{wt}) using the data in the \texttt{sd} and \texttt{n}
variables. We will use the \texttt{transform()} function to do this:

\begin{Shaded}
\begin{Highlighting}[]
\NormalTok{diets <-}\StringTok{ }\KeywordTok{transform}\NormalTok{(diets, }\DataTypeTok{lci =}\NormalTok{ wt }\OperatorTok{-}\StringTok{ }\NormalTok{sd }\OperatorTok{/}\StringTok{ }\KeywordTok{sqrt}\NormalTok{(n), }\DataTypeTok{uci =}\NormalTok{ wt }\OperatorTok{+}\StringTok{ }\NormalTok{sd }\OperatorTok{/}\StringTok{ }\KeywordTok{sqrt}\NormalTok{(n))}
\end{Highlighting}
\end{Shaded}

In this case we are calculating confidence intervals as plus or minus
one standard error of the mean.

The \texttt{transform()} function is very useful as it can add columns
directly to a data.frame or transform data already stored in a
data.frame.

Examine the \texttt{diets} data.frame:

\begin{Shaded}
\begin{Highlighting}[]
\NormalTok{diets}
\end{Highlighting}
\end{Shaded}

\begin{verbatim}
##    day oedema diet   wt   sd   n        lci         uci
## 1    3      1   HP  1.1 11.5  37 -0.7905884  2.99058835
## 2    6      1   HP  1.0 11.0  37 -0.8083889  2.80838886
## 3    9      1   HP  1.2  8.4  37 -0.1809515  2.58095149
## 4   12      1   HP -1.0  7.4  37 -2.2165525  0.21655251
## 5   15      1   HP -0.9  3.2  37 -1.4260768 -0.37392324
## 6   18      1   HP -1.7  5.8  37 -2.6535141 -0.74648587
## 7   21      1   HP -1.7  4.8  37 -2.4891151 -0.91088486
## 8   24      1   HP -1.7  4.8  37 -2.4891151 -0.91088486
## 9   27      1   HP -1.7  5.3  37 -2.5713146 -0.82868537
## 10  30      1   HP -2.5  6.8  37 -3.6179131 -1.38208689
## 11  33      1   HP -1.3  8.4  37 -2.6809515  0.08095149
## 12   3      2   HP  5.1 15.1 291  4.2148223  5.98517768
## 13   6      2   HP  5.1 13.4 291  4.3144781  5.88552191
## 14   9      2   HP  4.5  6.5 291  4.1189633  4.88103675
## 15  12      2   HP  5.2  9.9 291  4.6196517  5.78034828
## 16  15      2   HP  4.6  8.2 291  4.1193075  5.08069251
## 17  18      2   HP  5.1 13.4 291  4.3144781  5.88552191
## 18  21      2   HP  5.0 13.4 291  4.2144781  5.78552191
## 19  24      2   HP  5.2 15.1 291  4.3148223  6.08517768
## 20  27      2   HP  5.1 11.6 291  4.4199960  5.78000404
## 21  30      2   HP  5.0  9.2 291  4.4606864  5.53931355
## 22  33      2   HP  5.2  8.2 291  4.7193075  5.68069251
## 23   3      1   LP -3.0 15.7  65 -4.9473453 -1.05265467
## 24   6      1   LP -2.5 11.9  65 -3.9760133 -1.02398666
## 25   9      1   LP -2.0 13.7  65 -3.6992759 -0.30072414
## 26  12      1   LP  2.0 11.0  65  0.6356179  3.36438208
## 27  15      1   LP  2.1 13.2  65  0.4627415  3.73725850
## 28  18      1   LP  1.7 16.2  65 -0.3093627  3.70936270
## 29  21      1   LP  4.0 11.0  65  2.6356179  5.36438208
## 30  24      1   LP  6.5 14.7  65  4.6766894  8.32331060
## 31  27      1   LP  6.4 10.2  65  5.1348457  7.66515429
## 32  30      1   LP  6.6  8.0  65  5.6077221  7.59227788
## 33  33      1   LP  6.5 11.0  65  5.1356179  7.86438208
## 34   3      2   LP  5.1 11.1  86  3.9030562  6.29694378
## 35   6      2   LP  6.0  8.2  86  5.1157713  6.88422874
## 36   9      2   LP  5.1  8.2  86  4.2157713  5.98422874
## 37  12      2   LP  6.5 11.9  86  5.2167900  7.78321000
## 38  15      2   LP  6.4  9.1  86  5.4187218  7.38127824
## 39  18      2   LP  5.9 13.8  86  4.4119077  7.38809227
## 40  21      2   LP  6.1 16.5  86  4.3207592  7.87924076
## 41  24      2   LP  4.0  9.1  86  3.0187218  4.98127824
## 42  27      2   LP  3.1 12.8  86  1.7197405  4.48025950
## 43  30      2   LP  4.0  7.3  86  3.2128208  4.78717924
## 44  33      2   LP  5.0  9.1  86  4.0187218  5.98127824
\end{verbatim}

Two new columns (\texttt{lci} and \texttt{uci}) have been added.

Now that we have calculated the confidence intervals we should, for
convenience, split the diets data.frame into four separate data.frames
(one for each arm of the trial):

\begin{Shaded}
\begin{Highlighting}[]
\NormalTok{oed.hp <-}\StringTok{ }\KeywordTok{subset}\NormalTok{(diets, oedema }\OperatorTok{==}\StringTok{ }\DecValTok{1} \OperatorTok{&}\StringTok{ }\NormalTok{diet }\OperatorTok{==}\StringTok{ "HP"}\NormalTok{)}
\NormalTok{oed.lp <-}\StringTok{ }\KeywordTok{subset}\NormalTok{(diets, oedema }\OperatorTok{==}\StringTok{ }\DecValTok{1} \OperatorTok{&}\StringTok{ }\NormalTok{diet }\OperatorTok{==}\StringTok{ "LP"}\NormalTok{)}
\NormalTok{mar.hp <-}\StringTok{ }\KeywordTok{subset}\NormalTok{(diets, oedema }\OperatorTok{==}\StringTok{ }\DecValTok{2} \OperatorTok{&}\StringTok{ }\NormalTok{diet }\OperatorTok{==}\StringTok{ "HP"}\NormalTok{)}
\NormalTok{mar.lp <-}\StringTok{ }\KeywordTok{subset}\NormalTok{(diets, oedema }\OperatorTok{==}\StringTok{ }\DecValTok{2} \OperatorTok{&}\StringTok{ }\NormalTok{diet }\OperatorTok{==}\StringTok{ "LP"}\NormalTok{)}
\end{Highlighting}
\end{Shaded}

Check that each data.frame contains the data that you expect it to:

\begin{Shaded}
\begin{Highlighting}[]
\NormalTok{oed.hp}
\NormalTok{oed.lp}
\NormalTok{mar.hp}
\NormalTok{mar.lp}
\end{Highlighting}
\end{Shaded}

\begin{verbatim}
##    day oedema diet   wt   sd  n        lci         uci
## 1    3      1   HP  1.1 11.5 37 -0.7905884  2.99058835
## 2    6      1   HP  1.0 11.0 37 -0.8083889  2.80838886
## 3    9      1   HP  1.2  8.4 37 -0.1809515  2.58095149
## 4   12      1   HP -1.0  7.4 37 -2.2165525  0.21655251
## 5   15      1   HP -0.9  3.2 37 -1.4260768 -0.37392324
## 6   18      1   HP -1.7  5.8 37 -2.6535141 -0.74648587
## 7   21      1   HP -1.7  4.8 37 -2.4891151 -0.91088486
## 8   24      1   HP -1.7  4.8 37 -2.4891151 -0.91088486
## 9   27      1   HP -1.7  5.3 37 -2.5713146 -0.82868537
## 10  30      1   HP -2.5  6.8 37 -3.6179131 -1.38208689
## 11  33      1   HP -1.3  8.4 37 -2.6809515  0.08095149
\end{verbatim}

\begin{verbatim}
##    day oedema diet   wt   sd  n        lci        uci
## 23   3      1   LP -3.0 15.7 65 -4.9473453 -1.0526547
## 24   6      1   LP -2.5 11.9 65 -3.9760133 -1.0239867
## 25   9      1   LP -2.0 13.7 65 -3.6992759 -0.3007241
## 26  12      1   LP  2.0 11.0 65  0.6356179  3.3643821
## 27  15      1   LP  2.1 13.2 65  0.4627415  3.7372585
## 28  18      1   LP  1.7 16.2 65 -0.3093627  3.7093627
## 29  21      1   LP  4.0 11.0 65  2.6356179  5.3643821
## 30  24      1   LP  6.5 14.7 65  4.6766894  8.3233106
## 31  27      1   LP  6.4 10.2 65  5.1348457  7.6651543
## 32  30      1   LP  6.6  8.0 65  5.6077221  7.5922779
## 33  33      1   LP  6.5 11.0 65  5.1356179  7.8643821
\end{verbatim}

\begin{verbatim}
##    day oedema diet  wt   sd   n      lci      uci
## 12   3      2   HP 5.1 15.1 291 4.214822 5.985178
## 13   6      2   HP 5.1 13.4 291 4.314478 5.885522
## 14   9      2   HP 4.5  6.5 291 4.118963 4.881037
## 15  12      2   HP 5.2  9.9 291 4.619652 5.780348
## 16  15      2   HP 4.6  8.2 291 4.119307 5.080693
## 17  18      2   HP 5.1 13.4 291 4.314478 5.885522
## 18  21      2   HP 5.0 13.4 291 4.214478 5.785522
## 19  24      2   HP 5.2 15.1 291 4.314822 6.085178
## 20  27      2   HP 5.1 11.6 291 4.419996 5.780004
## 21  30      2   HP 5.0  9.2 291 4.460686 5.539314
## 22  33      2   HP 5.2  8.2 291 4.719307 5.680693
\end{verbatim}

\begin{verbatim}
##    day oedema diet  wt   sd  n      lci      uci
## 34   3      2   LP 5.1 11.1 86 3.903056 6.296944
## 35   6      2   LP 6.0  8.2 86 5.115771 6.884229
## 36   9      2   LP 5.1  8.2 86 4.215771 5.984229
## 37  12      2   LP 6.5 11.9 86 5.216790 7.783210
## 38  15      2   LP 6.4  9.1 86 5.418722 7.381278
## 39  18      2   LP 5.9 13.8 86 4.411908 7.388092
## 40  21      2   LP 6.1 16.5 86 4.320759 7.879241
## 41  24      2   LP 4.0  9.1 86 3.018722 4.981278
## 42  27      2   LP 3.1 12.8 86 1.719741 4.480259
## 43  30      2   LP 4.0  7.3 86 3.212821 4.787179
## 44  33      2   LP 5.0  9.1 86 4.018722 5.981278
\end{verbatim}

We can now plot the data for one arm of the trial:

\begin{Shaded}
\begin{Highlighting}[]
\KeywordTok{plot}\NormalTok{(oed.hp}\OperatorTok{$}\NormalTok{day, oed.hp}\OperatorTok{$}\NormalTok{wt, }\DataTypeTok{type =} \StringTok{"l"}\NormalTok{)}
\end{Highlighting}
\end{Shaded}

\includegraphics{prfe_files/figure-latex/unnamed-chunk-384-1.pdf}

We can add error bars using the \texttt{arrows()} function:

\begin{Shaded}
\begin{Highlighting}[]
\KeywordTok{arrows}\NormalTok{(oed.hp}\OperatorTok{$}\NormalTok{day, oed.hp}\OperatorTok{$}\NormalTok{lci, oed.hp}\OperatorTok{$}\NormalTok{day, oed.hp}\OperatorTok{$}\NormalTok{uci, }\DataTypeTok{code =} \DecValTok{3}\NormalTok{, }\DataTypeTok{angle =} \DecValTok{90}\NormalTok{, }\DataTypeTok{length =} \FloatTok{0.1}\NormalTok{)}
\end{Highlighting}
\end{Shaded}

\includegraphics{prfe_files/figure-latex/unnamed-chunk-386-1.pdf}

The scale of the y axis is wrong because the \texttt{plot()} function
automatically scales axes to the ranges of the x and y data it is given.
We can fix this by specifying a different set of limits (from
\texttt{lci} and \texttt{uci}) for the y axis using the \texttt{ylim}
parameter:

\begin{Shaded}
\begin{Highlighting}[]
\KeywordTok{plot}\NormalTok{(oed.hp}\OperatorTok{$}\NormalTok{day, oed.hp}\OperatorTok{$}\NormalTok{wt, }\DataTypeTok{type =} \StringTok{"l"}\NormalTok{, }\DataTypeTok{ylim =} \KeywordTok{c}\NormalTok{(}\KeywordTok{min}\NormalTok{(oed.hp}\OperatorTok{$}\NormalTok{lci), }\KeywordTok{max}\NormalTok{(oed.hp}\OperatorTok{$}\NormalTok{uci)))}

\KeywordTok{arrows}\NormalTok{(oed.hp}\OperatorTok{$}\NormalTok{day, oed.hp}\OperatorTok{$}\NormalTok{lci, oed.hp}\OperatorTok{$}\NormalTok{day, oed.hp}\OperatorTok{$}\NormalTok{uci, }\DataTypeTok{code =} \DecValTok{3}\NormalTok{, }\DataTypeTok{angle =} \DecValTok{90}\NormalTok{, }\DataTypeTok{length =} \FloatTok{0.1}\NormalTok{)}
\end{Highlighting}
\end{Shaded}

\includegraphics{prfe_files/figure-latex/unnamed-chunk-388-1.pdf}

The plot might also be improved by adding plotting symbols:

\begin{Shaded}
\begin{Highlighting}[]
\KeywordTok{points}\NormalTok{(oed.hp}\OperatorTok{$}\NormalTok{day, oed.hp}\OperatorTok{$}\NormalTok{wt, }\DataTypeTok{pch =} \DecValTok{21}\NormalTok{, }\DataTypeTok{bg =} \StringTok{"white"}\NormalTok{)}
\end{Highlighting}
\end{Shaded}

\includegraphics{prfe_files/figure-latex/unnamed-chunk-390-1.pdf}

The axis titles (\texttt{oed.hp\$day} and \texttt{oed.hp\$wt}) are the
column (variable) names.

We can specify titles and axes labels in the call to the \texttt{plot()}
function:

\begin{Shaded}
\begin{Highlighting}[]
\KeywordTok{plot}\NormalTok{(oed.hp}\OperatorTok{$}\NormalTok{day, oed.hp}\OperatorTok{$}\NormalTok{wt, }\DataTypeTok{type =} \StringTok{"l"}\NormalTok{,}
     \DataTypeTok{ylim =} \KeywordTok{c}\NormalTok{(}\KeywordTok{min}\NormalTok{(oed.hp}\OperatorTok{$}\NormalTok{lci), }\KeywordTok{max}\NormalTok{(oed.hp}\OperatorTok{$}\NormalTok{uci)),}
     \DataTypeTok{main =} \StringTok{"HP Diet. Oedematous Cases"}\NormalTok{,}
     \DataTypeTok{xlab =} \StringTok{"Days of treatment"}\NormalTok{, }\DataTypeTok{ylab =} \StringTok{"Weight gain (g/kg/d)"}\NormalTok{)}

\KeywordTok{arrows}\NormalTok{(oed.hp}\OperatorTok{$}\NormalTok{day, oed.hp}\OperatorTok{$}\NormalTok{lci, oed.hp}\OperatorTok{$}\NormalTok{day, oed.hp}\OperatorTok{$}\NormalTok{uci,}
       \DataTypeTok{code =} \DecValTok{3}\NormalTok{, }\DataTypeTok{angle =} \DecValTok{90}\NormalTok{, }\DataTypeTok{length =} \FloatTok{0.1}\NormalTok{)}

\KeywordTok{points}\NormalTok{(oed.hp}\OperatorTok{$}\NormalTok{day, oed.hp}\OperatorTok{$}\NormalTok{wt, }\DataTypeTok{pch =} \DecValTok{21}\NormalTok{, }\DataTypeTok{bg =} \StringTok{"white"}\NormalTok{)}
\end{Highlighting}
\end{Shaded}

\includegraphics{prfe_files/figure-latex/unnamed-chunk-392-1.pdf}

Now that we know how to plot error bars, we can write a function that we
will be able to use whenever we need to plot data with error bars.

Before continuing we will consider what the new function should be able
to do. This will help us when it comes to writing the function. Our new
function should:

\begin{enumerate}
\def\labelenumi{\arabic{enumi}.}
\item
  Take four numeric vectors (x, y, lower CI for y, and upper CI for y)
  and plot them.
\item
  Be able to plot the data points as unconnected points or as points
  joined by lines.
\item
  Calculateappropriatelimitsfortheyaxis.
\item
  Produce a plot without axes so that more than one data series may be
  plotted on the same chart. 5. Provide sensible default values for axis
  limits and labels.
\end{enumerate}

From this list we know that we need the function to take several
parameters:

\begin{longtable}[]{@{}lll@{}}
\toprule
\begin{minipage}[b]{0.15\columnwidth}\raggedright
\textbf{Name}\strut
\end{minipage} & \begin{minipage}[b]{0.32\columnwidth}\raggedright
\textbf{Purpose}\strut
\end{minipage} & \begin{minipage}[b]{0.39\columnwidth}\raggedright
\textbf{Default value}\strut
\end{minipage}\tabularnewline
\midrule
\endhead
\begin{minipage}[t]{0.15\columnwidth}\raggedright
\texttt{x}\strut
\end{minipage} & \begin{minipage}[t]{0.32\columnwidth}\raggedright
Data to plot\strut
\end{minipage} & \begin{minipage}[t]{0.39\columnwidth}\raggedright
None\strut
\end{minipage}\tabularnewline
\begin{minipage}[t]{0.15\columnwidth}\raggedright
\texttt{y}\strut
\end{minipage} & \begin{minipage}[t]{0.32\columnwidth}\raggedright
Data to plot\strut
\end{minipage} & \begin{minipage}[t]{0.39\columnwidth}\raggedright
None\strut
\end{minipage}\tabularnewline
\begin{minipage}[t]{0.15\columnwidth}\raggedright
\texttt{y.lci}\strut
\end{minipage} & \begin{minipage}[t]{0.32\columnwidth}\raggedright
Data to plot\strut
\end{minipage} & \begin{minipage}[t]{0.39\columnwidth}\raggedright
None\strut
\end{minipage}\tabularnewline
\begin{minipage}[t]{0.15\columnwidth}\raggedright
\texttt{y.uci}\strut
\end{minipage} & \begin{minipage}[t]{0.32\columnwidth}\raggedright
Data to plot\strut
\end{minipage} & \begin{minipage}[t]{0.39\columnwidth}\raggedright
None\strut
\end{minipage}\tabularnewline
\begin{minipage}[t]{0.15\columnwidth}\raggedright
\texttt{ylim}\strut
\end{minipage} & \begin{minipage}[t]{0.32\columnwidth}\raggedright
Limits for y axis\strut
\end{minipage} & \begin{minipage}[t]{0.39\columnwidth}\raggedright
\texttt{c(min(y.lci),\ max(y.uci))}\strut
\end{minipage}\tabularnewline
\begin{minipage}[t]{0.15\columnwidth}\raggedright
\texttt{xlab}\strut
\end{minipage} & \begin{minipage}[t]{0.32\columnwidth}\raggedright
Label for x axis\strut
\end{minipage} & \begin{minipage}[t]{0.39\columnwidth}\raggedright
\texttt{deparse(substitute(x))}\strut
\end{minipage}\tabularnewline
\begin{minipage}[t]{0.15\columnwidth}\raggedright
\texttt{ylab}\strut
\end{minipage} & \begin{minipage}[t]{0.32\columnwidth}\raggedright
Label for y axis\strut
\end{minipage} & \begin{minipage}[t]{0.39\columnwidth}\raggedright
\texttt{deparse(substitute(y))}\strut
\end{minipage}\tabularnewline
\begin{minipage}[t]{0.15\columnwidth}\raggedright
\texttt{main}\strut
\end{minipage} & \begin{minipage}[t]{0.32\columnwidth}\raggedright
Chart title\strut
\end{minipage} & \begin{minipage}[t]{0.39\columnwidth}\raggedright
\texttt{paste(ylab,\ "by",\ xlab)}\strut
\end{minipage}\tabularnewline
\begin{minipage}[t]{0.15\columnwidth}\raggedright
\texttt{type}\strut
\end{minipage} & \begin{minipage}[t]{0.32\columnwidth}\raggedright
Type of plot\strut
\end{minipage} & \begin{minipage}[t]{0.39\columnwidth}\raggedright
\texttt{"l"}\strut
\end{minipage}\tabularnewline
\begin{minipage}[t]{0.15\columnwidth}\raggedright
\texttt{lty}\strut
\end{minipage} & \begin{minipage}[t]{0.32\columnwidth}\raggedright
Line type\strut
\end{minipage} & \begin{minipage}[t]{0.39\columnwidth}\raggedright
\texttt{1}\strut
\end{minipage}\tabularnewline
\begin{minipage}[t]{0.15\columnwidth}\raggedright
\texttt{col}\strut
\end{minipage} & \begin{minipage}[t]{0.32\columnwidth}\raggedright
Line and point colour\strut
\end{minipage} & \begin{minipage}[t]{0.39\columnwidth}\raggedright
\texttt{"black"}\strut
\end{minipage}\tabularnewline
\begin{minipage}[t]{0.15\columnwidth}\raggedright
\texttt{axes}\strut
\end{minipage} & \begin{minipage}[t]{0.32\columnwidth}\raggedright
Draw x and y axes\strut
\end{minipage} & \begin{minipage}[t]{0.39\columnwidth}\raggedright
\texttt{TRUE}\strut
\end{minipage}\tabularnewline
\begin{minipage}[t]{0.15\columnwidth}\raggedright
\texttt{pch}\strut
\end{minipage} & \begin{minipage}[t]{0.32\columnwidth}\raggedright
Type of points to plot\strut
\end{minipage} & \begin{minipage}[t]{0.39\columnwidth}\raggedright
\texttt{1}\strut
\end{minipage}\tabularnewline
\begin{minipage}[t]{0.15\columnwidth}\raggedright
\texttt{bg}\strut
\end{minipage} & \begin{minipage}[t]{0.32\columnwidth}\raggedright
Fill colour of point\strut
\end{minipage} & \begin{minipage}[t]{0.39\columnwidth}\raggedright
\texttt{white}\strut
\end{minipage}\tabularnewline
\bottomrule
\end{longtable}

The parameter names have been chosen to be the same as the parameter
names to \texttt{plot()} and \texttt{points()}. This makes the function
easier to use. It also makes the function easier to write.

Create a new function called \texttt{plot.ci()}:

\begin{Shaded}
\begin{Highlighting}[]
\NormalTok{plot.ci <-}\StringTok{ }\ControlFlowTok{function}\NormalTok{() \{\}}
\end{Highlighting}
\end{Shaded}

This creates an empty function called \texttt{plot.ci()}.

Use the \texttt{fix()} function to edit the \texttt{plot.ci()} function:

\begin{Shaded}
\begin{Highlighting}[]
\KeywordTok{fix}\NormalTok{(plot.ci)}
\end{Highlighting}
\end{Shaded}

Edit the function to read:

\begin{Shaded}
\begin{Highlighting}[]
\ControlFlowTok{function}\NormalTok{(x,}
\NormalTok{         y, y.lci, y.uci,}
         \DataTypeTok{ylim =} \KeywordTok{c}\NormalTok{(}\KeywordTok{min}\NormalTok{(y.lci), }\KeywordTok{max}\NormalTok{(y.uci)),}
         \DataTypeTok{xlab =} \KeywordTok{deparse}\NormalTok{(}\KeywordTok{substitute}\NormalTok{(x)),}
         \DataTypeTok{ylab =} \KeywordTok{deparse}\NormalTok{(}\KeywordTok{substitute}\NormalTok{(y)),}
         \DataTypeTok{main =} \KeywordTok{paste}\NormalTok{(ylab, }\StringTok{"by"}\NormalTok{, xlab),}
         \DataTypeTok{type =} \StringTok{"l"}\NormalTok{,}
         \DataTypeTok{lty =} \DecValTok{1}\NormalTok{,}
         \DataTypeTok{col =} \StringTok{"black"}\NormalTok{,}
         \DataTypeTok{axes =} \OtherTok{TRUE}\NormalTok{,}
         \DataTypeTok{pch =} \DecValTok{21}\NormalTok{,}
         \DataTypeTok{bg =} \StringTok{"white"}\NormalTok{) \{}

  \KeywordTok{plot}\NormalTok{(x, y, }\DataTypeTok{type =}\NormalTok{ type, }\DataTypeTok{ylim =}\NormalTok{ ylim, }\DataTypeTok{xlab =}\NormalTok{ xlab, }\DataTypeTok{ylab =}\NormalTok{ ylab,}
       \DataTypeTok{main =}\NormalTok{ main, }\DataTypeTok{lty =}\NormalTok{ lty, }\DataTypeTok{col =}\NormalTok{ col, }\DataTypeTok{axes =}\NormalTok{ axes)}
  
  \KeywordTok{arrows}\NormalTok{(x, y.lci, x, y.uci, }\DataTypeTok{code =} \DecValTok{3}\NormalTok{, }\DataTypeTok{angle =} \DecValTok{90}\NormalTok{, }\DataTypeTok{length =} \FloatTok{0.1}\NormalTok{,}
         \DataTypeTok{lty =}\NormalTok{ lty, }\DataTypeTok{col =}\NormalTok{ col)}
  
  \KeywordTok{points}\NormalTok{(x, y, }\DataTypeTok{pch =}\NormalTok{ pch, }\DataTypeTok{bg =}\NormalTok{ bg, }\DataTypeTok{col =}\NormalTok{ col)}
\NormalTok{\}}
\end{Highlighting}
\end{Shaded}

Once you have made the changes shown above, check your work, save the
file, and quit the editor.

Let's try the \texttt{plot.ci()} function with the test data:

\begin{Shaded}
\begin{Highlighting}[]
\KeywordTok{plot.ci}\NormalTok{(oed.hp}\OperatorTok{$}\NormalTok{day, oed.hp}\OperatorTok{$}\NormalTok{wt, oed.hp}\OperatorTok{$}\NormalTok{lci, oed.hp}\OperatorTok{$}\NormalTok{uci)}
\end{Highlighting}
\end{Shaded}

\includegraphics{prfe_files/figure-latex/unnamed-chunk-397-1.pdf}

Note how the function has used default values for the axis labels, chart
titles, chart limits etc.

We can override these default values if we need to:

\begin{Shaded}
\begin{Highlighting}[]
\KeywordTok{plot.ci}\NormalTok{(oed.hp}\OperatorTok{$}\NormalTok{day, oed.hp}\OperatorTok{$}\NormalTok{wt, oed.hp}\OperatorTok{$}\NormalTok{lci, oed.hp}\OperatorTok{$}\NormalTok{uci,}
        \DataTypeTok{ylim =} \KeywordTok{c}\NormalTok{(}\OperatorTok{-}\DecValTok{6}\NormalTok{, }\DecValTok{10}\NormalTok{), }\DataTypeTok{xlab =} \StringTok{"Day"}\NormalTok{,}
        \DataTypeTok{ylab =} \StringTok{"Weight gain (g/kg/day)"}\NormalTok{,}
        \DataTypeTok{main =} \StringTok{"Oedematous"}\NormalTok{, }\DataTypeTok{col =} \StringTok{"red"}\NormalTok{)}
\end{Highlighting}
\end{Shaded}

\includegraphics{prfe_files/figure-latex/unnamed-chunk-398-1.pdf}

Do not close the plot window.

We should also check that we can plot another data series on this chart:

\begin{Shaded}
\begin{Highlighting}[]
\KeywordTok{par}\NormalTok{(}\DataTypeTok{new =} \OtherTok{TRUE}\NormalTok{)}
\KeywordTok{plot.ci}\NormalTok{(oed.lp}\OperatorTok{$}\NormalTok{day, oed.lp}\OperatorTok{$}\NormalTok{wt, oed.lp}\OperatorTok{$}\NormalTok{lci, oed.lp}\OperatorTok{$}\NormalTok{uci,}
        \DataTypeTok{axes =} \OtherTok{FALSE}\NormalTok{, }\DataTypeTok{pch =} \DecValTok{22}\NormalTok{, }\DataTypeTok{xlab =} \StringTok{""}\NormalTok{, }\DataTypeTok{ylab =} \StringTok{""}\NormalTok{, }\DataTypeTok{main =} \StringTok{""}\NormalTok{,}
        \DataTypeTok{col =} \StringTok{"darkgreen"}\NormalTok{)}
\end{Highlighting}
\end{Shaded}

\includegraphics{prfe_files/figure-latex/unnamed-chunk-400-1.pdf}

We can also add a legend:

\begin{Shaded}
\begin{Highlighting}[]
\KeywordTok{legend}\NormalTok{(}\DecValTok{5}\NormalTok{, }\DecValTok{8}\NormalTok{, }\DataTypeTok{legend =} \KeywordTok{c}\NormalTok{(}\StringTok{"HP"}\NormalTok{, }\StringTok{"LP"}\NormalTok{), }\DataTypeTok{lty =} \KeywordTok{c}\NormalTok{(}\DecValTok{1}\NormalTok{, }\DecValTok{1}\NormalTok{),}
       \DataTypeTok{pch =} \KeywordTok{c}\NormalTok{(}\DecValTok{21}\NormalTok{, }\DecValTok{22}\NormalTok{), }\DataTypeTok{col =} \KeywordTok{c}\NormalTok{(}\StringTok{"red"}\NormalTok{, }\StringTok{"darkgreen"}\NormalTok{))}
\end{Highlighting}
\end{Shaded}

\includegraphics{prfe_files/figure-latex/unnamed-chunk-402-1.pdf}

We should also check that we can produce plots of unconnected points:

\begin{Shaded}
\begin{Highlighting}[]
\KeywordTok{plot.ci}\NormalTok{(oed.hp}\OperatorTok{$}\NormalTok{day, oed.hp}\OperatorTok{$}\NormalTok{wt, oed.hp}\OperatorTok{$}\NormalTok{lci, oed.hp}\OperatorTok{$}\NormalTok{uci, }\DataTypeTok{type =} \StringTok{"p"}\NormalTok{)}
\end{Highlighting}
\end{Shaded}

\includegraphics{prfe_files/figure-latex/unnamed-chunk-403-1.pdf}

Try plotting the data for the marasmic patients using the
\texttt{plot.ci()} function.

You might like to use the \texttt{save()} function to save the
\texttt{plot.ci()} function.

We can use a similar technique to add error bars to different types of
plot. If we plot the weight velocities for oedematous patients receiving
the high protein diet as a bar chart:

\begin{Shaded}
\begin{Highlighting}[]
\KeywordTok{barplot}\NormalTok{(oed.hp}\OperatorTok{$}\NormalTok{wt, }\DataTypeTok{names.arg =}\NormalTok{ oed.hp}\OperatorTok{$}\NormalTok{day, }\DataTypeTok{col =} \StringTok{"white"}\NormalTok{, }
        \DataTypeTok{ylim =} \KeywordTok{c}\NormalTok{(}\KeywordTok{min}\NormalTok{(oed.hp}\OperatorTok{$}\NormalTok{lci), }\KeywordTok{max}\NormalTok{(oed.hp}\OperatorTok{$}\NormalTok{uci)))}
\end{Highlighting}
\end{Shaded}

\includegraphics{prfe_files/figure-latex/unnamed-chunk-404-1.pdf}

we could add error bars using the \texttt{arrows()} function as we did
with a line plot:

\begin{Shaded}
\begin{Highlighting}[]
\KeywordTok{arrows}\NormalTok{(oed.hp}\OperatorTok{$}\NormalTok{day, oed.hp}\OperatorTok{$}\NormalTok{lci, oed.hp}\OperatorTok{$}\NormalTok{day, oed.hp}\OperatorTok{$}\NormalTok{uci, }
       \DataTypeTok{code =} \DecValTok{3}\NormalTok{, }\DataTypeTok{angle =} \DecValTok{90}\NormalTok{, }\DataTypeTok{length =} \FloatTok{0.1}\NormalTok{)}
\end{Highlighting}
\end{Shaded}

\includegraphics{prfe_files/figure-latex/unnamed-chunk-406-1.pdf}

but this does not produce the expected results because the centres of
the bars are not placed on the chart at the positions held in
\texttt{oed.hp\$day}. This is easily fixed as \texttt{barplot()} returns
a numeric vector (or matrix, when \texttt{beside\ =\ TRUE}) containing
the co-ordinates of the bar midpoints:

\begin{Shaded}
\begin{Highlighting}[]
\NormalTok{bar.positions <-}\StringTok{ }\KeywordTok{barplot}\NormalTok{(oed.hp}\OperatorTok{$}\NormalTok{wt, }\DataTypeTok{names.arg =}\NormalTok{ oed.hp}\OperatorTok{$}\NormalTok{day,}
                         \DataTypeTok{ylim =} \KeywordTok{c}\NormalTok{(}\KeywordTok{min}\NormalTok{(oed.hp}\OperatorTok{$}\NormalTok{lci), }\KeywordTok{max}\NormalTok{(oed.hp}\OperatorTok{$}\NormalTok{uci)),}
                         \DataTypeTok{col =} \StringTok{"white"}\NormalTok{)}
\end{Highlighting}
\end{Shaded}

\includegraphics{prfe_files/figure-latex/unnamed-chunk-407-1.pdf}

\begin{Shaded}
\begin{Highlighting}[]
\NormalTok{bar.positions}
\end{Highlighting}
\end{Shaded}

\begin{verbatim}
##       [,1]
##  [1,]  0.7
##  [2,]  1.9
##  [3,]  3.1
##  [4,]  4.3
##  [5,]  5.5
##  [6,]  6.7
##  [7,]  7.9
##  [8,]  9.1
##  [9,] 10.3
## [10,] 11.5
## [11,] 12.7
\end{verbatim}

We can now use the information stored in \texttt{bar.positions} to
specify the positions of the error bars:

\begin{Shaded}
\begin{Highlighting}[]
\KeywordTok{arrows}\NormalTok{(bar.positions, oed.hp}\OperatorTok{$}\NormalTok{lci, bar.positions, oed.hp}\OperatorTok{$}\NormalTok{uci,}
       \DataTypeTok{code =} \DecValTok{3}\NormalTok{, }\DataTypeTok{angle =} \DecValTok{90}\NormalTok{, }\DataTypeTok{length =} \FloatTok{0.1}\NormalTok{)}
\end{Highlighting}
\end{Shaded}

\includegraphics{prfe_files/figure-latex/unnamed-chunk-409-1.pdf}

Armed with this information, we can write a function that we will be
able to use whenever we need to plot a bar chart with error bars.

Create a new function called \texttt{barplot.ci()}:

\begin{Shaded}
\begin{Highlighting}[]
\NormalTok{barplot.ci <-}\StringTok{ }\ControlFlowTok{function}\NormalTok{() \{\}}
\end{Highlighting}
\end{Shaded}

This creates an empty function called \texttt{barplot.ci()}.

Use the \texttt{fix()} function to edit the \texttt{barplot.ci()}
function:

\begin{Shaded}
\begin{Highlighting}[]
\KeywordTok{fix}\NormalTok{(barplot.ci)}
\end{Highlighting}
\end{Shaded}

Edit the function to read:

\begin{Shaded}
\begin{Highlighting}[]
\ControlFlowTok{function}\NormalTok{(y, bar.names, lci, uci,}
         \DataTypeTok{ylim =} \KeywordTok{c}\NormalTok{(}\KeywordTok{min}\NormalTok{(lci), }\KeywordTok{max}\NormalTok{(uci)),}
         \DataTypeTok{xlab =} \KeywordTok{deparse}\NormalTok{(}\KeywordTok{substitute}\NormalTok{(bar.names)),}
         \DataTypeTok{ylab =} \KeywordTok{deparse}\NormalTok{(}\KeywordTok{substitute}\NormalTok{(y)),}
         \DataTypeTok{main =} \KeywordTok{paste}\NormalTok{(ylab, }\StringTok{"by"}\NormalTok{, xlab)) \{}
     
\NormalTok{  bp <-}\StringTok{ }\KeywordTok{barplot}\NormalTok{(y, }\DataTypeTok{names.arg =}\NormalTok{ bar.names, }\DataTypeTok{ylim =}\NormalTok{ ylim, }\DataTypeTok{xlab =}\NormalTok{ xlab,}
                \DataTypeTok{ylab =}\NormalTok{ ylab, }\DataTypeTok{main =}\NormalTok{ main, }\DataTypeTok{col =} \StringTok{"white"}\NormalTok{)}
     
  \KeywordTok{arrows}\NormalTok{(bp, lci, bp, uci, }\DataTypeTok{code =} \DecValTok{3}\NormalTok{, }\DataTypeTok{angle =} \DecValTok{90}\NormalTok{, }\DataTypeTok{length =} \FloatTok{0.1}\NormalTok{)}
\NormalTok{\}}
\end{Highlighting}
\end{Shaded}

Once you have made the changes shown above, check your work, save the
file, and quit the editor.

Let's try the \texttt{barplot.ci()} function with the test data:

\begin{Shaded}
\begin{Highlighting}[]
\KeywordTok{barplot.ci}\NormalTok{(oed.hp}\OperatorTok{$}\NormalTok{wt, oed.hp}\OperatorTok{$}\NormalTok{day, oed.hp}\OperatorTok{$}\NormalTok{lci, oed.hp}\OperatorTok{$}\NormalTok{uci)}
\end{Highlighting}
\end{Shaded}

\includegraphics{prfe_files/figure-latex/unnamed-chunk-414-1.pdf}

Try plotting the weight velocities for the marasmic patients receiving
the high protein diet using the \texttt{barplot.ci()} function:

\begin{Shaded}
\begin{Highlighting}[]
\KeywordTok{barplot.ci}\NormalTok{(mar.hp}\OperatorTok{$}\NormalTok{wt, mar.hp}\OperatorTok{$}\NormalTok{day, mar.hp}\OperatorTok{$}\NormalTok{lci, mar.hp}\OperatorTok{$}\NormalTok{uci)}
\end{Highlighting}
\end{Shaded}

\includegraphics{prfe_files/figure-latex/unnamed-chunk-415-1.pdf}

The chart looks wrong. This is because we have set the wrong limits for
the y axis. The bars are drawn from zero to the data point but we have
specified a limit for the y axis that is not constrained to include
zero. This is easy to fix. Edit the \texttt{barplot.ci()} function to
read:

\begin{Shaded}
\begin{Highlighting}[]
\ControlFlowTok{function}\NormalTok{(y, bar.names, lci, uci,}
         \DataTypeTok{ylim =} \KeywordTok{c}\NormalTok{(}\KeywordTok{min}\NormalTok{(}\DecValTok{0}\NormalTok{, lci), }\KeywordTok{max}\NormalTok{(}\DecValTok{0}\NormalTok{, uci)),}
         \DataTypeTok{xlab =} \KeywordTok{deparse}\NormalTok{(}\KeywordTok{substitute}\NormalTok{(bar.names)),}
         \DataTypeTok{ylab =} \KeywordTok{deparse}\NormalTok{(}\KeywordTok{substitute}\NormalTok{(y)),}
         \DataTypeTok{main =} \KeywordTok{paste}\NormalTok{(ylab, }\StringTok{"by"}\NormalTok{, xlab)) \{}

\NormalTok{  bp <-}\StringTok{ }\KeywordTok{barplot}\NormalTok{(y, }\DataTypeTok{names.arg =}\NormalTok{ bar.names,  }\DataTypeTok{ylim =}\NormalTok{ ylim, }\DataTypeTok{xlab =}\NormalTok{ xlab,}
                \DataTypeTok{ylab =}\NormalTok{ ylab, }\DataTypeTok{main =}\NormalTok{ main, }\DataTypeTok{col =} \StringTok{"white"}\NormalTok{)}
     
  \KeywordTok{arrows}\NormalTok{(bp, lci, bp, uci, }\DataTypeTok{code =} \DecValTok{3}\NormalTok{, }\DataTypeTok{angle =} \DecValTok{90}\NormalTok{, }\DataTypeTok{length =} \FloatTok{0.1}\NormalTok{)}
\NormalTok{\}}
\end{Highlighting}
\end{Shaded}

Once you have made the changes shown above, check your work, save the
file, and quit the editor.

Try plotting the data for the marasmic patients using the
\texttt{barplot.ci()} function:

\begin{Shaded}
\begin{Highlighting}[]
\KeywordTok{barplot.ci}\NormalTok{(mar.hp}\OperatorTok{$}\NormalTok{wt, mar.hp}\OperatorTok{$}\NormalTok{day, mar.hp}\OperatorTok{$}\NormalTok{lci, mar.hp}\OperatorTok{$}\NormalTok{uci)}
\end{Highlighting}
\end{Shaded}

\includegraphics{prfe_files/figure-latex/unnamed-chunk-418-1.pdf}

Check that the function still operates as expected with the data for
oedematous patients:

\begin{Shaded}
\begin{Highlighting}[]
\KeywordTok{barplot.ci}\NormalTok{(oed.hp}\OperatorTok{$}\NormalTok{wt, oed.hp}\OperatorTok{$}\NormalTok{day, oed.hp}\OperatorTok{$}\NormalTok{lci, oed.hp}\OperatorTok{$}\NormalTok{uci)}
\end{Highlighting}
\end{Shaded}

\includegraphics{prfe_files/figure-latex/unnamed-chunk-419-1.pdf}

The end of one of the error bars touches the x axis. This can also be
fixed by slightly widening the limits for the y axis. It might also be
useful to plot the centre position of each error bar. We can use the
\texttt{points()} function to do this. Edit the \texttt{barplot.ci()}
function to read:

\begin{Shaded}
\begin{Highlighting}[]
\ControlFlowTok{function}\NormalTok{(y, bar.names, lci, uci,}
         \DataTypeTok{ylim =} \KeywordTok{c}\NormalTok{(}\KeywordTok{min}\NormalTok{(}\DecValTok{0}\NormalTok{, lci), }\KeywordTok{max}\NormalTok{(}\DecValTok{0}\NormalTok{, uci)),}
         \DataTypeTok{xlab =} \KeywordTok{deparse}\NormalTok{(}\KeywordTok{substitute}\NormalTok{(bar.names)),}
         \DataTypeTok{ylab =} \KeywordTok{deparse}\NormalTok{(}\KeywordTok{substitute}\NormalTok{(y)),}
         \DataTypeTok{main =} \KeywordTok{paste}\NormalTok{(ylab, }\StringTok{"by"}\NormalTok{, xlab)) \{}
     
\NormalTok{  ylim <-}\StringTok{ }\NormalTok{ylim }\OperatorTok{*}\StringTok{ }\FloatTok{1.1}
     
\NormalTok{  bp <-}\StringTok{ }\KeywordTok{barplot}\NormalTok{(y, }\DataTypeTok{names.arg =}\NormalTok{ bar.names, }\DataTypeTok{ylim =}\NormalTok{ ylim, }\DataTypeTok{xlab =}\NormalTok{ xlab,}
                \DataTypeTok{ylab =}\NormalTok{ ylab, }\DataTypeTok{main =}\NormalTok{ main, }\DataTypeTok{col =} \StringTok{"white"}\NormalTok{)}
     
  \KeywordTok{arrows}\NormalTok{(bp, lci, bp, uci, }\DataTypeTok{code =} \DecValTok{3}\NormalTok{, }\DataTypeTok{angle =} \DecValTok{90}\NormalTok{, }\DataTypeTok{length =} \FloatTok{0.1}\NormalTok{)}
     
  \KeywordTok{points}\NormalTok{(bp, y)}
\NormalTok{\}}
\end{Highlighting}
\end{Shaded}

Once you have made the changes shown above, check your work, save the
file, and quit the editor.

Check that the function works as expected:

\begin{Shaded}
\begin{Highlighting}[]
\KeywordTok{barplot.ci}\NormalTok{(oed.hp}\OperatorTok{$}\NormalTok{wt, oed.hp}\OperatorTok{$}\NormalTok{day, oed.hp}\OperatorTok{$}\NormalTok{lci, oed.hp}\OperatorTok{$}\NormalTok{uci)}
\KeywordTok{barplot.ci}\NormalTok{(mar.hp}\OperatorTok{$}\NormalTok{wt, mar.hp}\OperatorTok{$}\NormalTok{day, mar.hp}\OperatorTok{$}\NormalTok{lci, mar.hp}\OperatorTok{$}\NormalTok{uci)}
\end{Highlighting}
\end{Shaded}

\includegraphics{prfe_files/figure-latex/unnamed-chunk-423-1.pdf}
\includegraphics{prfe_files/figure-latex/unnamed-chunk-423-2.pdf}

The \texttt{barplot.ci()} function now works as expected with both sets
of data. It is important when developing your own functions, to test
them with different data so as to ensure that they work correctly with a
wide range if data.

You might like to use the \texttt{save()} function to save the
\texttt{barplot.ci()} function.

\hypertarget{mesh-map}{%
\section{Mesh map}\label{mesh-map}}

The fact that \texttt{R} provides flexible graphical functions means
that, with little extra work, you can use these functions to present
your data in appropriate and interesting ways rather than having to rely
on a limited set of basic chart types.

In this exercise we will use the \texttt{plot()} function to produce a
simple \emph{mesh-map}.

The file \texttt{cover.dat} contains data from a coverage survey for a
therapeutic feeding program (TFP) in central Malawi undertaken in March
2003. Data were collected using the \emph{centric systematic area
sampling} method to define sampling locations: A number of communities
located closest to the centres of thirty 10 x 10 kilometre grid squares
were sampled using active (investigative) case-finding.

The columns in this dataset are as follows:

\begin{longtable}[]{@{}ll@{}}
\toprule
\begin{minipage}[t]{0.09\columnwidth}\raggedright
\textbf{x}\strut
\end{minipage} & \begin{minipage}[t]{0.85\columnwidth}\raggedright
x position of grid square\strut
\end{minipage}\tabularnewline
\begin{minipage}[t]{0.09\columnwidth}\raggedright
\textbf{y}\strut
\end{minipage} & \begin{minipage}[t]{0.85\columnwidth}\raggedright
y position of grid square\strut
\end{minipage}\tabularnewline
\begin{minipage}[t]{0.09\columnwidth}\raggedright
\textbf{cases}\strut
\end{minipage} & \begin{minipage}[t]{0.85\columnwidth}\raggedright
Number of cases found in sampled communities in.program Number of cases
(from above) enrolled in the TFP\strut
\end{minipage}\tabularnewline
\bottomrule
\end{longtable}

Retrieve the dataset:

\begin{Shaded}
\begin{Highlighting}[]
\NormalTok{cs <-}\StringTok{ }\KeywordTok{read.table}\NormalTok{(}\StringTok{"cover.dat"}\NormalTok{, }\DataTypeTok{header =} \OtherTok{TRUE}\NormalTok{)}
\end{Highlighting}
\end{Shaded}

Examine the dataset:

\begin{Shaded}
\begin{Highlighting}[]
\NormalTok{cs}
\end{Highlighting}
\end{Shaded}

\begin{verbatim}
##    x  y cases in.program
## 1  1  7     7          2
## 2  2  5     4          0
## 3  2  6     4          1
## 4  2  7     3          1
## 5  2  8     3          1
## 6  2  9     5          1
## 7  3  3     3          0
## 8  3  4     2          0
## 9  3  5     3          0
## 10 3  6     3          1
## 11 3  7     5          2
## 12 3  8     2          0
## 13 3  9     4          1
## 14 3 10     5          0
## 15 4  4     5          2
## 16 4  5     8          1
## 17 4  6     6          0
## 18 4  7     6          1
## 19 4  8     3          1
## 20 4  9     5          1
## 21 4 10     6          2
## 22 5  4     5          1
## 23 5  5     3          1
## 24 5  6     4          0
## 25 5  7     6          3
## 26 5  8     4          0
## 27 6  3     8          2
## 28 6  4     5          2
## 29 6  6     6          1
## 30 6  7     3          1
\end{verbatim}

We should calculate the observed coverage for each grid square:

\begin{Shaded}
\begin{Highlighting}[]
\NormalTok{cs}\OperatorTok{$}\NormalTok{cvr <-}\StringTok{ }\NormalTok{cs}\OperatorTok{$}\NormalTok{in.program }\OperatorTok{/}\StringTok{ }\NormalTok{cs}\OperatorTok{$}\NormalTok{cases}
\NormalTok{cs}
\end{Highlighting}
\end{Shaded}

\begin{verbatim}
##    x  y cases in.program       cvr
## 1  1  7     7          2 0.2857143
## 2  2  5     4          0 0.0000000
## 3  2  6     4          1 0.2500000
## 4  2  7     3          1 0.3333333
## 5  2  8     3          1 0.3333333
## 6  2  9     5          1 0.2000000
## 7  3  3     3          0 0.0000000
## 8  3  4     2          0 0.0000000
## 9  3  5     3          0 0.0000000
## 10 3  6     3          1 0.3333333
## 11 3  7     5          2 0.4000000
## 12 3  8     2          0 0.0000000
## 13 3  9     4          1 0.2500000
## 14 3 10     5          0 0.0000000
## 15 4  4     5          2 0.4000000
## 16 4  5     8          1 0.1250000
## 17 4  6     6          0 0.0000000
## 18 4  7     6          1 0.1666667
## 19 4  8     3          1 0.3333333
## 20 4  9     5          1 0.2000000
## 21 4 10     6          2 0.3333333
## 22 5  4     5          1 0.2000000
## 23 5  5     3          1 0.3333333
## 24 5  6     4          0 0.0000000
## 25 5  7     6          3 0.5000000
## 26 5  8     4          0 0.0000000
## 27 6  3     8          2 0.2500000
## 28 6  4     5          2 0.4000000
## 29 6  6     6          1 0.1666667
## 30 6  7     3          1 0.3333333
\end{verbatim}

Note that some grid squares have zero coverage. It might be useful to
use specific plotting characters (e.g.~open and filled squares) to
indicate zero and non-zero coverage. We can use the \texttt{ifelse()}
function to do this:

\begin{Shaded}
\begin{Highlighting}[]
\NormalTok{cs}\OperatorTok{$}\NormalTok{cvr.pch <-}\StringTok{ }\KeywordTok{ifelse}\NormalTok{(cs}\OperatorTok{$}\NormalTok{cvr }\OperatorTok{==}\StringTok{ }\DecValTok{0}\NormalTok{, }\DecValTok{0}\NormalTok{, }\DecValTok{15}\NormalTok{)}
\NormalTok{cs}
\end{Highlighting}
\end{Shaded}

\begin{verbatim}
##    x  y cases in.program       cvr cvr.pch
## 1  1  7     7          2 0.2857143      15
## 2  2  5     4          0 0.0000000       0
## 3  2  6     4          1 0.2500000      15
## 4  2  7     3          1 0.3333333      15
## 5  2  8     3          1 0.3333333      15
## 6  2  9     5          1 0.2000000      15
## 7  3  3     3          0 0.0000000       0
## 8  3  4     2          0 0.0000000       0
## 9  3  5     3          0 0.0000000       0
## 10 3  6     3          1 0.3333333      15
## 11 3  7     5          2 0.4000000      15
## 12 3  8     2          0 0.0000000       0
## 13 3  9     4          1 0.2500000      15
## 14 3 10     5          0 0.0000000       0
## 15 4  4     5          2 0.4000000      15
## 16 4  5     8          1 0.1250000      15
## 17 4  6     6          0 0.0000000       0
## 18 4  7     6          1 0.1666667      15
## 19 4  8     3          1 0.3333333      15
## 20 4  9     5          1 0.2000000      15
## 21 4 10     6          2 0.3333333      15
## 22 5  4     5          1 0.2000000      15
## 23 5  5     3          1 0.3333333      15
## 24 5  6     4          0 0.0000000       0
## 25 5  7     6          3 0.5000000      15
## 26 5  8     4          0 0.0000000       0
## 27 6  3     8          2 0.2500000      15
## 28 6  4     5          2 0.4000000      15
## 29 6  6     6          1 0.1666667      15
## 30 6  7     3          1 0.3333333      15
\end{verbatim}

A quick way of seeing the code associated with each plotting symbol is:

\begin{Shaded}
\begin{Highlighting}[]
\KeywordTok{plot}\NormalTok{(}\DecValTok{0}\OperatorTok{:}\DecValTok{25}\NormalTok{, }\DecValTok{0}\OperatorTok{:}\DecValTok{25}\NormalTok{, }\DataTypeTok{pch =} \DecValTok{0}\OperatorTok{:}\DecValTok{25}\NormalTok{, }\DataTypeTok{cex =} \DecValTok{2}\NormalTok{)}
\end{Highlighting}
\end{Shaded}

\includegraphics{prfe_files/figure-latex/unnamed-chunk-428-1.pdf}

The size of the plotting symbol may be used to indicate the level of
coverage in each quadrat but we must ensure that the symbol used for
zero-coverage is not invisibly small:

\begin{Shaded}
\begin{Highlighting}[]
\NormalTok{cs}\OperatorTok{$}\NormalTok{cvr.cex <-}\StringTok{ }\KeywordTok{ifelse}\NormalTok{(cs}\OperatorTok{$}\NormalTok{cvr }\OperatorTok{==}\StringTok{ }\DecValTok{0}\NormalTok{, }\DecValTok{1}\NormalTok{, }\DecValTok{10} \OperatorTok{*}\StringTok{ }\NormalTok{cs}\OperatorTok{$}\NormalTok{cvr)}
\NormalTok{cs}
\end{Highlighting}
\end{Shaded}

\begin{verbatim}
##    x  y cases in.program       cvr cvr.pch  cvr.cex
## 1  1  7     7          2 0.2857143      15 2.857143
## 2  2  5     4          0 0.0000000       0 1.000000
## 3  2  6     4          1 0.2500000      15 2.500000
## 4  2  7     3          1 0.3333333      15 3.333333
## 5  2  8     3          1 0.3333333      15 3.333333
## 6  2  9     5          1 0.2000000      15 2.000000
## 7  3  3     3          0 0.0000000       0 1.000000
## 8  3  4     2          0 0.0000000       0 1.000000
## 9  3  5     3          0 0.0000000       0 1.000000
## 10 3  6     3          1 0.3333333      15 3.333333
## 11 3  7     5          2 0.4000000      15 4.000000
## 12 3  8     2          0 0.0000000       0 1.000000
## 13 3  9     4          1 0.2500000      15 2.500000
## 14 3 10     5          0 0.0000000       0 1.000000
## 15 4  4     5          2 0.4000000      15 4.000000
## 16 4  5     8          1 0.1250000      15 1.250000
## 17 4  6     6          0 0.0000000       0 1.000000
## 18 4  7     6          1 0.1666667      15 1.666667
## 19 4  8     3          1 0.3333333      15 3.333333
## 20 4  9     5          1 0.2000000      15 2.000000
## 21 4 10     6          2 0.3333333      15 3.333333
## 22 5  4     5          1 0.2000000      15 2.000000
## 23 5  5     3          1 0.3333333      15 3.333333
## 24 5  6     4          0 0.0000000       0 1.000000
## 25 5  7     6          3 0.5000000      15 5.000000
## 26 5  8     4          0 0.0000000       0 1.000000
## 27 6  3     8          2 0.2500000      15 2.500000
## 28 6  4     5          2 0.4000000      15 4.000000
## 29 6  6     6          1 0.1666667      15 1.666667
## 30 6  7     3          1 0.3333333      15 3.333333
\end{verbatim}

We can now plot the data:

\begin{Shaded}
\begin{Highlighting}[]
\KeywordTok{par}\NormalTok{(}\DataTypeTok{pty=}\StringTok{"s"}\NormalTok{)}
\KeywordTok{plot}\NormalTok{(cs}\OperatorTok{$}\NormalTok{x, cs}\OperatorTok{$}\NormalTok{y, }\DataTypeTok{cex =}\NormalTok{ cs}\OperatorTok{$}\NormalTok{cvr.cex, }\DataTypeTok{pch =}\NormalTok{ cs}\OperatorTok{$}\NormalTok{cvr.pch)}
\end{Highlighting}
\end{Shaded}

\includegraphics{prfe_files/figure-latex/unnamed-chunk-430-1.pdf}

There are some problems with this plot:

\begin{itemize}
\item
  The axes and labels distract from the data.
\item
  The distance between grid-square centres is wider in the \emph{x} than
  in the \emph{y} direction.
\item
  The colour (black) of the plotting symbols is too strong.
\end{itemize}

All of these problems can be fixed by specifying values for
\texttt{plot()} function parameters:

\begin{Shaded}
\begin{Highlighting}[]
\KeywordTok{plot}\NormalTok{(cs}\OperatorTok{$}\NormalTok{x, cs}\OperatorTok{$}\NormalTok{y,}
     \DataTypeTok{cex =}\NormalTok{ cs}\OperatorTok{$}\NormalTok{cvr.cex,}
     \DataTypeTok{pch =}\NormalTok{ cs}\OperatorTok{$}\NormalTok{cvr.pch,}
     \DataTypeTok{xlab =} \StringTok{""}\NormalTok{,}
     \DataTypeTok{ylab =} \StringTok{""}\NormalTok{,}
     \DataTypeTok{axes =} \OtherTok{FALSE}\NormalTok{,}
     \DataTypeTok{xlim =} \KeywordTok{c}\NormalTok{(}\DecValTok{0}\NormalTok{,}\DecValTok{10}\NormalTok{),}
     \DataTypeTok{ylim =} \KeywordTok{c}\NormalTok{(}\DecValTok{0}\NormalTok{,}\DecValTok{10}\NormalTok{),}
     \DataTypeTok{col =} \KeywordTok{gray}\NormalTok{(}\FloatTok{0.5}\NormalTok{))}
\end{Highlighting}
\end{Shaded}

\includegraphics{prfe_files/figure-latex/unnamed-chunk-431-1.pdf}

An alternative way of plotting this data is to use shades of grey
(rather than the size of the plotting symbol) to represent the level of
coverage in each grid-square:

\begin{Shaded}
\begin{Highlighting}[]
\KeywordTok{plot}\NormalTok{(cs}\OperatorTok{$}\NormalTok{x, cs}\OperatorTok{$}\NormalTok{y,}
     \DataTypeTok{cex =} \DecValTok{5}\NormalTok{,}
     \DataTypeTok{xlab =} \StringTok{""}\NormalTok{,}
     \DataTypeTok{ylab =} \StringTok{""}\NormalTok{,}
     \DataTypeTok{axes =} \OtherTok{FALSE}\NormalTok{,}
     \DataTypeTok{pch =} \DecValTok{15}\NormalTok{,}
     \DataTypeTok{xlim =} \KeywordTok{c}\NormalTok{(}\DecValTok{0}\NormalTok{, }\DecValTok{10}\NormalTok{),}
     \DataTypeTok{ylim =} \KeywordTok{c}\NormalTok{(}\DecValTok{0}\NormalTok{, }\DecValTok{10}\NormalTok{),}
     \DataTypeTok{col =} \KeywordTok{gray}\NormalTok{(}\DecValTok{1} \OperatorTok{-}\StringTok{ }\NormalTok{cs}\OperatorTok{$}\NormalTok{cvr))}
\end{Highlighting}
\end{Shaded}

\includegraphics{prfe_files/figure-latex/unnamed-chunk-432-1.pdf}

In this context it is useful to show the approximate location of
therapeutic feeding centres:

\begin{Shaded}
\begin{Highlighting}[]
\KeywordTok{plot}\NormalTok{(cs}\OperatorTok{$}\NormalTok{x, cs}\OperatorTok{$}\NormalTok{y,}
     \DataTypeTok{cex =}\NormalTok{ cs}\OperatorTok{$}\NormalTok{cvr.cex,}
     \DataTypeTok{pch =}\NormalTok{ cs}\OperatorTok{$}\NormalTok{cvr.pch,}
     \DataTypeTok{xlab =} \StringTok{""}\NormalTok{,}
     \DataTypeTok{ylab =} \StringTok{""}\NormalTok{,}
     \DataTypeTok{axes =} \OtherTok{FALSE}\NormalTok{,}
     \DataTypeTok{xlim =} \KeywordTok{c}\NormalTok{(}\DecValTok{0}\NormalTok{,}\DecValTok{10}\NormalTok{),}
     \DataTypeTok{ylim =} \KeywordTok{c}\NormalTok{(}\DecValTok{0}\NormalTok{,}\DecValTok{10}\NormalTok{),}
     \DataTypeTok{col =} \KeywordTok{gray}\NormalTok{(}\FloatTok{0.5}\NormalTok{))}

\KeywordTok{points}\NormalTok{(}\KeywordTok{c}\NormalTok{(}\FloatTok{2.5}\NormalTok{, }\FloatTok{4.5}\NormalTok{, }\FloatTok{5.5}\NormalTok{), }\KeywordTok{c}\NormalTok{(}\FloatTok{6.5}\NormalTok{, }\FloatTok{8.25}\NormalTok{, }\DecValTok{5}\NormalTok{), }\DataTypeTok{pch =} \DecValTok{19}\NormalTok{, }\DataTypeTok{cex =} \DecValTok{2}\NormalTok{)}
\end{Highlighting}
\end{Shaded}

\includegraphics{prfe_files/figure-latex/unnamed-chunk-433-1.pdf}

The techniques introduced in this section allow you to write custom
graphical functions but they can also be used to change the default
behaviour of standard graphical functions.

\hypertarget{combining-plots}{%
\section{Combining plots}\label{combining-plots}}

In \protect\hyperlink{exercise1}{\textbf{\emph{Exercise 1 Getting
acquainted with R}}}, we saw how the \texttt{plot()} function could be
applied to a fitted object:

\begin{Shaded}
\begin{Highlighting}[]
\NormalTok{fem <-}\StringTok{ }\KeywordTok{read.table}\NormalTok{(}\StringTok{"fem.dat"}\NormalTok{, }\DataTypeTok{header =} \OtherTok{TRUE}\NormalTok{)}
\KeywordTok{attach}\NormalTok{(fem)}
\end{Highlighting}
\end{Shaded}

\begin{verbatim}
## The following objects are masked from fem (pos = 4):
## 
##     AGE, ANX, DEP, ID, IQ, LIFE, SEX, SLP, WT
\end{verbatim}

\begin{verbatim}
## The following objects are masked from fem (pos = 5):
## 
##     AGE, ANX, DEP, ID, IQ, LIFE, SEX, SLP, WT
\end{verbatim}

\begin{verbatim}
## The following objects are masked from fem (pos = 9):
## 
##     AGE, ANX, DEP, ID, IQ, LIFE, SEX, SLP, WT
\end{verbatim}

\begin{verbatim}
## The following objects are masked from fem (pos = 10):
## 
##     AGE, ANX, DEP, ID, IQ, LIFE, SEX, SLP, WT
\end{verbatim}

\begin{verbatim}
## The following objects are masked from fem (pos = 12):
## 
##     AGE, ANX, DEP, ID, IQ, LIFE, SEX, SLP, WT
\end{verbatim}

\begin{verbatim}
## The following objects are masked from fem (pos = 13):
## 
##     AGE, ANX, DEP, ID, IQ, LIFE, SEX, SLP, WT
\end{verbatim}

\begin{verbatim}
## The following objects are masked from fem (pos = 18):
## 
##     AGE, ANX, DEP, ID, IQ, LIFE, SEX, SLP, WT
\end{verbatim}

\begin{Shaded}
\begin{Highlighting}[]
\NormalTok{fem.lm <-}\StringTok{ }\KeywordTok{lm}\NormalTok{(WT }\OperatorTok{~}\StringTok{ }\NormalTok{AGE)}
\KeywordTok{plot}\NormalTok{(fem.lm)}
\end{Highlighting}
\end{Shaded}

\includegraphics{prfe_files/figure-latex/unnamed-chunk-434-1.pdf}
\includegraphics{prfe_files/figure-latex/unnamed-chunk-434-2.pdf}
\includegraphics{prfe_files/figure-latex/unnamed-chunk-434-3.pdf}
\includegraphics{prfe_files/figure-latex/unnamed-chunk-434-4.pdf}

Each of the diagnostic plots are presented as a separate chart. We could
use the \texttt{mfrow} parameter of the \texttt{par()} function to
present all four diagnostic plots on a single chart:

\begin{Shaded}
\begin{Highlighting}[]
\KeywordTok{par}\NormalTok{(}\DataTypeTok{mfrow =} \KeywordTok{c}\NormalTok{(}\DecValTok{2}\NormalTok{, }\DecValTok{2}\NormalTok{))}
\KeywordTok{plot}\NormalTok{(fem.lm)}
\end{Highlighting}
\end{Shaded}

\includegraphics{prfe_files/figure-latex/unnamed-chunk-435-1.pdf}

It might improve the appearance of the chart if each of the diagnostic
plots were square rather than rectangular:

\begin{Shaded}
\begin{Highlighting}[]
\KeywordTok{par}\NormalTok{(}\DataTypeTok{mfrow =} \KeywordTok{c}\NormalTok{(}\DecValTok{2}\NormalTok{, }\DecValTok{2}\NormalTok{))}
\KeywordTok{par}\NormalTok{(}\DataTypeTok{pty =} \StringTok{"s"}\NormalTok{)}
\KeywordTok{plot}\NormalTok{(fem.lm)}
\end{Highlighting}
\end{Shaded}

\includegraphics{prfe_files/figure-latex/unnamed-chunk-436-1.pdf}

It might improve the appearance of the chart if smaller text and symbols
were used:

\begin{Shaded}
\begin{Highlighting}[]
\KeywordTok{par}\NormalTok{(}\DataTypeTok{mfrow =} \KeywordTok{c}\NormalTok{(}\DecValTok{2}\NormalTok{, }\DecValTok{2}\NormalTok{))}
\KeywordTok{par}\NormalTok{(}\DataTypeTok{pty =} \StringTok{"s"}\NormalTok{)}
\KeywordTok{par}\NormalTok{(}\DataTypeTok{cex =} \FloatTok{0.5}\NormalTok{)}
\KeywordTok{plot}\NormalTok{(fem.lm)}
\end{Highlighting}
\end{Shaded}

\includegraphics{prfe_files/figure-latex/unnamed-chunk-437-1.pdf}

Graphical parameters set using the \texttt{par()} function affect all
subsequent plot commands and must be reset explicitly:

\begin{Shaded}
\begin{Highlighting}[]
\KeywordTok{par}\NormalTok{(}\DataTypeTok{mfrow =} \KeywordTok{c}\NormalTok{(}\DecValTok{1}\NormalTok{, }\DecValTok{1}\NormalTok{), }\DataTypeTok{pty =} \StringTok{"m"}\NormalTok{, }\DataTypeTok{cex =} \DecValTok{1}\NormalTok{)}
\KeywordTok{plot}\NormalTok{(fem.lm)}
\end{Highlighting}
\end{Shaded}

\includegraphics{prfe_files/figure-latex/unnamed-chunk-438-1.pdf}
\includegraphics{prfe_files/figure-latex/unnamed-chunk-438-2.pdf}
\includegraphics{prfe_files/figure-latex/unnamed-chunk-438-3.pdf}
\includegraphics{prfe_files/figure-latex/unnamed-chunk-438-4.pdf}

It is possible to save graphical parameters into an \texttt{R} object
and use this object to restore original graphical parameters:

\begin{Shaded}
\begin{Highlighting}[]
\NormalTok{old.par <-}\StringTok{ }\KeywordTok{par}\NormalTok{()}
\KeywordTok{par}\NormalTok{(}\DataTypeTok{mfrow =} \KeywordTok{c}\NormalTok{(}\DecValTok{2}\NormalTok{, }\DecValTok{2}\NormalTok{), }\DataTypeTok{pty =} \StringTok{"s"}\NormalTok{)}
\KeywordTok{plot}\NormalTok{(fem.lm)}
\KeywordTok{par}\NormalTok{(old.par)}
\KeywordTok{plot}\NormalTok{(fem.lm)}
\end{Highlighting}
\end{Shaded}

\includegraphics{prfe_files/figure-latex/unnamed-chunk-440-1.pdf}

\includegraphics{prfe_files/figure-latex/unnamed-chunk-441-1.pdf}
\includegraphics{prfe_files/figure-latex/unnamed-chunk-441-2.pdf}
\includegraphics{prfe_files/figure-latex/unnamed-chunk-441-3.pdf}
\includegraphics{prfe_files/figure-latex/unnamed-chunk-441-4.pdf}

The ability to save and apply graphical parameters means that you can
create a library of graphical parameter sets that can be applied with
the \texttt{par()} function as required:

\begin{Shaded}
\begin{Highlighting}[]
\NormalTok{default.par <-}\StringTok{ }\KeywordTok{par}\NormalTok{()}
\KeywordTok{par}\NormalTok{(}\DataTypeTok{mfrow =} \KeywordTok{c}\NormalTok{(}\DecValTok{2}\NormalTok{, }\DecValTok{2}\NormalTok{), }\DataTypeTok{pty =} \StringTok{"s"}\NormalTok{)}
\NormalTok{plot.lm.par <-}\StringTok{ }\KeywordTok{par}\NormalTok{()}
\KeywordTok{par}\NormalTok{(default.par)}
\KeywordTok{plot}\NormalTok{(fem.lm)}
\KeywordTok{par}\NormalTok{(plot.lm.par)}
\KeywordTok{plot}\NormalTok{(fem.lm)}
\KeywordTok{par}\NormalTok{(default.par)}
\KeywordTok{plot}\NormalTok{(fem.lm)}
\end{Highlighting}
\end{Shaded}

\includegraphics{prfe_files/figure-latex/unnamed-chunk-443-1.pdf}
\includegraphics{prfe_files/figure-latex/unnamed-chunk-443-2.pdf}
\includegraphics{prfe_files/figure-latex/unnamed-chunk-443-3.pdf}
\includegraphics{prfe_files/figure-latex/unnamed-chunk-443-4.pdf}

\includegraphics{prfe_files/figure-latex/unnamed-chunk-444-1.pdf}

\includegraphics{prfe_files/figure-latex/unnamed-chunk-445-1.pdf}
\includegraphics{prfe_files/figure-latex/unnamed-chunk-445-2.pdf}
\includegraphics{prfe_files/figure-latex/unnamed-chunk-445-3.pdf}
\includegraphics{prfe_files/figure-latex/unnamed-chunk-445-4.pdf}

\texttt{R} produces warning messages when you save and restore graphical
parameters in this way. This is because some graphical parameters are
read only and cannot be changed using the \texttt{par()} function. This
has no effect other than to cause \texttt{R} to issue warning messages.

If you do not like the warning messages then you can use the
\texttt{par()} function with the \texttt{no.readonly} parameter set to
\texttt{TRUE}:

\begin{Shaded}
\begin{Highlighting}[]
\NormalTok{default.par <-}\StringTok{ }\KeywordTok{par}\NormalTok{(}\DataTypeTok{no.readonly =} \OtherTok{TRUE}\NormalTok{)}
\KeywordTok{par}\NormalTok{(}\DataTypeTok{mfrow =} \KeywordTok{c}\NormalTok{(}\DecValTok{2}\NormalTok{, }\DecValTok{2}\NormalTok{), }\DataTypeTok{pty =} \StringTok{"s"}\NormalTok{)}
\NormalTok{plot.lm.par <-}\StringTok{ }\KeywordTok{par}\NormalTok{(}\DataTypeTok{no.readonly =} \OtherTok{TRUE}\NormalTok{)}
\KeywordTok{par}\NormalTok{(default.par)}
\KeywordTok{plot}\NormalTok{(fem.lm)}
\end{Highlighting}
\end{Shaded}

\includegraphics{prfe_files/figure-latex/unnamed-chunk-446-1.pdf}
\includegraphics{prfe_files/figure-latex/unnamed-chunk-446-2.pdf}
\includegraphics{prfe_files/figure-latex/unnamed-chunk-446-3.pdf}
\includegraphics{prfe_files/figure-latex/unnamed-chunk-446-4.pdf}

\begin{Shaded}
\begin{Highlighting}[]
\KeywordTok{par}\NormalTok{(plot.lm.par)}
\KeywordTok{plot}\NormalTok{(fem.lm)}
\end{Highlighting}
\end{Shaded}

\includegraphics{prfe_files/figure-latex/unnamed-chunk-446-5.pdf}

\begin{Shaded}
\begin{Highlighting}[]
\KeywordTok{par}\NormalTok{(default.par)}
\KeywordTok{plot}\NormalTok{(fem.lm)}
\end{Highlighting}
\end{Shaded}

\includegraphics{prfe_files/figure-latex/unnamed-chunk-446-6.pdf}
\includegraphics{prfe_files/figure-latex/unnamed-chunk-446-7.pdf}
\includegraphics{prfe_files/figure-latex/unnamed-chunk-446-8.pdf}
\includegraphics{prfe_files/figure-latex/unnamed-chunk-446-9.pdf}

Graphical parameter sets, like any other \texttt{R} object, may be saved
and loaded using the \texttt{save()} and \texttt{load()} functions.

\hypertarget{summary-7}{%
\section{Summary}\label{summary-7}}

\begin{itemize}
\item
  \texttt{R} allows you to create functions that produce graphical
  output.
\item
  \texttt{R} graphical functions are flexible so that you can create
  functions that can produce chart types that are not available in
  \texttt{R} or many other statistical applications. Standard plots may
  also be customised using the \texttt{par()} function.
\item
  \texttt{R} allows you to specify default values for function
  parameters making functions calls easier by removing the requirement
  to specify values for every function parameter.
\end{itemize}

\hypertarget{exercise9}{%
\chapter{Computer intensive methods}\label{exercise9}}

\hypertarget{estimation}{%
\section{Estimation}\label{estimation}}

Estimation involves the calculation of a measure with some sense of
precision based upon sampling variation.

Only a few estimators (e.g.~the sample mean from a normal population)
have exact formulae that may be used to estimate sampling variation.
Typically, estimates of variability are based upon approximations
informed by expected or postulated properties of the sampled population.
The development of variance formulae for some measures may require
in-depth statistical and mathematical knowledge or may even be
impossible to derive.

\emph{Bootstrap} methods are computer-intensive methods that can provide
estimates and measures of precision (e.g.~confidence intervals) without
resort to theoretical models, higher mathematics, or assumptions about
the sampled population. They rely on repeated sampling, sometimes called
\emph{resampling}, of the observed data.

As a simple example of how such methods work, we will start by using
bootstrap methods to estimate the mean from a normal population. We will
work with a very simple dataset which we will enter directly:

\begin{Shaded}
\begin{Highlighting}[]
\NormalTok{x <-}\StringTok{ }\KeywordTok{c}\NormalTok{(}\FloatTok{7.3}\NormalTok{, }\FloatTok{10.4}\NormalTok{, }\FloatTok{14.0}\NormalTok{, }\FloatTok{12.2}\NormalTok{, }\FloatTok{8.4}\NormalTok{)}
\end{Highlighting}
\end{Shaded}

We can summarise this data quite easily:

\begin{Shaded}
\begin{Highlighting}[]
\KeywordTok{mean}\NormalTok{(x)}
\end{Highlighting}
\end{Shaded}

\begin{verbatim}
## [1] 10.46
\end{verbatim}

The \texttt{sample()} function can be used to select a bootstrap
\texttt{replicate}:

\begin{Shaded}
\begin{Highlighting}[]
\KeywordTok{sample}\NormalTok{(x, }\KeywordTok{length}\NormalTok{(x), }\DataTypeTok{replace =} \OtherTok{TRUE}\NormalTok{)}
\end{Highlighting}
\end{Shaded}

\begin{verbatim}
## [1] 14.0  8.4  8.4 10.4 12.2
\end{verbatim}

Enter this command several times to see some more bootstrap replicates.
Remember that previous commands can be recalled and edited using the up
and down arrow keys -- they do not need to be typed out in full each
time. The \texttt{length()} parameter is not required for taking
bootstrap replicates and can be omitted.

It is possible to apply a summary measure to a replicate:

\begin{Shaded}
\begin{Highlighting}[]
\KeywordTok{mean}\NormalTok{(}\KeywordTok{sample}\NormalTok{(x, }\DataTypeTok{replace =} \OtherTok{TRUE}\NormalTok{))}
\end{Highlighting}
\end{Shaded}

\begin{verbatim}
## [1] 10.06
\end{verbatim}

Enter this command several times. A bootstrap estimate of the mean of
\texttt{x} can be made by repeating this process many times and taking
the \texttt{median} of the means for each replicate.

One way of doing this is to create a matrix where each column contains a
bootstrap replicate and then use the \texttt{apply()} and
\texttt{mean()} functions to get at the estimate.

First create the matrix of replicates. Here we take ten replicates:

\begin{Shaded}
\begin{Highlighting}[]
\NormalTok{x1 <-}\StringTok{ }\KeywordTok{matrix}\NormalTok{(}\KeywordTok{sample}\NormalTok{(x, }\KeywordTok{length}\NormalTok{(x) }\OperatorTok{*}\StringTok{ }\DecValTok{10}\NormalTok{, }\DataTypeTok{replace =} \OtherTok{TRUE}\NormalTok{),}
             \DataTypeTok{nrow =} \KeywordTok{length}\NormalTok{(x), }\DataTypeTok{ncol =} \DecValTok{10}\NormalTok{)}
\NormalTok{x1}
\end{Highlighting}
\end{Shaded}

\begin{verbatim}
##      [,1] [,2] [,3] [,4] [,5] [,6] [,7] [,8] [,9] [,10]
## [1,] 14.0 14.0 10.4 10.4  7.3 12.2 12.2  8.4  8.4  10.4
## [2,] 14.0 10.4 12.2  7.3 10.4 14.0  7.3  8.4 12.2  12.2
## [3,] 12.2 12.2 12.2 12.2 10.4 12.2  8.4  8.4  7.3  10.4
## [4,] 10.4 12.2 12.2  7.3 10.4 10.4 12.2  8.4 12.2  12.2
## [5,] 12.2  8.4 10.4  7.3 10.4 10.4 10.4 10.4 12.2   7.3
\end{verbatim}

Then calculate and store the means of each replicate. We can do this
using the \texttt{apply()} function to apply the \texttt{mean()}
function to the columns of matrix \texttt{x1}:

\begin{Shaded}
\begin{Highlighting}[]
\NormalTok{x2 <-}\StringTok{ }\KeywordTok{apply}\NormalTok{(x1, }\DecValTok{2}\NormalTok{, mean)}
\NormalTok{x2}
\end{Highlighting}
\end{Shaded}

\begin{verbatim}
##  [1] 12.56 11.44 11.48  8.90  9.78 11.84 10.10  8.80 10.46 10.50
\end{verbatim}

The bootstrap estimate of the mean is:

\begin{Shaded}
\begin{Highlighting}[]
\KeywordTok{median}\NormalTok{(x2)}
\end{Highlighting}
\end{Shaded}

\begin{verbatim}
## [1] 10.48
\end{verbatim}

The bootstrap estimate may differ somewhat from the mean of \texttt{x}:

\begin{Shaded}
\begin{Highlighting}[]
\KeywordTok{mean}\NormalTok{(x)}
\end{Highlighting}
\end{Shaded}

\begin{verbatim}
## [1] 10.46
\end{verbatim}

The situation is improved by increasing the number of replicates. Here
we take 5000 replicates:

\begin{Shaded}
\begin{Highlighting}[]
\NormalTok{x1 <-}\StringTok{ }\KeywordTok{matrix}\NormalTok{(}\KeywordTok{sample}\NormalTok{(x, }\KeywordTok{length}\NormalTok{(x) }\OperatorTok{*}\StringTok{ }\DecValTok{5000}\NormalTok{, }\DataTypeTok{replace =} \OtherTok{TRUE}\NormalTok{),}
             \DataTypeTok{nrow =} \KeywordTok{length}\NormalTok{(x), }\DataTypeTok{ncol =} \DecValTok{5000}\NormalTok{)}
\NormalTok{x2 <-}\StringTok{ }\KeywordTok{apply}\NormalTok{(x1, }\DecValTok{2}\NormalTok{, mean)}
\KeywordTok{median}\NormalTok{(x2)}
\end{Highlighting}
\end{Shaded}

\begin{verbatim}
## [1] 10.46
\end{verbatim}

This is a pretty useless example as estimating the mean / standard
deviation, or standard error of the mean of a sample from a normal
population can be done using standard formulae.

The utility of bootstrap methods is that they can be applied to summary
measures that are not as well understood as the arithmetic mean. The
bootstrap method also has the advantage of retaining simplicity even
with complicated measures.

To illustrate this, we will work through an example of using the
bootstrap to estimate the harmonic mean.

Again, we will work with a simple dataset which we will enter directly:

\begin{Shaded}
\begin{Highlighting}[]
\NormalTok{d <-}\StringTok{ }\KeywordTok{c}\NormalTok{(}\FloatTok{43.64}\NormalTok{, }\FloatTok{50.67}\NormalTok{, }\FloatTok{33.56}\NormalTok{, }\FloatTok{27.75}\NormalTok{, }\FloatTok{43.35}\NormalTok{, }\FloatTok{29.56}\NormalTok{, }\FloatTok{38.83}\NormalTok{, }\FloatTok{35.95}\NormalTok{, }\FloatTok{20.01}\NormalTok{)}
\end{Highlighting}
\end{Shaded}

The data represents distance (in kilometres) from a point source of
environmental pollution for nine female patients with oral / pharyngeal
cancer.

The harmonic mean is considered to be a sensitive measure of spatial
clustering. The first step is to construct a function to calculate the
harmonic mean:

\begin{Shaded}
\begin{Highlighting}[]
\NormalTok{h.mean <-}\StringTok{ }\ControlFlowTok{function}\NormalTok{(x) \{}\KeywordTok{length}\NormalTok{(x) }\OperatorTok{/}\StringTok{ }\KeywordTok{sum}\NormalTok{(}\DecValTok{1} \OperatorTok{/}\StringTok{ }\NormalTok{x)\}}
\end{Highlighting}
\end{Shaded}

Calling this function with the sample data:

\begin{Shaded}
\begin{Highlighting}[]
\KeywordTok{h.mean}\NormalTok{(d)}
\end{Highlighting}
\end{Shaded}

\begin{verbatim}
## [1] 33.46646
\end{verbatim}

Should return an estimated harmonic mean distance of 33.47 kilometres.
This is simple. The problem is that calculating the variance of this
estimate is complicated using standard methods. This problem is
relatively simple to solve using bootstrap methods:

\begin{Shaded}
\begin{Highlighting}[]
\NormalTok{replicates <-}\StringTok{ }\DecValTok{5000}
\NormalTok{n <-}\StringTok{ }\KeywordTok{length}\NormalTok{(d)}
\NormalTok{x1 <-}\StringTok{ }\KeywordTok{matrix}\NormalTok{(}\KeywordTok{sample}\NormalTok{(d, n }\OperatorTok{*}\StringTok{ }\NormalTok{replicates, }\DataTypeTok{replace =} \OtherTok{TRUE}\NormalTok{),}
             \DataTypeTok{nrow =}\NormalTok{ n, }\DataTypeTok{ncol =}\NormalTok{ replicates)}
\NormalTok{x2 <-}\StringTok{ }\KeywordTok{apply}\NormalTok{(x1, }\DecValTok{2}\NormalTok{, h.mean)}
\KeywordTok{median}\NormalTok{(x2)}
\end{Highlighting}
\end{Shaded}

\begin{verbatim}
## [1] 33.59034
\end{verbatim}

A 95\% confidence interval can be extracted from \texttt{x2} using the
\texttt{quantile()} function:

\begin{Shaded}
\begin{Highlighting}[]
\KeywordTok{quantile}\NormalTok{(x2, }\KeywordTok{c}\NormalTok{(}\FloatTok{0.025}\NormalTok{, }\FloatTok{0.975}\NormalTok{))}
\end{Highlighting}
\end{Shaded}

\begin{verbatim}
##     2.5%    97.5% 
## 27.89177 40.34553
\end{verbatim}

A 99\% confidence interval can also be extracted from \texttt{x2} using
the \texttt{quantile()} function:

\begin{Shaded}
\begin{Highlighting}[]
\KeywordTok{quantile}\NormalTok{(x2, }\KeywordTok{c}\NormalTok{(}\FloatTok{0.005}\NormalTok{, }\FloatTok{0.995}\NormalTok{))}
\end{Highlighting}
\end{Shaded}

\begin{verbatim}
##     0.5%    99.5% 
## 26.37928 42.20182
\end{verbatim}

As a final example of the bootstrap method we will use the method to
obtain an estimate of an odds ratio from a two-by-two table. We will
work with the \texttt{salex} dataset which we used in exercise 2 and
exercise 3:

\begin{Shaded}
\begin{Highlighting}[]
\NormalTok{salex <-}\StringTok{ }\KeywordTok{read.table}\NormalTok{(}\StringTok{"salex.dat"}\NormalTok{,  }\DataTypeTok{header =} \OtherTok{TRUE}\NormalTok{, }\DataTypeTok{na.strings =} \StringTok{"9"}\NormalTok{)}
\KeywordTok{table}\NormalTok{(salex}\OperatorTok{$}\NormalTok{EGGS, salex}\OperatorTok{$}\NormalTok{ILL)}
\end{Highlighting}
\end{Shaded}

\begin{verbatim}
##    
##      1  2
##   1 40  6
##   2 10 20
\end{verbatim}

We should set up our estimator function to calculate an odds ratio from
a two-by-two table:

\begin{Shaded}
\begin{Highlighting}[]
\NormalTok{or <-}\StringTok{ }\ControlFlowTok{function}\NormalTok{(x) \{(x[}\DecValTok{1}\NormalTok{,}\DecValTok{1}\NormalTok{] }\OperatorTok{/}\StringTok{ }\NormalTok{x[}\DecValTok{1}\NormalTok{,}\DecValTok{2}\NormalTok{]) }\OperatorTok{/}\StringTok{ }\NormalTok{(x[}\DecValTok{2}\NormalTok{,}\DecValTok{1}\NormalTok{] }\OperatorTok{/}\StringTok{ }\NormalTok{x[}\DecValTok{2}\NormalTok{,}\DecValTok{2}\NormalTok{])\}}
\end{Highlighting}
\end{Shaded}

We should test this:

\begin{Shaded}
\begin{Highlighting}[]
\KeywordTok{or}\NormalTok{(}\KeywordTok{table}\NormalTok{(salex}\OperatorTok{$}\NormalTok{EGGS, salex}\OperatorTok{$}\NormalTok{ILL))}
\end{Highlighting}
\end{Shaded}

\begin{verbatim}
## [1] 13.33333
\end{verbatim}

The problem is to take a bootstrap replicate from two vectors in a
data.frame. This can be achieved by using \texttt{sample()} to create a
vector of row indices and then use this sample of indices to select
replicates from the data.frame:

\begin{Shaded}
\begin{Highlighting}[]
\NormalTok{boot <-}\StringTok{ }\OtherTok{NULL}
\ControlFlowTok{for}\NormalTok{(i }\ControlFlowTok{in} \DecValTok{1}\OperatorTok{:}\DecValTok{1000}\NormalTok{) \{}
\NormalTok{  sampled.rows <-}\StringTok{ }\KeywordTok{sample}\NormalTok{(}\DecValTok{1}\OperatorTok{:}\KeywordTok{nrow}\NormalTok{(salex), }\DataTypeTok{replace =} \OtherTok{TRUE}\NormalTok{)}
\NormalTok{  x <-}\StringTok{ }\NormalTok{salex[sampled.rows, }\StringTok{"EGGS"}\NormalTok{]}
\NormalTok{  y <-}\StringTok{ }\NormalTok{salex[sampled.rows, }\StringTok{"ILL"}\NormalTok{]}
\NormalTok{  boot[i] <-}\StringTok{ }\KeywordTok{or}\NormalTok{(}\KeywordTok{table}\NormalTok{(x, y))}
\NormalTok{\}}
\end{Highlighting}
\end{Shaded}

The vector \texttt{boot} now contains the odds ratios calculated from
1000 replicates. Estimates of the odds ratio and its 95\% confidence
interval may be obtained using the \texttt{median()} and
\texttt{quantile()} functions

\begin{Shaded}
\begin{Highlighting}[]
\KeywordTok{median}\NormalTok{(boot)}
\KeywordTok{quantile}\NormalTok{(boot, }\KeywordTok{c}\NormalTok{(}\FloatTok{0.025}\NormalTok{, }\FloatTok{0.975}\NormalTok{))}
\end{Highlighting}
\end{Shaded}

\begin{verbatim}
## [1] 14.61538
\end{verbatim}

\begin{verbatim}
##      2.5%     97.5% 
##  4.713095 61.888839
\end{verbatim}

This approach may fail when some tables have cells that contain zero.
Infinite values arise due to division by zero when calculating the odds
ratio for some replicates. We can avoid this problem by selecting only
those values of \texttt{boot} that are not (\texttt{!=}) infinite
(\texttt{Inf}):

\begin{Shaded}
\begin{Highlighting}[]
\NormalTok{boot <-}\StringTok{ }\NormalTok{boot[boot }\OperatorTok{!=}\StringTok{ }\OtherTok{Inf}\NormalTok{]}
\KeywordTok{median}\NormalTok{(boot)}
\KeywordTok{quantile}\NormalTok{(boot, }\KeywordTok{c}\NormalTok{(}\FloatTok{0.025}\NormalTok{, }\FloatTok{0.975}\NormalTok{))}
\end{Highlighting}
\end{Shaded}

\begin{verbatim}
## [1] 14.61538
\end{verbatim}

\begin{verbatim}
##      2.5%     97.5% 
##  4.711905 61.875000
\end{verbatim}

Another way to avoid this problem is to use an
\texttt{adjusted\ odds\ ratio} calculated by adding 0.5 to each cell of
the two-by-two table:

\begin{Shaded}
\begin{Highlighting}[]
\NormalTok{boot <-}\StringTok{ }\OtherTok{NULL}
\ControlFlowTok{for}\NormalTok{(i }\ControlFlowTok{in} \DecValTok{1}\OperatorTok{:}\DecValTok{1000}\NormalTok{) \{}
\NormalTok{  sampled.rows <-}\StringTok{ }\KeywordTok{sample}\NormalTok{(}\DecValTok{1}\OperatorTok{:}\KeywordTok{nrow}\NormalTok{(salex), }\DataTypeTok{replace =} \OtherTok{TRUE}\NormalTok{)}
\NormalTok{  x <-}\StringTok{ }\NormalTok{salex[sampled.rows, }\StringTok{"EGGS"}\NormalTok{]}
\NormalTok{  y <-}\StringTok{ }\NormalTok{salex[sampled.rows, }\StringTok{"ILL"}\NormalTok{]}
\NormalTok{  boot[i] <-}\StringTok{ }\KeywordTok{or}\NormalTok{(}\KeywordTok{table}\NormalTok{(x, y) }\OperatorTok{+}\StringTok{ }\FloatTok{0.5}\NormalTok{)}
\NormalTok{  \}}
\KeywordTok{median}\NormalTok{(boot)}
\KeywordTok{quantile}\NormalTok{(boot, }\KeywordTok{c}\NormalTok{(}\FloatTok{0.025}\NormalTok{, }\FloatTok{0.975}\NormalTok{))}
\end{Highlighting}
\end{Shaded}

\begin{verbatim}
## [1] 12.47368
\end{verbatim}

\begin{verbatim}
##      2.5%     97.5% 
##  4.550267 47.061816
\end{verbatim}

This procedure is preferred when working with sparse tables.

\hypertarget{hypothesis-testing}{%
\section{Hypothesis testing}\label{hypothesis-testing}}

Computer-intensive methods also offer a general approach to statistical
hypothesis testing. To illustrate this we will use \emph{computer based
simulation} to investigate spatial clustering around a point.

Before continuing, we will retrieve a dataset:

\begin{Shaded}
\begin{Highlighting}[]
\NormalTok{waste <-}\StringTok{ }\KeywordTok{read.table}\NormalTok{(}\StringTok{"waste.dat"}\NormalTok{, }\DataTypeTok{header =} \OtherTok{TRUE}\NormalTok{)}
\end{Highlighting}
\end{Shaded}

The file \texttt{waste.dat} contains the location of twenty-three recent
cases of childhood cancer in 5 by 5 km square surrounding an industrial
waste disposal site. The columns in the dataset are:

\begin{longtable}[]{@{}ll@{}}
\toprule
\begin{minipage}[t]{0.14\columnwidth}\raggedright
\textbf{x}\strut
\end{minipage} & \begin{minipage}[t]{0.41\columnwidth}\raggedright
The x location of cases\strut
\end{minipage}\tabularnewline
\begin{minipage}[t]{0.14\columnwidth}\raggedright
\textbf{y}\strut
\end{minipage} & \begin{minipage}[t]{0.41\columnwidth}\raggedright
The y location of cases\strut
\end{minipage}\tabularnewline
\bottomrule
\end{longtable}

The \texttt{x} and \texttt{y} variables have been transformed to lie
between 0 and 1 with the industrial waste disposal site centrally
located (i.e.~at \texttt{x} = 0.5, \texttt{y} = 0.5).

Plot the data and the location of the industrial waste disposal site on
the same chart:

\begin{Shaded}
\begin{Highlighting}[]
\KeywordTok{plot}\NormalTok{(waste}\OperatorTok{$}\NormalTok{x, waste}\OperatorTok{$}\NormalTok{y, }\DataTypeTok{xlim =} \KeywordTok{c}\NormalTok{(}\DecValTok{0}\NormalTok{, }\DecValTok{1}\NormalTok{), }\DataTypeTok{ylim =} \KeywordTok{c}\NormalTok{(}\DecValTok{0}\NormalTok{, }\DecValTok{1}\NormalTok{))}
\KeywordTok{points}\NormalTok{(}\FloatTok{0.5}\NormalTok{, }\FloatTok{0.5}\NormalTok{, }\DataTypeTok{pch =} \DecValTok{3}\NormalTok{)}
\end{Highlighting}
\end{Shaded}

\includegraphics{prfe_files/figure-latex/unnamed-chunk-473-1.pdf}

We can calculate the distance of each point from the industrial waste
disposal site using Pythagoras' Theorem:

\begin{Shaded}
\begin{Highlighting}[]
\NormalTok{distance.obs <-}\StringTok{ }\KeywordTok{sqrt}\NormalTok{((waste}\OperatorTok{$}\NormalTok{x }\OperatorTok{-}\StringTok{ }\FloatTok{0.5}\NormalTok{) }\OperatorTok{^}\StringTok{ }\DecValTok{2} \OperatorTok{+}\StringTok{ }\NormalTok{(waste}\OperatorTok{$}\NormalTok{y }\OperatorTok{-}\StringTok{ }\FloatTok{0.5}\NormalTok{) }\OperatorTok{^}\StringTok{ }\DecValTok{2}\NormalTok{)}
\end{Highlighting}
\end{Shaded}

The observed mean distance or each case from the industrial waste
disposal site is:

\begin{Shaded}
\begin{Highlighting}[]
\KeywordTok{mean}\NormalTok{(distance.obs)}
\end{Highlighting}
\end{Shaded}

\begin{verbatim}
## [1] 0.3444118
\end{verbatim}

To test whether this distance is unlikely to have arisen by chance
(i.e.~evidence of spatial clustering) we need to simulate the
distribution of distances when no spatial pattern exists:

\begin{Shaded}
\begin{Highlighting}[]
\NormalTok{r <-}\StringTok{ }\DecValTok{10000}
\NormalTok{x.sim <-}\StringTok{ }\KeywordTok{matrix}\NormalTok{(}\KeywordTok{runif}\NormalTok{(r }\OperatorTok{*}\StringTok{ }\DecValTok{23}\NormalTok{), }\DecValTok{23}\NormalTok{, r)}
\NormalTok{y.sim <-}\StringTok{ }\KeywordTok{matrix}\NormalTok{(}\KeywordTok{runif}\NormalTok{(r }\OperatorTok{*}\StringTok{ }\DecValTok{23}\NormalTok{), }\DecValTok{23}\NormalTok{, r)}
\NormalTok{distance.run <-}\StringTok{ }\KeywordTok{sqrt}\NormalTok{((x.sim }\OperatorTok{-}\StringTok{ }\FloatTok{0.5}\NormalTok{)}\OperatorTok{^}\DecValTok{2} \OperatorTok{+}\StringTok{ }\NormalTok{(y.sim }\OperatorTok{-}\StringTok{ }\FloatTok{0.5}\NormalTok{)}\OperatorTok{^}\DecValTok{2}\NormalTok{)}
\NormalTok{distance.sim <-}\StringTok{ }\KeywordTok{apply}\NormalTok{(distance.run, }\DecValTok{2}\NormalTok{, mean)}
\KeywordTok{hist}\NormalTok{(distance.sim, }\DataTypeTok{breaks =} \DecValTok{20}\NormalTok{)}
\KeywordTok{abline}\NormalTok{(}\DataTypeTok{v =} \KeywordTok{mean}\NormalTok{(distance.obs), }\DataTypeTok{lty =} \DecValTok{3}\NormalTok{)}
\end{Highlighting}
\end{Shaded}

\includegraphics{prfe_files/figure-latex/unnamed-chunk-476-1.pdf}

The probability (i.e.~the \emph{p-value}) of observing a mean distance
smaller than the \emph{observed mean} distance under the null hypothesis
can be estimated as the number of estimates of the mean distance under
the null hypothesis falling below the observed mean divided by the total
number of estimates of the mean distance under the null hypothesis:

\begin{Shaded}
\begin{Highlighting}[]
\NormalTok{m <-}\StringTok{ }\KeywordTok{mean}\NormalTok{(distance.obs)}
\NormalTok{z <-}\StringTok{ }\KeywordTok{ifelse}\NormalTok{(distance.sim }\OperatorTok{<}\StringTok{ }\NormalTok{m, }\DecValTok{1}\NormalTok{, }\DecValTok{0}\NormalTok{)}
\KeywordTok{sum}\NormalTok{(z) }\OperatorTok{/}\StringTok{ }\NormalTok{r}
\end{Highlighting}
\end{Shaded}

\begin{verbatim}
## [1] 0.1031
\end{verbatim}

You might like to repeat this exercise using the harmonic mean distance
and the median distance.

We can check if this method is capable of detecting a simple cluster
using simulated data:

\begin{Shaded}
\begin{Highlighting}[]
\NormalTok{x <-}\StringTok{ }\KeywordTok{rnorm}\NormalTok{(}\DecValTok{23}\NormalTok{, }\DataTypeTok{mean =} \FloatTok{0.5}\NormalTok{, }\DataTypeTok{sd =} \FloatTok{0.2}\NormalTok{)}
\NormalTok{y <-}\StringTok{ }\KeywordTok{rnorm}\NormalTok{(}\DecValTok{23}\NormalTok{, }\DataTypeTok{mean =} \FloatTok{0.5}\NormalTok{, }\DataTypeTok{sd =} \FloatTok{0.2}\NormalTok{)}
\KeywordTok{plot}\NormalTok{(x, y, }\DataTypeTok{xlim =} \KeywordTok{c}\NormalTok{(}\DecValTok{0}\NormalTok{, }\DecValTok{1}\NormalTok{), }\DataTypeTok{ylim =} \KeywordTok{c}\NormalTok{(}\DecValTok{0}\NormalTok{, }\DecValTok{1}\NormalTok{))}
\KeywordTok{points}\NormalTok{(}\FloatTok{0.5}\NormalTok{, }\FloatTok{0.5}\NormalTok{, }\DataTypeTok{pch =} \DecValTok{3}\NormalTok{)}
\end{Highlighting}
\end{Shaded}

\includegraphics{prfe_files/figure-latex/unnamed-chunk-478-1.pdf}

We need to recalculate the distance of each simulated case from the
industrial waste disposal site:

\begin{Shaded}
\begin{Highlighting}[]
\NormalTok{distance.obs <-}\StringTok{ }\KeywordTok{sqrt}\NormalTok{((x }\OperatorTok{-}\StringTok{ }\FloatTok{0.5}\NormalTok{) }\OperatorTok{^}\StringTok{ }\DecValTok{2} \OperatorTok{+}\StringTok{ }\NormalTok{(y }\OperatorTok{-}\StringTok{ }\FloatTok{0.5}\NormalTok{) }\OperatorTok{^}\StringTok{ }\DecValTok{2}\NormalTok{)}
\end{Highlighting}
\end{Shaded}

The observed mean distance of each case from the industrial waste
disposal site is:

\begin{Shaded}
\begin{Highlighting}[]
\KeywordTok{mean}\NormalTok{(distance.obs)}
\end{Highlighting}
\end{Shaded}

\begin{verbatim}
## [1] 0.3040121
\end{verbatim}

We can use the the simulated null hypothesis data to test for spatial
clustering:

\begin{Shaded}
\begin{Highlighting}[]
\NormalTok{m <-}\StringTok{ }\KeywordTok{mean}\NormalTok{(distance.obs)}
\NormalTok{z <-}\StringTok{ }\KeywordTok{ifelse}\NormalTok{(distance.sim }\OperatorTok{<}\StringTok{ }\NormalTok{m, }\DecValTok{1}\NormalTok{, }\DecValTok{0}\NormalTok{)}
\KeywordTok{sum}\NormalTok{(z) }\OperatorTok{/}\StringTok{ }\NormalTok{r}
\end{Highlighting}
\end{Shaded}

\begin{verbatim}
## [1] 0.0052
\end{verbatim}

We should also check if the procedure can detect a plume of cases, such
as might be created by a prevailing wind at a waste incineration site,
in a similar way:

\begin{Shaded}
\begin{Highlighting}[]
\NormalTok{x <-}\StringTok{ }\KeywordTok{rnorm}\NormalTok{(}\DecValTok{23}\NormalTok{, }\FloatTok{0.25}\NormalTok{, }\FloatTok{0.1}\NormalTok{) }\OperatorTok{+}\StringTok{ }\FloatTok{0.5}
\NormalTok{y <-}\StringTok{ }\KeywordTok{rnorm}\NormalTok{(}\DecValTok{23}\NormalTok{, }\FloatTok{0.25}\NormalTok{, }\FloatTok{0.1}\NormalTok{) }\OperatorTok{+}\StringTok{ }\FloatTok{0.5}
\KeywordTok{plot}\NormalTok{(x, y, }\DataTypeTok{xlim =} \KeywordTok{c}\NormalTok{(}\DecValTok{0}\NormalTok{, }\DecValTok{1}\NormalTok{), }\DataTypeTok{ylim =} \KeywordTok{c}\NormalTok{(}\DecValTok{0}\NormalTok{, }\DecValTok{1}\NormalTok{))}
\KeywordTok{points}\NormalTok{(}\FloatTok{0.5}\NormalTok{, }\FloatTok{0.5}\NormalTok{, }\DataTypeTok{pch =} \DecValTok{3}\NormalTok{)}
\end{Highlighting}
\end{Shaded}

\includegraphics{prfe_files/figure-latex/unnamed-chunk-482-1.pdf}

\begin{Shaded}
\begin{Highlighting}[]
\NormalTok{distance.obs <-}\StringTok{ }\KeywordTok{sqrt}\NormalTok{((x }\OperatorTok{-}\StringTok{ }\FloatTok{0.5}\NormalTok{)}\OperatorTok{^}\DecValTok{2} \OperatorTok{+}\StringTok{ }\NormalTok{(y }\OperatorTok{-}\StringTok{ }\FloatTok{0.5}\NormalTok{)}\OperatorTok{^}\DecValTok{2}\NormalTok{)}
\NormalTok{m <-}\StringTok{ }\KeywordTok{mean}\NormalTok{(distance.obs)}
\NormalTok{z <-}\StringTok{ }\KeywordTok{ifelse}\NormalTok{(distance.sim }\OperatorTok{<}\StringTok{ }\NormalTok{m, }\DecValTok{1}\NormalTok{, }\DecValTok{0}\NormalTok{)}
\KeywordTok{sum}\NormalTok{(z) }\OperatorTok{/}\StringTok{ }\NormalTok{r}
\end{Highlighting}
\end{Shaded}

\begin{verbatim}
## [1] 0.22
\end{verbatim}

The method is not capable of detecting plumes.

You might like to try adapting the simulation code presented here to
provide a method capable of detecting plumes of cases.

\hypertarget{simulating-processes}{%
\section{Simulating processes}\label{simulating-processes}}

In the previous example we simulated the expected distribution of data
under the \emph{null hypothesis}. Computer based simulations are not
limited to simulating data. They can also be used to simulate processes.

In this example we will simulate the behaviour of the \emph{lot quality
assurance sampling} (LQAS) survey method when sampling constraints lead
to a loss of sampling independence. In this example the sampling process
is simulated and applied to real-world data.

LQAS is a small-sample classification technique that is widely used in
manufacturing industry to judge the quality of a batch of manufactured
items. In this context, LQAS is used to identify batches that are likely
to contain an unacceptably large number of defective items. In the
public health context, LQAS may be used to identify communities with
unacceptably low levels of service (e.g.~vaccine) coverage or worrying
levels of disease prevalence.

The LQAS method produces data that is easy to analyse. Data analysis is
performed as data is collected and consists solely of counting the
number of \emph{defects} (e.g.~children with a specific disease) in the
sample and checking whether a predetermined threshold value has been
exceeded. This combination of data collection and data analysis is
called a \emph{sampling plan}. LQAS sampling plans are developed by
specifying:

\begin{itemize}
\item
  \textbf{A TRIAGE SYSTEM}: A classification system that defines
  \emph{high}, \emph{moderate}, and \emph{low} categories of the
  prevalence of the phenomenon of interest.
\item
  \textbf{ACCEPTABLE PROBABILITIES OF ERROR}: There are two
  probabilities of error. These are termed provider \emph{probability of
  error} (PPE) and \emph{consumer probability of error} (CPE):

  \begin{itemize}
  \item
    \textbf{Provider Probability of Error (PPE)}: The risk that the
    survey will indicate that prevalence is \emph{high} when it is, in
    fact, \emph{low}. PPE is analogous to \emph{type I} (\(\alpha\))
    error in statistical hypothesis testing.
  \item
    \textbf{Consumer Probability of Error (CPE)}: The risk that the
    survey will indicate that prevalence is \emph{low} when it is, in
    fact, \emph{high}. CPE is analogous to \emph{type II} (\(\beta\))
    error in statistical hypothesis testing.
  \end{itemize}
\end{itemize}

Once the upper and lower levels of the triage system and acceptable
levels of error have been decided, a set of probability tables are
constructed that are used to select a maximum sample size (\texttt{n})
and the number of defects or cases (\texttt{d}) that are allowed in the
sample of \texttt{n} subjects before deciding that a population is a
high prevalence population. The combination of maximum sample size
(\texttt{n}) and number of defects (\texttt{d}) form the stopping rules
of the sampling plan. Sampling stops when either the maximum sample size
(\texttt{n}) is met or the allowable number of defects (\texttt{d}) is
exceeded:

\begin{itemize}
\item
  If \texttt{d} is exceeded then the population is classified as high
  prevalence.
\item
  If \texttt{n} is met without d being exceeded then the population is
  classified as low prevalence.
\end{itemize}

The values of \texttt{n} and \texttt{d} used in a sampling plan depend
upon the threshold values used in the triage system and the acceptable
levels of error. The values of \texttt{n} and \texttt{d} used in a
sampling plan are calculated using binomial probabilities. For example,
the probabilities of finding 14 or fewer cases (\(d = 14\)) in a sample
of 50 individuals (\(n = 50\)) from populations with prevalences of
either 20\% or 40\% are:

\begin{Shaded}
\begin{Highlighting}[]
\KeywordTok{pbinom}\NormalTok{(}\DataTypeTok{q =} \DecValTok{14}\NormalTok{, }\DataTypeTok{size =} \DecValTok{50}\NormalTok{, }\DataTypeTok{prob =} \FloatTok{0.2}\NormalTok{)}
\KeywordTok{pbinom}\NormalTok{(}\DataTypeTok{q =} \DecValTok{14}\NormalTok{, }\DataTypeTok{size =} \DecValTok{50}\NormalTok{, }\DataTypeTok{prob =} \FloatTok{0.4}\NormalTok{)}
\end{Highlighting}
\end{Shaded}

\begin{verbatim}
## [1] 0.9392779
\end{verbatim}

\begin{verbatim}
## [1] 0.05395503
\end{verbatim}

The sampling plan with \(n = 50\) and \(d = 14\) is, therefore, a
reasonable candidate for a sampling plan intended to distinguish between
populations with prevalences of less than or equal to 20\% and
populations with prevalences greater than or equal to 40\%.

There is no middle ground with LQAS sampling plans. Population are
always classified as either high or low prevalence. Populations with
prevalences between the upper and lower standards of the triage system
are classified as high or low prevalence populations. The probability of
a moderate prevalence population being classified as high or low
prevalence is proportional to the proximity of the prevalence in that
population to the triage standards. Moderate prevalence populations
close to the upper standard will tend to be classified as high
prevalence populations. Moderate prevalence populations close to the
lower standard will tend to be classified as low prevalence populations.
This behaviour is summarised by the operating characteristic (OC) curve
for the sampling plan. For example:

\begin{Shaded}
\begin{Highlighting}[]
\KeywordTok{plot}\NormalTok{(}\KeywordTok{seq}\NormalTok{(}\DecValTok{0}\NormalTok{, }\FloatTok{0.6}\NormalTok{, }\FloatTok{0.01}\NormalTok{),}
     \KeywordTok{pbinom}\NormalTok{(}\DecValTok{14}\NormalTok{, }\DecValTok{50}\NormalTok{, }\KeywordTok{seq}\NormalTok{(}\DecValTok{0}\NormalTok{, }\FloatTok{0.6}\NormalTok{, }\FloatTok{0.01}\NormalTok{), }\DataTypeTok{lower.tail =} \OtherTok{FALSE}\NormalTok{),}
     \DataTypeTok{main =} \StringTok{"OC Curve for n = 50, d = 14"}\NormalTok{,}
     \DataTypeTok{xlab =} \StringTok{"Proportion diseased"}\NormalTok{,}
     \DataTypeTok{ylab =} \StringTok{"Probability"}\NormalTok{,}
     \DataTypeTok{type =} \StringTok{"l"}\NormalTok{, }\DataTypeTok{lty =} \DecValTok{2}\NormalTok{)}
\end{Highlighting}
\end{Shaded}

\includegraphics{prfe_files/figure-latex/unnamed-chunk-485-1.pdf}

The data we will use for the simulation is stored in forty-eight
separate files. These files contain the returns from whole community
screens for active trachoma (TF/TI) in children undertaken as part of
trachoma control activities in five African countries. Each file has the
file suffix .sim. The name of the file reflects the country in which the
data were collected. The data files are:

\begin{longtable}[]{@{}lrl@{}}
\toprule
\begin{minipage}[b]{0.27\columnwidth}\raggedright
\textbf{File name}\strut
\end{minipage} & \begin{minipage}[b]{0.15\columnwidth}\raggedleft
\textbf{Files}\strut
\end{minipage} & \begin{minipage}[b]{0.27\columnwidth}\raggedright
\textbf{Origin}\strut
\end{minipage}\tabularnewline
\midrule
\endhead
\begin{minipage}[t]{0.27\columnwidth}\raggedright
\textbf{egyptXX.sim}\strut
\end{minipage} & \begin{minipage}[t]{0.15\columnwidth}\raggedleft
10\strut
\end{minipage} & \begin{minipage}[t]{0.27\columnwidth}\raggedright
Egypt\strut
\end{minipage}\tabularnewline
\begin{minipage}[t]{0.27\columnwidth}\raggedright
\textbf{gambiaXX.sim}\strut
\end{minipage} & \begin{minipage}[t]{0.15\columnwidth}\raggedleft
10\strut
\end{minipage} & \begin{minipage}[t]{0.27\columnwidth}\raggedright
Gambia\strut
\end{minipage}\tabularnewline
\begin{minipage}[t]{0.27\columnwidth}\raggedright
\textbf{ghanaXX.sim}\strut
\end{minipage} & \begin{minipage}[t]{0.15\columnwidth}\raggedleft
3\strut
\end{minipage} & \begin{minipage}[t]{0.27\columnwidth}\raggedright
Ghana\strut
\end{minipage}\tabularnewline
\begin{minipage}[t]{0.27\columnwidth}\raggedright
\textbf{tanzaniaXX.sim}\strut
\end{minipage} & \begin{minipage}[t]{0.15\columnwidth}\raggedleft
14\strut
\end{minipage} & \begin{minipage}[t]{0.27\columnwidth}\raggedright
Tanzania\strut
\end{minipage}\tabularnewline
\begin{minipage}[t]{0.27\columnwidth}\raggedright
\textbf{togoXX.sim}\strut
\end{minipage} & \begin{minipage}[t]{0.15\columnwidth}\raggedleft
11\strut
\end{minipage} & \begin{minipage}[t]{0.27\columnwidth}\raggedright
Togo\strut
\end{minipage}\tabularnewline
\bottomrule
\end{longtable}

All of these data files have the same structure. The variables in the
data files are:

\begin{longtable}[]{@{}ll@{}}
\toprule
\begin{minipage}[t]{0.21\columnwidth}\raggedright
\textbf{hh}\strut
\end{minipage} & \begin{minipage}[t]{0.67\columnwidth}\raggedright
Household identifier\strut
\end{minipage}\tabularnewline
\begin{minipage}[t]{0.21\columnwidth}\raggedright
\textbf{sex}\strut
\end{minipage} & \begin{minipage}[t]{0.67\columnwidth}\raggedright
Sex of child (1=male, 2=female)\strut
\end{minipage}\tabularnewline
\begin{minipage}[t]{0.21\columnwidth}\raggedright
\textbf{age}\strut
\end{minipage} & \begin{minipage}[t]{0.67\columnwidth}\raggedright
Age of child in years\strut
\end{minipage}\tabularnewline
\begin{minipage}[t]{0.21\columnwidth}\raggedright
\textbf{tfti}\strut
\end{minipage} & \begin{minipage}[t]{0.67\columnwidth}\raggedright
Child has active (TF/TI) trachoma (0=no, 1=yes)\strut
\end{minipage}\tabularnewline
\bottomrule
\end{longtable}

Each row in these files represents a single child. For example:

\begin{Shaded}
\begin{Highlighting}[]
\NormalTok{x <-}\StringTok{ }\KeywordTok{read.table}\NormalTok{(}\StringTok{"gambia09.sim"}\NormalTok{, }\DataTypeTok{header =} \OtherTok{TRUE}\NormalTok{)}
\NormalTok{x[}\DecValTok{1}\OperatorTok{:}\DecValTok{10}\NormalTok{, ]}
\end{Highlighting}
\end{Shaded}

\begin{verbatim}
##         hh sex age tfti
## 1  4008001   1   6    1
## 2  4008001   1   3    1
## 3  4008002   1   8    1
## 4  4008002   1   6    0
## 5  4008003   1   8    0
## 6  4008003   1   1    0
## 7  4008004   1   7    0
## 8  4008004   1   4    0
## 9  4008004   1   2    0
## 10 4008004   1   2    0
\end{verbatim}

Any rapid survey method that is appropriate for general use in
developing countries is restricted to sampling \texttt{households}
rather than \texttt{individuals}. Sampling households in order to sample
individuals violates a principal requirement for a sample to be
representative of the population from which it is drawn (i.e.~that
individuals are selected \texttt{independently} of each other). This
lack of statistical independence amongst sampled individuals may
invalidate standard approaches to selecting sampling plans leading to
increased probabilities of error. This is likely to be a particular
problem if cases tend to be clustered within households. Trachoma is an
infectious disease that confers no lasting immunity in the host. Cases
are, therefore, very likely to be clustered within households. One
solution to this problem would be to sample (i.e.~at random) a single
child from each of the sampled households. This is not appropriate for
active trachoma as the examination procedure often causes distress to
younger children. This may influence survey staff to select older
children, who tend to have a lower risk of infection, for examination.
Sampling is, therefore, constrained to sampling all children in selected
households.

The purpose of the simulations presented here is to determine whether
the LQAS method is robust to:

\begin{enumerate}
\def\labelenumi{\arabic{enumi}.}
\tightlist
\item
  The loss of sampling independence introduced by sampling households at
  random and examining all children in selected households for active
  trachoma.
\end{enumerate}

And:

\begin{enumerate}
\def\labelenumi{\arabic{enumi}.}
\setcounter{enumi}{1}
\tightlist
\item
  The slight increase in the maximum sample size (\texttt{n}) introduced
  by examining all children in selected households for active trachoma.
\end{enumerate}

Each row in the datasets we will be using represents an individual
child. Since we will simulate sampling households rather than individual
children we need to be able to convert the datasets from one row per
child to one row per household. We will write a function to do this.

Create a new function called \texttt{ind2hh()}:

\begin{Shaded}
\begin{Highlighting}[]
\NormalTok{ind2hh <-}\StringTok{ }\ControlFlowTok{function}\NormalTok{() \{\}}
\end{Highlighting}
\end{Shaded}

This creates an empty function called \texttt{ind2hh()}. Use the
\texttt{fix()} function to edit the \texttt{ind2hh()} function:

\begin{Shaded}
\begin{Highlighting}[]
\KeywordTok{fix}\NormalTok{(ind2hh)}
\end{Highlighting}
\end{Shaded}

Edit the function to read:

\begin{Shaded}
\begin{Highlighting}[]
\ControlFlowTok{function}\NormalTok{(data) \{}
\NormalTok{  n.kids <-}\StringTok{ }\NormalTok{n.cases <-}\StringTok{ }\OtherTok{NULL}
\NormalTok{  id <-}\StringTok{ }\KeywordTok{unique}\NormalTok{(data}\OperatorTok{$}\NormalTok{hh)}
  \ControlFlowTok{for}\NormalTok{(household }\ControlFlowTok{in}\NormalTok{ id) \{}
\NormalTok{    temp <-}\StringTok{ }\KeywordTok{subset}\NormalTok{(data, data}\OperatorTok{$}\NormalTok{hh }\OperatorTok{==}\StringTok{ }\NormalTok{household)}
\NormalTok{    n.kids <-}\StringTok{ }\KeywordTok{c}\NormalTok{(n.kids, }\KeywordTok{nrow}\NormalTok{(temp))}
\NormalTok{    n.cases <-}\StringTok{ }\KeywordTok{c}\NormalTok{(n.cases, }\KeywordTok{sum}\NormalTok{(temp}\OperatorTok{$}\NormalTok{tfti))}
\NormalTok{  \}}
\NormalTok{  result <-}\StringTok{ }\KeywordTok{as.data.frame}\NormalTok{(}\KeywordTok{cbind}\NormalTok{(id, n.kids, n.cases))}
  \KeywordTok{return}\NormalTok{(result)}
\NormalTok{\}}
\end{Highlighting}
\end{Shaded}

Once you have made the changes shown above, check your work, save the
file, and quit the editor.

Now we have created the \texttt{ind2hh()} function we should test it for
correct operation. We will create a simple test data.frame
(\texttt{test.df}) for this purpose:

\begin{Shaded}
\begin{Highlighting}[]
\NormalTok{test.df <-}\StringTok{ }\KeywordTok{as.data.frame}\NormalTok{(}\KeywordTok{cbind}\NormalTok{(}\KeywordTok{c}\NormalTok{(}\DecValTok{1}\NormalTok{, }\DecValTok{1}\NormalTok{, }\DecValTok{2}\NormalTok{, }\DecValTok{2}\NormalTok{, }\DecValTok{2}\NormalTok{),  }\KeywordTok{c}\NormalTok{(}\DecValTok{1}\NormalTok{, }\DecValTok{1}\NormalTok{, }\DecValTok{1}\NormalTok{, }\DecValTok{0}\NormalTok{ ,}\DecValTok{0}\NormalTok{)))}
\KeywordTok{names}\NormalTok{(test.df) <-}\StringTok{ }\KeywordTok{c}\NormalTok{(}\StringTok{"hh"}\NormalTok{, }\StringTok{"tfti"}\NormalTok{)}
\NormalTok{test.df}
\end{Highlighting}
\end{Shaded}

\begin{verbatim}
##   hh tfti
## 1  1    1
## 2  1    1
## 3  2    1
## 4  2    0
## 5  2    0
\end{verbatim}

The expected operation of the \texttt{ind2hh()} function given
\texttt{test.df} as input is:

\begin{longtable}[]{@{}ll@{}}
\toprule
\begin{minipage}[t]{0.08\columnwidth}\raggedright
hh\strut
\end{minipage} & \begin{minipage}[t]{0.14\columnwidth}\raggedright
tfti\strut
\end{minipage}\tabularnewline
\begin{minipage}[t]{0.08\columnwidth}\raggedright
1\strut
\end{minipage} & \begin{minipage}[t]{0.14\columnwidth}\raggedright
1\strut
\end{minipage}\tabularnewline
\begin{minipage}[t]{0.08\columnwidth}\raggedright
1\strut
\end{minipage} & \begin{minipage}[t]{0.14\columnwidth}\raggedright
1\strut
\end{minipage}\tabularnewline
\begin{minipage}[t]{0.08\columnwidth}\raggedright
2\strut
\end{minipage} & \begin{minipage}[t]{0.14\columnwidth}\raggedright
1\strut
\end{minipage}\tabularnewline
\begin{minipage}[t]{0.08\columnwidth}\raggedright
2\strut
\end{minipage} & \begin{minipage}[t]{0.14\columnwidth}\raggedright
0\strut
\end{minipage}\tabularnewline
\begin{minipage}[t]{0.08\columnwidth}\raggedright
2\strut
\end{minipage} & \begin{minipage}[t]{0.14\columnwidth}\raggedright
0\strut
\end{minipage}\tabularnewline
\bottomrule
\end{longtable}

becomes

\begin{longtable}[]{@{}lll@{}}
\toprule
\begin{minipage}[t]{0.08\columnwidth}\raggedright
id\strut
\end{minipage} & \begin{minipage}[t]{0.14\columnwidth}\raggedright
n.kids\strut
\end{minipage} & \begin{minipage}[t]{0.14\columnwidth}\raggedright
n.case\strut
\end{minipage}\tabularnewline
\begin{minipage}[t]{0.08\columnwidth}\raggedright
1\strut
\end{minipage} & \begin{minipage}[t]{0.14\columnwidth}\raggedright
2\strut
\end{minipage} & \begin{minipage}[t]{0.14\columnwidth}\raggedright
2\strut
\end{minipage}\tabularnewline
\begin{minipage}[t]{0.08\columnwidth}\raggedright
2\strut
\end{minipage} & \begin{minipage}[t]{0.14\columnwidth}\raggedright
3\strut
\end{minipage} & \begin{minipage}[t]{0.14\columnwidth}\raggedright
1\strut
\end{minipage}\tabularnewline
\bottomrule
\end{longtable}

Confirm this behaviour:

\begin{Shaded}
\begin{Highlighting}[]
\NormalTok{test.df}
\KeywordTok{ind2hh}\NormalTok{(test.df)}
\end{Highlighting}
\end{Shaded}

\begin{verbatim}
##   hh tfti
## 1  1    1
## 2  1    1
## 3  2    1
## 4  2    0
## 5  2    0
\end{verbatim}

\begin{verbatim}
##   id n.kids n.cases
## 1  1      2       2
## 2  2      3       1
\end{verbatim}

We can apply this function to the datasets as required. For example:

\begin{Shaded}
\begin{Highlighting}[]
\NormalTok{x <-}\StringTok{ }\KeywordTok{read.table}\NormalTok{(}\StringTok{"gambia09.sim"}\NormalTok{, }\DataTypeTok{header =} \OtherTok{TRUE}\NormalTok{)}
\NormalTok{x}
\NormalTok{x.hh <-}\StringTok{ }\KeywordTok{ind2hh}\NormalTok{(x)}
\NormalTok{x.hh}
\end{Highlighting}
\end{Shaded}

\begin{verbatim}
##          hh sex age tfti
## 1   4008001   1   6    1
## 2   4008001   1   3    1
## 3   4008002   1   8    1
## 4   4008002   1   6    0
## 5   4008003   1   8    0
## 6   4008003   1   1    0
## 7   4008004   1   7    0
## 8   4008004   1   4    0
## 9   4008004   1   2    0
## 10  4008004   1   2    0
## 11  4008004   1   4    0
## 12  4008005   1   1    0
## 13  4008006   1   7    0
## 14  4008006   1   5    0
## 15  4008007   1   5    0
## 16  4008007   1   9    0
## 17  4008007   1   9    0
## 18  4008008   1   9    1
## 19  4008008   1   8    0
## 20  4008008   1   1    0
## 21  4008009   1   4    1
## 22  4008009   1   2    0
## 23  4008010   1   7    1
## 24  4008010   1   7    0
## 25  4008010   1   9    0
## 26  4008010   1   2    0
## 27  4008011   1   5    0
## 28  4008011   1   8    0
## 29  4008011   1   5    0
## 30  4008012   1   9    1
## 31  4008013   1   3    1
## 32  4008013   1   7    1
## 33  4008013   1   5    1
## 34  4008013   1   2    0
## 35  4008013   1   3    0
## 36  4008013   1   7    0
## 37  4008014   1   9    1
## 38  4008014   1   1    1
## 39  4008014   1   7    1
## 40  4008015   1   5    1
## 41  4008015   1   6    1
## 42  4008015   1   1    1
## 43  4008015   1   5    1
## 44  4008015   1   7    0
## 45  4008015   1   2    0
## 46  4008015   1   9    0
## 47  4008016   1   9    1
## 48  4008016   1   7    1
## 49  4008016   1   9    1
## 50  4008016   1   4    1
## 51  4008016   1   7    1
## 52  4008016   1   5    1
## 53  4008016   1   8    1
## 54  4008017   1   5    1
## 55  4008017   1   8    0
## 56  4008018   1   5    0
## 57  4008018   1   2    0
## 58  4008019   1   4    1
## 59  4008019   1   6    0
## 60  4008020   1   1    0
## 61  4008020   1   4    1
## 62  4008020   1   2    0
## 63  4008021   1   3    0
## 64  4008021   1   7    0
## 65  4008021   1   9    1
## 66  4008021   1   2    0
## 67  4008022   1   5    0
## 68  4008022   1   8    0
## 69  4008022   1   5    0
## 70  4008023   1   6    1
## 71  4008023   1   3    1
## 72  4008024   1   2    1
## 73  4008024   1   5    1
## 74  4008024   1   2    0
## 75  4008024   1   3    0
## 76  4008024   1   8    0
## 77  4008025   1   3    0
## 78  4008025   1   8    0
## 79  4008026   1   6    1
## 80  4008026   1   8    0
## 81  4008026   1   1    0
## 82  4008026   1   7    0
## 83  4008027   1   4    0
## 84  4008027   1   7    0
## 85  4008027   1   2    0
## 86  4008027   1   2    0
## 87  4008027   1   1    0
## 88  4008028   1   4    0
## 89  4008028   1   5    0
## 90  4008028   1   7    0
## 91  4008029   1   5    0
## 92  4008029   1   3    0
## 93  4008029   1   6    0
## 94  4108001   1   6    1
## 95  4108001   1   3    1
## 96  4108002   1   8    1
## 97  4108002   1   6    0
## 98  4108003   1   8    0
## 99  4108003   1   1    0
## 100 4108004   1   7    0
## 101 4108004   1   4    0
## 102 4108004   1   2    0
## 103 4108004   1   2    0
## 104 4108004   1   4    0
## 105 4108005   1   1    0
## 106 4108006   1   7    0
## 107 4108006   1   5    0
## 108 4108007   1   5    0
## 109 4108007   1   9    0
## 110 4108007   1   9    0
## 111 4108008   1   9    1
## 112 4108008   1   8    0
## 113 4108008   1   1    0
## 114 4108009   1   4    1
## 115 4108009   1   2    0
## 116 4108010   1   7    1
## 117 4108010   1   7    0
## 118 4108010   1   9    0
## 119 4108010   1   2    0
## 120 4108011   1   5    0
## 121 4108011   1   8    0
## 122 4108011   1   5    0
## 123 4108012   1   9    1
## 124 4108013   1   3    1
## 125 4108013   1   7    1
## 126 4108013   1   5    1
## 127 4108013   1   2    0
## 128 4108013   1   3    0
## 129 4108013   1   7    0
## 130 4108014   1   9    1
## 131 4108014   1   1    1
## 132 4108014   1   7    1
## 133 4108015   1   5    1
## 134 4108015   1   6    1
## 135 4108015   1   1    1
## 136 4108015   1   5    1
## 137 4108015   1   7    0
## 138 4108015   1   2    0
## 139 4108015   1   9    0
## 140 4108016   1   9    1
## 141 4108016   1   7    1
## 142 4108016   1   9    1
## 143 4108016   1   4    1
## 144 4108016   1   7    1
## 145 4108016   1   5    1
## 146 4108016   1   8    1
## 147 4108017   1   5    1
## 148 4108017   1   8    0
## 149 4108018   1   5    0
## 150 4108018   1   2    0
## 151 4108019   1   4    1
## 152 4108019   1   6    0
## 153 4108020   1   1    0
## 154 4108020   1   4    1
## 155 4108020   1   2    0
## 156 4108021   1   3    0
## 157 4108021   1   7    0
## 158 4108021   1   9    1
## 159 4108021   1   2    0
## 160 4108022   1   5    0
## 161 4108022   1   8    0
## 162 4108022   1   5    0
## 163 4108023   1   6    1
## 164 4108023   1   3    1
## 165 4108024   1   2    1
## 166 4108024   1   5    1
## 167 4108024   1   2    0
## 168 4108024   1   3    0
## 169 4108024   1   8    0
## 170 4108025   1   3    0
## 171 4108025   1   8    0
## 172 4108026   1   6    1
## 173 4108026   1   8    0
## 174 4108026   1   1    0
## 175 4108026   1   7    0
## 176 4108027   1   4    0
## 177 4108027   1   7    0
## 178 4108027   1   2    0
## 179 4108027   1   2    0
## 180 4108027   1   1    0
## 181 4108028   1   4    0
## 182 4108028   1   5    0
## 183 4108028   1   7    0
## 184 4108029   1   5    0
## 185 4108029   1   3    0
## 186 4108029   1   6    0
## 187 4208001   1   6    1
## 188 4208001   1   3    1
## 189 4208002   1   8    1
## 190 4208002   1   6    0
## 191 4208003   1   8    0
## 192 4208003   1   1    0
## 193 4208004   1   7    0
## 194 4208004   1   4    0
## 195 4008004   1   2    0
## 196 4208004   1   2    0
## 197 4208004   1   4    0
## 198 4208005   1   1    0
## 199 4208006   1   7    0
## 200 4208006   1   5    0
## 201 4208007   1   5    0
## 202 4208007   1   9    0
## 203 4208007   1   9    0
## 204 4208008   1   9    1
## 205 4208008   1   8    0
## 206 4208008   1   1    0
## 207 4208009   1   4    1
## 208 4208009   1   2    0
## 209 4208010   1   7    1
## 210 4208010   1   7    0
## 211 4208010   1   9    0
## 212 4208010   1   2    0
## 213 4208011   1   5    0
## 214 4208011   1   8    0
## 215 4208011   1   5    0
## 216 4208012   1   9    1
## 217 4208013   1   3    1
## 218 4208013   1   7    1
## 219 4208013   1   5    1
## 220 4208013   1   2    0
## 221 4208013   1   3    0
## 222 4208013   1   7    0
## 223 4208014   1   9    1
## 224 4208014   1   1    1
## 225 4208014   1   7    1
## 226 4208015   1   5    1
## 227 4208015   1   6    1
## 228 4208015   1   1    1
## 229 4208015   1   5    1
## 230 4208015   1   7    0
## 231 4208015   1   2    0
## 232 4208015   1   9    0
## 233 4208016   1   9    1
## 234 4208016   1   7    1
## 235 4208016   1   9    1
## 236 4208016   1   4    1
## 237 4208016   1   7    1
## 238 4208016   1   5    1
## 239 4208016   1   8    1
## 240 4208017   1   5    1
## 241 4208017   1   8    0
## 242 4208018   1   5    0
## 243 4208018   1   2    0
## 244 4208019   1   4    1
## 245 4208019   1   6    0
## 246 4208020   1   1    0
## 247 4208020   1   4    1
## 248 4208020   1   2    0
## 249 4208021   1   3    0
## 250 4208021   1   7    0
## 251 4208021   1   9    1
## 252 4208021   1   2    0
## 253 4208022   1   5    0
## 254 4208022   1   8    0
## 255 4208022   1   5    0
## 256 4208023   1   6    1
## 257 4208023   1   3    1
## 258 4208024   1   2    1
## 259 4208024   1   5    1
## 260 4208024   1   2    0
## 261 4208024   1   3    0
## 262 4208024   1   8    0
## 263 4208025   1   3    0
## 264 4208025   1   8    0
## 265 4208026   1   6    1
## 266 4208026   1   8    0
## 267 4208026   1   1    0
## 268 4208026   1   7    0
## 269 4208027   1   4    0
## 270 4208027   1   7    0
## 271 4208027   1   2    0
## 272 4208027   1   2    0
## 273 4208027   1   1    0
## 274 4208028   1   4    0
## 275 4208028   1   5    0
## 276 4208028   1   7    0
## 277 4208029   1   5    0
## 278 4208029   1   3    0
## 279 4208029   1   6    0
\end{verbatim}

\begin{verbatim}
##         id n.kids n.cases
## 1  4008001      2       2
## 2  4008002      2       1
## 3  4008003      2       0
## 4  4008004      6       0
## 5  4008005      1       0
## 6  4008006      2       0
## 7  4008007      3       0
## 8  4008008      3       1
## 9  4008009      2       1
## 10 4008010      4       1
## 11 4008011      3       0
## 12 4008012      1       1
## 13 4008013      6       3
## 14 4008014      3       3
## 15 4008015      7       4
## 16 4008016      7       7
## 17 4008017      2       1
## 18 4008018      2       0
## 19 4008019      2       1
## 20 4008020      3       1
## 21 4008021      4       1
## 22 4008022      3       0
## 23 4008023      2       2
## 24 4008024      5       2
## 25 4008025      2       0
## 26 4008026      4       1
## 27 4008027      5       0
## 28 4008028      3       0
## 29 4008029      3       0
## 30 4108001      2       2
## 31 4108002      2       1
## 32 4108003      2       0
## 33 4108004      5       0
## 34 4108005      1       0
## 35 4108006      2       0
## 36 4108007      3       0
## 37 4108008      3       1
## 38 4108009      2       1
## 39 4108010      4       1
## 40 4108011      3       0
## 41 4108012      1       1
## 42 4108013      6       3
## 43 4108014      3       3
## 44 4108015      7       4
## 45 4108016      7       7
## 46 4108017      2       1
## 47 4108018      2       0
## 48 4108019      2       1
## 49 4108020      3       1
## 50 4108021      4       1
## 51 4108022      3       0
## 52 4108023      2       2
## 53 4108024      5       2
## 54 4108025      2       0
## 55 4108026      4       1
## 56 4108027      5       0
## 57 4108028      3       0
## 58 4108029      3       0
## 59 4208001      2       2
## 60 4208002      2       1
## 61 4208003      2       0
## 62 4208004      4       0
## 63 4208005      1       0
## 64 4208006      2       0
## 65 4208007      3       0
## 66 4208008      3       1
## 67 4208009      2       1
## 68 4208010      4       1
## 69 4208011      3       0
## 70 4208012      1       1
## 71 4208013      6       3
## 72 4208014      3       3
## 73 4208015      7       4
## 74 4208016      7       7
## 75 4208017      2       1
## 76 4208018      2       0
## 77 4208019      2       1
## 78 4208020      3       1
## 79 4208021      4       1
## 80 4208022      3       0
## 81 4208023      2       2
## 82 4208024      5       2
## 83 4208025      2       0
## 84 4208026      4       1
## 85 4208027      5       0
## 86 4208028      3       0
## 87 4208029      3       0
\end{verbatim}

We will now write a function that will simulate a single LQAS survey.
Create a new function called \texttt{lqas.run()}:

\begin{Shaded}
\begin{Highlighting}[]
\NormalTok{lqas.run <-}\StringTok{ }\ControlFlowTok{function}\NormalTok{() \{\}}
\end{Highlighting}
\end{Shaded}

This creates an empty function called \texttt{lqas.run()}.

Use the \texttt{fix()} function to edit the \texttt{lqas.run()}
function:

\begin{Shaded}
\begin{Highlighting}[]
\KeywordTok{fix}\NormalTok{(lqas.run)}
\end{Highlighting}
\end{Shaded}

Edit the function to read:

Once you have made the changes shown above, check your work, save the
file, and quit the editor.

We should try this function on a low, a moderate, and a high prevalence
dataset. To select suitable test datasets we need to know the prevalence
in each dataset:

\begin{Shaded}
\begin{Highlighting}[]
\ControlFlowTok{for}\NormalTok{(i }\ControlFlowTok{in} \KeywordTok{dir}\NormalTok{(}\DataTypeTok{pattern =} \StringTok{"}\CharTok{\textbackslash{}\textbackslash{}}\StringTok{.sim$"}\NormalTok{)) \{}
\NormalTok{  data <-}\StringTok{ }\KeywordTok{read.table}\NormalTok{(i, }\DataTypeTok{header =} \OtherTok{TRUE}\NormalTok{)}
  \KeywordTok{cat}\NormalTok{(i, }\StringTok{":"}\NormalTok{, }\KeywordTok{mean}\NormalTok{(data}\OperatorTok{$}\NormalTok{tfti), }\StringTok{"}\CharTok{\textbackslash{}n}\StringTok{"}\NormalTok{)}
\NormalTok{\}}
\end{Highlighting}
\end{Shaded}

\begin{verbatim}
## egypt01.sim : 0.5913978 
## egypt02.sim : 0.09243697 
## egypt03.sim : 0.343949 
## egypt04.sim : 0.2546584 
## egypt05.sim : 0.1354839 
## egypt06.sim : 0.6993464 
## egypt07.sim : 0.09815951 
## egypt08.sim : 0.5668449 
## egypt09.sim : 0.5251799 
## egypt10.sim : 0.4343434 
## gambia01.sim : 0.08510638 
## gambia02.sim : 0.2517483 
## gambia03.sim : 0.1730769 
## gambia04.sim : 0.1866667 
## gambia05.sim : 0.1264368 
## gambia06.sim : 0.5180952 
## gambia07.sim : 0.3876652 
## gambia08.sim : 0.310559 
## gambia09.sim : 0.3548387 
## gambia10.sim : 0.3358491 
## ghana01.sim : 0.0990099 
## ghana02.sim : 0.2162162 
## ghana03.sim : 0.2781955 
## tanzania01.sim : 0.4672897 
## tanzania02.sim : 0.4563107 
## tanzania03.sim : 0.1025641 
## tanzania04.sim : 0.1724138 
## tanzania05.sim : 0.6962963 
## tanzania06.sim : 0.3092784 
## tanzania07.sim : 0.3727273 
## tanzania08.sim : 0.4454545 
## tanzania09.sim : 0.2146893 
## tanzania10.sim : 0.5925926 
## tanzania11.sim : 0.2544379 
## tanzania12.sim : 0.5619835 
## tanzania13.sim : 0.5086207 
## tanzania14.sim : 0.4105263 
## togo01.sim : 0.254902 
## togo02.sim : 0.1692308 
## togo03.sim : 0.2147651 
## togo04.sim : 0.1265823 
## togo05.sim : 0.1098266 
## togo06.sim : 0.2677165 
## togo07.sim : 0.3244681 
## togo08.sim : 0.3125 
## togo09.sim : 0.125 
## togo10.sim : 0.04385965 
## togo11.sim : 0.08609272
\end{verbatim}

The coding scheme used for the \texttt{tfti} variable (0=no, 1=yes)
allows us to use the \texttt{mean()} function to calculate prevalence in
these datasets.

The pattern \texttt{"\textbackslash{}\textbackslash{}.sim\$"} is a
regular expression for files ending in .sim.

If you want to use the \texttt{dir()} function to list files stored
outside of the current working directory you will need to specify an
appropriate value for the \texttt{path} parameter. On some systems you
may also need to set the value of the \texttt{full.names} parameter to
\texttt{TRUE}. For example:

\begin{Shaded}
\begin{Highlighting}[]
\ControlFlowTok{for}\NormalTok{(i }\ControlFlowTok{in} \KeywordTok{dir}\NormalTok{(}\DataTypeTok{path =} \StringTok{"~/prfe"}\NormalTok{, }\DataTypeTok{pattern =} \StringTok{"}\CharTok{\textbackslash{}\textbackslash{}}\StringTok{.sim$"}\NormalTok{, }\DataTypeTok{full.names =} \OtherTok{TRUE}\NormalTok{)) \{}
\NormalTok{  data <-}\StringTok{ }\KeywordTok{read.table}\NormalTok{(i, }\DataTypeTok{header =} \OtherTok{TRUE}\NormalTok{)}
  \KeywordTok{cat}\NormalTok{(i, }\StringTok{":"}\NormalTok{, }\KeywordTok{mean}\NormalTok{(data}\OperatorTok{$}\NormalTok{tfti), }\StringTok{"}\CharTok{\textbackslash{}n}\StringTok{"}\NormalTok{)}
\NormalTok{\}}
\end{Highlighting}
\end{Shaded}

cycles through all files ending in \texttt{.sim} (specified by giving
the value \texttt{"\textbackslash{}\textbackslash{}.sim\$"} to the
\texttt{pattern} parameter) that are stored the \texttt{prfe} directory
under the users home directory (specified by giving the value
\texttt{"\textasciitilde{}/prfe"} to the \texttt{path} parameter) on
UNIX systems. You \textbf{cannot} usefully specify a URL for the
\texttt{path} parameter of the \texttt{dir()} function.

We will use \texttt{tanzania04.sim} as an example of a low prevalence
dataset:

\begin{Shaded}
\begin{Highlighting}[]
\NormalTok{x <-}\StringTok{ }\KeywordTok{read.table}\NormalTok{(}\StringTok{"tanzania04.sim"}\NormalTok{, }\DataTypeTok{header =} \OtherTok{TRUE}\NormalTok{)}
\NormalTok{x.hh <-}\StringTok{ }\KeywordTok{ind2hh}\NormalTok{(x)}
\KeywordTok{lqas.run}\NormalTok{(}\DataTypeTok{x =}\NormalTok{ x.hh, }\DataTypeTok{n =} \DecValTok{50}\NormalTok{, }\DataTypeTok{d =} \DecValTok{14}\NormalTok{)}
\end{Highlighting}
\end{Shaded}

\begin{verbatim}
## $kids
## [1] 51
## 
## $cases
## [1] 9
## 
## $outcome
## [1] 0
\end{verbatim}

Repeat the last function call several times. Remember that previous
commands can be recalled and edited using the up and down arrow keys --
they do not need to be typed out in full each time.

The function should, for most calls, return:

\begin{verbatim}
$outcome
[1] 0
\end{verbatim}

We will use \texttt{tanzania08.sim} as an example of a high prevalence
dataset:

\begin{Shaded}
\begin{Highlighting}[]
\NormalTok{x <-}\StringTok{ }\KeywordTok{read.table}\NormalTok{(}\StringTok{"tanzania08.sim"}\NormalTok{, }\DataTypeTok{header =} \OtherTok{TRUE}\NormalTok{)}
\NormalTok{x.hh <-}\StringTok{ }\KeywordTok{ind2hh}\NormalTok{(x)}
\KeywordTok{lqas.run}\NormalTok{(}\DataTypeTok{x =}\NormalTok{ x.hh, }\DataTypeTok{n =} \DecValTok{50}\NormalTok{, }\DataTypeTok{d =} \DecValTok{14}\NormalTok{)}
\end{Highlighting}
\end{Shaded}

\begin{verbatim}
## $kids
## [1] 36
## 
## $cases
## [1] 16
## 
## $outcome
## [1] 1
\end{verbatim}

Repeat the last function call several times. The function should, for
most calls, return:

\begin{verbatim}
$outcome
[1] 1
\end{verbatim}

We will use \texttt{tanzania06.sim} as an example of a moderate
prevalence dataset:

\begin{Shaded}
\begin{Highlighting}[]
\NormalTok{x <-}\StringTok{ }\KeywordTok{read.table}\NormalTok{(}\StringTok{"tanzania06.sim"}\NormalTok{, }\DataTypeTok{header =} \OtherTok{TRUE}\NormalTok{)}
\NormalTok{x.hh <-}\StringTok{ }\KeywordTok{ind2hh}\NormalTok{(x)}
\KeywordTok{lqas.run}\NormalTok{(}\DataTypeTok{x =}\NormalTok{ x.hh, }\DataTypeTok{n =} \DecValTok{50}\NormalTok{, }\DataTypeTok{d =} \DecValTok{14}\NormalTok{)}
\end{Highlighting}
\end{Shaded}

\begin{verbatim}
## $kids
## [1] 53
## 
## $cases
## [1] 17
## 
## $outcome
## [1] 1
\end{verbatim}

Repeat the last function call several times. The function should return:

\begin{verbatim}
$outcome
[1] 0
\end{verbatim}

And:

\begin{verbatim}
$outcome
[1] 1
\end{verbatim}

In roughly equal proportion.

The simulation will require repeated sampling from the same dataset. We
need to write a function to do this.

Create a new function called \texttt{lqas.simul()}:

\begin{Shaded}
\begin{Highlighting}[]
\NormalTok{lqas.simul <-}\StringTok{ }\ControlFlowTok{function}\NormalTok{() \{\}}
\end{Highlighting}
\end{Shaded}

This creates an empty function called \texttt{lqas.simul()}.

Use the \texttt{fix()} function to edit the \texttt{lqas.simul()}
function:

\begin{Shaded}
\begin{Highlighting}[]
\KeywordTok{fix}\NormalTok{(lqas.simul)}
\end{Highlighting}
\end{Shaded}

Edit the function to read:

\begin{Shaded}
\begin{Highlighting}[]
\ControlFlowTok{function}\NormalTok{(x, n, d, runs) \{}
\NormalTok{  all <-}\StringTok{ }\KeywordTok{data.frame}\NormalTok{()}
  \ControlFlowTok{for}\NormalTok{(i }\ControlFlowTok{in} \DecValTok{1}\OperatorTok{:}\NormalTok{runs) \{}
\NormalTok{    run <-}\StringTok{ }\KeywordTok{data.frame}\NormalTok{(}\KeywordTok{lqas.run}\NormalTok{(x, n ,d))}
\NormalTok{    all <-}\StringTok{ }\KeywordTok{rbind}\NormalTok{(all, run)}
\NormalTok{  \}}
\NormalTok{  p <-}\StringTok{ }\KeywordTok{sum}\NormalTok{(x}\OperatorTok{$}\NormalTok{n.cases) }\OperatorTok{/}\StringTok{ }\KeywordTok{sum}\NormalTok{(x}\OperatorTok{$}\NormalTok{n.kids)}
\NormalTok{  asn <-}\StringTok{ }\KeywordTok{mean}\NormalTok{(all}\OperatorTok{$}\NormalTok{kids)}
\NormalTok{  p.high <-}\StringTok{ }\KeywordTok{mean}\NormalTok{(all}\OperatorTok{$}\NormalTok{outcome)}
\NormalTok{  result <-}\StringTok{ }\KeywordTok{list}\NormalTok{(}\DataTypeTok{p =}\NormalTok{ p, }\DataTypeTok{asn =}\NormalTok{ asn, }\DataTypeTok{p.high =}\NormalTok{ p.high)}
  \KeywordTok{return}\NormalTok{(result)}
\NormalTok{\}}
\end{Highlighting}
\end{Shaded}

Once you have made the changes shown above, check your work, save the
file, and quit the editor.

We can test this function with the same three test datasets:

\begin{Shaded}
\begin{Highlighting}[]
\NormalTok{x <-}\StringTok{ }\KeywordTok{read.table}\NormalTok{(}\StringTok{"tanzania04.sim"}\NormalTok{, }\DataTypeTok{header =} \OtherTok{TRUE}\NormalTok{)}
\NormalTok{x.hh <-}\StringTok{ }\KeywordTok{ind2hh}\NormalTok{(x)}
\KeywordTok{lqas.simul}\NormalTok{(}\DataTypeTok{x =}\NormalTok{ x.hh, }\DataTypeTok{n =} \DecValTok{50}\NormalTok{, }\DataTypeTok{d =} \DecValTok{14}\NormalTok{, }\DataTypeTok{runs =} \DecValTok{250}\NormalTok{)}

\NormalTok{x <-}\StringTok{ }\KeywordTok{read.table}\NormalTok{(}\StringTok{"tanzania08.sim"}\NormalTok{, }\DataTypeTok{header =} \OtherTok{TRUE}\NormalTok{)}
\NormalTok{x.hh <-}\StringTok{ }\KeywordTok{ind2hh}\NormalTok{(x)}
\KeywordTok{lqas.simul}\NormalTok{(}\DataTypeTok{x =}\NormalTok{ x.hh, }\DataTypeTok{n =} \DecValTok{50}\NormalTok{, }\DataTypeTok{d =} \DecValTok{14}\NormalTok{, }\DataTypeTok{runs =} \DecValTok{250}\NormalTok{)}

\NormalTok{x <-}\StringTok{ }\KeywordTok{read.table}\NormalTok{(}\StringTok{"tanzania06.sim"}\NormalTok{, }\DataTypeTok{header =} \OtherTok{TRUE}\NormalTok{)}
\NormalTok{x.hh <-}\StringTok{ }\KeywordTok{ind2hh}\NormalTok{(x)}
\KeywordTok{lqas.simul}\NormalTok{( }\DataTypeTok{x =}\NormalTok{ x.hh, }\DataTypeTok{n =} \DecValTok{50}\NormalTok{, }\DataTypeTok{d =} \DecValTok{14}\NormalTok{, }\DataTypeTok{runs =} \DecValTok{250}\NormalTok{)}
\end{Highlighting}
\end{Shaded}

\begin{verbatim}
## $p
## [1] 0.1724138
## 
## $asn
## [1] 50.8
## 
## $p.high
## [1] 0.028
\end{verbatim}

\begin{verbatim}
## $p
## [1] 0.4454545
## 
## $asn
## [1] 33.672
## 
## $p.high
## [1] 0.984
\end{verbatim}

\begin{verbatim}
## $p
## [1] 0.3092784
## 
## $asn
## [1] 44.896
## 
## $p.high
## [1] 0.66
\end{verbatim}

The simulation consists of applying this function to each of the
datasets in turn and collating the results. We will create a function to
do this.

Create a new function called \texttt{main.simul()}:

\begin{Shaded}
\begin{Highlighting}[]
\NormalTok{main.simul <-}\StringTok{ }\ControlFlowTok{function}\NormalTok{() \{\}}
\end{Highlighting}
\end{Shaded}

This creates an empty function called \texttt{main.simul()}.

Use the \texttt{fix()} function to edit the \texttt{main.simul()}
function:

\begin{Shaded}
\begin{Highlighting}[]
\KeywordTok{fix}\NormalTok{(main.simul)}
\end{Highlighting}
\end{Shaded}

Edit the function to read:

\begin{Shaded}
\begin{Highlighting}[]
\ControlFlowTok{function}\NormalTok{(n, d, runs) \{}
\NormalTok{  result <-}\StringTok{ }\KeywordTok{data.frame}\NormalTok{()}
  \ControlFlowTok{for}\NormalTok{(i }\ControlFlowTok{in} \KeywordTok{dir}\NormalTok{(}\DataTypeTok{pattern =} \StringTok{"}\CharTok{\textbackslash{}\textbackslash{}}\StringTok{.sim$"}\NormalTok{)) \{}
    \KeywordTok{cat}\NormalTok{(}\StringTok{"."}\NormalTok{, }\DataTypeTok{sep =} \StringTok{""}\NormalTok{)}
\NormalTok{    x <-}\StringTok{ }\KeywordTok{read.table}\NormalTok{(i, }\DataTypeTok{header =} \OtherTok{TRUE}\NormalTok{)}
\NormalTok{    y <-}\StringTok{ }\KeywordTok{ind2hh}\NormalTok{(x)}
\NormalTok{    z <-}\StringTok{ }\KeywordTok{lqas.simul}\NormalTok{(y, n, d, runs)}
\NormalTok{    z}\OperatorTok{$}\NormalTok{file.name <-}\StringTok{ }\NormalTok{i}
\NormalTok{    result <-}\StringTok{ }\KeywordTok{rbind}\NormalTok{(result, }\KeywordTok{as.data.frame}\NormalTok{(z))}
\NormalTok{  \}}
  \KeywordTok{return}\NormalTok{(result)}
\NormalTok{\}}
\end{Highlighting}
\end{Shaded}

Once you have made the changes shown above, check your work, save the
file, and quit the editor.

We are now ready to run the simulation:

\begin{Shaded}
\begin{Highlighting}[]
\NormalTok{z1 <-}\StringTok{ }\KeywordTok{main.simul}\NormalTok{(}\DataTypeTok{n =} \DecValTok{50}\NormalTok{, }\DataTypeTok{d =} \DecValTok{14}\NormalTok{, }\DataTypeTok{runs =} \DecValTok{250}\NormalTok{)}
\end{Highlighting}
\end{Shaded}

\begin{verbatim}
## ................................................
\end{verbatim}

Progress of the simulation is shown by a lengthening line of dots. Each
dot represents one community screening file processed. The
\texttt{\textgreater{}} prompt will be displayed when the simulation has
finished running.

The returned data.frame object (\texttt{z1}) contains the results of the
simulation:

\begin{Shaded}
\begin{Highlighting}[]
\NormalTok{z1}
\end{Highlighting}
\end{Shaded}

\begin{verbatim}
##             p    asn p.high      file.name
## 1  0.59139785 26.300  1.000    egypt01.sim
## 2  0.09243697 50.848  0.000    egypt02.sim
## 3  0.34394904 43.028  0.824    egypt03.sim
## 4  0.25465839 48.980  0.348    egypt04.sim
## 5  0.13548387 50.688  0.004    egypt05.sim
## 6  0.69934641 22.368  1.000    egypt06.sim
## 7  0.09815951 50.716  0.000    egypt07.sim
## 8  0.56684492 27.020  1.000    egypt08.sim
## 9  0.52517986 29.060  1.000    egypt09.sim
## 10 0.43434343 35.532  0.992    egypt10.sim
## 11 0.08510638 51.604  0.000   gambia01.sim
## 12 0.25174825 49.076  0.360   gambia02.sim
## 13 0.17307692 51.880  0.068   gambia03.sim
## 14 0.18666667 51.792  0.076   gambia04.sim
## 15 0.12643678 52.288  0.024   gambia05.sim
## 16 0.51809524 32.616  1.000   gambia06.sim
## 17 0.38766520 41.236  0.920   gambia07.sim
## 18 0.31055901 46.884  0.600   gambia08.sim
## 19 0.35483871 41.216  0.760   gambia09.sim
## 20 0.33584906 45.164  0.672   gambia10.sim
## 21 0.09900990 52.044  0.004    ghana01.sim
## 22 0.21621622 51.468  0.164    ghana02.sim
## 23 0.27819549 48.644  0.504    ghana03.sim
## 24 0.46728972 33.044  0.980 tanzania01.sim
## 25 0.45631068 33.780  0.992 tanzania02.sim
## 26 0.10256410 50.676  0.000 tanzania03.sim
## 27 0.17241379 50.624  0.036 tanzania04.sim
## 28 0.69629630 22.604  1.000 tanzania05.sim
## 29 0.30927835 45.648  0.644 tanzania06.sim
## 30 0.37272727 39.060  0.892 tanzania07.sim
## 31 0.44545455 34.592  0.980 tanzania08.sim
## 32 0.21468927 50.392  0.120 tanzania09.sim
## 33 0.59259259 25.932  1.000 tanzania10.sim
## 34 0.25443787 48.528  0.312 tanzania11.sim
## 35 0.56198347 29.184  1.000 tanzania12.sim
## 36 0.50862069 30.168  0.992 tanzania13.sim
## 37 0.41052632 36.852  0.964 tanzania14.sim
## 38 0.25490196 48.552  0.364     togo01.sim
## 39 0.16923077 55.304  0.048     togo02.sim
## 40 0.21476510 50.804  0.216     togo03.sim
## 41 0.12658228 51.952  0.000     togo04.sim
## 42 0.10982659 53.404  0.000     togo05.sim
## 43 0.26771654 47.516  0.440     togo06.sim
## 44 0.32446809 45.416  0.644     togo07.sim
## 45 0.31250000 45.708  0.620     togo08.sim
## 46 0.12500000 51.340  0.000     togo09.sim
## 47 0.04385965 52.612  0.000     togo10.sim
## 48 0.08609272 52.124  0.000     togo11.sim
\end{verbatim}

We can examine the prevalences in the test datasets:

\begin{Shaded}
\begin{Highlighting}[]
\KeywordTok{x11}\NormalTok{()}

\KeywordTok{hist}\NormalTok{(z1}\OperatorTok{$}\NormalTok{p,}
     \DataTypeTok{main =} \StringTok{"Prevalence in test datasets"}\NormalTok{,}
     \DataTypeTok{xlab =} \StringTok{"Proportion TF/TI"}\NormalTok{)}
\end{Highlighting}
\end{Shaded}

\includegraphics{prfe_files/figure-latex/unnamed-chunk-517-1.pdf}

If you are using a Macintosh computer then you can use \texttt{quartz()}
instead of \texttt{x11()}. This will give better results.

We examine the sample size required to make classifications at different
levels of prevalence as \emph{anaverage sample number} (ASN) curve:

\begin{Shaded}
\begin{Highlighting}[]
\KeywordTok{x11}\NormalTok{()}

\KeywordTok{plot}\NormalTok{(z1}\OperatorTok{$}\NormalTok{p,}
\NormalTok{     z1}\OperatorTok{$}\NormalTok{asn,}
     \DataTypeTok{main =} \StringTok{"ASN Curve"}\NormalTok{,}
     \DataTypeTok{xlab =} \StringTok{"Proportion TF/TI"}\NormalTok{,}
     \DataTypeTok{ylab =} \StringTok{"Sample size required"}\NormalTok{)}
\end{Highlighting}
\end{Shaded}

\includegraphics{prfe_files/figure-latex/unnamed-chunk-519-1.pdf}

We can examine the performance of the sampling plan by plotting its
\emph{operating characteristic} (OC) curve:

\begin{Shaded}
\begin{Highlighting}[]
\KeywordTok{x11}\NormalTok{()}

\KeywordTok{plot}\NormalTok{(z1}\OperatorTok{$}\NormalTok{p,}
\NormalTok{     z1}\OperatorTok{$}\NormalTok{p.high,}
     \DataTypeTok{main =} \StringTok{"OC Curve"}\NormalTok{,}
     \DataTypeTok{xlab =} \StringTok{"Proportion TF/TI"}\NormalTok{,}
     \DataTypeTok{ylab =} \StringTok{"Probability"}\NormalTok{)}
\end{Highlighting}
\end{Shaded}

\includegraphics{prfe_files/figure-latex/unnamed-chunk-521-1.pdf}

Before closing the OC curve plot, we can compare the simulation results
with the expected \emph{operating characteristic} (OC) curve under ideal
sampling conditions:

\begin{Shaded}
\begin{Highlighting}[]
\KeywordTok{lines}\NormalTok{(}\KeywordTok{seq}\NormalTok{(}\DecValTok{0}\NormalTok{, }\KeywordTok{max}\NormalTok{(z1}\OperatorTok{$}\NormalTok{p), }\FloatTok{0.01}\NormalTok{),}
      \KeywordTok{pbinom}\NormalTok{(}\DecValTok{14}\NormalTok{, }\DecValTok{50}\NormalTok{,}\KeywordTok{seq}\NormalTok{(}\DecValTok{0}\NormalTok{, }\KeywordTok{max}\NormalTok{(z1}\OperatorTok{$}\NormalTok{p),}\FloatTok{0.01}\NormalTok{), }\DataTypeTok{lower.tail =} \OtherTok{FALSE}\NormalTok{),}
      \DataTypeTok{lty =} \DecValTok{3}\NormalTok{)}
\end{Highlighting}
\end{Shaded}

\includegraphics{prfe_files/figure-latex/unnamed-chunk-523-1.pdf}

The LQAS method appears to be robust to the loss of sampling variation
introduced by the proposed sampling constraints. There is, however, some
deviation from the expected \emph{operating characteristic} (OC) curve
at lower prevalences:

\begin{Shaded}
\begin{Highlighting}[]
\KeywordTok{x11}\NormalTok{()}
\KeywordTok{plot}\NormalTok{(z1}\OperatorTok{$}\NormalTok{p, z1}\OperatorTok{$}\NormalTok{p.high,}
     \DataTypeTok{xlim =} \KeywordTok{c}\NormalTok{(}\FloatTok{0.10}\NormalTok{, }\FloatTok{0.35}\NormalTok{),}
     \DataTypeTok{main =} \StringTok{"OC Curve"}\NormalTok{,}
     \DataTypeTok{xlab =} \StringTok{"Proportion TF/TI"}\NormalTok{,}
     \DataTypeTok{ylab =} \StringTok{"Probability"}\NormalTok{)}

\KeywordTok{lines}\NormalTok{(}\KeywordTok{seq}\NormalTok{(}\FloatTok{0.10}\NormalTok{, }\FloatTok{0.35}\NormalTok{, }\FloatTok{0.01}\NormalTok{),}
      \KeywordTok{pbinom}\NormalTok{(}\DecValTok{14}\NormalTok{, }\DecValTok{50}\NormalTok{, }\KeywordTok{seq}\NormalTok{(}\FloatTok{0.10}\NormalTok{, }\FloatTok{0.35}\NormalTok{, }\FloatTok{0.01}\NormalTok{), }\DataTypeTok{lower.tail =} \OtherTok{FALSE}\NormalTok{),}
      \DataTypeTok{lty =} \DecValTok{3}\NormalTok{)}
\end{Highlighting}
\end{Shaded}

\includegraphics{prfe_files/figure-latex/unnamed-chunk-525-1.pdf}

This deviation from the expected \emph{operating characteristic} (OC)
curve is likely to be due to a few very large households in which many
of the children have active trachoma. You can check this by examining
the \texttt{p.high} (i.e.~the probability of a classification as high
prevalence) column in \texttt{z1}, and household size and trachoma
status in the individual data files that return higher than expected
values for \texttt{p.high}. The \texttt{ind2hh()} function is likely to
prove useful in this context.

The observed deviation from the expected \emph{operating characteristic}
(OC) curve is small but it is important, in the resource-constrained
context of trachoma-endemic countries, to minimise the false positive
rate in order to ensure that resources are devoted to communities that
need them most.

We might be able to improve the performance of the survey method in this
regard by restricting the sample so that only younger children are
examined. This will have the effect of reducing the number of children
examined in each household. It also has the benefit of making the
surveys simpler and quicker since younger children will tend to be
closer to home than older children.

The range of ages in the datasets can be found using:

\begin{Shaded}
\begin{Highlighting}[]
\ControlFlowTok{for}\NormalTok{(i }\ControlFlowTok{in} \KeywordTok{dir}\NormalTok{(}\DataTypeTok{pattern =} \StringTok{"}\CharTok{\textbackslash{}\textbackslash{}}\StringTok{.sim$"}\NormalTok{)) \{}
\NormalTok{  x <-}\StringTok{ }\KeywordTok{read.table}\NormalTok{(i, }\DataTypeTok{header =} \OtherTok{TRUE}\NormalTok{)}
  \KeywordTok{cat}\NormalTok{(i, }\StringTok{":"}\NormalTok{, }\KeywordTok{range}\NormalTok{(x}\OperatorTok{$}\NormalTok{age), }\StringTok{"}\CharTok{\textbackslash{}n}\StringTok{"}\NormalTok{)}
\NormalTok{\}}
\end{Highlighting}
\end{Shaded}

\begin{verbatim}
## egypt01.sim : 1 10 
## egypt02.sim : 2 8 
## egypt03.sim : 2 10 
## egypt04.sim : 1 10 
## egypt05.sim : 2 6 
## egypt06.sim : 2 10 
## egypt07.sim : 2 6 
## egypt08.sim : 2 10 
## egypt09.sim : 2 10 
## egypt10.sim : 2 10 
## gambia01.sim : 1 9 
## gambia02.sim : 1 9 
## gambia03.sim : 1 9 
## gambia04.sim : 1 9 
## gambia05.sim : 1 9 
## gambia06.sim : 1 9 
## gambia07.sim : 1 9 
## gambia08.sim : 1 9 
## gambia09.sim : 1 9 
## gambia10.sim : 1 9 
## ghana01.sim : 1 10 
## ghana02.sim : 1 10 
## ghana03.sim : 1 10 
## tanzania01.sim : 1 10 
## tanzania02.sim : 1 10 
## tanzania03.sim : 1 10 
## tanzania04.sim : 1 10 
## tanzania05.sim : 1 10 
## tanzania06.sim : 1 10 
## tanzania07.sim : 1 10 
## tanzania08.sim : 1 10 
## tanzania09.sim : 1 10 
## tanzania10.sim : 1 10 
## tanzania11.sim : 1 10 
## tanzania12.sim : 1 10 
## tanzania13.sim : 1 10 
## tanzania14.sim : 1 10 
## togo01.sim : 1 10 
## togo02.sim : 1 10 
## togo03.sim : 1 10 
## togo04.sim : 1 10 
## togo05.sim : 1 10 
## togo06.sim : 1 10 
## togo07.sim : 1 10 
## togo08.sim : 1 10 
## togo09.sim : 1 10 
## togo10.sim : 1 10 
## togo11.sim : 1 10
\end{verbatim}

We will investigate the effect of restricting the sample to children
aged between two and five years. Use the fix() function to edit the
\texttt{main.simul()} function:

\begin{Shaded}
\begin{Highlighting}[]
\KeywordTok{fix}\NormalTok{(main.simul)}
\end{Highlighting}
\end{Shaded}

Edit the function to read:

\begin{Shaded}
\begin{Highlighting}[]
\ControlFlowTok{function}\NormalTok{(n, d, runs) \{}
\NormalTok{  result <-}\StringTok{ }\KeywordTok{data.frame}\NormalTok{()}
  \ControlFlowTok{for}\NormalTok{(i }\ControlFlowTok{in} \KeywordTok{dir}\NormalTok{(}\DataTypeTok{pattern =} \StringTok{"}\CharTok{\textbackslash{}\textbackslash{}}\StringTok{.sim$"}\NormalTok{)) \{}
    \KeywordTok{cat}\NormalTok{(}\StringTok{"."}\NormalTok{, }\DataTypeTok{sep =} \StringTok{""}\NormalTok{)}
\NormalTok{    x <-}\StringTok{ }\KeywordTok{read.table}\NormalTok{(i, }\DataTypeTok{header =} \OtherTok{TRUE}\NormalTok{)}
\NormalTok{    x <-}\StringTok{ }\KeywordTok{subset}\NormalTok{(x, age }\OperatorTok{>=}\StringTok{ }\DecValTok{2} \OperatorTok{&}\StringTok{ }\NormalTok{age }\OperatorTok{<=}\StringTok{ }\DecValTok{5}\NormalTok{)}
\NormalTok{    y <-}\StringTok{ }\KeywordTok{ind2hh}\NormalTok{(x)}
\NormalTok{    z <-}\StringTok{ }\KeywordTok{lqas.simul}\NormalTok{(y, n, d, runs) z}\OperatorTok{$}\NormalTok{file.name <-}\StringTok{ }\NormalTok{i}
\NormalTok{    result <-}\StringTok{ }\KeywordTok{rbind}\NormalTok{(result, }\KeywordTok{as.data.frame}\NormalTok{(z))}
\NormalTok{  \}}
  \KeywordTok{return}\NormalTok{(result)}
\NormalTok{\}}
\end{Highlighting}
\end{Shaded}

Once you have made the changes shown above, check your work, save the
file, and quit the editor.

We are now ready to run the simulation again:

\begin{Shaded}
\begin{Highlighting}[]
\NormalTok{z2 <-}\StringTok{ }\KeywordTok{main.simul}\NormalTok{(}\DataTypeTok{n =} \DecValTok{50}\NormalTok{, }\DataTypeTok{d =} \DecValTok{14}\NormalTok{, }\DataTypeTok{runs =} \DecValTok{250}\NormalTok{)}
\end{Highlighting}
\end{Shaded}

\begin{verbatim}
## ................................................
\end{verbatim}

The data.frame object \texttt{z2} contains the results of the
simulation:

\begin{Shaded}
\begin{Highlighting}[]
\NormalTok{z2}
\end{Highlighting}
\end{Shaded}

\begin{verbatim}
##             p    asn p.high      file.name
## 1  0.59139785 26.428  1.000    egypt01.sim
## 2  0.09243697 50.880  0.000    egypt02.sim
## 3  0.34394904 42.548  0.760    egypt03.sim
## 4  0.25465839 48.952  0.312    egypt04.sim
## 5  0.13548387 50.632  0.016    egypt05.sim
## 6  0.69934641 22.580  1.000    egypt06.sim
## 7  0.09815951 50.680  0.000    egypt07.sim
## 8  0.56684492 26.680  1.000    egypt08.sim
## 9  0.52517986 29.516  1.000    egypt09.sim
## 10 0.43434343 34.936  1.000    egypt10.sim
## 11 0.08510638 51.612  0.000   gambia01.sim
## 12 0.25174825 49.824  0.324   gambia02.sim
## 13 0.17307692 51.912  0.072   gambia03.sim
## 14 0.18666667 51.780  0.056   gambia04.sim
## 15 0.12643678 52.088  0.008   gambia05.sim
## 16 0.51809524 32.800  1.000   gambia06.sim
## 17 0.38766520 41.164  0.928   gambia07.sim
## 18 0.31055901 47.072  0.628   gambia08.sim
## 19 0.35483871 43.120  0.748   gambia09.sim
## 20 0.33584906 45.096  0.660   gambia10.sim
## 21 0.09900990 52.116  0.000    ghana01.sim
## 22 0.21621622 51.504  0.196    ghana02.sim
## 23 0.27819549 49.000  0.528    ghana03.sim
## 24 0.46728972 32.976  0.988 tanzania01.sim
## 25 0.45631068 34.104  0.992 tanzania02.sim
## 26 0.10256410 50.792  0.000 tanzania03.sim
## 27 0.17241379 50.608  0.032 tanzania04.sim
## 28 0.69629630 23.244  1.000 tanzania05.sim
## 29 0.30927835 44.996  0.612 tanzania06.sim
## 30 0.37272727 41.176  0.872 tanzania07.sim
## 31 0.44545455 34.444  0.972 tanzania08.sim
## 32 0.21468927 50.580  0.120 tanzania09.sim
## 33 0.59259259 25.936  1.000 tanzania10.sim
## 34 0.25443787 48.852  0.292 tanzania11.sim
## 35 0.56198347 29.160  1.000 tanzania12.sim
## 36 0.50862069 30.872  1.000 tanzania13.sim
## 37 0.41052632 37.240  0.980 tanzania14.sim
## 38 0.25490196 49.500  0.328     togo01.sim
## 39 0.16923077 55.640  0.084     togo02.sim
## 40 0.21476510 51.032  0.176     togo03.sim
## 41 0.12658228 52.252  0.004     togo04.sim
## 42 0.10982659 52.976  0.000     togo05.sim
## 43 0.26771654 47.936  0.424     togo06.sim
## 44 0.32446809 44.680  0.724     togo07.sim
## 45 0.31250000 45.564  0.660     togo08.sim
## 46 0.12500000 51.400  0.000     togo09.sim
## 47 0.04385965 52.408  0.000     togo10.sim
## 48 0.08609272 52.488  0.000     togo11.sim
\end{verbatim}

We can examine the performance of the sampling plan on the
age-restricted datasets by plotting its \emph{operating characteristic}
(OC) curve and comparing the simulation results with the expected
\emph{operating characteristic} (OC) curve under ideal sampling
conditions:

\begin{Shaded}
\begin{Highlighting}[]
\KeywordTok{x11}\NormalTok{()}

\KeywordTok{plot}\NormalTok{(z2}\OperatorTok{$}\NormalTok{p, z2}\OperatorTok{$}\NormalTok{p.high,}
     \DataTypeTok{main =} \StringTok{"OC Curve"}\NormalTok{,}
     \DataTypeTok{xlab =} \StringTok{"Proportion TF/TI"}\NormalTok{,}
     \DataTypeTok{ylab =} \StringTok{"Probability "}\NormalTok{)}

\KeywordTok{lines}\NormalTok{(}\KeywordTok{seq}\NormalTok{(}\DecValTok{0}\NormalTok{, }\KeywordTok{max}\NormalTok{(z2}\OperatorTok{$}\NormalTok{p), }\FloatTok{0.01}\NormalTok{),}
      \KeywordTok{pbinom}\NormalTok{(}\DecValTok{14}\NormalTok{, }\DecValTok{50}\NormalTok{, }\KeywordTok{seq}\NormalTok{(}\DecValTok{0}\NormalTok{, }\KeywordTok{max}\NormalTok{(z2}\OperatorTok{$}\NormalTok{p), }\FloatTok{0.01}\NormalTok{), }\DataTypeTok{lower.tail =} \OtherTok{FALSE}\NormalTok{),}
      \DataTypeTok{lty =} \DecValTok{3}\NormalTok{)}
\end{Highlighting}
\end{Shaded}

\includegraphics{prfe_files/figure-latex/unnamed-chunk-532-1.pdf}

Remember that if you are using a Macintosh computer then you can use
\texttt{quartz()} instead of \texttt{x11()}. This will give better
results.

We should also take a closer look at the range of prevalences where the
deviation from the expected \emph{operating characteristic} (OC) curve
was largest and most problematic in the previous simulation:

\begin{Shaded}
\begin{Highlighting}[]
\KeywordTok{x11}\NormalTok{()}

\KeywordTok{plot}\NormalTok{(z2}\OperatorTok{$}\NormalTok{p, z2}\OperatorTok{$}\NormalTok{p.high,}
     \DataTypeTok{xlim =} \KeywordTok{c}\NormalTok{(}\FloatTok{0.10}\NormalTok{, }\FloatTok{0.35}\NormalTok{),}
     \DataTypeTok{main =} \StringTok{"OC Curve"}\NormalTok{,}
     \DataTypeTok{xlab =} \StringTok{"Proportion TF/TI"}\NormalTok{,}
     \DataTypeTok{ylab =} \StringTok{"Probability"}\NormalTok{)}

\KeywordTok{lines}\NormalTok{(}\KeywordTok{seq}\NormalTok{(}\FloatTok{0.10}\NormalTok{, }\FloatTok{0.35}\NormalTok{, }\FloatTok{0.01}\NormalTok{),}
      \KeywordTok{pbinom}\NormalTok{(}\DecValTok{14}\NormalTok{, }\DecValTok{50}\NormalTok{, }\KeywordTok{seq}\NormalTok{(}\FloatTok{0.10}\NormalTok{, }\FloatTok{0.35}\NormalTok{, }\FloatTok{0.01}\NormalTok{), }\DataTypeTok{lower.tail =} \OtherTok{FALSE}\NormalTok{),}
      \DataTypeTok{lty =} \DecValTok{3}\NormalTok{)}
\end{Highlighting}
\end{Shaded}

\includegraphics{prfe_files/figure-latex/unnamed-chunk-534-1.pdf}

We can compare the behaviour of the sampling plan in the unrestricted
and age-restricted datasets:

\begin{Shaded}
\begin{Highlighting}[]
\KeywordTok{x11}\NormalTok{()}

\KeywordTok{plot}\NormalTok{(z1}\OperatorTok{$}\NormalTok{p, z1}\OperatorTok{$}\NormalTok{p.high,}
     \DataTypeTok{main =} \StringTok{""}\NormalTok{,}
     \DataTypeTok{xlab =} \StringTok{""}\NormalTok{,}
     \DataTypeTok{ylab =} \StringTok{""}\NormalTok{,}
     \DataTypeTok{axes =} \OtherTok{FALSE}\NormalTok{,}
     \DataTypeTok{xlim =} \KeywordTok{c}\NormalTok{(}\DecValTok{0}\NormalTok{, }\FloatTok{0.8}\NormalTok{))}

\KeywordTok{par}\NormalTok{(}\DataTypeTok{new =} \OtherTok{TRUE}\NormalTok{)}

\KeywordTok{plot}\NormalTok{(z2}\OperatorTok{$}\NormalTok{p, z2}\OperatorTok{$}\NormalTok{p.high,}
     \DataTypeTok{main =} \StringTok{"OC Curve"}\NormalTok{,}
     \DataTypeTok{xlab =} \StringTok{"Proportion TF/TI"}\NormalTok{,}
     \DataTypeTok{ylab =} \StringTok{"Probability"}\NormalTok{,}
     \DataTypeTok{xlim =} \KeywordTok{c}\NormalTok{(}\DecValTok{0}\NormalTok{, }\FloatTok{0.8}\NormalTok{),}
     \DataTypeTok{pch =} \DecValTok{3}\NormalTok{)}
   
\KeywordTok{lines}\NormalTok{(}\KeywordTok{seq}\NormalTok{(}\DecValTok{0}\NormalTok{, }\FloatTok{0.8}\NormalTok{, }\FloatTok{0.01}\NormalTok{),}
      \KeywordTok{pbinom}\NormalTok{(}\DecValTok{14}\NormalTok{, }\DecValTok{50}\NormalTok{, }\KeywordTok{seq}\NormalTok{(}\DecValTok{0}\NormalTok{, }\FloatTok{0.8}\NormalTok{, }\FloatTok{0.01}\NormalTok{), }\DataTypeTok{lower.tail =} \OtherTok{FALSE}\NormalTok{),}
      \DataTypeTok{lty =} \DecValTok{3}\NormalTok{)}
\end{Highlighting}
\end{Shaded}

Restricting the sample to children aged between two and five years
(inclusive) appears to have improved the behaviour of the survey method
by lowering the false positive rate to close to expected behaviour under
ideal sampling condition. Process simulation has allowed us to improve
the performance of the survey method without expensive and lengthy
field-work. In practice, the method would now be validated in the field
probably by repeated sampling of communities in which prevalence is
known from house-to-house screening.

\hypertarget{cellular-automata-machines}{%
\section{Cellular automata machines}\label{cellular-automata-machines}}

\emph{Cellular automata machines} are simple computing devices that are
commonly used to simulate social, biological, and physical processes.
Despite their simplicity, cellular automata machines are general purpose
computing devices. This means that they may be used for any
\emph{computable} problem.

The way that problems are specified to cellular automata machines make
them simple to program for some types of problem and difficult to
program for other types of problem. In this exercise we will explore the
use of cellular automata machines to create a simple model of epidemic
spread.

Cellular automata machines model a universe in which space is
represented by a uniform grid, time advances in steps, and the laws of
the universe are represented by a set of rules which compute the future
state of each cell of the grid from its current state and from the
current state of its neighbouring cells.

Typically, a cellular automata machine has the following features:

\begin{enumerate}
\def\labelenumi{\arabic{enumi}.}
\item
  It consists of a large number of identical cells arranged in a regular
  grid. The grid is a two-dimensional projection of a torus (a
  ring-doughnut shaped surface) and has no edges.
\item
  Each cell can be in one of a limited number of states.
\item
  Time advances through the simulation in steps. At each time-step, the
  state of a cell may change.
\item
  The state of a cell after each time-step is determined by a set of
  rules that define how the future state of a cell depends on the
  current state of the cell and the current state of its immediate
  neighbours. This set of rules is used to update the state of every
  cell in the grid at each time-step. Since the rules refer only to the
  state of an individual cell and its immediate neighbours, cellular
  automata machines are best suited to modelling situations where local
  interactions give rise to global phenomena.
\end{enumerate}

In this exercise we will simulate a cellular automata machine using
\texttt{R} and then use the simulated machine to simulate epidemic
spread. We will use three functions to simulate the cellular automata
machine (CAM):

\begin{itemize}
\item
  \textbf{cam.run():} This function will display the initial state of
  the CAM, examine each cell in the CAM grid, apply the CAM rule-set to
  each cell, update the CAM grid, and display the state of the CAM at
  each time-step.
\item
  \textbf{cam.state.display():} This function will display the state of
  the CAM at each time-step. It will be implemented using the
  \texttt{image()} function to plot the contents of matrices held in the
  \texttt{cam.state} list object (see below). This function will be
  developed as we refine the epidemic model.
\item
  \textbf{cam.rule():} This function will contain the rule-set. It will
  be implemented using \texttt{ifelse()} functions to codify rules. This
  function will also be developed as we refine the epidemic model.
\end{itemize}

The current and future states of the CAM will be held in two
\emph{global} list objects:

\begin{itemize}
\item
  \textbf{cam.state:} This object will contain the current state of the
  CAM and must contain a matrix object called grid.
\item
  \textbf{cam.state.new:} This object will contain the the state of the
  CAM at the next time-step as defined by the the rule-set.
\end{itemize}

Other \emph{global} objects will be defined as required.

Create a new function called \texttt{cam.run():}

\begin{Shaded}
\begin{Highlighting}[]
\NormalTok{cam.run <-}\StringTok{ }\ControlFlowTok{function}\NormalTok{() \{\}}
\end{Highlighting}
\end{Shaded}

This creates an empty function called \texttt{cam.run()}.

Use the \texttt{fix()} function to edit the \texttt{cam.run()} function:

\begin{Shaded}
\begin{Highlighting}[]
\KeywordTok{fix}\NormalTok{(cam.run)}
\end{Highlighting}
\end{Shaded}

Edit the function to read:

\begin{Shaded}
\begin{Highlighting}[]
\ControlFlowTok{function}\NormalTok{(steps) \{}
\NormalTok{  cam.state.new <<-}\StringTok{ }\NormalTok{cam.state}
  \KeywordTok{cam.state.display}\NormalTok{(}\DataTypeTok{t =} \DecValTok{0}\NormalTok{)}
\NormalTok{  mx <-}\StringTok{ }\KeywordTok{nrow}\NormalTok{(cam.state}\OperatorTok{$}\NormalTok{grid)}
\NormalTok{  my <-}\StringTok{ }\KeywordTok{ncol}\NormalTok{(cam.state}\OperatorTok{$}\NormalTok{grid)}
  \ControlFlowTok{for}\NormalTok{(t }\ControlFlowTok{in} \DecValTok{1}\OperatorTok{:}\NormalTok{steps) \{}
    \ControlFlowTok{for}\NormalTok{(y }\ControlFlowTok{in} \DecValTok{1}\OperatorTok{:}\NormalTok{my) \{}
      \ControlFlowTok{for}\NormalTok{(x }\ControlFlowTok{in} \DecValTok{1}\OperatorTok{:}\NormalTok{mx) \{}
\NormalTok{        V <-}\StringTok{ }\NormalTok{cam.state}\OperatorTok{$}\NormalTok{grid[x, y]}
\NormalTok{        N <-}\StringTok{ }\NormalTok{cam.state}\OperatorTok{$}\NormalTok{grid[x, }\KeywordTok{ifelse}\NormalTok{(y }\OperatorTok{==}\StringTok{ }\DecValTok{1}\NormalTok{, my, y }\OperatorTok{-}\StringTok{ }\DecValTok{1}\NormalTok{)]}
\NormalTok{        S <-}\StringTok{ }\NormalTok{cam.state}\OperatorTok{$}\NormalTok{grid[x, }\KeywordTok{ifelse}\NormalTok{(y }\OperatorTok{==}\StringTok{ }\NormalTok{my, }\DecValTok{1}\NormalTok{, y }\OperatorTok{+}\StringTok{ }\DecValTok{1}\NormalTok{)]}
\NormalTok{        E <-}\StringTok{ }\NormalTok{cam.state}\OperatorTok{$}\NormalTok{grid[}\KeywordTok{ifelse}\NormalTok{(x }\OperatorTok{==}\StringTok{ }\NormalTok{mx, }\DecValTok{1}\NormalTok{, x }\OperatorTok{+}\StringTok{ }\DecValTok{1}\NormalTok{), y]}
\NormalTok{        W <-}\StringTok{ }\NormalTok{cam.state}\OperatorTok{$}\NormalTok{grid[}\KeywordTok{ifelse}\NormalTok{(x }\OperatorTok{==}\StringTok{ }\DecValTok{1}\NormalTok{, mx, x }\OperatorTok{-}\StringTok{ }\DecValTok{1}\NormalTok{), y]}
        \KeywordTok{cam.rule}\NormalTok{(V, N, S, E, W, x, y, t)}
\NormalTok{        \}}
\NormalTok{      \}}
\NormalTok{      cam.state <<-}\StringTok{ }\NormalTok{cam.state.new}
      \KeywordTok{cam.state.display}\NormalTok{(t)}
\NormalTok{  \}}
\NormalTok{\}}
\end{Highlighting}
\end{Shaded}

Once you have made the changes shown above, check your work, save the
file, and quit the

Note that when we assign anything to the state of the CAM (held in
\texttt{cam.state} and \texttt{cam.state.new}) we use the
\texttt{\textless{}\textless{}-} (instead of the usual
\texttt{\textless{}-}) assignment operator. This operator allows
assignment to objects outside of the function in the global environment.

Objects that are stored in the \emph{global} environment are available
to all functions.

Create a new function called \texttt{cam.state.display()}:

\begin{Shaded}
\begin{Highlighting}[]
\NormalTok{cam.state.display <-}\StringTok{ }\ControlFlowTok{function}\NormalTok{() \{\}}
\end{Highlighting}
\end{Shaded}

This creates an empty function called \texttt{cam.state.display()}.

Use the \texttt{fix()} function to edit the \texttt{cam.state.display()}
function:

\begin{Shaded}
\begin{Highlighting}[]
\KeywordTok{fix}\NormalTok{(cam.state.display)}
\end{Highlighting}
\end{Shaded}

Edit the function to read:

\begin{Shaded}
\begin{Highlighting}[]
\ControlFlowTok{function}\NormalTok{(t) \{}
  \ControlFlowTok{if}\NormalTok{(t }\OperatorTok{==}\StringTok{ }\DecValTok{0}\NormalTok{) \{}
    \KeywordTok{x11}\NormalTok{(); }\KeywordTok{par}\NormalTok{(}\DataTypeTok{pty =} \StringTok{"s"}\NormalTok{)}
\NormalTok{  \}}
  \KeywordTok{image}\NormalTok{(cam.state}\OperatorTok{$}\NormalTok{grid, }\DataTypeTok{main =} \KeywordTok{paste}\NormalTok{(}\StringTok{"Infected at :"}\NormalTok{, t),}
        \DataTypeTok{col =} \KeywordTok{c}\NormalTok{(}\StringTok{"wheat"}\NormalTok{, }\StringTok{"navy"}\NormalTok{), }\DataTypeTok{axes =} \OtherTok{FALSE}\NormalTok{)}
\NormalTok{\}}
\end{Highlighting}
\end{Shaded}

Remember that if you are using a Macintosh computer then you can use
\texttt{quartz()} instead of \texttt{x11()}. This will give better
results.

Once you have made the changes shown above, check your work, save the
file, and quit the editor.

The \texttt{cam.rule()} function contains the rules of the CAM universe.

We will start with a very simple \emph{infection} rule:

\begin{itemize}
\item
  Each cell can be either \emph{infected} or \emph{not-infected}.
\item
  If a cell is already \emph{infected} it will remain \emph{infected}.
\item
  If a cell is \emph{not-infected} then it will change its state to
  \emph{infected} based on the state of its neighbours: If a
  neighbouring cell is \emph{infected} it will infect the cell with a
  fixed probability or \emph{transmission} pressure.
\end{itemize}

Create a new function called \texttt{cam.rule()}:

\begin{Shaded}
\begin{Highlighting}[]
\NormalTok{cam.rule <-}\StringTok{ }\ControlFlowTok{function}\NormalTok{() \{\}}
\end{Highlighting}
\end{Shaded}

This creates an empty function called \texttt{cam.rule()}. Use the
\texttt{fix()} function to edit the \texttt{cam.rule()} function:

\begin{Shaded}
\begin{Highlighting}[]
\KeywordTok{fix}\NormalTok{(cam.rule)}
\end{Highlighting}
\end{Shaded}

Edit the function to read:

\begin{Shaded}
\begin{Highlighting}[]
\ControlFlowTok{function}\NormalTok{(V, N, S, E, W, x, y, t) \{}
\NormalTok{  tp <-}\StringTok{ }\KeywordTok{c}\NormalTok{(N, S, E, W) }\OperatorTok{*}\StringTok{ }\KeywordTok{rbinom}\NormalTok{(}\DecValTok{4}\NormalTok{, }\DecValTok{1}\NormalTok{, TP)}
\NormalTok{  cam.state.new}\OperatorTok{$}\NormalTok{grid[x, y] <<-}\StringTok{ }\KeywordTok{ifelse}\NormalTok{(V }\OperatorTok{==}\StringTok{ }\DecValTok{1} \OperatorTok{|}\StringTok{ }\KeywordTok{sum}\NormalTok{(tp) }\OperatorTok{>}\StringTok{ }\DecValTok{0}\NormalTok{, }\DecValTok{1}\NormalTok{, }\DecValTok{0}\NormalTok{)}
\NormalTok{\}}
\end{Highlighting}
\end{Shaded}

Once you have made the changes shown above, check your work, save the
file, and quit the editor. The basic CAM machine is now complete.

We need to specify a value for the transmission pressure (\texttt{TP}):

\begin{Shaded}
\begin{Highlighting}[]
\NormalTok{TP <-}\StringTok{ }\FloatTok{0.2}
\end{Highlighting}
\end{Shaded}

And define the initial state of the CAM:

\begin{Shaded}
\begin{Highlighting}[]
\NormalTok{cases <-}\StringTok{ }\KeywordTok{matrix}\NormalTok{(}\DecValTok{0}\NormalTok{, }\DataTypeTok{nrow =} \DecValTok{19}\NormalTok{, }\DataTypeTok{ncol =} \DecValTok{19}\NormalTok{)}
\NormalTok{cases[}\DecValTok{10}\NormalTok{, }\DecValTok{10}\NormalTok{] <-}\StringTok{ }\DecValTok{1}
\NormalTok{cam.state <-}\StringTok{ }\KeywordTok{list}\NormalTok{(}\DataTypeTok{grid =}\NormalTok{ cases)}
\end{Highlighting}
\end{Shaded}

We can now run the simulation:

\begin{Shaded}
\begin{Highlighting}[]
\KeywordTok{cam.run}\NormalTok{(}\DataTypeTok{steps =} \DecValTok{20}\NormalTok{)}
\end{Highlighting}
\end{Shaded}

The number of infected cells is:

\begin{Shaded}
\begin{Highlighting}[]
\KeywordTok{sum}\NormalTok{(cam.state}\OperatorTok{$}\NormalTok{grid)}
\end{Highlighting}
\end{Shaded}

\begin{verbatim}
## [1] 128
\end{verbatim}

We can use this model to investigate the effect of different
transmission pressures by systematically altering the transmission
pressure specified in \texttt{TP}:

\begin{Shaded}
\begin{Highlighting}[]
\NormalTok{cases <-}\StringTok{ }\KeywordTok{matrix}\NormalTok{(}\DecValTok{0}\NormalTok{, }\DataTypeTok{nrow =} \DecValTok{19}\NormalTok{, }\DataTypeTok{ncol =} \DecValTok{19}\NormalTok{)}
\NormalTok{cases[}\DecValTok{10}\NormalTok{, }\DecValTok{10}\NormalTok{] <-}\StringTok{ }\DecValTok{1}
\NormalTok{cam.state.initial <-}\StringTok{ }\KeywordTok{list}\NormalTok{(}\DataTypeTok{grid =}\NormalTok{ cases)}
\NormalTok{pressure <-}\StringTok{ }\KeywordTok{vector}\NormalTok{(}\DataTypeTok{mode =} \StringTok{"numeric"}\NormalTok{)}
\NormalTok{infected <-}\StringTok{ }\KeywordTok{vector}\NormalTok{(}\DataTypeTok{mode =} \StringTok{"numeric"}\NormalTok{)}
   
\ControlFlowTok{for}\NormalTok{(TP }\ControlFlowTok{in} \KeywordTok{seq}\NormalTok{(}\FloatTok{0.05}\NormalTok{, }\FloatTok{0.25}\NormalTok{, }\FloatTok{0.05}\NormalTok{)) \{}
\NormalTok{  cam.state <-}\StringTok{ }\NormalTok{cam.state.initial}
  \KeywordTok{cam.run}\NormalTok{(}\DataTypeTok{steps =} \DecValTok{20}\NormalTok{)}
  \KeywordTok{graphics.off}\NormalTok{()}
\NormalTok{  pressure <-}\StringTok{ }\KeywordTok{c}\NormalTok{(pressure, TP)}
\NormalTok{  infected <-}\StringTok{ }\KeywordTok{c}\NormalTok{(infected, }\KeywordTok{sum}\NormalTok{(cam.state}\OperatorTok{$}\NormalTok{grid))}
\NormalTok{  \}}

\KeywordTok{plot}\NormalTok{(pressure, infected)}
\end{Highlighting}
\end{Shaded}

In practice we would run the simulation many times for each transmission
pressure and plot (e.g.) the median number of infected cells found at
the end of each run of the model.

We can extend the model to include host immunity by specifying a new
layer of cells (i.e.~for host immunity) and modifying the CAM rule-set
appropriately.

Use the \texttt{fix()} function to edit the \texttt{cam.rule()}
function:

\begin{Shaded}
\begin{Highlighting}[]
\KeywordTok{fix}\NormalTok{(cam.rule)}
\end{Highlighting}
\end{Shaded}

Edit the function to read:

\begin{Shaded}
\begin{Highlighting}[]
\ControlFlowTok{function}\NormalTok{(V, N, S, E, W, x, y, t) \{}
\NormalTok{  tp <-}\StringTok{ }\KeywordTok{c}\NormalTok{(N, S, E, W) }\OperatorTok{*}\StringTok{ }\KeywordTok{rbinom}\NormalTok{(}\DecValTok{4}\NormalTok{, }\DecValTok{1}\NormalTok{, TP)}
\NormalTok{  cam.state.new}\OperatorTok{$}\NormalTok{grid[x, y] <<-}\StringTok{ }\KeywordTok{ifelse}\NormalTok{(V }\OperatorTok{==}\StringTok{ }\DecValTok{1} \OperatorTok{|}\StringTok{ }\NormalTok{(}\KeywordTok{sum}\NormalTok{(tp) }\OperatorTok{>}\StringTok{ }\DecValTok{0} \OperatorTok{&}\StringTok{ }\NormalTok{cam.state}\OperatorTok{$}\NormalTok{immune[x, y] }\OperatorTok{!=}\StringTok{ }\DecValTok{1}\NormalTok{), }\DecValTok{1}\NormalTok{, }\DecValTok{0}\NormalTok{)}
\NormalTok{\}}
\end{Highlighting}
\end{Shaded}

Once you have made the changes shown above, check your work, save the
file, and quit the editor.

We need to specify values for the transmission pressure (\texttt{TP})
and the proportion of the population that is immune (\texttt{IM}):

\begin{Shaded}
\begin{Highlighting}[]
\NormalTok{TP <-}\StringTok{ }\FloatTok{0.2}
\NormalTok{IM <-}\StringTok{ }\FloatTok{0.4}
\end{Highlighting}
\end{Shaded}

And define the initial state of the CAM:

\begin{Shaded}
\begin{Highlighting}[]
\NormalTok{cases <-}\StringTok{ }\KeywordTok{matrix}\NormalTok{(}\DecValTok{0}\NormalTok{, }\DataTypeTok{nrow =} \DecValTok{19}\NormalTok{, }\DataTypeTok{ncol =} \DecValTok{19}\NormalTok{)}
\NormalTok{cases[}\DecValTok{10}\NormalTok{, }\DecValTok{10}\NormalTok{] <-}\StringTok{ }\DecValTok{1}
\NormalTok{immune <-}\StringTok{ }\KeywordTok{matrix}\NormalTok{(}\KeywordTok{rbinom}\NormalTok{(}\DecValTok{361}\NormalTok{, }\DecValTok{1}\NormalTok{, IM), }\DataTypeTok{nrow =} \DecValTok{19}\NormalTok{, }\DataTypeTok{ncol =} \DecValTok{19}\NormalTok{)}
\NormalTok{immune[}\DecValTok{10}\NormalTok{,}\DecValTok{10}\NormalTok{] <-}\StringTok{ }\DecValTok{0}
\NormalTok{cam.state <-}\StringTok{ }\KeywordTok{list}\NormalTok{(}\DataTypeTok{grid =}\NormalTok{ cases, }\DataTypeTok{immune =}\NormalTok{ immune)}
\end{Highlighting}
\end{Shaded}

We can display the distribution of immune cells using the
\texttt{image()} function:

\begin{Shaded}
\begin{Highlighting}[]
\KeywordTok{image}\NormalTok{(cam.state}\OperatorTok{$}\NormalTok{immune, }\DataTypeTok{main =} \StringTok{"Immune"}\NormalTok{,}
      \DataTypeTok{col =} \KeywordTok{c}\NormalTok{(}\StringTok{"wheat"}\NormalTok{, }\StringTok{"navy"}\NormalTok{), }\DataTypeTok{axes =} \OtherTok{FALSE}\NormalTok{)}
\end{Highlighting}
\end{Shaded}

\includegraphics{prfe_files/figure-latex/unnamed-chunk-558-1.pdf}

We can now run the simulation:

\begin{Shaded}
\begin{Highlighting}[]
\KeywordTok{cam.run}\NormalTok{(}\DataTypeTok{steps =} \DecValTok{30}\NormalTok{)}
\end{Highlighting}
\end{Shaded}

The number of infected cells is:

\begin{Shaded}
\begin{Highlighting}[]
\KeywordTok{sum}\NormalTok{(cam.state}\OperatorTok{$}\NormalTok{grid)}
\end{Highlighting}
\end{Shaded}

\begin{verbatim}
## [1] 49
\end{verbatim}

A better summary is the proportion of susceptible (i.e.~non-immune)
cells that become infected during a run:

\begin{Shaded}
\begin{Highlighting}[]
\KeywordTok{sum}\NormalTok{(cam.state}\OperatorTok{$}\NormalTok{grid) }\OperatorTok{/}\StringTok{ }\NormalTok{(}\DecValTok{361} \OperatorTok{-}\StringTok{ }\KeywordTok{sum}\NormalTok{(cam.state}\OperatorTok{$}\NormalTok{immune))}
\end{Highlighting}
\end{Shaded}

\begin{verbatim}
## [1] 0.2300469
\end{verbatim}

We can use this model to investigate the effect of different proportions
of the population that are immune by systematically altering the value
assigned to \texttt{IM}:

\begin{Shaded}
\begin{Highlighting}[]
\NormalTok{immune.p <-}\StringTok{ }\KeywordTok{vector}\NormalTok{(}\DataTypeTok{mode =} \StringTok{"numeric"}\NormalTok{)}
\NormalTok{infected.p <-}\StringTok{ }\KeywordTok{vector}\NormalTok{(}\DataTypeTok{mode =} \StringTok{"numeric"}\NormalTok{)}

\ControlFlowTok{for}\NormalTok{(IM }\ControlFlowTok{in} \KeywordTok{seq}\NormalTok{(}\DecValTok{0}\NormalTok{, }\FloatTok{0.8}\NormalTok{, }\FloatTok{0.05}\NormalTok{)) \{}
\NormalTok{  cases <-}\StringTok{  }\KeywordTok{matrix}\NormalTok{(}\DecValTok{0}\NormalTok{, }\DataTypeTok{nrow =} \DecValTok{19}\NormalTok{, }\DataTypeTok{ncol =} \DecValTok{19}\NormalTok{)}
\NormalTok{  cases[}\DecValTok{10}\NormalTok{, }\DecValTok{10}\NormalTok{] <-}\StringTok{ }\DecValTok{1}
\NormalTok{  immune <-}\StringTok{ }\KeywordTok{matrix}\NormalTok{(}\KeywordTok{rbinom}\NormalTok{(}\DecValTok{361}\NormalTok{, }\DecValTok{1}\NormalTok{, IM), }\DataTypeTok{nrow =} \DecValTok{19}\NormalTok{, }\DataTypeTok{ncol =} \DecValTok{19}\NormalTok{)}
\NormalTok{  immune[}\DecValTok{10}\NormalTok{,}\DecValTok{10}\NormalTok{] <-}\StringTok{ }\DecValTok{0}
\NormalTok{  cam.state <-}\StringTok{ }\KeywordTok{list}\NormalTok{(}\DataTypeTok{grid =}\NormalTok{ cases, }\DataTypeTok{immune =}\NormalTok{ immune)}
  \KeywordTok{cam.run}\NormalTok{(}\DataTypeTok{steps =} \DecValTok{30}\NormalTok{)}
  \KeywordTok{graphics.off}\NormalTok{()}
\NormalTok{  immune.p <-}\StringTok{ }\KeywordTok{c}\NormalTok{(immune.p, }\KeywordTok{sum}\NormalTok{(cam.state}\OperatorTok{$}\NormalTok{immune) }\OperatorTok{/}\StringTok{ }\DecValTok{361}\NormalTok{)}
\NormalTok{  infected.p <-}\StringTok{ }\KeywordTok{c}\NormalTok{(infected.p,}
                  \KeywordTok{sum}\NormalTok{(cam.state}\OperatorTok{$}\NormalTok{grid) }\OperatorTok{/}\StringTok{ }\NormalTok{(}\DecValTok{361} \OperatorTok{-}\StringTok{ }\KeywordTok{sum}\NormalTok{(cam.state}\OperatorTok{$}\NormalTok{immune)))}
\NormalTok{\}}

\KeywordTok{plot}\NormalTok{(immune.p, infected.p)}
\end{Highlighting}
\end{Shaded}

In practice we would run the simulation many times for each value of
\texttt{IM} and plot (e.g.) the median proportion of susceptible
(i.e.~non-immune) cells that become infected.

A simple modification to this model would be to record the time-step at
which individual cells become infected. We can do this by adding a new
layer of cells (i.e.~to record the time-step at which a cell becomes
infected) and modifying the CAM rule-set appropriately.

Use the \texttt{fix()} function to edit the \texttt{cam.rule()}
function:

\begin{Shaded}
\begin{Highlighting}[]
\KeywordTok{fix}\NormalTok{(cam.rule)}
\end{Highlighting}
\end{Shaded}

Edit the function to read:

\begin{Shaded}
\begin{Highlighting}[]
\ControlFlowTok{function}\NormalTok{(V, N, S, E, W, x, y, t) \{}
\NormalTok{  tp <-}\StringTok{ }\KeywordTok{c}\NormalTok{(N, S, E, W) }\OperatorTok{*}\StringTok{ }\KeywordTok{rbinom}\NormalTok{(}\DecValTok{4}\NormalTok{, }\DecValTok{1}\NormalTok{, TP)}
\NormalTok{  cam.state.new}\OperatorTok{$}\NormalTok{grid[x, y] <<-}\StringTok{ }\KeywordTok{ifelse}\NormalTok{(V }\OperatorTok{==}\StringTok{ }\DecValTok{1} \OperatorTok{|}\StringTok{ }\NormalTok{(}\KeywordTok{sum}\NormalTok{(tp) }\OperatorTok{>}\StringTok{ }\DecValTok{0} \OperatorTok{&}\StringTok{ }\NormalTok{cam.state}\OperatorTok{$}\NormalTok{immune[x, y] }\OperatorTok{!=}\StringTok{ }\DecValTok{1}\NormalTok{), }\DecValTok{1}\NormalTok{, }\DecValTok{0}\NormalTok{)}
  \ControlFlowTok{if}\NormalTok{(V }\OperatorTok{!=}\StringTok{ }\DecValTok{1} \OperatorTok{&}\StringTok{ }\NormalTok{cam.state.new}\OperatorTok{$}\NormalTok{grid[x, y] }\OperatorTok{==}\StringTok{ }\DecValTok{1}\NormalTok{) \{}
\NormalTok{    cam.state.new}\OperatorTok{$}\NormalTok{ti[x, y] <<-}\StringTok{ }\NormalTok{t}
\NormalTok{  \}}
\NormalTok{\}}
\end{Highlighting}
\end{Shaded}

Once you have made the changes shown above, check your work, save the
file, and quit the editor.

We need to specify values for the transmission pressure (\texttt{TP})
and the proportion of the population that is immune (\texttt{IM}):

\begin{Shaded}
\begin{Highlighting}[]
\NormalTok{TP <-}\StringTok{ }\FloatTok{0.2}
\NormalTok{IM <-}\StringTok{ }\FloatTok{0.4}
\end{Highlighting}
\end{Shaded}

And define the initial state of the CAM:

\begin{Shaded}
\begin{Highlighting}[]
\NormalTok{cases <-}\StringTok{ }\KeywordTok{matrix}\NormalTok{(}\DecValTok{0}\NormalTok{, }\DataTypeTok{nrow =} \DecValTok{19}\NormalTok{, }\DataTypeTok{ncol =} \DecValTok{19}\NormalTok{)}
\NormalTok{cases[}\DecValTok{10}\NormalTok{, }\DecValTok{10}\NormalTok{] <-}\StringTok{ }\DecValTok{1}
\NormalTok{immune <-}\StringTok{ }\KeywordTok{matrix}\NormalTok{(}\KeywordTok{rbinom}\NormalTok{(}\DecValTok{361}\NormalTok{, }\DecValTok{1}\NormalTok{, IM), }\DataTypeTok{nrow =} \DecValTok{19}\NormalTok{, }\DataTypeTok{ncol =} \DecValTok{19}\NormalTok{)}
\NormalTok{immune[}\DecValTok{10}\NormalTok{,}\DecValTok{10}\NormalTok{] <-}\StringTok{ }\DecValTok{0}
\NormalTok{ti <-}\StringTok{ }\KeywordTok{matrix}\NormalTok{(}\OtherTok{NA}\NormalTok{, }\DataTypeTok{nrow =} \DecValTok{19}\NormalTok{, }\DataTypeTok{ncol =} \DecValTok{19}\NormalTok{)}
\NormalTok{ti[}\DecValTok{10}\NormalTok{,}\DecValTok{10}\NormalTok{] <-}\StringTok{ }\DecValTok{1}
\NormalTok{cam.state <-}\StringTok{ }\KeywordTok{list}\NormalTok{(}\DataTypeTok{grid =}\NormalTok{ cases, }\DataTypeTok{immune =}\NormalTok{ immune, }\DataTypeTok{ti =}\NormalTok{ ti)}
\end{Highlighting}
\end{Shaded}

We can now run the simulation:

\begin{Shaded}
\begin{Highlighting}[]
\KeywordTok{cam.run}\NormalTok{(}\DataTypeTok{steps =} \DecValTok{120}\NormalTok{)}
\end{Highlighting}
\end{Shaded}

Recording the time-step at which individual cells become infected allows
us to plot an epidemic curve from the model:

\begin{Shaded}
\begin{Highlighting}[]
\KeywordTok{x11}\NormalTok{()}
\KeywordTok{hist}\NormalTok{(cam.state}\OperatorTok{$}\NormalTok{ti)}
\end{Highlighting}
\end{Shaded}

\includegraphics{prfe_files/figure-latex/unnamed-chunk-570-1.pdf}

Remember, if you are using a Macintosh computer then you can use
\texttt{quartz()} instead of \texttt{x11()}. This will give better
results.

The \texttt{image()} function can provide an alternative view of the
same data:

\begin{Shaded}
\begin{Highlighting}[]
\KeywordTok{image}\NormalTok{(cam.state}\OperatorTok{$}\NormalTok{ti, }\DataTypeTok{axes =} \OtherTok{FALSE}\NormalTok{)}
\end{Highlighting}
\end{Shaded}

\includegraphics{prfe_files/figure-latex/unnamed-chunk-571-1.pdf}

The colour of each cell reflects the time-step of infection (i.e.~the
darker cells were infected before the lighter cells).

The CAM models that we have developed are general models of an
infectious phenomenon. They could, for example, be models of the spread
of an item of gossip, a forest fire, or an ink-spot. They are, however,
poorly specified models for the epidemic spread of an infectious
disease. In particular, they assume that a cell is infectious to other
cells immediately after infection, that an infected cell never loses its
ability to infect other cells, recovery never takes place, and immunity
is never acquired. These deficits in the models may be addressed by
appropriate modification of the CAM rule-set.

A simple and useful model of epidemic spread is the \emph{SIR} model.
The letters in \emph{SIR} refer to the three states that influence
epidemic spread that an individual can exist in. The three states are
\textbf{S}usceptible, \textbf{I}nfectious, and \textbf{R}ecovered.

We will now modify our CAM model to follow the \emph{SIR} model using
the following parameters:

\begin{itemize}
\item
  \textbf{Susceptible}: A cell may be immune or non-immune. A cell may
  be immune prior to the epidemic or acquire immunity fourteen
  time-steps after infection. Once a cell is immune it remains immune.
\item
  \textbf{Infectious}: An infected cell is infectious from eight to
  fourteen time-steps after being infected.
\item
  \textbf{Recovered}: A cell is clinically sick from ten to twenty
  time-steps after being infected. If one time-step is taken to equal
  one day, these parameters provide a coarse simulation of the course of
  a measles infection.
\end{itemize}

Use the \texttt{fix()} function to edit the \texttt{cam.rule()}
function:

\begin{Shaded}
\begin{Highlighting}[]
\KeywordTok{fix}\NormalTok{(cam.rule)}
\end{Highlighting}
\end{Shaded}

Edit the function to read:

\begin{Shaded}
\begin{Highlighting}[]
\ControlFlowTok{function}\NormalTok{(V, N, S, E, W, x, y, t) \{}
\NormalTok{  tsi <-}\StringTok{ }\NormalTok{t }\OperatorTok{-}\StringTok{ }\NormalTok{cam.state}\OperatorTok{$}\NormalTok{ti[x, y]}
\NormalTok{  cam.state.new}\OperatorTok{$}\NormalTok{grid[x, y] <<-}\StringTok{ }\KeywordTok{ifelse}\NormalTok{(tsi }\OperatorTok\StringTok{ }\NormalTok{INFECTIOUS, }\DecValTok{1}\NormalTok{, }\DecValTok{0}\NormalTok{)}
\NormalTok{  cam.state.new}\OperatorTok{$}\NormalTok{cf[x, y] <<-}\StringTok{ }\KeywordTok{ifelse}\NormalTok{(tsi }\OperatorTok\StringTok{ }\NormalTok{CLINICAL, }\DecValTok{1}\NormalTok{, }\DecValTok{0}\NormalTok{)}
\NormalTok{  cam.state.new}\OperatorTok{$}\NormalTok{immune[x, y] <<-}\StringTok{ }\KeywordTok{ifelse}\NormalTok{(}\OperatorTok{!}\KeywordTok{is.na}\NormalTok{(tsi) }\OperatorTok{&}\StringTok{  }\NormalTok{tsi }\OperatorTok{>}\StringTok{ }\NormalTok{IMMUNITY, }\DecValTok{1}\NormalTok{, cam.state}\OperatorTok{$}\NormalTok{immune[x,y])}
     
  \ControlFlowTok{if}\NormalTok{(cam.state}\OperatorTok{$}\NormalTok{infected[x, y] }\OperatorTok{==}\StringTok{ }\DecValTok{1}\NormalTok{) \{}
\NormalTok{    cam.state.new}\OperatorTok{$}\NormalTok{infected[x, y] <<-}\StringTok{ }\DecValTok{1}
\NormalTok{    cam.state.new}\OperatorTok{$}\NormalTok{ti[x, y] <<-}\StringTok{ }\NormalTok{cam.state}\OperatorTok{$}\NormalTok{ti[x, y]}
\NormalTok{  \} }\ControlFlowTok{else}\NormalTok{ \{}
\NormalTok{    tp <-}\StringTok{ }\KeywordTok{c}\NormalTok{(N, S, E, W) }\OperatorTok{*}\StringTok{ }\KeywordTok{rbinom}\NormalTok{(}\DecValTok{4}\NormalTok{, }\DecValTok{1}\NormalTok{, TP)}
    
    \ControlFlowTok{if}\NormalTok{(}\KeywordTok{sum}\NormalTok{(tp) }\OperatorTok{>}\StringTok{ }\DecValTok{0} \OperatorTok{&}\StringTok{ }\NormalTok{cam.state}\OperatorTok{$}\NormalTok{immune[x, y] }\OperatorTok{!=}\StringTok{ }\DecValTok{1}\NormalTok{) \{}
\NormalTok{      cam.state.new}\OperatorTok{$}\NormalTok{infected[x, y] <<-}\StringTok{ }\DecValTok{1}
\NormalTok{      cam.state.new}\OperatorTok{$}\NormalTok{ti[x, y] <<-}\StringTok{ }\NormalTok{t}
\NormalTok{    \}}
\NormalTok{  \} }
\NormalTok{\}}
\end{Highlighting}
\end{Shaded}

Once you have made the changes shown above, check your work, save the
file, and quit the editor.

It will also be useful to have a more detailed report of the state of
the CAM at each time-step.

Use the \texttt{fix()} function to edit the \texttt{cam.state.display()}
function:

\begin{Shaded}
\begin{Highlighting}[]
\KeywordTok{fix}\NormalTok{(cam.state.display)}
\end{Highlighting}
\end{Shaded}

Edit the function to read:

\begin{Shaded}
\begin{Highlighting}[]
\ControlFlowTok{function}\NormalTok{(t) \{}
  \ControlFlowTok{if}\NormalTok{(t }\OperatorTok{==}\StringTok{ }\DecValTok{0}\NormalTok{) \{}
    \KeywordTok{x11}\NormalTok{(}\DataTypeTok{width =} \DecValTok{9}\NormalTok{, }\DataTypeTok{height =} \DecValTok{9}\NormalTok{)}
    \KeywordTok{par}\NormalTok{(}\DataTypeTok{mfrow =} \KeywordTok{c}\NormalTok{(}\DecValTok{2}\NormalTok{, }\DecValTok{2}\NormalTok{))}
    \KeywordTok{par}\NormalTok{(}\DataTypeTok{pty =} \StringTok{"s"}\NormalTok{)}
\NormalTok{  \}}
     
  \KeywordTok{image}\NormalTok{(cam.state}\OperatorTok{$}\NormalTok{grid,}
        \DataTypeTok{main =} \KeywordTok{paste}\NormalTok{(}\StringTok{"Infectious at :"}\NormalTok{, t),}
        \DataTypeTok{col =} \KeywordTok{c}\NormalTok{(}\StringTok{"wheat"}\NormalTok{, }\StringTok{"navy"}\NormalTok{), }\DataTypeTok{axes =} \OtherTok{FALSE}\NormalTok{)}
     
  \KeywordTok{image}\NormalTok{(cam.state}\OperatorTok{$}\NormalTok{cf,}
        \DataTypeTok{main =} \KeywordTok{paste}\NormalTok{(}\StringTok{"Clinical at :"}\NormalTok{, t),}
        \DataTypeTok{col =} \KeywordTok{c}\NormalTok{(}\StringTok{"wheat"}\NormalTok{, }\StringTok{"navy"}\NormalTok{), }\DataTypeTok{axes =} \OtherTok{FALSE}\NormalTok{)}
     
  \KeywordTok{image}\NormalTok{(cam.state}\OperatorTok{$}\NormalTok{ti,}
        \DataTypeTok{main =} \KeywordTok{paste}\NormalTok{(}\StringTok{"Infected at :"}\NormalTok{, t), }\DataTypeTok{axes =} \OtherTok{FALSE}\NormalTok{)}
     
  \KeywordTok{image}\NormalTok{(cam.state}\OperatorTok{$}\NormalTok{immune,}
        \DataTypeTok{main =} \KeywordTok{paste}\NormalTok{(}\StringTok{"Immune at :"}\NormalTok{, t),}
        \DataTypeTok{col =} \KeywordTok{c}\NormalTok{(}\StringTok{"wheat"}\NormalTok{, }\StringTok{"navy"}\NormalTok{), }\DataTypeTok{axes =} \OtherTok{FALSE}\NormalTok{)}
\NormalTok{\}}
\end{Highlighting}
\end{Shaded}

Remember, if you are using a Macintosh computer then you can use
\texttt{quartz()} instead of \texttt{x11()}. This will give better
results.

Once you have made the changes shown above, check your work, save the
file, and quit the editor.

We need to specify values for the transmission pressure (\texttt{TP})
and the proportion of the population that is immune (\texttt{IM}):

\begin{Shaded}
\begin{Highlighting}[]
\NormalTok{TP <-}\StringTok{ }\FloatTok{0.2}
\NormalTok{IM <-}\StringTok{ }\FloatTok{0.2}
\end{Highlighting}
\end{Shaded}

And the \emph{SIR} parameters:

\begin{Shaded}
\begin{Highlighting}[]
\NormalTok{CLINICAL <-}\StringTok{ }\DecValTok{10}\OperatorTok{:}\DecValTok{20}
\NormalTok{INFECTIOUS <-}\StringTok{ }\DecValTok{8}\OperatorTok{:}\DecValTok{14}
\NormalTok{IMMUNITY <-}\StringTok{ }\DecValTok{14}
\end{Highlighting}
\end{Shaded}

And define the initial state of the CAM:

\begin{Shaded}
\begin{Highlighting}[]
\NormalTok{infected <-}\StringTok{ }\KeywordTok{matrix}\NormalTok{(}\DecValTok{0}\NormalTok{, }\DataTypeTok{nrow =} \DecValTok{19}\NormalTok{, }\DataTypeTok{ncol =} \DecValTok{19}\NormalTok{)}
\NormalTok{infected[}\DecValTok{10}\NormalTok{, }\DecValTok{10}\NormalTok{] <-}\StringTok{ }\DecValTok{1}
\NormalTok{ti <-}\StringTok{ }\KeywordTok{matrix}\NormalTok{(}\OtherTok{NA}\NormalTok{, }\DataTypeTok{nrow =} \DecValTok{19}\NormalTok{, }\DataTypeTok{ncol =} \DecValTok{19}\NormalTok{)}
\NormalTok{ti[}\DecValTok{10}\NormalTok{,}\DecValTok{10}\NormalTok{] <-}\StringTok{ }\DecValTok{0}
\NormalTok{infectious <-}\StringTok{ }\KeywordTok{matrix}\NormalTok{(}\DecValTok{0}\NormalTok{, }\DataTypeTok{nrow =} \DecValTok{19}\NormalTok{, }\DataTypeTok{ncol =} \DecValTok{19}\NormalTok{)}
\NormalTok{immune <-}\StringTok{ }\KeywordTok{matrix}\NormalTok{(}\KeywordTok{rbinom}\NormalTok{(}\DecValTok{361}\NormalTok{, }\DecValTok{1}\NormalTok{, IM), }\DataTypeTok{nrow =} \DecValTok{19}\NormalTok{, }\DataTypeTok{ncol =} \DecValTok{19}\NormalTok{)}
\NormalTok{immune[}\DecValTok{10}\NormalTok{,}\DecValTok{10}\NormalTok{] <-}\StringTok{ }\DecValTok{0}
\NormalTok{cf <-}\StringTok{ }\KeywordTok{matrix}\NormalTok{(}\DecValTok{0}\NormalTok{, }\DataTypeTok{nrow =} \DecValTok{19}\NormalTok{, }\DataTypeTok{ncol =} \DecValTok{19}\NormalTok{)}
\NormalTok{cam.state <-}\StringTok{  }\KeywordTok{list}\NormalTok{(}\DataTypeTok{grid =}\NormalTok{ infectious, }\DataTypeTok{infected =}\NormalTok{ infected, }\DataTypeTok{ti =}\NormalTok{ ti,}
                   \DataTypeTok{immune =}\NormalTok{ immune, }\DataTypeTok{cf =}\NormalTok{ cf)}
\end{Highlighting}
\end{Shaded}

We should also record the number of susceptible cells for later use:

\begin{Shaded}
\begin{Highlighting}[]
\NormalTok{susceptibles <-}\StringTok{ }\DecValTok{361} \OperatorTok{-}\StringTok{ }\KeywordTok{sum}\NormalTok{(cam.state}\OperatorTok{$}\NormalTok{immune)}
\end{Highlighting}
\end{Shaded}

We can now run the simulation:

\begin{Shaded}
\begin{Highlighting}[]
\KeywordTok{cam.run}\NormalTok{(}\DataTypeTok{steps =} \DecValTok{200}\NormalTok{)}
\end{Highlighting}
\end{Shaded}

We can now calculate the proportion of susceptible (i.e.~non-immune)
cells that become infected:

\begin{Shaded}
\begin{Highlighting}[]
\KeywordTok{sum}\NormalTok{(cam.state}\OperatorTok{$}\NormalTok{infected) }\OperatorTok{/}\StringTok{ }\NormalTok{susceptibles}
\end{Highlighting}
\end{Shaded}

\begin{verbatim}
## [1] 0.8585859
\end{verbatim}

We can use this model to test the effect of an intervention such as
isolating an infected cell for a short period after clinical features
first appear. We can do this, imperfectly because it does not allow us
to specify compliance, by shortening the infectious period to include
only the non-symptomatic time-steps and one time- step after clinical
features have appeared:

\begin{Shaded}
\begin{Highlighting}[]
\NormalTok{INFECTIOUS <-}\StringTok{ }\DecValTok{8}\OperatorTok{:}\DecValTok{11}
\end{Highlighting}
\end{Shaded}

resetting the initial state of the CAM:

\begin{Shaded}
\begin{Highlighting}[]
\NormalTok{infected <-}\StringTok{ }\KeywordTok{matrix}\NormalTok{(}\DecValTok{0}\NormalTok{, }\DataTypeTok{nrow =} \DecValTok{19}\NormalTok{, }\DataTypeTok{ncol =} \DecValTok{19}\NormalTok{)}
\NormalTok{infected[}\DecValTok{10}\NormalTok{, }\DecValTok{10}\NormalTok{] <-}\StringTok{ }\DecValTok{1}
\NormalTok{ti <-}\StringTok{ }\KeywordTok{matrix}\NormalTok{(}\OtherTok{NA}\NormalTok{, }\DataTypeTok{nrow =} \DecValTok{19}\NormalTok{, }\DataTypeTok{ncol =} \DecValTok{19}\NormalTok{)}
\NormalTok{ti[}\DecValTok{10}\NormalTok{,}\DecValTok{10}\NormalTok{] <-}\StringTok{ }\DecValTok{0}
\NormalTok{infectious <-}\StringTok{ }\KeywordTok{matrix}\NormalTok{(}\DecValTok{0}\NormalTok{, }\DataTypeTok{nrow =} \DecValTok{19}\NormalTok{, }\DataTypeTok{ncol =} \DecValTok{19}\NormalTok{)}
\NormalTok{immune <-}\StringTok{ }\KeywordTok{matrix}\NormalTok{(}\KeywordTok{rbinom}\NormalTok{(}\DecValTok{361}\NormalTok{, }\DecValTok{1}\NormalTok{, IM), }\DataTypeTok{nrow =} \DecValTok{19}\NormalTok{, }\DataTypeTok{ncol =} \DecValTok{19}\NormalTok{)}
\NormalTok{immune[}\DecValTok{10}\NormalTok{,}\DecValTok{10}\NormalTok{] <-}\StringTok{ }\DecValTok{0}
\NormalTok{cf <-}\StringTok{ }\KeywordTok{matrix}\NormalTok{(}\DecValTok{0}\NormalTok{, }\DataTypeTok{nrow =} \DecValTok{19}\NormalTok{, }\DataTypeTok{ncol =} \DecValTok{19}\NormalTok{)}
\NormalTok{cam.state <-}\StringTok{ }\KeywordTok{list}\NormalTok{(}\DataTypeTok{grid =}\NormalTok{ infectious, }\DataTypeTok{infected =}\NormalTok{ infected, }\DataTypeTok{ti =}\NormalTok{ ti,}
                  \DataTypeTok{immune =}\NormalTok{ immune, }\DataTypeTok{cf =}\NormalTok{ cf)}
\end{Highlighting}
\end{Shaded}

Recording the number of susceptible cells:

\begin{Shaded}
\begin{Highlighting}[]
\NormalTok{susceptibles <-}\StringTok{ }\DecValTok{361} \OperatorTok{-}\StringTok{ }\KeywordTok{sum}\NormalTok{(cam.state}\OperatorTok{$}\NormalTok{immune)}
\end{Highlighting}
\end{Shaded}

Running the simulation:

\begin{Shaded}
\begin{Highlighting}[]
\KeywordTok{cam.run}\NormalTok{(}\DataTypeTok{steps =} \DecValTok{200}\NormalTok{)}
\end{Highlighting}
\end{Shaded}

And recalculating the proportion of susceptible (i.e.~non-immune) cells
that become infected:

\begin{Shaded}
\begin{Highlighting}[]
\KeywordTok{sum}\NormalTok{(cam.state}\OperatorTok{$}\NormalTok{infected) }\OperatorTok{/}\StringTok{ }\NormalTok{susceptibles}
\end{Highlighting}
\end{Shaded}

\begin{verbatim}
## [1] 0.01754386
\end{verbatim}

We have developed a simple but realistic model of epidemic spread using
a cellular automata machine. Such a model could be used to investigate
the relative effect of model parameters (e.g.~initial proportion immune,
initial number of infective cells, transmission pressure, etc.) on
epidemic spread by changing a single parameter at a time and running the
simulation. Since the model is \emph{stochastic}, the effect of each
parameter change would be simulated many times and suitable summaries
calculated.

The model could be improved by, for example:

\begin{itemize}
\item
  Allowing for an \emph{open population} with births (or immigration)
  and deaths (or emigration). This could be implemented by allowing
  immune cells to become susceptible after a specified number of
  time-steps. The ratio of births to deaths could then be modelled as
  the ratio of the length of the infectious period to the length of the
  immune period.
\item
  Specifying a non-uniform distribution of transmission pressure during
  the infectious period.
\item
  Allowing for individual variation in susceptibility.
\item
  Allowing for individual variation in the duration of the infectious
  period.
\item
  Simulating a non-uniform population density by allowing cells to be
  empty. Note that an immune cell is the same as an empty cell in the
  current model.
\item
  Simulating a clustered distribution of immunity in the initial state
  of the CAM.
\item
  Allowing a small proportion of infections to be infectious without
  exhibiting clinical features.
\item
  Specifying rules that are applied at different time-points such as
  introducing isolation only after a certain number of clinical cases
  have appeared (i.e.~after an epidemic has been detected).
\item
  Allowing the coverage of interventions (e.g.~the coverage of a
  vaccination campaign or compliance with isolation instructions) that
  are introduced after an epidemic has been detected to be specified.
\item
  Simulating a more complex social structure. This could be implemented
  by allowing cells to belong to one of a finite set of castes with
  different initial conditions (e.g.~immunity) and having rules that
  specify the level of interaction (i.e.~the transmission pressure)
  between members of separate castes and the levels of intervention
  coverage achievable in the separate castes.
\item
  Extending the neighbourhood definition by using the corner
  (i.e.~north-east, south-east, south-west, and north-west) cells as
  neighbours for consideration in the CAM rule-set.
\end{itemize}

We can now quit \texttt{R}:

\begin{Shaded}
\begin{Highlighting}[]
\KeywordTok{q}\NormalTok{()}
\end{Highlighting}
\end{Shaded}

For this exercise there is no need to save the workspace image so click
the \textbf{No} button (GUI) or enter \texttt{n} when prompted to save
the workspace image (terminal).

\hypertarget{summary-8}{%
\section{Summary}\label{summary-8}}

\begin{itemize}
\item
  Computer intensive methods provide an alternative to classical
  statistical techniques for both estimation and statistical hypothesis
  testing. They have the advantage of being simple to implement and they
  remain simple even with complex estimators.
\item
  Computer based simulation can simulate both data and processes.
  Process simulations maybe arbitrarily complex. Process simulation is a
  useful development tool that can save considerable time and expense
  when developing systems and methods.
\item
  Process simulation can also be used to model complex social phenomena
  such as epidemic spread. Such simulations allow (e.g.) the the
  relative efficacy of interventions to be evaluated.
\item
  \texttt{R} provides functions that allow you to implement computer
  intensive methods such as the bootstrap and computer based simulation.
\end{itemize}

\hypertarget{what-now}{%
\chapter*{What now?}\label{what-now}}
\addcontentsline{toc}{chapter}{What now?}

Now that you have had a taste of using \texttt{R} you will be able to
decide whether it meets your requirements for a data-analysis system.

The file \texttt{R-intro.pdf} which is installed with the \texttt{R}
system contains the document \emph{`An Introduction to \texttt{R}'}.
This document provides a solid introduction to \texttt{R}.

The file \texttt{refman.pdf} which is also installed with the \texttt{R}
system contains the document \emph{`The \texttt{R} reference index'}.
This document provides a complete function-by-function reference to the
\texttt{R} base system and several standard function libraries
(packages).

Other documents are available from the \texttt{R} Website:
\url{http://www.r-project.org/}

\bibliography{book.bib}


\end{document}
